\index{CMP}

This chapter describes the configuration of a chip multiprocessor
(CMP) version of JOP. The various SimpCon based arbiters have been
developed by Christof Pitter and are described in \cite{jop:dma,
jop:cmp, jop:cmp:eval}.

The project file to start with is \code{cyccmp}, a configuration for
three processors with a TDMA based arbiter in the Cycore board with
the EP1C12.

\section{Booting a CMP System}

\index{CMP!booting}

One interesting issue for a CMP system is the question how the
startup or boot-up is performed. Before we explain the CMP solution,
we need an understanding of the boot-up sequence of JOP in an FPGA.
On power-up, the FPGA starts the configuration state machine to read
the FPGA configuration data either from a Flash or via a download
cable (for development). When the configuration has finished an
internal reset is generated. After that reset, microcode instructions
are executed starting from address 0. At this stage, we have not yet
loaded any application program (Java bytecode). The first sequence in
microcode performs this task. The Java application can be loaded from
an external Flash or via a serial line (or USB port) from a PC. The
microcode assembly configured the mode. Consequently, the Java
application is loaded into the main memory. To simplify the startup
code we perform the rest of the startup in Java itself, even when
some parts of the JVM are not yet setup.

In the next step, a minimal stack frame is generated and the special
method \code{Startup.boot()} is invoked. From now on JOP runs in Java
mode. The method \code{boot()} performs the following steps:
\begin{samepage}
\begin{itemize}
    \item Send a greeting message to \emph{stdout}
    \item Detect the size of the main memory
    \item Initialize the data structures for the garbage
        collector
    \item Initialize \code{java.lang.System}
    \item Print out JOP's version number, detected clock speed,
        and memory size
    \item Invoke the static class initializers in a predefined
        order
    \item Invoke the \code{main} method of the application class
\end{itemize}
\end{samepage}

The boot-up process is the same for all processors until the
generation of the internal reset and the execution of the first
microcode instruction. From that point on, we have to take care that
\emph{only one} processor performs the initialization steps.

All processors in the CMP are functionally identical. Only one
processor is designated to boot-up and initialize the whole system.
Therefore, it is necessary to distinguish between the different CPUs.
We assign a unique CPU identity number (CPU ID) to each processor.
Only processor CPU0 is designated to do all the boot-up and
initialization work. The other CPUs have to wait until CPU0 completes
the boot-up and initialization sequence. At the beginning of the
booting sequence, CPU0 loads the Java application. Meanwhile, all
other processors are waiting for an \emph{initialization finished}
signal of CPU0. This busy wait is performed in microcode. When the
other CPUs are enabled, they will run the same sequence as CPU0.
Therefore, the initialization steps are guarded by a condition on the
CPU ID.

In our current prototype, we let all additional CPUs also invoke the
main method of the application. This is a shortcut for a simple
evaluation of the system\footnote{In the main method we execute
different applications based on the CPU ID.}. In a future version,
the additional CPUs will invoke a system method to be integrated into
the normal scheduling system.


\section{CMP Scheduling}

\index{CMP!scheduling}

There are two possibilities to run multiple threads on the CMP
system:

\begin{enumerate}
  \item A single thread per processor
  \item Several threads on each processor
\end{enumerate}

For the configuration of one thread per processor the scheduler does
not need to be started. Running several threads on each core is
managed via the JOP real-time threads \code{RtThread}. On each core a
local preemptive, priority based scheduler runs. Threads cannot
migrate from one core to another one.

\subsection{One Thread per Core}


The first processor executes, as usual, \code{main()}. To execute
code on the other cores a \code{Runnable} has to be registered for
each core. After registering those Runnables the other cores need to
be started. The following code shows an example that can be found in
\dirent{test/cmp/HelloCMP.java}.

\begin{verbatim}
public class HelloCMP implements Runnable {

    int id;

    static Vector msg;

    public HelloCMP(int i) {
        id = i;
    }

    /**
     * @param args
     */
    public static void main(String[] args) {

        msg = new Vector();

        System.out.println("Hello World from CPU 0");

        SysDevice sys = IOFactory.getFactory().getSysDevice();
        for (int i=0; i<sys.nrCpu-1; ++i) {
            Runnable r = new HelloCMP(i+1);
            Startup.setRunnable(r, i);
        }

        // start the other CPUs
        sys.signal = 1;

        // print their messages
        for (;;) {
            int size = msg.size();
            if (size!=0) {
                StringBuffer sb = (StringBuffer) msg.remove(0);
                System.out.println(sb);
            }
        }
    }

    public void run() {

        StringBuffer sb = new StringBuffer();
        sb.append("Hello World from CPU ");
        sb.append(id);
        msg.addElement(sb);
    }

}
\end{verbatim}

\subsection{Scheduling on the CMP System}

Running several threads on each core is possible with \code{RtThread}
and setting the core for each thread with
\code{RtThread.setProcessor(nr)}. The following example
(\code{test/cmp/RtHelloCMP.java}) shows registering 50 threads on
three cores. On \code{missionStart()} the threads are distributed to
the cores, a scheduler for each core registered as timer interrupt
handler, and the other cores started.

\begin{verbatim}
public class RtHelloCMP extends RtThread {

    public RtHelloCMP(int prio, int us) {
        super(prio, us);
    }

    int id;

    public static Vector msg;

    final static int NR_THREADS = 50;

    /**
     * @param args
     */
    public static void main(String[] args) {

        msg = new Vector();

        System.out.println("Hello World from CPU 0");

        SysDevice sys = IOFactory.getFactory().getSysDevice();
        for (int i=0; i<NR_THREADS; ++i) {
            RtHelloCMP th = new RtHelloCMP(1, 1000*1000);
            th.id = i;
            th.setProcessor(i%sys.nrCpu);
        }

        System.out.println("Start mission");
        // start mission and other CPUs
        RtThread.startMission();
        System.out.println("Mission started");

        // print their messages
        for (;;) {
            RtThread.sleepMs(5);
            int size = msg.size();
            if (size!=0) {
                StringBuffer sb = (StringBuffer) msg.remove(0);
                // System.out.print(sb);
                // converts StringBuffer to a String and creates garbage
                for (int i=0; i<sb.length(); ++i) {
                    System.out.print(sb.charAt(i));
                }
            }
        }
    }

    public void run() {

        StringBuffer sb = new StringBuffer();
        StringBuffer ping = new StringBuffer();

        sb.append("Thread ");
        sb.append((char) ('A'+id));
        sb.append(" start on CPU ");
        sb.append(IOFactory.getFactory().getSysDevice().cpuId);
        sb.append("\r\n");
        msg.addElement(sb);
        waitForNextPeriod();

        for (;;) {
            ping.setLength(0);
            ping.append((char) ('A'+id));
            msg.addElement(ping);
            waitForNextPeriod();
        }
    }

}
\end{verbatim}
