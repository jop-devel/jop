
\documentclass[%draft,
    11pt, % use explicit paper size
    headinclude, footexclude,
    twoside, % this produces strange margins!
    openright, % for new chapters
    notitlepage,
    cleardoubleempty,
    headsepline,
    pointlessnumbers,
    bibtotoc, idxtotoc,
    ]{scrbook}


% that's for virtualbookworm trim size
%\setlength{\paperwidth}{6in} \setlength{\paperheight}{9in}
% that's for CreateSpace trim size
\setlength{\paperwidth}{7.5in} \setlength{\paperheight}{9.25in}
\usepackage{pslatex} % -- times instead of computer modern
% pslatex should be replaced by this:
%\usepackage{mathptmx}
%\usepackage[scaled=.70]{helvet}
%\usepackage{courier}
% pslatex does not work with T1 encoding. <> Problem?

% typeare{calc} without BCOR results to a DIV of 8 for 11pt
%\typearea[0.465in]{15}
% the decision is 14
\typearea[0.5in]{14}
%\typearea[0.535in]{13}
%\typearea[0.57in]{12}

% headings
\usepackage{scrpage2} % for headers
 \setkomafont{pagehead}{\scshape\small}
 \setkomafont{pagenumber}{\scshape\small}
 \automark[section]{chapter}
 \ohead[]{\pagemark}
 \chead[]{}
 \ihead[]{\headmark}
 \ofoot[]{} \cfoot[]{} \ifoot[]{}

%\tolerance=500 % to avoid lines sticking out into the margin
               % needed for 'high-performance' in Intro - contributions
\emergencystretch=2em
% or tol. to 500 and emerg. to 1em?
% pagebreak was ok with 500 and 1em
\interfootnotelinepenalty=10000


% use BCOR = (paperwidth-textwidth)/4
% A4: 210mm x 297mm
% B5: 176mm x 250mm
% Java book: 185mm x 232mm
% Engblom: 120x188 (without head)
% Java: 127x187 (without head)
% 1pt = 1/72.27 in = 0.351 mm
% Thesis is 232mm x 185mm (9,1in x 7,28in) with 11pt
% Springer is 9,2in x 6in with 10pt
% Altera Quaruts handbook is 9in x 7in with 9pt


% for book
% 'Java-format' 526pt x 660pt (Ghostscript)
%\setlength{\paperwidth}{185mm} \setlength{\paperheight}{232mm}
% use that BCOR setting with twoside to compensate the margin
%\areaset[13.75mm]{130mm}{200mm} % Java book format



% use 10pt for code instead of 11pt - but I still would prefer Lucida Typewriter
%\newfont{\myttfont}{cmss10 scaled 1000}
%\newfont{\myttbfont}{cmssdc10 scaled 1000}
%
% This IS Lucida Typewriter
%\newfont{\myttfont}{plsr8r scaled 950}
%\newfont{\myttbfont}{plsb8r scaled 950}
%\newfont{\myttifont}{plsro8r scaled 950}
%%\newfont{\mytttextfont}{plsr8r}

% Lucida is perhaps available in the new Tex installation!!!!
% does not really work!!!
%\newfont{\myttfont}{hlsrt8r scaled 950}
%\newfont{\myttbfont}{hlsbt8r scaled 950}
%\newfont{\myttifont}{hlsrot8r scaled 950}

% I used these .ttf for the official Thesis
%..\ttf2pt1 -e -b LucidaTypewriterRegular.ttf plsr8a
%..\ttf2pt1 -e -b LucidaTypewriterBold.ttf plsb8a
%..\ttf2pt1 -e -b LucidaTypewriterOblique.ttf plsro8a
%..\ttf2pt1 -e -b LucidaTypewriterBoldOblique.ttf plsbo8a

%\newcommand{\javatt}{\myttfont}
%\newcommand{\javattb}{\myttbfont}
%\newcommand{\javatti}{\myttifont}
%\newcommand{\javatext}{\myttfont}
%
%\newcommand{\picscale}{0.909}
%\newcommand{\excelwidth}{11cm}

% end book

% for B5
%\newfont{\javatt}{cmss10}
%\newfont{\javattb}{cmssdc10}
%\newcommand{\picscale}{0.833}
%\newcommand{\excelwidth}{10cm}


%% was set to 9 for the green book
%% should be ok with 11 for the orange book
%\newfont{\javatt}{cmss11}
%\newfont{\javattb}{cmssdc11}
%% TODO find an italic
%\newfont{\javatti}{cmss11}
%\newcommand{\javatext}{\javatt}

\newcommand{\picscale}{1}
\newcommand{\excelwidth}{12cm}

% for SimpCon description
\newcommand{\sign}[1]{{\code{#1}}}

% for chapter head without a number
% \renewcommand{\chaptermark}[1]{\def\myleftmark{#1}}
% \ihead{\myleftmark} \chead{} \ohead{{\rightmark}}

\setkomafont{captionlabel}{\sffamily\bfseries}

\usepackage{latexsym}
\usepackage{graphicx}
\usepackage{amsmath}
\usepackage{amsthm}
\usepackage{longtable}
\usepackage{booktabs}

% I would need Lucida Console!!!
%
%\newfont{\javatt}{pltt12} % lucida teletype, better than normal but with serifs
%\newfont{\javatt}{plss12} % lucida no serifes, but variable spacing
%\newfont{\javatt}{plss10 scaled 1200}
%\newfont{\javattb}{plssdc10 scaled 1200}
% cmss is NOT a tt font....

\usepackage{listings}
%\lstset{language=Java,keywordstyle=,
%basicstyle=\small\javatt,emphstyle=\small\javattb,commentstyle=\small\javatti,
%%basicstyle=\small,
%%
%showstringspaces=false,captionpos=b,columns=flexible}
%%\lstset{basicstyle=\small,language=Java,columns=flexible}

\lstset{basicstyle=\sffamily\small,keywordstyle=\sffamily\small,language=Java,captionpos=b,columns=flexible,showstringspaces=false}

\usepackage{array}
\usepackage{dcolumn}
\newcommand{\cc}[1]{\multicolumn{1}{c}{#1}}
\newcolumntype{d}[1]{D{.}{.}{#1}}

\usepackage{capt-of}
\usepackage[colorlinks=true,urlcolor=black,linkcolor=black,citecolor=black]{hyperref}

% ----------------------

\usepackage{makeidx}
\makeindex


\usepackage{import} % for subimport text and graphics from subdirectory
% does not work with latex2html!


%\newcommand{\codefoot}{\textsf}
%\newcommand{\code}[1]{{\javatext#1}}
%\newcommand{\codeb}[1]{{\javattb#1}}

\newcommand{\codefoot}{\sffamily}
\newcommand{\code}[1]{{\small\sffamily{#1}}}


%\newcommand{\cmd}[1]{{\texttt{#1}}}
%\newcommand{\dirent}[1]{{\texttt{#1}}}
%\newcommand{\menuitem}[1]{\textsf{\textbf{#1}}}
\newcommand{\cmd}[1]{{\code{#1}}\index{#1}}
\newcommand{\cmdb}[1]{{\textsf{\textbf{#1}}}\index{#1}}
\newcommand{\dirent}[1]{{\code{#1}}}
\newcommand{\menuitem}[1]{\textsf{\textsl{#1}}}

\newcommand{\idx}[1]{#1\index{#1}}

% for flow.tex - part of index helper
\newcommand{\eei}[1]{%
\index{extension!\texttt{#1}}\texttt{#1}}

% JVs et al
%\newcommand{\ea}{et al.\xspace}
\newcommand{\ea}{et al.\ }

% used in GC period calculation
\newtheorem{lemma}{Lemma}
\newtheorem{theorem}{Theorem}

%\begin{htmlonly}
%\renewcommand{\code}[1]{{\texttt{#1}}} % for html2LaTeX
%\newcommand{\toprule}{\hline}
%\newcommand{\midrule}{\hline}
%\newcommand{\bottomrule}{\hline}
%\end{htmlonly}

% net wirklich notwendig -- h�ngt von code generierung ab
%\begin{htmlonly}
%\renewcommand{\javatt}{\texttt}
%\renewcommand{\javattb}{\texttt\bfseries}
%\end{htmlonly}

%\code{\hyphenchar\font=-1}

\newcommand{\mycomment}[1]{}

\newcommand{\instr}[6]{
    \begin{table}
    \index{microcode!{\textsf{#1}}}
        \begin{tabular}{ll}
            \emph{\large\textbf{#1}} & \\
            \\ \\
            \textbf{Operation} & #2 \\ \\
            \textbf{Opcode} & \texttt{#3} \\ \\
            \textbf{Dataflow} & \parbox[t]{10.5cm}{\(#4\)}\\ \\
            \textbf{JVM equivalent} & \parbox[t]{10.5cm}{\code{#5}} \\ \\
            \textbf{Description} & \parbox[t]{10.5cm}{#6}\\
        \end{tabular}
    \end{table}
}


\def\bl{\mbox{}\newline\mbox{}\newline{}}
\usepackage{ifthen}
\newcommand{\hide}[2]{
\ifthenelse{\equal{#1}{inherited}}%
{}%
{}%
}
\newcommand{\entityintro}[3]{%
  \hbox to \hsize{%
    \vbox{%
      \hbox to .2in{}%
    }%
    {\bf #1}%
    \dotfill\pageref{#2}%
  }
  \makebox[\hsize]{%
    \parbox{.4in}{}%
    \parbox[l]{5in}{%
      \vspace{1mm}\it%
      #3%
      \vspace{1mm}%
    }%
  }%
}
\newcommand{\isep}[0]{%
\setlength{\itemsep}{-.4ex}
}
\newcommand{\sld}[0]{%
\setlength{\topsep}{0em}
\setlength{\partopsep}{0em}
\setlength{\parskip}{0em}
\setlength{\parsep}{-1em}
}
\newcommand{\headref}[3]{%
\ifthenelse{#1 = 1}{%
\addcontentsline{toc}{section}{\hspace{\qquad}\protect\numberline{}{#3}}%
}{}%
\ifthenelse{#1 = 2}{%
\addcontentsline{toc}{subsection}{\hspace{\qquad}\protect\numerline{}{#3}}%
}{}%
\ifthenelse{#1 = 3}{%
\addcontentsline{toc}{subsubsection}{\hspace{\qquad}\protect\numerline{}{#3}}%
}{}%
\label{#3}%
\makebox[\textwidth][l]{#2 #3}%
}%
\newcommand{\membername}[1]{{\it #1}\linebreak}
\newcommand{\divideents}[1]{\vskip -1em\indent\rule{2in}{.2mm}}
\newcommand{\refdefined}[1]{
\expandafter\ifx\csname r@#1\endcsname\relax
\relax\else
{$($ in \ref{#1}, page \pageref{#1}$)$}
\fi}
\newcommand{\startsection}[4]{
\gdef\classname{#2}
\subsection{\label{#3}{\bf {\sc #1} #2}}{
\rule[1em]{\hsize}{1.5pt}\vskip -1em \vskip .1in #4
}%
}
\newcommand{\startsubsubsection}[2]{
\subsubsection{\sc #1}{%
\rule[1em]{\hsize}{1pt}%
#2}
}
\usepackage{color}
\addtocontents{toc}{\protect\def\protect\packagename{}}
\addtocontents{toc}{\protect\def\protect\classname{}}
\markboth{\protect\packagename --
\protect\classname}{\protect\packagename -- \protect\classname}
% \topmargin -.8in
\chardef\bslash=`\\



\begin{document}

\title{JOP User Manual}
\author{Martin Schoeberl\\martin@jopdesign.com}
\maketitle \thispagestyle{empty}

%\input{intro/title}


%\input{intro/abstract}

\pagestyle{scrheadings} \pagenumbering{roman}

\tableofcontents \cleardoublepage
 \listoftables \newpage \listoffigures
 \newpage \lstlistoflistings \newpage

\chapter*{Foreword}

This book is about JOP, the Java Optimized Processor. JOP began as
research project for a PhD thesis. JOP has been used in several
industrial applications and, due to the fact that it is an
open-source project, has a growing user base. This book is written
for all of you who build this lively community. The book is based to
some extent on the PhD thesis. For a long time the thesis, some
research papers, and the web site  have been the only available
documentation for JOP. A thesis is quite different to a user manual.
Its focus is on research results and implementation details are
usually omitted. This book fills the gap and provides inside into
the implementation of JOP and the accompanying Java virtual machine
(JVM). It also gives you an idea how to build an embedded real-time
system based on JOP.


\pagestyle{scrheadings} \pagenumbering{arabic}


%-----------------------------------------------------------------
% here we start to work
%-----------------------------------------------------------------

\chapter{Introduction}
\label{chap:intro}

\emph{Copy some stuff from the thesis and rewrite it.}
    

This handbook introduces the concept of a Java processor for
embedded real-time systems, in particular the design of a small
processor for resource-constrained devices with time-predictable
execution of Java programs. This Java processor is called JOP --
which stands for Java Optimized Processor --, based on the
assumption that a full native implementation of all Java bytecode
instructions is not a useful approach.

%\section{Motivation}
\section{Justification for Development}

To justify Java's use in embedded real-time systems we quote from a
document published by the National Institute of Standards and
Technology \cite{nist99}:

\begin{itemize}
    \item Java's higher level of abstraction allows for increased programmer
productivity (although recognizing that the tradeoff is runtime
efficiency)
    \item Java is relatively easier to master than C++
    \item Java is relatively secure, keeping software components (including
the JVM itself) protected from one another
    \item Java supports dynamic loading of new classes
    \item Java is highly dynamic, supporting object and thread creation at
runtime
    \item Java is designed to support component integration and reuse
    \item The Java technologies have been developed with careful
consideration, erring on the conservative side using concepts and
techniques that have been scrutinized by the community
    \item The Java programming language and Java platforms support
application portability
    \item The Java technologies support distributed applications
    \item Java provides well-defined execution semantics
\end{itemize}

Based on the NIST document, the Real-Time for Java Experts Group has
published the Real Time Specification for Java (RTSJ) \cite{rtsj} to
add real-time extensions to Java.

Despite the above, to date Java is rarely used in embedded real-time
systems. High resource requirements for the Java virtual machine and
unpredictable real-time behavior are the main issues surrounding the
use of Java for embedded systems. JOP addresses both issues, and the
proposed Java processor makes a strong case for the use of Java in
embedded systems.

\section{Embedded Real-Time Systems}

An embedded system is a special-purpose computer system that is part
of a larger system or machine. An embedded system is designed to
perform a narrow range of functions with no, or minimal user
intervention.

Since many embedded systems are produced in large quantities, the
need to reduce costs is a major concern. Embedded systems often have
significant energy constraints, and many are battery-powered. As a
result of these constraints, embedded systems use a slow processor
and small memory size to minimize costs and energy consumption.

Embedded systems interact with the environment and often have to
produce output within a given timeframe. Therefore, most embedded
systems are real-time systems. Here is a general definition of a
real-time system (John A. Stankovic \cite{50811}):

\begin{quote}
In real-time computing the correctness of the system depends not
only on the logical result of the computation but also on the time
at which the result is produced.
\end{quote}

%(Donald Gillies \cite{rt:definition}):
%% realtime FAQ - Donald Gillies
%A real-time system is one in which the correctness of the
%computations not only depends upon the logical correctness of the
%computation but also upon the time at which the result is produced.
%If the timing constraints of the system are not met, system failure
%is said to have occurred.
% Donald W. Gillies website
%In a hard real-time system the correctness of a computation depends
%not only on the computed results but also on the time at which they
%are produced. A result produced on or after the deadline is
%typically useless.
%
However, it should be noted that `real-time' does not mean `really
fast'. In pure real-time systems (i.e.\ without non real-time
tasks), there is no additional value in producing results earlier
than required.

Embedded real-time systems often have to handle concurrent tasks,
such as communication, calculating values for a control loop, user
interface and supervision. A natural way to handle these concurrent
jobs is to model them as individual tasks. These tasks are executed
on a preemptive multi-tasking system. Each task is assigned a
priority and the multi-tasking system is responsible for scheduling
individual tasks according to their priority.
%A schedulability test shows that all tasks would meet their deadlines.

%\subsection{Scheduling}

To fulfil the time constraints for a real-time system, an
appropriate schedule needs to be found. This problem was solved in
the classic paper by Liu and Layland \cite{321743} on independent
periodic tasks. The optimal priority assignment for a set of tasks
is called the rate monotonic priority order, in which a task with a
shorter period is assigned a higher priority. If the Worst-Case
Execution Time (WCET) of each task is known, the schedule is
feasible and all tasks will meet their deadline\footnote{The period
of a periodic task is the time between consecutive activations of
the task. The deadline of the task is assumed to be at the end of
the tasks period.}, if:

\begin{samepage}
\begin{equation}
\nonumber
    \frac{C_1}{T_1}+\dots+\frac{C_n}{T_n} \le U(n) = n(2^{\frac{1}{n}}-1)
\end{equation}
%
where
\begin{equation}
\nonumber
    \begin{split}
        C_i & = \mbox{worst-case execution time of } task_i \\
        T_i & = \mbox{period of } task_i \\
        U(n) & = \mbox{utilization bound for $n$ tasks.}
    \end{split}
\end{equation}
\end{samepage}
%
In theory, this test is both elegant and simple. For concrete
systems, two issues have to be solved:
%
\begin{itemize}
    \item There are very few systems in existence that do not require
    communication between tasks.
    As a result, tasks cannot be seen as independent and blocking
    needs to be incorporated into the schedulability analysis.
    \item The WCET of each task has to be known. This is not a
    trivial task. Simple measurements of execution times never fully
    guarantee a correct value. The tasks therefore have to be analyzed
    using  the correct model of the target system. It is almost
    impossible to provide an accurate and correct model of modern
    processors and memory systems.
\end{itemize}
%
Several standard textbooks on real-time systems \cite{
book:klein-real-time-analysis-ratetm, 558498} deal with the first
issue. JOP is intended to resolve the second issue. It should be
noted that there are a number of scheduling approaches and
schedulability tests. However, as a rule, these approaches all
assume that the WCET of each task is known.


\section{Research Objectives and Contributions}


This book presents a hardware implementation of the Java Virtual
Machine (JVM), targeting small embedded systems with real-time
constraints. The processor is designed from the ground up for low
WCET of bytecodes, in order to give tasks low WCET values. The
following list summarizes the research objectives for the proposed
Java processor:
%
\paragraph{Primary Objectives:}
    \begin{itemize}
        \item Time-predictable Java platform for embedded real-time
        systems
        \item Small design that fits into a low-cost FPGA
        \item A working processor, not merely a proposed architecture
    \end{itemize}
\paragraph{Secondary Objectives:}
    \begin{itemize}
        \item Acceptable performance compared with mainstream non
        real-time Java systems
        \item A flexible architecture that allows different
        configurations for different application domains
        \item Definition of a real-time profile for Java
    \end{itemize}

\subsubsection{Contributions:}

JOP is a stack computer with its own instruction set, called
microcode in this book. Java bytecodes are translated into microcode
instructions or sequences of microcode. The difference between the
JVM and JOP is best described as the following:
\begin{quote}
The JVM is a CISC stack architecture, whereas JOP is a RISC stack
architecture.
\end{quote}


JOP will help to increase the acceptance of Java for embedded
real-time systems. JOP is implemented as a soft-core in a Field
Programmable Gate Array (FPGA). Using an FPGA as the processor for
embedded systems is uncommon, because of the high costs, compared
with a microcontroller. However, if the core is small enough, unused
FPGA resources can be used to implement periphery in the FPGA,
resulting in a lower chip count and hence lower overall costs.

The main contributions are as follows:

\begin{itemize}

    \item
The execution time for Java bytecodes can be exactly predicted in
terms of the number of clock cycles.
%The execution time for Java bytecodes is known cycle-accurate.
There is no mutual dependency between consecutive bytecodes.
Therefore, no pipeline analysis -- with possible unbound timing
effects -- is necessary. These properties greatly simplify low-level
WCET analysis.

In order to fill the gap between processor speed and the memory
access time, caches are mandatory. In Section~\ref{sec:cache}, a
novel way to organize an instruction cache, as \emph{method cache},
is provided. This method cache is simple to analyze with respect to
worst-case behavior and still provides a substantial performance
gain when compared against a solution without an instruction cache.

The proposed processor architecture results in a predictable and
high-performance
% fast
execution of real-time tasks in Java, without the resource
implications and unpredictability of a JIT-compiler.

    \item
JOP is microprogrammed using a novel way of mapping bytecodes to
microcode addresses. This mapping has zero overheads, even for
complex bytecodes.

A two-level stack cache, described in Section~\ref{sec:stack}, which
fits to the embedded memory technologies of current FPGAs and ASICs,
ensures the fast execution of basic instructions with minimum
resource requirements. Fill and spill of the stack cache is
subjected to microcode control and therefore time-predictable.

JOP is the smallest hardware implementation of the JVM available to
date. This fact enables low-cost FPGAs to be used in embedded
systems. The resource usage of JOP can be configured to trade size
against performance for different application domains.

    \item
The definition of standard Java does not fit hard real-time
applications. Therefore, a real-time profile for Java (with
restrictions) is defined in Section~\ref{sec:rtprof} and implemented
on JOP. Tight integration of the scheduler and the hardware that
generates schedule events results in low latency and low jitter of
the task dispatch.

In this profile, hardware interrupts are represented as asynchronous
events with associated threads. These events are subject to the
control of the scheduler and can be incorporated into the priority
assignment and schedulability analysis in the same way as normal
application tasks.

    \item
One contribution made as part of this work is the concrete
implementation of the proposed architecture. The author is aware
that it is not usually considered necessary to provide a complete
implementation in a research project. However, it is the opinion of
the author that a simulation-only approach would lead to mistakes or
small glitches. By providing a concrete implementation, we are not
only confronted with the full complexity of real-life processes, but
also with one or more major issues that would often be generously
overlooked in a simulation. In Section~\ref{sec:applications}, the
usage of JOP in a real-world application is described.

\end{itemize}

\section{Outline of the Book}

Chapter~\ref{chap:java} provides background information on the Java
programming language and the execution environment, the Java virtual
machine, for Java applications.

The related work is presented in Chapter~\ref{chap:related}.
Different hardware solutions from both academia and industry for
accelerating Java in embedded systems are analyzed. This chapter
concludes with the research question.

Standard Java is not suitable for the resource-constrained world of
embedded systems. Chapter~\ref{chap:rtjava} gives an overview of the
different restrictions of Java for embedded and real-time systems.

Chapter~\ref{chap:arch} is the main chapter in which the
architecture of JOP is described. The motivation behind different
design decisions is given.

A Java processor alone is not a complete JVM.
Chapter~\ref{chap:runtime} describes the runtime environment on top
of JOP, including the definition of a real-time profile for Java and
a framework for a user-defined scheduler in Java.

In Chapter~\ref{chap:results}, JOP is evaluated with respect to
size, performance and WCET. This is followed by a description of the
first commercial real-world application of JOP.

Finally, in Chapter~\ref{chap:conclusions}, the work undertaken is
reviewed and the major contributions are presented. This chapter
concludes with directions for future research using JOP and
real-time Java.


\section{Is JOP the Solution for Your Problem?}

\emph{TODO: move is to the correct place.}

I had a lot of fun, and still have, developing and using JOP, but
should you use JOP? JOP is a processor design intended as a time
predictable solution for hard real-time systems. If your application
or research focus is on those systems and you prefer Java as
programming language JOP is the right choice. If you are interested
in larger, dynamic systems JOP is the wrong choice. If average
performance is important for you and you do not care about
worst-case performance other solutions will probably do a better
job.


\section{Directory Structure}

The top-level directories of the distribution are:

\begin{description}
    \item[asm] Microcode source files. The microcode part of the JVM
    and test files.
    \item[bat] Old batch files - \emph{not used}
    \item[boards] Pictures and text for the Eclipse plugin
    \item[c\_src] Some utilities in C (e.g.\ \cmd{down.exe} and
    \cmd{e.exe}.
    \item[doc] \LaTeX sources for this handbook and short notes.
    \item[eclipse] Eclipse project files
    \item[ext] External VHDL and Verilog sources
    \item[java] All Java files
    \begin{description}
        \item[lib] External .jar files
        \item[pc] Tools on the PC
        \item[pcsim] High-level simulation on the PC
        \item[target] THE Java sources for JOP
        \item[tools] All Java tools
    \end{description}
    \item[jbc] (generated)
    \item[jopc]
    \item[linux]
    \item[modelsim]
    \item[pins]
    \item[quartus]
    \item[rbf]
    \item[sopc]
    \item[support]
    \item[ttf] (generated)
    \item[vhdl]
    \item[xilinx]
\end{description}



\chapter{Introduction into the Design Flow}
\label{chap:build}

This section describes the design flow for JOP --- how to build the
Java processor and a Java application from scratch (the VHDL and
Java sources) and download the processor to an FPGA and the Java
application to the processor.


\section{Introduction}

JOP \cite{jop:thesis}, the Java optimized processor, is an
open-source development platform available for different targets
(Altera and Xilinx FPGAs and various types of FPGA boards). To
support several targets, the design-flow is a little bit complicated.
There is a \code{Makefile} available and when everything is set up
correctly, a simple
%
\begin{verbatim}
    make
\end{verbatim}
%
should build everything from the sources and download a \emph{Hello
World} example. However, to customize the \code{Makefile} for a
different target it is necessary to understand the complete design
flow. It should be noted that an
Ant\footnote{\url{http://ant.apache.org/}} based build process is
also available.

\subsection{Tools}

All needed tools are freely available.
%
\begin{itemize}
    \item  \href{http://java.sun.com/javase/downloads/index.jsp}%
{Java SE Development Kit (JDK)}  Java compiler and runtime
    \item  \href{http://www.cygwin.com/}%
{Cygwin} GNU tools for Windows. Packages cvs, gcc and make are
needed
    \item  \href{https://www.altera.com/support/software/download/altera_design/quartus_we/dnl-quartus_we.jsp}%
{Quarts II Web Edition} VHDL synthesis, place and route for Altera
FPGAs
%    \item  \href{https://www.altera.com/support/software/download/programming/jam/dnl-byte_code_player.jsp}%
%{Jam STAPL Byte-Code Player} FPGA configuration in batch mode
%(\cmd{jbi32.exe})

\end{itemize}
%
The PATH variable should contain entries to the executables of all
packages (java and javac, Cygwin bin, and Quartus executables). Check
the PATH at the command prompt with:
%
\begin{verbatim}
    javac
    gcc
    make
    cvs
    quartus_map
\end{verbatim}
%
All the executables should be found and usually report their usage.

\subsection{Getting Started}

\label{sec:started}

This section shows a quick step-by-step build of JOP for the Cyclone
target in the minimal configuration. All directory paths are given
relative to the JOP root directory \dirent{jop}. The build process is
explained in more detail in one of the following sections.

\subsubsection{Download the Source}

Create a working directory and download JOP from the
\url{www.opencores.org} CVS server:

\begin{verbatim}
 cvs -d :pserver:anonymous@cvs.opencores.org:/cvsroot/anonymous \
    -z9 co -P jop
\end{verbatim}

All sources are downloaded to a directory \dirent{jop}. For the
following command change to this directory. Create the needed
directories with:
\begin{verbatim}
    make directories
\end{verbatim}

\subsubsection{Tools}

The tools contain \cmd{Jopa}, the microcode assembler, \cmd{JopSim},
a Java based simulation of JOP, and \cmd{JOPizer}, the application
builder. The tools are built with following make command:

\begin{verbatim}
    make tools
\end{verbatim}

\subsubsection{Assemble the Microcode JVM, Compile the Processor}

The JVM configured to download the Java application from the serial
interface is built with:

\begin{verbatim}
    make jopser
\end{verbatim}

This command also invokes Quartus to build the processer. If you
want to build it within Quartus follow the following instructions:

\label{subsubsec:quartus}

Start Quartus II and open the project \code{jop.qpf} from directory
\dirent{quartus/cycmin} in Quartus with \menuitem{File -- Open
Project...}. Start the compiler and fitter with \menuitem{Processing
-- Start Compilation}. After successful compilation the FPGA is
configured with \menuitem{Tools -- Programmer} and \menuitem{Start}.

\subsubsection{Compiling and Downloading the Java Application}

A simple \emph{Hello World} application is the default application
in the Makefile. It is built and downloaded to JOP with:

\begin{verbatim}
    make japp
\end{verbatim}

The ``Hello World" message should be printed in the command window.

For a different application change the Makefile targets or override
the \code{make} variables at the command line. The following example
builds and runs some benchmarks on JOP:

\begin{verbatim}
    make japp -e P1=bench P2=jbe P3=DoAll
\end{verbatim}

The three variables \code{P1}, \code{P2}, and \code{P3} are a
shortcut to set the directory, the package name, and the main class
of the application.

\subsubsection{USB based Boards}

Several Altera based boards use an FTDI FT2232 USB chip for the FPGA
and Java program download. To change the download flow for those
boards change the value of the following variable in the Makefile to
\code{true}:

\begin{verbatim}
    USB=true
\end{verbatim}

The Java download channel is mapped to a virtual serial port on the
PC. Check the port number in the system properties and set the
variable \code{COM\_PORT} accordingly.

\subsection{Xilinx Spartan-3 Starter Kit}

\index{Xilinx} The Xilinx tool chain is still not well supported by
the Makefile or the Ant design flow. Here is a short list on how to
build JOP for a Xilinx board:

\begin{verbatim}
    make tools
    cd asm
    jopser
    cd ..
\end{verbatim}


Now start the Xilinx IDE wirh the project file \code{jop.npl}. It
will be converted to a new (binary) \code{jop.ise} project. The
\code{.npl} project file is used as it is simple to edit (ASCII).

\begin{itemize}
    \item Generate JOP by double clicking 'Generate PROM, ACE, or JTAG File'
    \item Configure the FPGA according to the board type
\end{itemize}

The above is a one step build for the processor. The Java
application is built and downloaded by:

\begin{verbatim}
    make java_app
    make download
\end{verbatim}

Now your first Java program runs on JOP/Spartan-3!

\section{Booting JOP --- How Your Application Starts}

Basically this is a two step process: (a) configuration of the FPGA
and (b) downloading the Java application. There are different
possibilities to perform these steps.

\subsection{FPGA Configuration}

FPGAs are usually SRAM based and \emph{lose} their configuration
after power down. Therefore the configuration has to be loaded on
power up. For development the FPGA can be configured via a download
cable (with JTAG commands). This can be done within the IDEs from
Altera and Xilinx or with command line tools such as
\cmd{quartus\_pgm} or \cmd{jbi32}.

For the device to boot automatically, the configuration has to be
stored in non volatile memory such as Flash. Serial Flash is directly
supported by an FPGA to boot on power up. Another method is to use a
standard parallel Flash to store the configuration and additional
data (e.g. the Java application). A small PLD reads the configuration
data from the Flash and shifts it into the FPGA. This method is used
on the Cyclone and ACEX boards.

\subsection{Java Download}

\index{application!download} When the FPGA is configured the Java
application has to be downloaded into the main memory. This download
is performed in microcode as part of the JVM startup sequence. The
application is a \code{.jop} file generated by \cmd{JOPizer}. At the
moment there are three options:

\begin{description}
    \item[Serial line] JOP listens to the serial line and the
        data is written into the main memory. A simple echo
        protocol performs the flow control. The baud rate is
        usually 115~kBaud.
    \item[USB] Similar to the serial line version, JOP listens to
        the parallel interface of the FTDI FT2232 USB chip. The
        FT2232 performs the flow control at the USB level and the
        echo protocol is omitted.
    \item[Flash] For stand alone applications the Java program is
    copied from the Flash (relative Flash address 0, mapped Flash
    address is 0x80000\footnote{All addresses in JOP are counted in
    32-bit quantities. However, the Flash is connected only to the
    lower 8 bits of the data bus. Therefore a store of one word in
    the main memory needs four loads from the Flash.}) to the main
    memory (usually a 32-bit SRAM).
\end{description}


The mode of downloading is defined in the JVM (\code{jvm.asm}). To
select a new mode, the JVM has to be assembled and the complete
processor has to be rebuilt -- a full \code{make} run. The generation
is performed by the C preprocessor (\cmd{gcc}) on \code{jvm.asm}. The
serial version is generated by default; the USB or Flash version are
generated by defining the preprocessor variables \code{USB} or
\code{FLASH}.

\paragraph{VHDL Simulation}

\index{simulation!VHDL}To speed up the VHDL simulation in ModelSim
there is a forth method where the Java application is loaded by the
test bench instead of JOP. This version is generated by defining
\code{SIMULATION}. The actual Java application is written by
\cmd{jop2dat} into a plain text file (\code{mem\_main.dat}) and read
by the simulation test bench into the simulated main memory.

There are four small batch-files in directory \dirent{asm} that
perform the JVM generation: \cmd{jopser}, \cmd{jopusb},
\cmd{jopflash}, and \cmd{jopsim}.

\subsection{Combinations}

Theoretically all variants to configure the FPGA can be combined with
all variations to download the Java application. However, only two
combinations are useful:

\begin{enumerate}
    \item For VHDL or Java development configure the FPGA
    via the download cable and download the Java application
    via the serial line or USB.
    \item For a stand-alone application load the configuration and
    the Java program from the Flash.
\end{enumerate}



\section{The Design Flow}

This section describes the design flow to build JOP in greater
detail.

\subsection{Tools}

There are a few tools necessary to build and download JOP to the FPGA
boards. Most of them are written in Java. Only the tools that access
the serial line are written in C.\footnote{The Java JDK still comes
without the javax.comm package and getting this optional package
correctly installed is not that easy.}

\subsubsection{Downloading}

These little programs are already compiled and the binaries are
checked in into the repository. The sources can be found in directory
\dirent{c\_src}.

\begin{description}
    \item[\eei{down.exe}] The workhorse to download Java programs. The
    mandatory argument is the COM-port. Optional switch \code{-e}
    keeps the program running after the download and echoes the
    characters from the serial line (\code{System.out} in JOP) to
    stdout. Switch \code{-usb} disables the echo protocol to speed up the
    download over USB.
    \item[\eei{e.exe}] Echoes the characters from the serial line
        to stdout. Parameter is the COM-port.
    \item[\eei{amd.exe}] A utility to send data over the serial
        line to program the on-board Flash. The complementary
        Java program \code{util.Amd} must be running on JOP.
    \item[\eei{USBRunner.exe}] Download the FPGA configuration via
    USB with the FTDI2232C chip (dpsio board).
\end{description}

\subsubsection{Generation of Files}

These tools are written in Java and are delivered in source form.
The source can be found under \dirent{java/tools/src} and the class
files are in \code{jop-tools.jar} in directory
\dirent{java/tools/dist/lib}.

\begin{description}
    \item[\eei{Jopa}] The JOP assembler. Assembles the microcoded
    JVM and produces on-chip memory initialization files and VHDL
    files.
    \item[\eei{BlockGen}] converts Altera memory initialization
        files to VHDL files for a Xilinx FPGA.
    \item[\eei{JOPizer}] links a Java application and converts the
    class information to the format that JOP expects (a \code{.jop} file).
    \cmd{JOPizer} uses the bytecode engineering library\footnote{\url{http://jakarta.apache.org/bcel/}} (BCEL).

\end{description}

\subsubsection{Simulation}

\index{simulation!JopSim}
\begin{description}
    \item[\eei{JopSim}] reads a \code{.jop} file and executes it in
    a debug JVM written in Java. Command line option
    \code{-Dlog="true"} prints a log entry for each executed JVM
    bytecode.
    \item[\eei{pcsim}] simulates the BaseIO expansion board for Java
    debugging on a PC (using the JVM on the PC).
\end{description}

\subsection{Targets}

JOP has been successfully ported to several different FPGAs and
boards. The main distribution contains the ports for the FPGAs:

\begin{itemize}
    \item Altera Cyclone EP1C6 or EP1C12
    \item Xilinx Spartan-3
    \item Altera Cyclone-II (Altera DE2 board)
    \item Xilinx Virtex-4 (ML40x board)
    \item Xilinx Spartan-3E (Digilent Nexys 2 board)
\end{itemize}

For the current list of the supported FPGA boards see the list at the
web site.\footnote{\url{http://www.jopwiki.com/FPGA_boards}} Besides
the ports to different FPGAs there are ports to different boards.

\subsubsection{Cyclone EP1C6/12}

This board is the workhorse for the JOP development and comes in two
versions: with an Cyclone EP1C6 or EP1C12. The schematics can be
found in Appendix~\ref{appx:cycore}. The board contains:

\begin{itemize}
    \item Altera Cyclone EP1C6Q240 or EP1C12Q240 FPGA
    \item 1~MB fast SRAM
    \item 512~KB Flash (for FPGA configuration and program code)
    \item 32~MB NAND Flash
    \item ByteBlasterMV port
    \item Watchdog with LED
    \item EPM7064 PLD to configure the FPGA from the Flash (on watchdog reset)
    \item Voltage regulator (1V5)
    \item Crystal clock (20 MHz) at the PLL input (up to 640 MHz internal)
    \item Serial interface (MAX3232)
    \item 56 general purpose I/O pins
\end{itemize}

The Cyclone specific files are \code{jopcyc.vhd} or \code{jopcyc12}
and \code{mem32.vhd}. This FPGA board is designed as a module to be
integrated with an application specific I/O-board. There exist
following I/O-boards:
%
\begin{description}
    \item[simpexp] A simple bread board with a voltage regulator and
    a SUBD connector for the serial line
    \item[baseio] A board with Ethernet connection and EMC
        protected digital I/O and analog input
    \item[bg263] Interface to a GPS receiver, a GPRS modem, keyboard
    and a display for a railway application
    \item[lego] Interface to the sensors and motors of the LEGO
    Mindstorms. This board is a substitute for the LEGO RCX.
    \item[dspio] Developed at the University of Technology Vienna, Austria for
    digital signal processing related work. All design files for this
    board are open-source.
\end{description}
%
Table~\ref{tab:cycio} lists the related VHDL files and Quartus
project directories for each I/O board.

\begin{table}
    \centering

    \begin{tabular}{lll}
        \toprule
        I/O board & Quartus & I/O top level \\
        \midrule
        simpexp, baseio  & \dirent{cycmin} & \code{scio\_min.vhd} \\
        dspio  & \dirent{usbmin} & \code{scio\_dspiomin.vhd} \\
        baseio  & \dirent{cycbaseio} & \code{scio\_baseio.vhd} \\
        bg263  & \dirent{cybg} & \code{scio\_bg.vhd} \\
        lego  & \dirent{cyclego} & \code{scio\_lego.vhd} \\
        dspio  & \dirent{dspio} & \code{scio\_dspio.vhd} \\
        \bottomrule

    \end{tabular}
    \caption{Quartus project directories and VHDL files for the different I/O boards}
    \label{tab:cycio}

\end{table}


\subsubsection{Xilinx Spartan-3}

\index{Xilinx} The Spartan-3 specific files are \code{jop\_xs3.vhd}
and \code{mem\_xs3.vhd} for the Xilinx Spartan-3 Starter Kit and
\code{jop\_trenz.vhd} and \code{mem\_trenz.vhd} for the Trenz
Retrocomputing board.


\section{Eclipse}

In folder \dirent{eclipse} there are four Eclipse projects that you
can import into your Eclipse workspace. However, do not use
\emph{that} directory as your workspace directory. Choose a directory
outside of the JOP source tree for the workspace.

All projects use the Eclipse path variable\footnote{Eclipse (path)
variables are workspace specific.} \code{JOP\_HOME} that has to point
to the root directory (\dirent{.../jop}) of the JOP sources. Under
\menuitem{Window -- Preferences...} select \menuitem{General --
Workspace -- Linked Resources} and create the path variable
\code{JOP\_HOME} with \menuitem{New...}.

Import the projects with \menuitem{File -- Import..} and
\menuitem{Existing Projects into Workspace}. It is suggested to an
Eclipse workspace that is not part of the jop source tree. Select as
root directory \dirent{.../jop/eclipse}, select the projects you want
to import, select \menuitem{Copy projects into workspace}, and press
\menuitem{Finish}. Table~\ref{tab:eclipse} shows all available
projects.

\begin{table}
    \centering

    \begin{tabular}{ll}
        \toprule
        Project & Content \\
        \midrule
        \dirent{jop} & The target sources \\
        \dirent{joptools} & Tools such as \code{Jopa}, \code{JopSim}, and \code{JOPizer} \\
        \dirent{pc} & Some PC utilities (e.g.\ Flash programming via UDP/IP) \\
        \dirent{pcsim} & Simulation of the basio hardware on the PC \\
        \bottomrule

    \end{tabular}
    \caption{Eclipse projects}
    \label{tab:eclipse}

\end{table}

Add the libraries from \dirent{.../jop/java/lib} (as external
archives) to the build path (right click on the \dirent{joptools}
project) of the project \dirent{joptools}.\footnote{Eclipse can't use
path variables for external .jar files.}

\section{Simulation}

This section contains the information you need to get a simulation
of JOP running. There are two ways to simulate JOP:
%
{\samepage
\begin{itemize}
    \item High-level JVM simulation with \cmd{JopSim}
    \item VHDL simulation (e.g. with ModelSim)
\end{itemize}
}
%

\subsection{JopSim Simulation}

\index{simulation!JopSim}

The high level simulation with \cmd{JopSim} is a simple JVM written
in Java that can execute the JOP specific application (the
\code{.jop} file). It is started with:
\begin{verbatim}
    make jsim
\end{verbatim}

To output each executing bytecode during the simulation run change in
the Makefile the logging parameter to \code{-Dlog="true"}.


\subsection{VHDL Simulation}

\index{simulation!VHDL}

This section is about running a VHDL simulation with ModelSim. All
simulation files are vendor independent and should run on any
versions of ModelSim or a different VHDL simulator. You can simulate
JOP even with the free ModelSim XE II Starter Xilinx version, the
ModelSim Altera version or the ModelSim Actel version.

To simulate JOP, or any other processor design, in a vendor neutral
way, models of the internal memories (block RAM) and the external
main memory are necessary. Beside this, only a simple clock driver is
necessary. To speed-up the simulation a little bit, a simulation of
the UART output, which is used for \code{System.out.print()}, is also
part of the package.

Table~\ref{tab:simfiles} lists the simulation files for JOP and the
programs that generates the initialization data. The non-generated
VHDL files can be found in directory \dirent{vhdl/simulation}.
%
\begin{table}
\small
    \centering

    \begin{tabular}{llll}
        \toprule
        VHDL file & Function & Initialization file & Generator \\
        \midrule
        \code{sim\_jop\_types\_100.vhd} & JOP constant definitions & - & - \\
        \code{sim\_rom.vhd} & JVM microcode ROM & \code{mem\_rom.dat} & \cmd{Jopa} \\
        \code{sim\_ram.vhd} & Stack RAM & \code{mem\_ram.dat} & \cmd{Jopa} \\
        \code{sim\_jbc.vhd} & Bytecode memory (cache) & - & - \\
        \code{sim\_memory.vhd} & Main memory & \code{mem\_main.dat} & \cmd{jop2dat} \\
        \code{sim\_pll.vhd} & A dummy entity for the PLL & - & - \\
        \code{sim\_uart.vhd} & Print characters to stdio & - & - \\
        \bottomrule

    \end{tabular}
    \caption{Simulation specific VHDL files}
    \label{tab:simfiles}

\end{table}
%
The needed VHDL files and the compile order can be found in
\code{sim.bat} under \dirent{modelsim}.


The actual version of JOP contains all necessary files to run a
simulation with ModelSim. In directory \dirent{vhdl/simulation} you
will find:
%
\begin{itemize}
    \item A test bench: \code{tb\_jop.vhd} with a serial receiver to
    print out the messages from JOP during the simulation
    \item Simulation versions of all memory components (vendor neutral)
    \item Simulation of the main memory
\end{itemize}
%
\cmd{Jopa} generates various \code{mem\_xxx.dat} files that are read
by the simulation. The JVM that is generated with \code{jopsim.bat}
assumes that the Java application is preloaded in the main memory.
\cmd{jop2dat} generates a memory initialization file from the Java
application file (\code{MainClass.jop}) that is read by the
simulation of the main memory (\code{sim\_memory.vhd}).

In directory \dirent{modelsim} you will find a small batch file
(\cmd{sim.bat}) that compiles JOP and the test bench in the correct
order and starts ModelSim. The whole simulation process (including
generation of the correct microcode) is started with:

\begin{verbatim}
    make sim
\end{verbatim}

After a few seconds you should see the startup message from JOP
printed in ModelSim's command window. The simulation can be continued
with \cmd{run -all} and after around 6~ms \emph{simulation time} the
actual Java \code{main()} method is executed. During those 6~ms,
which will probably be minutes of simulation, the memory is
initialized for the garbage collector.

\section{Files Types You Might Encounter}

As there are various tools involved in the complete build process,
you will find files with various extensions. The following list
explains the file types you might encounter when changing and
building JOP.

The following files are the \emph{source} files:

\begin{description}

\item[\eei{.vhd}] VHDL files describe the hardware part and are
compiled with either Quartus or Xilinx ISE. Simulation in ModelSim
is also based on VHDL files.
\item[\eei{.v}] Verilog HDL. Another hardware description language.
Used more in the US.

\item[\eei{.java}] Java --- the language that runs native on JOP.

\item[\eei{.c}] There are still some tools written in C.

\item[\eei{.asm}] JOP microcode. The JVM is written in this stack
oriented assembler. Files are assembled with \cmd{Jopa}. The result
are VHDL files, \code{.mif} files, and \code{.dat} files for
ModelSim.

\item[\eei{.bat}] Usage of these DOS batch files still prohibit
running the JOP build under Unix. However, these files get less used
as the \code{Makefile} progresses.

\item[\eei{.xml}] Project files for Ant. Ant is an attractive
substitution to \cmd{make}. Future distributions on JOP will be
\cmd{ant} based.

\end{description}


Quartus II and Xilinx ISE need configuration files that describe
your project. All files are usually ASCII text files.

\begin{description}

\item[\eei{.qpf}] Quartus II Project File. Contains almost no
information.
\item[\eei{.qsf}] Quartus II Settings File defines the project.
    VHDL files that make up your project are listed. Constraints
    such as pin assignments and timing constraints are set here.
\item[\eei{.cdf}] Chain Description File. This file
stores device name, device order, and programming file name
information for the programmer.
\item[\eei{.tcl}] Tool Command Language. Can be used in Quartus to
automate parts of the design flow (e.g. pin assignment).

\item[\eei{.npl}] Xilinx ISE project. VHDL files that make up
    your project are listed. The actual version of Xilinx ISE
    converts this project file to a new format that is not in
    ASCII anymore.
\item[\eei{.ucf}] Xilinx Foundation User Constraint File.
    Constraints such as pin assignments and timing constraints
    are set here.

\end{description}

The Java tools \cmd{javac} and \cmd{jar} produce following file types
from the Java sources:

\begin{description}

\item[\eei{.class}] A class file contains the bytecodes, a symbol table and other
ancillary information and is executed by the JVM.

\item[\eei{.jar}] The Java Archive file format enables you to bundle multiple files
into a single archive file. Typically a \code{.jar} file contains
the class files and auxiliary resources. A \code{.jar} file is
essentially a zip file that contains an optional \dirent{META-INF}
directory.

\end{description}

The following files are generated by the various tools from the
source files:

\begin{description}

\item[\eei{.jop}] This file makes up the linked Java application
    that runns on JOP. It is generated by \cmd{JOPizer} and can
    be either downloaded (serial line or USB) or stored in the
    Flash (or used by the simulation with \cmd{JopSim} or
    ModelSim)

\item[\eei{.mif}] Memory Initialization File. Defines the initial
content of on-chip block memories for Altera devices.

\item[\eei{.dat}] memory initialization files for the simulation
with ModelSim.

\item[\eei{.sof}] SRAM Output File. Configuration file for Altera
devices. Used by the Quartus programmer or by \cmd{quartus\_pgm}.
Can be converted to various (or too many) different format. Some are
listed below.

\item[\eei{.pof}] Programmer Object File. Configuration for Altera
devices. Used for the Flash loader PLDs.

\item[\eei{.jbc}] JamTM STAPL Byte Code 2.0. Configuration for Altera
devices. Input file for \cmd{jbi32}.

\item[\eei{.ttf}] Tabular Text File. Configuration for Altera
devices. Used by flash programming utilities (\cmd{amd} and
\cmd{udp.Flash} to store the FPGA configuration in the boards Flash.

\item[\eei{.rbf}] Raw Binary File. Configuration for Altera
devices. Used by the USB download utility (\cmd{USBRunner}) to
configure the dspio board via the USB connection.

\item[\eei{.bit}] Bitstream File. Configuration file for Xilinx
devices.

\end{description}

\section{Information on the Web}

Further information on JOP and the build process can be found on the
Internet at the following places:

\begin{itemize}
    \item \url{http://www.jopdesign.com/} is the main web site
        for JOP
    \item \url{http://www.jopwiki.com/} is a Wiki that can be
        freely edited by JOP users.
    \item
        \url{http://tech.groups.yahoo.com/group/java-processor/}
        hosts a mailing list for discussions on Java processors
        in general and mostly on JOP related topics
\end{itemize}


\section{Porting JOP}

Porting JOP to a different FPGA platform or board usually consists
of adapting pin definitions and selection of the correct memory
interface. Memory interfaces for the SimpCon interconnect can be
found in directory \dirent{vhdl/memory}.

\subsection{Test Utilities}

To verify that the port of JOP is successful there are some small
test programs in \dirent{asm/src}. To run the JVM on JOP the
microcode \code{jvm.asm} is assembled and will be stored in an
on-chip ROM. The Java application will then be loaded by the first
microcode instructions in \code{jvm.asm} into an external memory.
However, to verify that JOP and the serial line are working
correctly, it is possible to run small test programs directly in
microcode.

One test program (\code{blink.asm}) does not need the main memory and
is a first test step before testing the possibly changed memory
interface. \code{testmon.asm} can be used to debug the main memory
interface. Both test programs can be built with the \cmd{make}
targets \cmd{jop\_blink\_test} and \cmd{jop\_testmon}.

\subsubsection{Blinking LED and UART output}

The test is built with:
%
\begin{verbatim}
    make jop_blink_test
\end{verbatim}
%
After download, the watchdog LED should blink and the FPGA will print
out 0 and 1 on the serial line. Use a terminal program or the utility
\cmd{e.exe} to check the output from the serial line.

\subsubsection{Test Monitor}

Start a terminal program (e.g. HyperTerm) to communicate with the
monitor program and build the test monitor with:
%
\begin{verbatim}
    make jop_testmon
\end{verbatim}
%
After download the program prints the content of the memory at
address 0. The program understands following \emph{commands}:

\begin{itemize}
    \item A single CR reads the memory at the current addres
    and prints out the address and memory content
    \item \code{addr=val;} writes $val$ into the memory location at
    address $addr$
\end{itemize}

One tip: Take care that your terminal program does not send an LF
after the CR.


\section{Extending JOP}

%% Trevor: This section seems like esoteric information that might be better as an appendix.

JOP is a soft-core processor and customizing it for an application
is an interesting opportunity.

\subsection{Native Methods}

\index{native method} The \emph{native} language of JOP is microcode.
A native method is implemented in JOP microcode. The interface to
this native method is through a \emph{special} bytecode. The mapping
between native methods and the special bytecode is performed by
\code{JOPizer}. When adding a new (\emph{special}) bytecode to JOP,
the following files have to be changed:
\begin{enumerate}
    \item \code{jvm.asm} implementation
    \item \code{Native.java} method signature
    \item \code{JopInstr.java} mapping of the signature to the name
    \item \code{JopSim.java} simulation of the bytecode
    \item \code{JVM.java} (just rename the method name)
    \item \code{Startup.java} (only when needed in a class
        initializer)
    \item \code{WCETInstruction.java} timing information
\end{enumerate}

First implement the native code in \code{JopSim.java} for easy
debugging. The \emph{real} microcode is added in \code{jvm.asm} with
a label for the special byctecode. The naming convention is
\code{jopsys\_name}. In \code{Native.java} provide a method
signature for the native method and enter the mapping between this
signature and the name in \code{jvm.asm} and in
\code{JopInstr.java}. Provide the execution time in
\code{WCETInstruction.java} for the WCET analysis.

The native method is accessed by the method provided in
\code{Native.java}. There is no calling overhead involved in the
mechanism. The \emph{native} method gets substituted by
\cmd{JOPizer} with a \emph{special} bytecode.

\subsection{A new Peripheral Device}

Creation of a new peripheral devices involves some VHDL coding.
However, there are several examples in \dirent{jop/vhdl/scio}
available.

All peripheral components in JOP are connected with the SimpCon
\cite{simpcon} interface. For a device that implements the Wishbone
\cite{soc:wishbone} bus, a SimpCon-Wishbone bridge (\code{sc2wb.vhd})
is available (e.g., it is used to connect the AC97 interface in the
\code{dspio} project).

For an easy start use an existing example and change it to your
needs. Take a look into \code{sc\_test\_slave.vhd}. All peripheral
components (SimpCon slaves) are connected in one module usually named
\code{scio\_xxx.vhd}. Browse the examples and copy one that best fits
your needs. In this module the address of your peripheral device is
defined (e.g. 0x10 for the primary UART). This I/O address is mapped
to a negative memory address for JOP. That means 0xffffff80 is added
as a base to the I/O address.

By convention this address mapping is defined in
\code{com.jopdesign.sys.Const}. Here is the UART example:

\begin{verbatim}
    // use negative base address for fast constant load
    // with bipush
    public static final int IO_BASE = 0xffffff80;
    ...
    public static final int IO_STATUS = IO_BASE+0x10;
    public static final int IO_UART = IO_BASE+0x10+1;
\end{verbatim}

The I/O devices are accessed from Java by
\emph{native}\footnote{These are not real functions and are
substituted by special bytecodes on application building with
JOPizer.} functions: \code{Native.rdMem()} and \code{Native.wrMem()}
in pacakge \code{com.jopdesign.sys}. Again an example with the UART:

\begin{verbatim}
    // busy wait on free tx buffer
    // no wait on an open serial line, just wait
    // on the baud rate
    while ((Native.rdMem(Const.IO_STATUS)&1)==0) {
        ;
    }
    Native.wrMem(c, Const.IO_UART);
\end{verbatim}

Best practise is to create a new I/O configuration
\code{scio\_xxx.vhdl} and a new Quartus project for this
configuration. This avoids the mixup of the changes with a new
version of JOP. For the new Quartus project only the three files
\code{jop.cdf}, \code{jop.qpf}, and \code{jop.qsf} have to be copied
in a new directory under \dirent{quartus}. This new directory is the
project name that has to be set in the Makefile:

\begin{verbatim}
    QPROJ=yourproject
\end{verbatim}

The new VHDL module and the \code{scio\_xxx.vhdl} are added in
\code{jop.qsf}. This file is a plain ASCII file and can be edited
with a standard editor or within Quartus.

\subsection{A Customized Instruction}

A customized instruction can be simply added by implementing it in
microcode and mapping it to a native function as described before. If
you want to include a hardware module that implements this
instruction a new microinstruction has to be introduced. Besides
mapping this instruction to a native method the instruction has also
be added to the microcode assembler \cmd{Jopa}.

\subsection{Dependencies and Configurations}

As JOP and the JVM are a mix of VHDL and Java files, changes in the
central data structures or some configurations needs an update in
several files.

\subsubsection{Stack Size}

The on-chip stack size can be configured by changing following
constants:

\begin{itemize}
    \item \code{ram\_width} in \code{jop\_config\_xx.vhd}
    \item \code{STACK\_SIZE} in \code{com.jopdesign.sys.Const}
    \item \code{RAM\_LEN} in \code{com.jopdesign.sys.Jopa}
\end{itemize}

\subsubsection{Changing the Class Format}

\begin{itemize}
    \item JOPizer: CLS\_HEAD, dump()
    \item GC.java uses CLASS\_HEADR
    \item JMV.java uses CLASS\_HEADR + offset (checkcast, instanceof)
\end{itemize}




%\section{Notes}
%
%TODO:
%
%\begin{itemize}
%    \item pcsim
%    \item JopSim
%\end{itemize}
%Formating ideas -- see Latax intro p 27
%\url{http://www.ctan.org/tex-archive/info/lshort/english/lshort.pdf}\\


\chapter{Java and the Java Virtual Machine}
\label{chap:java}
Java technology consists of the Java language definition, a
definition of the standard library, and the definition of an
intermediate instruction set with an accompanying execution
environment. This combination helps to make \emph{write once, run
anywhere} possible.

The following chapter gives a short overview of the Java programming
language. A more detailed description of the Java Virtual Machine
(JVM) and the explanation of the JVM instruction set, the so-called
bytecodes, follows. The exploration of dynamic instruction counts of
typical Java programs can be found in Section~\ref{sec:bench:jvm}.

\section{Java}

Java is a relatively new and popular programming language. The main
features that have helped Java achieve success are listed below:
%
\begin{description}
    \item[Simple and object oriented:] Java is a simple
        programming language that appears very similar to C. This
        `look and feel' of C means that programmers who know C,
        can switch to Java without difficulty. Java provides a
        simplified object model with single
        inheritance\footnote{Java has \emph{single inheritance}
        of \emph{implementation} -- only one class can be
        extended. However, a class can implement several
        interfaces, which means that Java has \emph{multiple
        interface inheritance}.}.

    \item[Portability:]
To accommodate the diversity of operating environments, the Java
compiler generates bytecodes -- an architecture neutral intermediate
format. To guarantee platform independence, Java specifies the sizes
of its basic data types and the behavior of its arithmetic
operators. A Java interpreter, the Java virtual machine, is
available on various platforms to help make `write once, run
anywhere' possible.

    \item[Availability:] Java is not only available for different
        operating systems, it is available at no cost. The
        runtime system and the compiler can be downloaded from
        Sun's website for Windows, Linux, and Solaris.
        Sophisticated development environments, such as Netbeans
        or Eclipse, are available under the GNU Public License.

    \item[Library:] The complete Java system includes a rich
        class library to increase programming productivity.
        Besides the functionality of a C standard library, it
        also contains other tools, such as collection classes and
        a GUI toolkit.

    \item[Built-in multithreading:]
Java supports multithreading at the language level: the library
provides the \code{Thread} class, the language provides the keyword
\code{synchronized} for critical sections and the runtime system
provides monitor and condition lock primitives. The system libraries
have been written to be thread-safe: the functionality provided by
the libraries is available without conflicts due to multiple
concurrent threads of execution.

    \item[Safety:]
Java provides extensive compile-time checking, followed by a second
level of runtime checking. The memory management model is simple --
objects are created with the \code{new} operator. There are no
explicit pointer data types and no pointer arithmetic, but there is
automatic garbage collection. This simple memory management model
eliminates a large number of the programming errors found in C and
C++ programs. A restricted runtime environment, the so-called
\emph{sandbox}, is available when executing small Java applications
in Web browsers.

\end{description}
%
\begin{figure*}
    \centering
    \includegraphics[scale=\picscale]{intro/java_overview}
    \caption{Java system overview}
    \label{fig:java:overview}
\end{figure*}
%
As can be seen in \figurename~\ref{fig:java:overview}, Java consists
of three main components:
%
\begin{enumerate}
    \item The Java programming language as defined in
    \cite{JavaLangSpec2}
    \item The class library, defined as part of the Java
        specification. All implementations of Java have to
        contain the library as defined by Sun
    \item The Java virtual machine (defined in \cite{jvm}) that loads,
     verifies and executes the binary representation (the
\emph{class file}) of a Java program
\end{enumerate}
%
The Java native interface supports functions written in C or C++.
This combination is sometimes called \emph{Java technology} to
emphasize the fact that Java is more than just another
object-oriented language.

However, a number of issues have hindered a broad acceptance of
Java. The original presentation of Java as an Internet language led
to the misconception that Java was not a general-purpose programming
language. Another obstacle was the first implementation of the JVM
as an interpreter. Execution of Java programs was \emph{very} slow
compared to compiled C/C++ programs. Although advances in its
runtime technology, in particular the just-in-time compiler, have
closed the performance gap, it is still a commonly held view that
Java is slow.

\subsection{History}

The Java programming language originated as part of a research
project to develop software for network devices and embedded
systems. In the early '90s, Java, which was originally known as Oak
\cite{java:oak, java:oak2}, was created as a programming tool for a
consumer device that we would today call a PDA. The device (known as
*7) was a small SPARC-based hardware device with a tiny embedded OS.
However, the *7 was not issued as a product and Java was officially
released in 1995 as a new language for the Internet (to be
integrated into Netscape's browser). Over the years, Java technology
has become a programming tool for desktop applications, web servers
and server applications. These application domains resulted in the
split of the Java platform into the Java standard edition (J2SE) and
the enterprise edition (J2EE) in 1999. With every new release, the
library (defined as part of the language) continued to grow. Java
for embedded systems was clearly not an area Sun was interested in
pursuing. However, with the arrival of mobile phones, Sun again
became interested in this embedded market. Sun defined different
subsets of Java, which have now been combined into the Java Micro
Edition (J2ME). A detailed description of the J2ME follows in
Section~\ref{sec:j2me}.


\subsection{The Java Programming Language}

The Java programming language is a general-purpose object-oriented
language. Java is related to C and C++, but with a number of aspects
omitted. Java is a strongly typed language, which means that type
errors can be detected at compile time. Other errors, such as wrong
indices in an array, are checked at runtime. The
problematic\footnote{C pointers represent memory addresses as data.
Pointer arithmetic and direct access to memory leads to common and
hard-to-find program errors.} \emph{pointer} in C and explicit
deallocation of memory is completely avoided. The pointer is
replaced by a \emph{reference}, i.e.\ an abstract pointer to an
object. Storage for an object is allocated from the heap during
creation of the object with \code{new}. Memory is freed by automatic
storage management, typically using a garbage collector. The garbage
collector avoids memory leaks from a missing \code{free()} and the
safety problems exposed by dangling pointers.

The types in Java are divided into two categories: primitive types
and reference types. \tablename~\ref{tab:java:primitive} lists the
available primitive types. Method local variables, class fields and
object fields contain either a primitive type value or a reference
to an object.

\begin{table}
    \centering
    \begin{tabular}{ll}
        \toprule
        Type & Description \\
        \midrule
        \code{boolean} & either \code{true} or \code{false} \\
        \code{char} & 16-bit Unicode character (unsigned) \\
        \code{byte} & 8-bit integer (signed) \\
        \code{short} & 16-bit integer (signed) \\
        \code{int} & 32-bit integer (signed) \\
        \code{long} & 64-bit integer (signed) \\
        \code{float} & 32-bit floating-point (IEEE 754-1985) \\
        \code{double} & 64-bit floating-point (IEEE 754-1985) \\
        \bottomrule
    \end{tabular}
    \caption{Java primitive data types}
    \label{tab:java:primitive}
\end{table}

Classes and class instances, the objects, are the fundamental data
and code organization structures in Java. There are no global
variables or functions as there are in C/C++. Each method belongs to
a class. This `everything belongs to a class or an object' combined
with the class naming convention, as suggested by Sun, avoids name
conflicts in even the largest applications.

New classes can extend exactly one superclass. Classes that do not
explicitly extend a superclass become direct subclasses of
\code{Object}, the root of the whole class tree. This single
inheritance model is extended by \emph{interfaces}. Interfaces are
abstract classes that only define method signatures and provide no
implementation. A concrete class can implement several interfaces.
This model provides a simplified form of multiple inheritance.

Java supports multitasking through \emph{threads}. Each thread is a
separate flow of control, executing concurrently with all other
threads. A thread contains the method stack as thread local data --
all objects are shared between threads. Access conflicts to shared
data are avoided by the proper use of \code{synchronized} methods or
code blocks.

Java programs are compiled to a machine-independent bytecode
representation as defined in \cite{jvm}. Although this intermediate
representation is defined for Java, other programming languages
(e.g.\ ADA \cite{269646}) can also be compiled into Java bytecodes.

\section{The Java Virtual Machine}

The Java virtual machine (JVM) is a definition of an abstract
computing machine that executes bytecode programs. The JVM
specification \cite{jvm} defines three elements:
\begin{itemize}
    \item An instruction set and the meaning of those instructions
    -- the \emph{bytecodes}
    \item A binary format -- the \emph{class file} format. A
    class file contains the bytecodes, a symbol table and other
    ancillary information
    \item An algorithm to \emph{verify} that a class file
    contains valid programs
\end{itemize}
%
In the solution presented in this book, the class files are
verified, linked and transformed into an internal representation
before being executed on JOP. This transformation is performed with
\cmd{JOPizer} and is not executed on JOP. We will therefore omit the
description of the class file and the verification process.

The instruction set of the JVM is stack-based. All operations take
their arguments from the stack and put the result onto the stack.
Values are transferred between the stack and various memory areas.
We will discuss these memory areas first, followed by an explanation
of the instruction set.

\subsection{Memory Areas}

The JVM contains various runtime data areas. Some of these areas are
shared between threads, whereas other data areas exist separately
for each thread.

\begin{description}
    \item[Method area:]
The method area is shared among all threads. It contains static
class information such as field and method data, the code for the
methods and the constant pool. The constant pool is a per-class
table, containing various kinds of constants such as numeric values
or method and field references. The constant pool is similar to a
symbol table.

Part of this area, the code for the methods, is very frequently
accessed (during instruction fetch) and therefore is a good
candidate for caching.

    \item[Heap:]
The heap is the data area where all objects and arrays are
allocated. The heap is shared among all threads. A garbage collector
reclaims storage for objects.

    \item[JVM stack:]
Each thread has a private stack area that is created at the same
time as the thread. The JVM stack is a logical stack that contains
following elements:
\begin{enumerate}
    \item A frame that contains return information for a method
    \item A local variable area to hold local values inside a method
    \item The operand stack, where all operations are performed
\end{enumerate}
%
Although it is not strictly necessary to allocate all three elements
to the same type of memory we will see in Section~\ref{sec:stack}
that the argument-passing mechanism regulates the layout of the JVM
stack.

Local variables and the operand stack are accessed as frequently
as registers in a standard processor. A Java processor should
provide some caching mechanism of this data area.

\end{description}
%
The memory areas are similar to the various segments in conventional
processes (e.g.\ the method code is analogous to the `text'
segment). However, the operand stack replaces the registers in a
conventional processor.

\subsection{JVM Instruction Set}

The instruction set of the JVM contains 201 different instructions
\cite{jvm}. This \emph{bytecodes} can be grouped into the following
categories:
%
\begin{description}
    \item[Load and store:]
Load instructions push values from the local variables onto the
operand stack. Store instructions transfer values from the stack
back to local variables. 70 different instructions belong to this
category. Short versions (single byte) exist to access the first
four local variables. There are unique instructions for each basic
type (\code{int}, \code{long}, \code{float}, \code{double} and
\code{reference}). This differentiation is necessary for the
bytecode verifier, but is not needed during execution. For example
\code{iload}, \code{fload} and \code{aload} all transfer one 32-bit
word from a local variable to the operand stack.

    \item[Arithmetic:]
The arithmetic instructions operate on the values found on the stack
and push the result back onto the operand stack. There are
arithmetic instructions for \code{int}, \code{float} and
\code{double}. There is no direct support for \code{byte},
\code{short} or \code{char} types. These values are handled by
\code{int} operations and have to be converted back before being
stored in a local variable or an object field.

    \item[Type conversion:]
The type conversion instructions perform numerical conversions
between all Java types: as implicit widening conversions (e.g.\
\code{int} to \code{long}, \code{float} or \code{double}) or
explicit (by casting to a type) narrowing conversions.

    \item[Object creation and manipulation:]
Class instances and arrays (that are also objects) are created and
manipulated with different instructions. Objects and class fields
are accessed with type-less instructions.

    \item[Operand stack manipulation:]
All direct stack manipulation instructions are type-less and
operate on 32-bit or 64-bit entities on the stack. Examples of these
instructions are \code{dup}, to duplicate the top operand stack
value, and \code{pop}, to remove the top operand stack value.

    \item[Control transfer:]
Conditional and unconditional branches cause the JVM to continue
execution with an instruction other than the one immediately
following. Branch target addresses are specified relative to the
current address with a signed 16-bit offset. The JVM provides a
complete set of branch conditions for \code{int} values and
references. Floating-point values and type \code{long} are
supported through compare instructions. These compare instructions
result in an \code{int} value on the operand stack.

    \item[Method invocation and return:]
The different types of methods are supported by four instructions:
invoke a class method, invoke an instance method, invoke a method
that implements an interface and an \code{invokespecial} for an
instance method that requires special handling, such as
\code{private} methods or a superclass method.


\end{description}
%
A bytecode consists of one instruction byte followed by optional
operand bytes. The length of the operand is one or two bytes, with
the following exceptions: \code{multianewarray} contains 3 operand
bytes; \code{invokeinterface} contains 4 operand bytes, where one is
redundant and one is always zero; \code{lookupswitch} and
\code{tableswitch} (used to implement the Java \code{switch}
statement) are variable-length instructions; and \code{goto\_w} and
\code{jsr\_w} are followed by a 4 byte branch offset, but neither is
used in practice as other factors limit the method size to 65535
bytes.

\subsection{Methods}

A Java \emph{method} is equivalent to a \emph{function} or
\emph{procedure} in other languages. In object oriented terminology
this \emph{method} is \emph{invoked} instead of \emph{called}. We
will use \emph{method} and \emph{invoke} in the remainder of this
text. In Java and the JVM, there are five types of methods:
%
\begin{itemize}
    \item Static or class methods
    \item Virtual methods
    \item Interface methods
    \item Class initialization
    \item Constructor of the parent class (\code{super()})
\end{itemize}
%
For these five types there are only four different bytecodes:
\begin{description}
    \item[\code{invokestatic}:] A class method (declared \code{static})
    is invoked. As the target does not depend on an object, the
    method reference can be resolved at load/link time.

    \item[\code{invokevirtual}:] An object reference is resolved and
    the corresponding method is invoked. The resolution is usually
    done with a dispatch table per class containing all implemented and
    inherited methods. With this dispatch table, the resolution can
    be performed in constant time.

    \item[\code{invokeinterface}:] An interface allows Java
    to emulate multiple inheritance. A class can implement several
    interfaces, and different classes (that have no inheritance
    relation) can implement the same interface. This flexibility
    results in a more complex resolution process. One method of
    resolution is a search through the class hierarchy that results
    in a variable, and possibly lengthy, execution time. A constant time
    resolution is possible by assigning every interface method a
    unique number. Each class that implements an interface needs its
    own table with unique positions for each interface method of
    the \emph{whole} application.

    \item[\code{invokespecial}:] Invokes an instance method with
    special handling for superclass, \code{private}, and instance
    initialization. This bytecode catches many different cases.
    This results in expensive checks for common \code{private} instance
    methods.
\end{description}
%

\subsection{Implementation of the JVM}

There are several different ways to implement a virtual machine. The
following list presents these possibilities and analyses how
appropriate they are for embedded devices.
%
\begin{description}
    \item[Interpreter:]
The simplest realization of the JVM is a program that interprets the
bytecode instructions. The interpreter itself is usually written in
C and is therefore easy to port to a new computer system. The
interpreter is very compact, making this solution a primary choice
for resource-constrained systems. The main disadvantage is the high
execution overhead. From a code fragment of the typical interpreter
loop, as shown in Listing~\ref{lst:intro:java:intprt}, we can
examine the overhead: The emulation of the stack in a high-level
language results in three memory accesses for a simple \code{iadd}
bytecode. The instruction is decoded through an indirect jump.
Indirect jumps are still a burden for standard branch prediction
logic.

\begin{lstlisting}[float,caption={Typical JVM interpreter loop},
label=lst:intro:java:intprt]
    for (;;) {
        instr = bcode[pc++];
        switch (instr) {
            ...
            case IADD:
                tos = stack[sp]+stack[sp-1];
                --sp;
                stack[sp] = tos;
                break;
            ...
        }
    }
\end{lstlisting}

    \item[Just-In-Time Compilation:]
Interpreting JVMs can be enhanced with just-in-time (JIT) compilers.
A JIT compiler translates Java bytecodes to native instructions
during runtime. The time spent on compilation is part of the
application execution time. JIT compilers are therefore restricted
in their optimization capacity. To reduce the compilation overhead,
current JVMs operate in mixed mode: Java methods are executed in
interpreter mode and the call frequency is monitored. Often-called
methods, the hot spots, are then compiled to native code.

JIT compilation has several disadvantages for embedded systems,
notably that a compiler (with the intrinsic memory overhead) is
necessary on the target system. Due to compilation during
runtime, execution times are hardly predictable.\footnote{Even if
the time for the compilation is known, the WCET for a method has
to include the compile time! Furthermore, WCET analysis has to
know in advance what code will be produced by JIT compilation.}

    \item[Batch Compilation:]
Java can be compiled, in advance, to the native instruction set of
the target. Precompiled libraries are linked with the application
during runtime. This is quite similar to C/C++ applications with
shared libraries. This solution undermines the flexibility of Java:
dynamic class loading during runtime. However, this is not a major
concern for embedded systems.


    \item[Hardware Implementation:]
A Java processor is the implementation of the JVM in hardware. The
JVM bytecode is the native instruction set of such a processor. This
solution can result in quite a small processor, as a stack
architecture can be implemented very efficiently. A Java processor
is memory-efficient as an interpreting JVM, but avoids the execution
overhead. The main disadvantage of a Java processor is the lack of
capability to execute C/C++ programs.

\end{description}

\subsection{Embedded Java}

\begin{figure}
    \centering
    \includegraphics[width=\textwidth]{intro/jvmall}
    \caption{Implementation variations for an embedded JVM: (a) standard layers
    for Java with an operating system -- equivalent to desktop configurations, (b) a JVM on the bare metal,
    and (c) a JVM as a Java processor.}\label{fig:java:embedded}
\end{figure}

In embedded systems the architecture of JVMs are more diverse than
on desktop or server systems. Figure~\ref{fig:java:embedded} shows
variations of Java implementations in embedded systems and an
example of the control flow for a web server application. The
standard approach of a JVM running on top of an operating system
(OS) is shown in sub-figure (a). A network connection bypasses the
JVM via native functions and uses the TCP/IP stack implementation
and the device drivers of the OS.

A JVM without and OS is shown in sub-figure (b). This solution is
often called \emph{running on the bare metal}. The JVM acts as the OS
and provides the thread scheduling and the low-level access to the
hardware. In that case the network stack can be written entirely in
Java. JNode\footnote{\url{http://www.jnode.org/}} is an approach to
implement the OS entirely in Java. This solution becomes popular even
in server applications.\footnote{BEA System offers the JVM LiquidVM
that includes basic OS functions and does not need a guest OS.
%\url{http://www.bea.com/framework.jsp?CNT=index.htm&FP=/content/products/weblogic/virtual_server}
}

Sub-figure (c) shows an embedded solution where the JVM is part of
the hardware layer. That means it is implemented in a Java
processor. With this solution the native layer can be completely
avoided and all code (application and system code) is written
entirely in Java.

Figure~\ref{fig:java:embedded} shows how the flow from the
application goes down to the hardware. The example consists of a web
server and an Internet connection via Ethernet. In case (a) the
application web server talks with \code{java.net} in the JDK. The
flow goes through a native interface to the TCP/IP implementation and
the Ethernet device driver within the OS (usually written in C). The
device driver talks with the Ethernet chip. In (b) the OS layer is
omitted: the TCP/IP layer and the Ethernet device driver are now part
of the Java library. In (c) the JVM is part of the hardware layer and
a direct access from the Ethernet driver to the Ethernet hardware is
mandatory. Note how part of the network stack moves up from the OS
layer to the Java library. Version (c) shows a pure Java
implementation of the whole network stack.


\section{Summary}

Java is a unique combination of the language definition, a rich
class library and a runtime environment. A Java program is compiled
to bytecodes that are executed by a Java virtual machine. Strong
typing, runtime checks and avoidance of pointers make Java a
\emph{safe} language. The intermediate bytecode representation
simplifies porting of Java to different computer systems. An
interpreting JVM is easy to implement and needs few system
resources. However, the execution speed suffers from interpreting.
JVMs with a just-in-time compiler are state-of-the-art for desktop
and server systems. These compilers require large amounts of memory
and have to be ported for each processor architecture, which means
they are not the best choice for embedded systems. A Java processor
is the implementation of the JVM as a concrete machine. A Java
processor avoids the slow execution model of an interpreting JVM and
the memory requirements of a compiler, thus making it an interesting
execution system for Java in embedded systems.


\chapter{Restrictions of Java for Embedded Real-Time Systems}
\emph{Shall we keep this one? Or better SCJ/RtThread description
(with some of general Java ME related stuff -- shorter.}

\label{chap:rtjava}
    Java was created as a part of the Green project specifically for an
embedded device, a handheld wireless PDA. The device was never
released as a product and Java was launched as the \emph{new}
language for the Internet. Over time, Java become very popular to
build desktop applications and web services. However, embedded
systems are still programmed in C or C++. The pragmatic approach of
Java to object orientation, the huge standard library and
enhancements over C lead to a productivity increase, which now also
attracts embedded system programmers. A built-in concurrency model
and an elegant language construct to express synchronization between
threads also simplify typical programming idioms in this area.

On the other hand, there are some issues with Java in an embedded
system. Embedded systems are usually too small for JIT-compilation
resulting in a slow interpreting execution model. Moreover, a major
problem for embedded systems, which are usually also real-time
systems, is the under specification of the scheduler. Even an
implementation without preemption is allowed. The intention for this
\textit{loose} definition of the scheduler is to be able to
implement the JVM on many platforms where no good multitasking
support is available. The Real Time Specification for Java (RTSJ)
\cite{rtsj}  addresses many of these problems.

This section summarizes the issues with standard Java on embedded
systems and describes various definitions for small devices given by
Sun. It is followed by an overview of the two real-time extensions of
Java and approaches for restricting the RTSJ for high-integrity
applications. If, and how, these specifications are sufficient for
small embedded systems in general, and specifically for JOP, is
analyzed. The missing definition for small embedded real-time systems
is provided in Section~\ref{sec:rtprof}.

\section{Java Support for Embedded Systems}

When not using the cyclic executive approach, programming of
embedded (real-time) systems is all about concurrent programming
with time constraints. The basic functions can be summarized as:

\begin{itemize}
    \item Threads
    \item Communication
    \item Activation
    \item Low level hardware access
\end{itemize}

\paragraph{Threads and Communication}

Java has a built-in model for concurrency, the class \code{Thread}.
All threads share the same heap resulting in a shared memory
communication model. Mutual exclusion can be defined on methods or
code blocks with the keyword \code{synchronized}. Synchronized
methods acquire a lock on the object of the method. For synchronized
code blocks, the object to be locked is explicitly stated.


\paragraph{Activation}

Every object inherits the methods \code{wait()}, \code{notify()} and
\code{notifyAll()} from \code{Object}. These methods in conjunction
with synchronization on the object support activation. The classes
\code{java.util.TimerTask} and \code{java.util.Timer} (since JDK
1.3) can be used to schedule tasks for future execution in a
background thread.

\section{Issues with Java in Embedded Systems}

Although Java has language features that simplify concurrent
programming, the definition of these features is too vague for
real-time systems.


\paragraph{Threads and Synchronization}

Java, as described in \cite{JavaLangSpec2}, defines a very loose
behavior of threads and scheduling. For example, the specification
allows even low priority threads to preempt high priority threads.
This protects threads from starvation in general purpose
applications, but is not acceptable in real-time programming. Wakeup
of a single thread with \code{notify()} is not precisely defined:
\textit{the choice is arbitrary and occurs at the discretion of the
implementation.} It is not mandatory for a JVM to deal with the
priority inversion problem. No notation of periodic activities,
which are common in embedded systems programming, is available with
the standard \code{Thread} class.

\paragraph{Garbage Collector}

Garbage collection greatly simplifies programming and helps to avoid
classic programming errors (e.g.\ memory leaks). Although real-time
garbage collectors evolve, they are usually avoided in hard
real-time systems. A more conservative approach to memory allocation
is necessary.

\paragraph{WCET on Interfaces (OOP)}

Method overriding and Interfaces, the simplified concept of multiple
inheritance in Java, are the key concepts in Java to support object
oriented programming. Like function pointers in C, the dynamic
selection of the actual function at runtime complicates WCET
analysis. Implementation of interface look up usually requires a
search of the class hierarchy at runtime or very large dispatch
tables.

\paragraph{Dynamic Class Loading}

Dynamic class loading requires the resolution and verification of
classes. This is a function that is usually too complex (and
consumes too much memory) for embedded devices. An upper bound of
execution time for this function is almost impossible to predict (or
too large). This results in the complete avoidance of dynamic class
loading in real-time systems.

\paragraph{Standard Library}

For an implementation to be Java-conformant, it must include the
full library (JDK). The JAR files for this library constitute about
15~MB (in JDK 1.3, without native libraries), which is far too large
for many embedded systems. Since Java was designed to be a safe
language with a safe execution environment, no classes are defined
for low-level access of hardware features. The standard library was
not defined and coded with real-time applications in mind.

\paragraph{Execution Model}

The first execution model for the JVM was an interpreter. The
interpreter is now enhanced with Just-In-Time (JIT) compilation.
Interpreting Java bytecodes is too slow and JIT compilation is not
applicable in real-time systems. The time for the compilation
process had to be included in the WCET, resulting in impracticable
values.

\paragraph{Class Initialization Issues}
\label{para:restrict:clinit}

According to \cite{jvm} the static initializers of a class C are
executed immediately before one of the following occurs: (i) an
instance of C is created; (ii) a static method of C is invoked or
(iii) a static field of C is used or assigned. The issue with this
definition is that it is not allowed to invoke the static
initializers at JVM startup and it is not so obvious when it gets
invoked.

It follows that the bytecodes \code{getstatic}, \code{putstatic},
\code{invokestatic} and \code{new} can lead to class initialization
and the possibility of high WCET values. In the JVM, it is necessary
to check every execution of these bytecodes if the class is already
initialized. This leads to a loss of performance and is violated in
some existing implementations of the JVM. For example, the first
version of CACAO \cite{cacao} invokes the static initializer of a
class at compilation time. Listing~\ref{lst:retrict:clinit} shows an
example of this problem.

\cmd{JOPizer} tries to find a correct order of the class
initializers and puts this list into the application file. If a
circular dependency is detected the application will not be built.
The class initializers are invoked at JVM startup.

\begin{lstlisting}[float,caption={Class initialization can occur very late},
label=lst:retrict:clinit]
    public class Problem {

        private static Abc a;
        public static int cnt; // implicitly set to 0

        static {
            // do some class initializaion
            a = new Abc();  //even this is ok.
        }

        public Problem() {
            ++cnt;
        }
    }

    // anywhere in some other class, in situation,
    // when no instance of Problem has been created
    // the following code can lead to
    // the execution of the initializer
    int nrOfProblems = Problem.cnt;

\end{lstlisting}

\paragraph{Synchronization Issue}

Synchronization is possible with methods and on code blocks. Each
object has a monitor associated with it and there are two different
ways to gain and release ownership of a monitor. Bytecodes
\code{monitorenter} and \code{monitorexit} explicitly handle
synchronization. In other cases, synchronized methods are marked in
the class file with the access flags. This means that all bytecodes
for method invocation and return must check this access flag. This
results in an unnecessary overhead on methods without
synchronization. It would be preferable to encapsulate the bytecode
of synchronized methods with bytecodes \code{monitorenter} and
\code{monitorexit}. This solution is used in Suns picoJava-II
\cite{pjProgRef}. The code is manipulated in the class loader. Two
different ways of coding synchronization, in the bytecode stream and
as access flags, are inconsistent.

\section{Java Micro Edition}
\label{sec:j2me}

The definition of Java also includes the definition of the class
library (JDK). This is a huge library\footnote{In JDK 1.4 the main
runtime library, rt.jar, is 25~MB.} and too large for some systems.
To compensate for this Sun has defined the \textit{Java 2 Platform,
Micro Edition} (J2ME) \cite{J2ME}. As Sun has changed the focus of
Java targets several times, the specifications reflect this through
their slightly chaotic manner. J2ME reduces the function of the JVM
(e.g. no floating point support) to make implementation easier on
smaller processors. It also reduces the library (API). J2ME defines
three layers of software built upon the host operating system of the
device:
%
\begin{description}
    \item[Java Virtual Machine:] This layer is just the JVM as in every Java
implementation. Sun has assumed that the JVM will be implemented on
top of a host operating system. There are no additional definitions
for the J2ME in this layer.

    \item[Configuration:] The configuration defines the minimum set of JVM features
and Java class libraries available on a particular category of
devices. In a way, a configuration defines the lowest common
denominator of the Java platform features and libraries that the
developers can assume to be available on all devices.

    \item[Profile:] The profile defines the minimum set of Application
Programming Interfaces (APIs) available on a particular family of
devices. Profiles are implemented upon a particular configuration.
Applications are written for a particular profile and are thus
portable to any device that supports that profile. A device can
support multiple profiles.

\end{description}
%
There is an overlap of the layers \textit{configuration} and
\textit{profile}: Both define/restrict Java class libraries. Sun
states: `\textit{A profile is an additional way of specifying the
subset of Java APIs, class libraries, and virtual machine features
that targets a specific family of devices.'} However, in the current
available definitions JVM features are only specified in
\textit{configurations}.

\subsection{Connected Limited Device Configuration (CLDC)}
\label{subsec:cldc}

CLDC is a configuration for connected devices with at least 192~KB
of total memory and a 16-bit or 32-bit processor. As the main target
devices are cellular phones, this configuration has become very
popular (Sun: `\textit{CLDC was designed to meet the rigorous memory
footprint requirements of cellular phones.}'). The CLDC is composed
of the K Virtual Machine (KVM) and core class libraries. The
following features have been removed from the Java language
definition:
%
\begin{itemize}
    \item Floating point support
    \item Finalization
\end{itemize}
%
Error handling has been altered so that the JVM halts in an
implementation-specific manner. The following features have been
removed from the JVM:
%
\begin{itemize}
    \item Floating point support
    \item Java Native Interface (JNI)
    \item Reflection
    \item Finalization
    \item Weak references
    \item User-defined class loaders
    \item Thread groups and daemon threads
    \item Asynchronous exceptions
    \item Data type \code{long} is optional
    \item \code{wait()}, \code{notify()}, and \code{notifyAll()}
\end{itemize}
%
These restrictions are defined in the final version 1.0 of CLDC. A
newer version (1.1) again adds floating-point support.

The CLDC defines a subset of the following Java class libraries:
\code{java.io}, \code{java.lang}, \code{java.lang.ref} and
\code{java.util}. An additional library
(\code{javax.\linebreak[4]microedition.io}) defines a simpler
interface for communication than \code{java.io} and \code{java.net}.
Examples of connections are: HTTP, datagrams, sockets and
communication ports.

A small-footprint JVM, known as the K Virtual Machine (KVM), is part
of the CLDC distribution. KVM is suitable for 16/32-bit
microprocessors with a total memory budget of about 128~KB.

When implementing CLDC, one may choose to preload/prelink some
classes. A utility (\textit{JavaCodeCompact}) combines one or more
Java class files and produces a C file that can be compiled and
linked directly with the KVM.

There is only one profile defined under CLDC: the Mobile Information
Device Profile (MIDP) defines a user interface for LC displays,
a media player and a game API.

\subsection{Connected Device Configuration (CDC)}

The CDC defines a configuration for devices with a network connection
and assumes a minimum of a 32-bit processor and 2~MB memory. CDC
defines no restrictions for the JVM. A virtual machine, the CVM, is
part of the distribution. The CVM expects the following functionality
from the underlying OS:
%
\begin{itemize}
    \item Threads
    \item Synchronization (mutexes and condition variables)
    \item Dynamic linking
    \item malloc (POSIX memory allocation utility) or equivalent
    \item Input/output (I/O) functions
    \item Berkeley Standard Distribution (BSD) sockets
    \item File system support
    \item Function libraries must be thread-safe. A thread blocking in a library should not block any other VM threads.
\end{itemize}
%
The tools \textit{JavaCodeCompact} and \textit{JavaMemberDepend} are
part of the distribution. \textit{JavaMemberDepend} generates lists
of dependencies at the class member level. The existence of
\textit{JavaCodeCompact} implies that preloading of classes is
allowed in CDC. Three profiles are defined for CDC:
%
\begin{description}
    \item[Foundation Profile] is a set of Java APIs that support resource-constrained
devices without a standards-based GUI system. The basic class
libraries from the Java standard edition (\code{java.io},
\code{java.lang} and \code{java.net}) are supported and a connection
framework (\code{javax.microedition.io}) is added.

    \item[Personal Basis Profile] is a set of Java APIs that support
resource-constrained devices with a standards-based GUI framework
based on lightweight components. It adds some parts of the Abstract
Window Toolkit (AWT) support (relative to JDK 1.1 AWT).

    \item[Personal Profile] completes the AWT libraries and includes support for the
applet interface.

\end{description}
%
Although a device can support multiple profiles, additional libraries
for RMI and ODBC are known as \textit{optional packages}.

\subsection{Additional Specifications}

The following specifications do not fit into the layer scheme of
J2ME. However, they are defined in the same way as the above:
subsets of the JVM and subsets/extensions of Java classes (API):
%
\begin{description}
    \item[Java Card] is a definition for the resource-constrained world of smart
cards. The execution lifetime of the JVM is the lifetime of the
card. The JVM is highly restricted (e.g. no threads, data type
\code{int} is optional) and defines a different instructions set
(i.e. new bytecodes to support smaller integer types).

    \item[Java Embedded Server] is an API definition for services such as HTTP.

    \item[Personal Java] was intended as a Java platform on Windows CE
and is now marked as end of life.

    \item[Java TV] is an extension to produce interactive
television content and manage digital media. The description states
that the JVM runs on top of an RTOS, but no real-time specific
extensions are defined.

\end{description}
%
Other than Sun's, the few specifications that exist for embedded
Java are:
%
\begin{description}
    \item[leJOS] \cite{lejos} is a JVM for Lego Mindstorm with stronger restrictions
on the core classes than the CLDC.

    \item[RTDA] \cite{rtda01} although named `Real-Time Data Access' the
definition consists of two parts:

    \begin{itemize}
        \item An I/O data access API specification applicable
        for real-time and non real-time applications.
        \item A minimal set of real-time extensions to enable
        the I/O data access also to cover hard real-time capable
        response handling.
    \end{itemize}
\end{description}

\subsection{Discussion}

Many of the specifications (i.e.\ \textit{configurations} and
\textit{profiles}) are developed using the Java Community Process
(JCP). Although the acronym J2ME implies Java version 2 (i.e. JDK
1.2 and later) almost all technologies under J2ME are still based on
JDK 1.1.

Besides Java Card, CLDC is the `smallest' definition from Sun. It
assumes an operating system and is quite large (the JAR file for the
classes is about 450KB). There are no API definitions for low-level
hardware access. CLDC is not suitable for small embedded devices.
Java Card defines a different JVM instruction set and thus
compromises basic ideas of Java. A more restricted definition with
following features is needed:
%
\begin{itemize}
    \item JVM restrictions, such as in CLDC 1.0
    \item A package for low-level hardware access
    \item A minimum subset of core libraries
    \item Additional profiles for different application domains
\end{itemize}


\section{Real-Time Extensions}

In 1999, a document defining the requirements for real-time Java was
published by NIST \cite{nist99}. Based on these requirements, two
groups defined specifications for real-time Java. A comparison of
these two specifications and a comparison with Ada 95's Real-Time
Annex can be found in \cite{507579}. The following section gives an
overview of these specifications and additional defined restrictions
of the RTSJ.

\subsection{Real-Time Core Extension}

The Real-Time Core Extension \cite{JCons00} is a specification
published under the J Consortium. Two execution environments are
defined: the \textit{Core} environment is the special real-time
component. It can be combined with a traditional JVM, the
\textit{Baseline}. For communication between these two domains,
every Core object has two APIs, one for the Core domain and one for
the Baseline domain. Baseline components can synchronize with Core
components via semaphores.

Two forms of source code are supported to annotate attributes:
\textit{stylized} code with calls of static methods of special
classes and \textit{syntactic} code with new keywords. Syntactic
code has to be processed by a special compiler or preprocessor.

The RT-Core as a specification faded into history. Many of the
proposed concepts can be found in NewMonics (now Aonix) PERC systems
\cite{PERC}. A more detailed description of the RT-Core extensions
can be found in \cite{jop:rtjava}.

\subsection{Real-Time Specification for Java}
\label{sec:rtsj}

The Real-Time Specification for Java (RTSJ) defines a new API with
support from the JVM \cite{rtsj}. The following guiding principles
led to the definition:
%
\begin{itemize}
    \item No restriction of the Java runtime environment
    \item Backward compatibility for non-real-time Java programs
    \item No syntactic extension to the Java
language or new keywords
    \item Predictable execution
    \item Address current real-time system practice
    \item Allow future implementations to add advanced features
\end{itemize}
%
A reference implementation (RI) of the RTSJ forms part of the
specification. The RTSJ is backward compatible with existing
non-real-time Java programs, which implies that the RTSJ is intended
to run on top of J2SE (and not on J2ME). The following section
presents an overview of the RTSJ.

\paragraph{Threads and Scheduling}

The behavior of the scheduler is more clearly defined than in
standard Java. A priority-based, preemptive scheduler with at least
28 real-time priorities is defined as the base scheduler. Additional
levels (ten) for the traditional Java threads need to be available.
Threads with the same priority are queued in FIFO order. Additional
schedulers (e.g. EDF) can be provided by the RTSJ implementation. The
class \code{Scheduler} and associated classes provide optional
support for feasibility analysis.


Any instances of classes that implement the interface
\code{Schedulable} are scheduled. In the RTSJ \code{RealtimeThread},
\code{NoHeapRealtimeThread}, and \code{AsyncEventHandler} are
\textit{schedulable objects}. \code{NoHeapRealtimeThread} has and
\code{AsyncEventHandler} can have a priority higher than the garbage
collector. As the available release-parameters indicate, threads are
either periodic or bound to asynchronous events. Threads can be
grouped together to bind the execution cost and deadline for a
period.

\paragraph{Memory}

As garbage collection is problematic in real-time applications, the
RTSJ defines new memory areas:
%
\begin{description}
    \item[Scoped memory] is a memory area with bounded lifetime. When a scope is
entered (with a new thread or through \code{enter()}), all new
objects are allocated in this memory area. Scoped memory areas can
be nested and shared among threads. On exit of the last thread from
a scope, all finalizers of the allocated objects are invoked and the
memory area is freed.
    \item[Physical memory] is used to control allocation in memories with
different access time.
    \item[Raw memory] allows byte-level access to physical memory or memory-mapped I/O.
    \item[Immortal memory] is a memory area shared between
all threads without a garbage collector. All objects created in this
memory area have the same lifetime as the application (a new
definition of \textit{immortal}).
    \item[Heap memory] is the traditional garbage collected memory area.
\end{description}
%
Maximum memory usage and the maximum allocation rate per thread can
be limited. Strict assignment rules between the different memory
areas have to be checked at runtime by the implementation.

\paragraph{Synchronization}

The implementation of \code{synchronized} has to include an
algorithm to prevent priority inversion. The priority inheritance
protocol is the default and the priority ceiling emulation protocol
can be used on request. Threads waiting to enter a synchronized
block are priority ordered and FIFO ordered within each priority.
Wait free queues are provided for communication between instances of
\code{java.lang.Thread} and \code{RealtimeThread}.

\paragraph{Time and Timers}

Classes to represent relative and absolute time with nanosecond
accuracy are defined. All time parameters are split to a \code{long}
for milliseconds and an \code{int} for nanoseconds within those
milliseconds. Each time object has an associated \code{Clock}
object. Multiple clocks can represent different sources of time and
resolution. This allows for the reduction of queue management
overheads for tasks with different tolerance for jitter. A new type,
rationale time, can be used to describe periods with a requested
resolution over a longer period (i.e. allowing release jitter
between the points of the \textit{outer} period)\footnote{Rationale
time is depreciated in the current version of the specification}.
Timer classes can generate time-triggered events (one shot and
periodic).

\paragraph{Asynchrony}

Program logic representing external world events is scheduled and
dispatched by the scheduler. An \code{AsyncEvent} object represents
an external event (such as a POSIX signal or a hardware interrupt)
or an internal event (through call of \code{fire()}). Event handlers
are associated to these events and can be bound to a regular
real-time thread or represent something \textit{similar} to a
thread. The relationship between events and handlers can be
many-to-many. Release of handlers can be restricted to a minimum
interarrival time.


Java's exception handling is extended to represent asynchronous
transfer of control (ATC). \code{RealtimeThread} overloads
\code{interrupt()} to generate an
\code{AsynchronousInterruptedException} (AIE). The AIE is deferred
until the execution of a method that is willing to accept an ATC.
The method indicates this by including AIE in its throw clause. The
semantics of \code{catch} is changed so that, even when it catches
an AIE, the AIE is still propagated until the \code{happened()}
method of the AIE is invoked. \code{Timed}, a subclass of AIE,
simplifies the programming of timeouts.

\paragraph{Support for the RTSJ}

Implementations of the RTSJ are still rare and under development:
%
\begin{description}
    \item[RI]
is the freely available reference implementation for a Linux system
\cite{rtsj-ri}.

    \item[jRate]
is an open-source implementation \cite{701668} based on
ahead-of-time compilation with the GNU compiler for Java.

    \item[FLEX]
is a compiler infrastructure for embedded systems developed at MIT
\cite{flex}. Real-time Java is implemented with region-based memory
management and a scheduler framework.

    \item[OVM]
is an open-source framework for Java \cite{ovmir2003}. The emphasis
is on a JVM that is compliant with the RTSJ. RTSJ support is based
on the translation of the complete Java application (including the
library) to C and then compiling it into a native executable.

    \item[aJile] (member of the initial JSR-1 expert group) announced the support the RTSJ with CLDC 1.0 on top of
    the aJ-80 and aJ-100 chips. However, the RTSJ was never
    implemented by aJile.

    \item[JamaicaVM] is a RTSJ implementation with a real-time GC by
    aicas.

    \item[IBM] provides a RTSJ implementation combined with the
    Metronome GC \cite{gc:bacon03}.

    \item[Sun's] latest implementation of the RTSJ contains a
    real-time GC developed at Lund University.
\end{description}


\subsection{Discussion of the RTSJ}

The RTSJ is a complex specification leading to a big memory
footprint. The following list shows the size of the main components
of the first RI on Linux:
%
\begin{itemize}
    \item Classes in javax/realtime: 343 KB
    \item All classes in library foundation.jar: 2 MB
    \item Timesys JVM executable: 2.6 MB
\end{itemize}
%
The RTSJ assumes an RTOS and the RI runs on a heavyweight RT-Linux
system. The RTSJ is too complex for low-end embedded systems. This
complexity also hampers programming of high-integrity applications.
The runtime memory allocation of the RTSJ classes is not documented.

\paragraph{Threads and Scheduling}

If a real-time thread is preempted by a higher priority thread, it
is not defined if the preempted thread is placed in front or back of
the waiting queue. The current version (1.0.2) of the
RTSJ\footnote{\url{http://www.rtsj.org/specjavadoc/book_index.html}}
states:
\begin{quote}
A schedulable object that is preempted by a higher priority
schedulable object is placed in the queue for its active priority,
at a position determined by the implementation.
\end{quote}
It is not specified whether the default scheduler performs, or has
to perform, time slicing between threads of equal priority.

\paragraph{Memory}


It would be ideal if real-time systems were able to allocate all
memory during the initialization phase and forbid dynamic memory
allocation in the mission phase. However, this restricts many of
Java's library functions.


The solution to this problem in the RTSJ is \code{ScopedMemory}, a
memory space with limited lifetime. However, it can only be used as
a parameter for thread creation or with \code{enter(Runnable r)}. In
a system without dynamic thread creation, using scoped memory at
creation time of the thread leads to the same behavior as using
immortal memory.

The syntax with \code{enter()} leads to a cumbersome programming
style: for each code part where limited lifetime memory is needed, a
new class has to be defined and a single instance of this class
allocated at initialization time. Trying to solve this problem
elegantly with anonymous classes, as in
Listing~\ref{lst:restr:wrong:scoped} (example from \cite{558498}, p.
623), leads to an error.

\begin{lstlisting}[float,caption={Scoped memory usage with a memory leak},label=lst:restr:wrong:scoped]
    import javax.realtime.*;
    public class ThreadCode implements Runnable
    {
        private void computation()
        {
            final int min = 1*1024;
            final int max = 1*1024;
            final LTMemory myMem = new LTMemeory(min, max);
            myMem.enter(new Runnable()
            {
                public void run()
                {
                    // access to temporary memory
                {
            } );
        }

        public void run()
        {
            ...
            computation();
            ...
        }
    }
\end{lstlisting}


On every call of \code{computation()}, an object of the anonymous
class (and a \code{LTMemory} object) is allocated in immortal
memory, leading to a memory leak. The correct usage of scoped memory
is shown as a code fragment in
Listing~\ref{lst:restr:correct:scoped}. The class \code{UseMem} only
exists to execute the method \code{run()} in scoped memory. One
instance of this class is created outside of the scoped memory.

\begin{lstlisting}[float,caption={Correct usage of scoped memory in the RTSJ},
label=lst:restr:correct:scoped]
    class UseMem implements Runnable {

        public void run() {
            // inside scoped memory
            Integer[] = new Integer[100];
            ...
        }
    }

    // outside of scoped memory
    // in immortal? at initialization?
    LTMemory mem = new LTMemory(1024, 1024);
    UseMem um = new UseMem();

    // usage
    computation() {
        mem.enter(um);
    }
\end{lstlisting}



A simpler\footnote{This syntax is \emph{not} part of the RTSJ. Is is
a suggested change and part of the real-time profile defined in
Section~\ref{sec:rtprof}.} syntax is shown in
Listing~\ref{lst:restr:simple:scoped}. The main drawback of this
syntax is that the programmer is responsible for its correct usage.

\begin{lstlisting}[float,caption={Simpler syntax for scoped memory}
,label=lst:restr:simple:scoped]
    LTMemory myMem;

    // Create the memory object once
    // in the constructor
    MyThread() {
        myMem = new LTMemeory(min, max);
        ...
    }

    public void run() {
        ...
        myMem.enter();
        {   // A new code block disables access
            // to new objects in outer scope.
            // Access to temporary memory:
            Abc a = new Abc();
            ...
        }
        myMem.exit();
        ...
    }
\end{lstlisting}

New objects and arrays of objects have to be initialized to their
default value after allocation \cite{jvm}. This usually results in
zeroing the memory at the JVM level and leads to variable (but
linear) allocation time. This is the reason for the type
\code{LTMemoryArea} in the RTSJ. As suggested in \cite{701668}, this
initialization could be lumped together with the creation time and
exit time of the scoped memory. This results in constant time for
allocation (and usually faster zeroing of the memory).


With the RTSJ memory areas, it is difficult to move data from one
area to another \cite{Niessner03}. This results in a completely
different programming model from that of standard Java. This can
result in the programmer developing his/her own memory management.

One solution to use scoped memory for communication is proposed in
\cite{conf/isorc/PizloFHV04}. A \emph{wedge} thread has to keep the
scope alive when data is added by the producer. The consumer has to
notify this wedge thread when all data is consumed the exit the scope
for the memory recycle. However, there is no single instant available
where we can \emph{guarantee} that the list is empty. A possible
solution for this problem is described in
\cite{conf/isorc/PizloFHV04} as the \emph{handoff} pattern. The
pattern is similar to double buffering, but with an explicit copy of
the data.


\paragraph{Time and Timers }

High resolution time is split into milliseconds and nanoseconds.
Calculation of time differences (an important operation for a
real-time system) is complex with this representation of the time.
In the RI, it is converted to ns for add/subtract. After all mapping
and converting (\code{AbsoluteTime}, \code{HighResolutionTime},
\code{Clock} and \code{RealTimeClock}) the
\code{System.currentTimeMillis()} time, with a ms resolution, is
used.

Since time triggered release of tasks can be modeled with periodic
threads, the additional concept of timers is superfluous.

\paragraph{Asynchronous Events}

An unbound \code{AsyncEventHandler} is not allowed to \code{enter()}
a scoped memory. However, it is not clear if scoped memory is
allowed as a parameter in the construction of a handler.

An unbound \code{AsyncEventHandler} leads to the implicit start of a
thread on an event. This can (and, in the RI, does -- see
\cite{701668}) lead to substantial overheads. From the application
perspective, bound and unbound event handlers behave in the same
way. This is an implementation hint expressed through different
classes. A consistent way to express the \textit{importance} of
events would be a scheduling parameter for the minimum allowed
latency of the handler.

\paragraph{Asynchronous Transfer of Control (ATC)}

The syntax that is used in the throws clause of a method to state
that ATC will be accepted is misleading. Exceptions in \code{throws}
clauses of a method are usually \emph{generated} in that method and
not \textit{accepted}.

\paragraph{J2SE Library}

It is not specified which classes are safe to be used in
\linebreak[4]\code{RealTimeThread} and \code{NoHeapRealTimeThread}.
Several operating system functions can cause unbound blocking and
their usage should be avoided. The memory allocation in standard JDK
methods is not documented and their use in immortal memory context
can lead to memory leaks.

\paragraph{Missing Features}

There is no concept such as start mission. Changing scheduling
parameters during runtime can lead to inconsistent scheduling
behavior.

There is no provision for low-level blocking such as disabling
interrupts. This is a common technique in device drivers where some
hardware operations have to be atomic without affecting the priority
level of the requesting thread (e.g.\ a low priority thread for a
flash file system shall not get preempted during sector write as
some chips internally start the write after a timeout).

\paragraph{On Small Systems}

Many embedded systems are still built with 8 or 16-bit CPUs. 32-bit
processors are seldom used. Java's default integer type is 32-bit,
still large enough for almost all data types needed in embedded
systems. The design decision in the RTSJ to use (often expensive)
64-bit \code{long} data is questionable.

\subsection{Subsets of the RTSJ}
\label{subsec:restr:rtsj}

The RTSJ is complex to implement and applications developed with the
RTSJ are difficult to analyze (because of some of the sophisticated
features of the RTSJ). Various profiles have been suggested for
high-integrity real-time applications that result in restrictions of
the RTSJ.

\subsubsection{A Profile for High-Integrity Real-Time Java Programs}


In \cite{Pusch01}, a subset of the RTSJ for the high-integrity
application domain with hard real-time constraints is proposed. It
is inspired by the Ravenscar profile for Ada \cite{289525} and
focuses on exact temporal predictability.
%
\begin{description}
\item[Application structure:] The application is divided into two
    different phases: \textit{initialization} and
    \textit{mission}. All non time-critical initialization,
    global object allocations, thread creation and startup are
    performed in the initialization phase. All classes need to be
    loaded and initialized in this phase. The mission phase
    starts after returning from \code{main()}, which is assumed
    to execute with maximum priority. The number of threads is
    fixed and the assigned priorities remain unchanged.

\item[Threads:] Two types of tasks are defined: \textit{Periodic
time-triggered activities} execute an infinite loop with at least
one call of \code{waitForNextPeriod()}. \textit{Sporadic activities}
are modeled with a new class \code{SporadicEvent}. A
\code{SporadicEvent} is bound to a thread and an external event on
creation. Unbound event handlers are not allowed. It is not clear if
the event can also be triggered by software (invocation of
\code{fire()}). A restriction for a minimum interarrival time of
events is not defined. Timers are not supported as time-triggered
activities are well supported by periodic threads. Asynchronous
transfers of control, overrun and miss handles and calls to
\code{sleep()} are not allowed.

\item[Concurrency:] Synchronized methods with priority ceiling emulation protocol
provide mutual exclusion to shared resources. Threads are dispatched
in FIFO order within each priority level. Sporadic events are used
instead of \code{wait()}, \code{notify()} and \code{notifyAll()} for
signaling.

\item[Memory:] Since garbage collection is still not
time-predictable, it is not supported. This implicitly converts the
traditional heap to immortal memory. Scoped memory (\code{LTMemory})
is provided for object allocation during the mission phase.
\code{LTMemory} has to be created during the initialization phase
with initial size equal maximum size.

\item[Implementation:] For each thread and for the operations
of the JVM the WCET must be computable. Code is restricted to bound
loops and bound recursions. Annotations for WCET analysis are
suggested. The JVM needs to check the timing of events and thread
execution. It is not stated how the JVM should react to a timing
error.

\end{description}

\subsubsection{Ravenscar-Java}
\label{subsec:rj}

The Ravenscar-Java (RJ) profile \cite{583825} is a restricted subset
of the RTSJ and is based on the work mentioned above. As the name
implies it resembles Ravenscar Ada \cite{289525} concepts in Java.


To simplify the initialization phase, RJ defines \code{Initializer},
a class that has to be extended by the application class which
contains \code{main()}. The use of time scoped memory is further
restricted. \code{LTMemory} areas are not allowed to be nested nor
shared between threads. Traditional Java threads are disallowed by
changing the class \code{java.lang.Thread}. The same is true for all
schedulable objects from the RTSJ. Two new classes are defined:
\begin{itemize}
    \item \code{PeriodicThread} where \code{run()} gets called periodically,
removing the loop construct with \code{waitForNextPeriod()}.
    \item \code{SporadicEventHandler} binds a single thread with a
single event. The event can be an interrupt or a software event.

\end{itemize}

\subsubsection{Safety Critical Java Technology}

A subset of the RTSJ based on the former ideas is under development
under JSR 302\footnote{\url{http://jcp.org/en/jsr/detail?id=302}}.
The subset is intended for Java in safety critical applications that
have to be certified.

\subsubsection{Criticisms on Subsets of the RTSJ}

If a new real-time profile is defined as a subset of the RTSJ it is
harder for the programmer to find out which functions are available
or not. This form of \textit{compatibility} causes confusion. The
use of different classes for a different specification is clearer
and less error prone.

Ravenscar-Java, as a subset of the RTSJ, claims to be compatible
with the RTSJ, in the sense that programs written according to the
profile are valid RTSJ programs. However, mandatory usages of new
classes such as \code{PeriodicThread} need an emulation layer to run
on an RTSJ system. In this case, it is better to define complete new
classes for a subset and provide the mapping to the RTSJ. This
allows a clearer distinction to be made between the two definitions.

It is not necessary to distinguish between heap and immortal memory.
Without a garbage collector, the heap implicitly equals to immortal
memory.

Objects are allocated in immortal memory in the initialization phase.
In the mission phase, no objects should be allocated in immortal
memory. Scoped memory can be entered and subsequent new objects are
allocated in the scoped memory area. Since there are no circumstances
in which allocation in these two memory areas are mixed, no
\code{newInstance()}as defined in the RTSJ is not necessary in a
Ravenscar-Java style profile.

\subsection{Extensions to the RTSJ}

The Distributed Real-Time Specification for Java \cite{Jensen00}
extends RMI within the RTSJ. In 2000, it was accepted in the Sun
Community Process as JSR-50. This specification is still under
development. According to \cite{WellRTSJRMI}, three levels of
integration between the RTSJ and RMI are defined:
%
\begin{description}
    \item[Level 0:] No changes in RMI and the RTSJ are necessary. The proxy thread
on the server acts as an ordinary Java thread. Real-time threads
cannot assume timely delivery of the RMI request.

    \item[Level 1:] RMI is extended to Real-Time RMI. The server
thread is a real-time thread that inherits scheduling parameters
from the calling client.

    \item[Level 2:] RMI and the RTSJ are extended to form the
concept of \textit{distributed real-time threads}. These threads
have a unique system-wide identifier and can move freely in the
distributed system.

\end{description}

JSR-50 is still not finalized. Only minor increments have been
announced at JTRES~2006 and JTRES~2007.

\section{Summary}

In this section, we described definitions for embedded devices given
by Sun. Most of these definitions are targeted for the mobile phone
market and not for classical embedded systems.

Standard Java is under-specified for real-time systems. Two
competing definitions, the `Real-Time Core Extension' and the `Real
Time Specification for Java', address this problem. The RTSJ has
been further restricted for high-integrity applications. A similar
definition that avoids inheritance of complex RTSJ classes is
provided in Section~\ref{sec:rtprof}.


\chapter{JOP Architecture}
\label{chap:arch}

    This chapter presents the architecture of JOP and the motivation
behind the various different design decisions we faced. First, we
benchmark the JVM, in order to extract execution frequencies for the
different bytecodes. These values will then guide the processor
design.

Pipelined instruction processing calls for high memory bandwidth.
Caches are needed in order to avoid bottlenecks resulting from the
main memory bandwidth. As seen in Chapter~\ref{chap:java}, there are
two memory areas that are frequently accessed by the JVM: the stack
and the method area. In this chapter, we will present
time-predictable cache solutions for both areas.

    \section{Benchmarking the JVM}
\label{sec:bench:jvm}

The rationale behind this section is best introduced with a warning
from Computer Architecture: A Quantitative Approach \cite{Hennessy02}
p. 63:

\begin{quote}
Virtually every practicing computer architect knows Amdahl�s Law.
Despite this, we almost all occasionally fall into the trap of
expending tremendous effort optimizing some aspect of a system
before we measure its usage. Only when the overall speedup is
unrewarding do we recall that we should have measured the usage of
that feature before we spent so much effort enhancing it!
\end{quote}
%
We measured how Java programs use the bytecode instruction set and
explored the typical and worst-case method sizes. Our measurements
and other reports are presented in the following sections.

\subsection{Bytecode Frequency}

The dynamic instruction frequency is the main measurement for
determining a processor implementation. We can identify those
instructions that should be fast. For seldom-used instructions, a
trade-off can be made between performance and hardware resources.

Many reports have been written about JVM bytecode frequencies (e.g.\
\cite{Greg2002, 365338, 624084}). Most of these reports provide only
a coarse categorization of the bytecodes. For example, the bytecodes
\code{iload\_n} (load an \code{int} from a local variable) and
\code{getfield} (fetch a field from an object) are combined in one
instruction category. However, these instructions are very different
in terms of their implementation complexity. We have chosen a
fine-grained categorization of the bytecodes to gain greater insight
into the bytecode usage. In Table~\ref{tab_java_instr_cat} all 201
bytecode instructions are listed by category.


\begin{table}
    \centering
    \begin{tabular}{ll}
    \toprule
    Type & Bytecode \\
    \midrule

    load
    & aload, dload, fload, iload, lload \\

    load (short)
    & aload\_0, aload\_1, aload\_2, aload\_3, \\
    & dload\_0, dload\_1, dload\_2, dload\_3, \\
    & fload\_0, fload\_1, fload\_2, fload\_3, \\
    & iload\_0, iload\_1, iload\_2, iload\_3, \\
    & lload\_0, lload\_1, lload\_2, lload\_3 \\

    store
    & astore, dstore, fstore, istore, lstore \\

    store (short)
    & astore\_0, astore\_1, astore\_2, astore\_3, \\
    & dstore\_0, dstore\_1, dstore\_2, dstore\_3, \\
    & fstore\_0, fstore\_1, fstore\_2, fstore\_3, \\
    & istore\_0, istore\_1, istore\_2, istore\_3, \\
    & lstore\_0, lstore\_1, lstore\_2, lstore\_3 \\

    const
    & bipush, ldc, ldc\_w, ldc2\_w, sipush \\

    const (short)
    & aconst\_null, dconst\_0, dconst\_1, fconst\_0, fconst\_1, fconst\_2, \\
    & iconst\_0, iconst\_1, iconst\_2, iconst\_3, iconst\_4, iconst\_5, \\
    & iconst\_m1, lconst\_0, lconst\_1 \\

    get
    & getfield, getstatic \\

    put
    & putfield, putstatic \\

    alu
    & dadd, ddiv, dmul, dneg, drem, dsub, \\
    & fadd, fdiv, fmul, fneg, frem, fsub, \\
    & iadd, iand, idiv, imul, ineg, ior, irem, ishl, ishr, isub, iushr, ixor, \\
    & ladd, land, ldiv, lmul, lneg, lor, lrem, lshl, lshr, lsub, lushr, lxor \\

    iinc
    & iinc \\

    stack
    & dup, dup\_x1, dup\_x2, dup2, dup2\_x1, dup2\_x2, pop, pop2, swap \\

    array
    & aaload, aastore, baload, bastore, caload, castore, daload, dastore, \\
    & faload, fastore, iaload, iastore, laload, lastore, saload, sastore \\

    branch
    & goto, goto\_w, if\_acmpeq, if\_acmpne, if\_icmpeq, \\
    & if\_icmpge, if\_icmpgt, if\_icmple, if\_icmplt, if\_icmpne, \\
    & ifeq, ifge, ifgt, ifle, iflt, ifne, ifnonnull, ifnull \\

    compare
    & dcmpg, dcmpl, fcmpg, fcmpl, lcmp \\

    switch
    & lookupswitch, tableswitch \\

    call
    & invokeinterface, invokespecial, invokestatic, invokevirtual \\

    return
    & areturn, dreturn, freturn, ireturn, lreturn, return \\

    conversion
    & d2f, d2i, d2l, f2d, f2i, f2l, i2b, i2c, i2d, i2f, i2l, i2s, l2d, l2f, l2i \\

    new
    & anewarray, multianewarray, new, newarray \\

    other
    & arraylength, athrow, checkcast, instanceof, jsr, jsr\_w, \\
    & monitorenter, monitorexit, nop, ret, wide \\


    \bottomrule
    \end{tabular}
    \caption[Categories of JVM bytecodes]{The 201 Java bytecodes and
    their assignment to different categories}
    \label{tab_java_instr_cat}
\end{table}


Three different applications were run on an instrumented JVM to
measure dynamic bytecode frequency. The results were compared with
the results from the above-mentioned reports. In
Table~\ref{tab_java_instr_frequ} the dynamic instruction count for
the three different benchmarks is shown. The last column is the
average of the three tests weighted by the individual instructions
count.

Kaffe \cite{kaffe} is an independent implementation of the JVM
distributed under the GNU Public License. Kaffe was instrumented to
collect data on dynamic bytecode usage. Three different applications
were used as benchmarks to obtain the dynamic instruction count:
JLex, KCJ and javac. JLex \cite{jlex} is a lexical analyzer
generator, written for Java in Java. The data was collected by
running JLex with the provided \code{sample.lex} as the input file.
KJC \cite{kcj} is a Java compiler in Java, freely available under the
terms of the GNU General Public License. javac is the Sun Java
compiler. Both compilers were compiling part of the KJC sources
during the benchmark. These benchmarks are similar to the benchmarks
used in other reports and the results are therefore comparable.
However, typical embedded applications can result in a slightly
different instruction set usage pattern. Embedded applications are
usually tightly connected with the environment and are therefore not
available as stand-alone programs to serve as benchmarks. An embedded
application developed on JOP was adapted to serve as a benchmark for
Section~\ref{sec:cache} and Chapter~\ref{chap:results}.

%19,572,165 + 951,138,375 + 341,926,231 = 1,312,636,771

% JLex.Main   kaffe JLex.Main JLex/sample.lex
% at.dms.kjc.Main:
%    kaffe -cp kjc-2.1B-bin.jar at.dms.kjc.Main -d classes kopi-2.1B/src/kjc/*.java
% com.sun.tools.javac.Main:
%    kaffe -cp /usr/lib/java/lib/tools.jar com.sun.tools.javac.Main -d classes kopi-2.1B/src/kjc/*.java
%  stops with compile error


\begin{table}
    \centering
    \begin{tabular}{lrrrr}
        \toprule
        & JLex & KJC & javac & Average \\
        \midrule
        load (short) & 32.72 & 31.45 & 27.24 & 30.37 \\
        get & 12.02 & 14.39 & 17.04 & 15.04 \\
        branch & 11.26 & 10.40 & 10.71 & 10.49 \\
        invoke & 6.87 & 6.31 & 4.24 & 5.77 \\
        return & 6.82 & 6.20 & 4.17 & 5.68 \\
        load & 7.59 & 4.19 & 7.48 & 5.09 \\
        alu & 2.60 & 4.43 & 4.74 & 4.48 \\
        const (short) & 4.61 & 4.26 & 4.74 & 4.39 \\
        array & 4.22 & 4.07 & 3.22 & 3.85 \\
        put & 0.78 & 2.14 & 3.65 & 2.52 \\
        iinc & 1.81 & 2.38 & 1.41 & 2.12 \\
        stack & 1.30 & 2.11 & 2.11 & 2.10 \\
        store (short) & 2.61 & 2.18 & 1.71 & 2.06 \\
        other & 1.63 & 2.22 & 1.21 & 1.95 \\
        const & 0.85 & 1.56 & 2.80 & 1.87 \\
        store & 2.05 & 0.85 & 1.94 & 1.15 \\
        conversion & 0.02 & 0.36 & 0.58 & 0.42 \\
        switch & 0.00 & 0.20 & 0.60 & 0.30 \\
        new & 0.08 & 0.28 & 0.20 & 0.25 \\
        compare & 0.14 & 0.03 & 0.22 & 0.08 \\
        \bottomrule
    \end{tabular}
    \caption{Dynamic bytecode frequency in \%}
    \label{tab_java_instr_frequ}
\end{table}

\begin{table}
    \centering
    \begin{tabular}{lrlr}
        \toprule
        \multicolumn{2}{c}{JLex, KJC and javac} &
        \multicolumn{2}{c}{SPEC and Java Grande} \\
        \cmidrule(lr){1-2} \cmidrule(lr){3-4}
        Instruction & Frequency & Instruction & Frequency \\
        \midrule
        load (short)  & 30.37 & acnst &  0.07 \\
        load          &  5.09 & aload & 16.23 \\
        const (short) &  4.39 & fcnst &  0.33 \\
        const         &  1.87 & fload &  6.33 \\
                      &       & icnst &  3.21 \\
                      &       & iload & 18.06 \\
        \midrule
        load \& const & \textbf{41.72}& & \textbf{44.77} \\
        \midrule
        get           & 15.04 & field & 11.12 \\
        put           &  2.52 &       &       \\
        \midrule
        field access & \textbf{17.56} & & \textbf{11.12} \\
        \midrule
        branch        & 10.49 & cjump &  5.67 \\
        compare       &  0.08 & ujump &  0.51 \\
        \midrule
        control & \textbf{10.57}& & \textbf{6.18} \\
        \midrule
        invoke & \textbf{5.77} & fcall &  \textbf{3.63} \\
        \midrule
        return &  \textbf{5.68} & retrn &  \textbf{2.07} \\
        \bottomrule
    \end{tabular}
    \caption[Dynamic bytecode frequency comared]
    {Dynamic bytecode frequency compared with the
    measurements from \cite{Dowling2002}}
    \label{tab:java:instr:frequ:comp}
\end{table}

In \cite{Dowling2002} the relationship between static and dynamic
instruction frequency of 19 programs from the SPECjvm98
\cite{SPECJvm98} and Java Grande benchmark suites were measured. The
bytecodes categories were chosen differently from the above
measurements, but detailed enough to verify our own measurements.
\tablename~\ref{tab:java:instr:frequ:comp} shows the average dynamic
execution frequency in percent\footnote{The values do not add up to
100\% as only the most significant bytecode categories are shown} of
selected bytecode categories from the SPEC and Java Grande
benchmarks, compared with the results obtained by our measurements.
The numbers in bold are categories or sums of categories that are
comparable. The frequency of the ``load \& const" instructions is
very similar to that in our measurements. However, field access,
control instructions and method invocations are more frequent in our
measurements. The higher count on field access instructions and
method invocation can result from a more object oriented programming
style in our selected applications than in the SPEC and Java Grande
benchmarks. The big difference, not seen in our measurements, between
the invoke and return frequency in the SPEC and Java Grande
benchmarks is not explained in \cite{Dowling2002}.

In all measurements, the load of local variables and constants onto
the stack accounts for more than 40\% of instructions executed. This
feature shows that an efficient realization of the local variable
memory area, the stack and the transfer between these memory areas
is mandatory.

The next most executed bytecodes (\code{getfield} and
\code{getstatic}) are the instructions that load an object or class
field onto the operand stack. To account for these frequent
instructions, the class layout for the runtime system has to be
optimized for quick resolution of field addresses (i.e.\ minimum
memory indirections).

The frequency of branches is comparable with the SPECint2000
measurements on RISC processors \cite{Hennessy02}. With such a high
branch frequency, a processor without branch prediction logic is put
under pressure in terms of pipeline length.

It is interesting to note that there are more method invoke
instructions than return instructions. Two facts are responsible for
this difference: native methods are invoked by a bytecode, but the
return is inside the native methods; and an exception can result in
a method exit without return.


\subsection{Methods Types and Length}
\label{sec:bench:jvm:methods}

\tablename~\ref{tab_java_meth_type} shows the number of dynamic
method calls of the Java Grande and \linebreak[4]SPECjvm98
benchmarks. It can be seen that the distribution of method types
depends on the application type. Usage of virtual methods and
interfaces is common in OO programming. Static methods result from
the simple translation of procedural programs to Java.

\begin{table}
    \centering
    \begin{tabular}{lrrrr}
        \toprule
         & virtual & special & static & interface \\
        \midrule
        Java Grande & 57.1 & 8.7 & 34.2 & 0.0 \\
        SPEC JVM98 & 81.0 & 10.9 & 2.9 & 5.2 \\
        \bottomrule
    \end{tabular}
    \caption{Types of different dynamic method calls for two benchmarks (from \cite{Power2002})}
    \label{tab_java_meth_type}
\end{table}

As a basis for the proposed cache solution in
Section~\ref{sec:cache}, we will explore static distribution of
method sizes. In the JVM, only relative branches are defined. The
conditional branches and goto have an offset of 16 bits, resulting
in a practical limit of the method length of 32KB. Although there is
a goto instruction with a wide index (\emph{goto\_w}) that takes a
4-byte branch offset, other factors (e.g.\ indices in the exception
table) limit the size of a method to 65535 bytes.

Radhakrishnan et al.\ \cite{365338} measured the dynamic method size
of the SPEC suite. They observed a `tri-nodal' distribution, where
most of the methods were 1, 9, or 26 bytecodes long. No explanation
is given for the sizes of 9 or 26. The explanation of the 1 bytecode
long methods as \emph{wrapper methods} is wrong. For a wrapper
method, the method needs to contain a minimum of two instructions (an
invoke and a return). A single instruction method can \emph{only}
contain a return. However, this observation is in sharp contrast to
the measurements obtained by Power and Waldron in \cite{Power2002}.




\begin{table}
    \centering
    \begin{tabular}{rrrr}
        \toprule
        Length & Methods & Percentage & Cumulative \\
        & & & percentage \\
        \midrule
        1 & 1,388 & 1.94 & 1.94 \\
        2 & 1,580 & 2.21 & 4.16 \\
        4 & 1,871 & 2.62 & 6.78 \\
        8 & 16,192 & 22.67 & 29.45 \\
        16 & 12,363 & 17.31 & 46.76 \\
        32 & 12,638 & 17.70 & 64.45 \\
        64 & 11,178 & 15.65 & 80.10 \\
        128 & 7,287 & 10.20 & 90.31 \\
        256 & 4,304 & 6.03 & 96.33\\
        512 & 1,727 & 2.42 & 98.75 \\
        1,024 & 592 & 0.83 & 99.58 \\
        2,048 & 175 & 0.25 & 99.83 \\
        4,096 & 75 & 0.11 & 99.93 \\
        8,192 & 37 & 0.05 & 99.98 \\
        16,384 & 11 & 0.02 & 100.00 \\
        32,768 & 1 & 0.00 & 100.00 \\
        65,536 & 0 & 0.00 & 100.00 \\
        \bottomrule
    \end{tabular}
    \caption{Static method count of different sizes from the runtime library (JDK 1.4).}
    \label{tab_java_jdk_static_size}
\end{table}

In Table~\ref{tab_java_jdk_static_size}, the number of methods of
different sizes in the Java runtime library (JDK 1.4) is shown. The
library consists of 71419 methods, the largest being 16706 bytes.
The size is classified by powers of 2 because we are interested in
the size of cache memory for complete methods. In the table, the row
of, for example, size 32 includes all methods of a size from 17 to
32 bytes. It can be seen that methods are typically very short. In
fact, 99\% of the methods are less than 513 bytes in size. This
property is important for the proposed method cache in
Section~\ref{sec:cache}, where a complete method has to fit into the
instruction cache.

All larger methods are different kinds of initialization functions,
in most cases \code{$<$clinit$>$()}\footnote{The class or interface
initialization method is static and the special name
\codefoot{$<$clinit$>$} is supplied by the compiler. These
initialization methods are invoked implicitly by the JVM. The
definition when these methods get invoked is problematic for the
WCET analysis (see Section~\ref{para:restrict:clinit}).}. The large
class initialization methods typically result from the
initialization of arrays with constant data. This is necessary
because of the lack of initialized data segments, such as the BSS in
C, in the Java class file. These initialization methods contain
straight-line code and can therefore be split to smaller methods
automatically, if necessary.

\begin{figure}
    \centering
    \includegraphics[width=\excelwidth]{arch/arch_meth32}
    \caption[Static method count for methods of size up to 32 bytes]
    {Static method count for methods of size up to 32 bytes in the JDK 1.4 runtime library.
    The horizontal axis indicates the method size.}
    \label{fig_java_meth32}
\end{figure}

\begin{figure}
    \centering
    \includegraphics[width=\excelwidth]{arch/arch_meth300}
    \caption[Static method count from the JDK 1.4 runtime library]
    {Static method count from the JDK 1.4 runtime library.
    The horizontal axis indicates the method size in bytes.}
    \label{fig_java_meth300}
\end{figure}

In Figure~\ref{fig_java_meth32}, the distribution of small methods
up to a size of 32 bytes is shown.
\figurename~\ref{fig_java_meth300} shows the method count for
methods up to 300 bytes. As expected, we see fewer methods as size
increases. We observed no surprise in the distribution, unlike the
`tri-nodal' distribution in \cite{365338}. The only method size that
is very common is 5 bytes. These methods are the typical setter and
getter methods in object-oriented programming as shown in
Listing~\ref{lst:arch:java:getval}.

\begin{lstlisting}[float, caption={Bytecodes for a getter method},label=lst:arch:java:getval]
    private int val;

    public int getVal() {
        return val;
    }

    public int getVal();
    Code:
    0:   aload_0
    1:   getfield        #2; //Field val:I
    4:   ireturn
\end{lstlisting}

The method \code{getVal()} translates to three bytecodes of 1, 3 and
1 bytes in length respectively. These methods should show up in
\cite{365338} as a peak at 3 bytecodes.

The static distribution of method sizes in an application (javac,
the Java compiler) is quite similar to the distribution in the
library. In the class file that contains the Java compiler, 98\% of
the methods are smaller than 513 bytes, and the larger methods are
class initializers.

\subsection{Summary}

In this section, we performed dynamic measurements on the JVM
instruction set. We saw that more than 40\% of the executed
instructions are local variables or constants loads onto the stack.
This high frequency of stack access calls for an efficient
implementation of the stack, as described in
Section~\ref{sec:stack}.

In addition, we have statically measured method sizes. Methods are
typically very short. 30\% of the methods are shorter than 9 bytes
and 99\% account for methods of up to 512 bytes. The maximum length
is further limited by the definition of the class file. We will use
this property in the proposed \emph{method cache} in
Section~\ref{sec:cache}.

Instruction-usage data is an important input for the design of a
processor architecture, as seen in the following sections.


%\subsection{Some more properties}
%Number of local variables per method.


%\subsubsection{Some Comments}
%
%* Clarification of the quick instructions

%* A word about javac: absolute not optimized code generations.
%Simplifies JIT, but not so good for Java processors or an
%interpreting JVM

%\dots CPI is not so easy to measure for the JVM. JVM = runtime
%infrastructure (GC, new). What is measured with JIT?

%\clearpage or pagebreak[x]
    This chapter presents the architecture of JOP and the motivation
behind the various different design decisions we faced. The first
sections give an overview of JOP, describe the microcode and the
pipeline.

Pipelined instruction processing calls for high memory bandwidth.
Caches are needed in order to avoid bottlenecks resulting from the
main memory bandwidth. As seen in Chapter~\ref{chap:java}, there are
two memory areas that are frequently accessed by the JVM: the stack
and the method area. In this chapter, time-predictable cache
solutions for both areas that are implemented in JOP are presented.

\section{Overview of JOP}

This section gives an overview of JOP architecture.
Figure~\ref{fig:arch:jop:block} shows JOP's major function units. A
typical configuration of JOP contains the processor core, a memory
interface and a number of I/O devices. The module extension provides
the link between the processor core, and the memory and I/O modules.

\begin{figure}[t]
    \centering
    \includegraphics{arch/arch_jop_block}
    \caption{Block diagram of JOP}
    \label{fig:arch:jop:block}
\end{figure}


The processor core contains the three microcode pipeline stages
\emph{microcode fetch}, \emph{decode} and \emph{execute} and an
additional translation stage \emph{bytecode fetch}. The ports to the
other modules are the two top elements of the stack (TOS and NOS),
input to the top-of-stack (Data), bytecode cache address and data,
and a number of control signals. There is no direct connection
between the processor core and the external world.

The memory controller implements the simple memory load and store
operations and the field and array access bytecodes. It also contains
the method cache. The memory interface provides a connection between
the main memory and the memory controller. The extension module
controls data read and write. The \emph{busy} signal is used by the
microcode instruction \code{wait}\footnote{The busy signal can also
be used to stall the whole processor pipeline. This was the change
made to JOP by Flavius Gruian \cite{jop:sac05}. However, in this
synchronization mode, the concurrency between the memory access
module and the main pipeline is lost.} to synchronize the processor
core with the memory unit. The core reads bytecode instructions
through dedicated buses (BC address and BC data) from the memory
controller. The request for a method to be placed in the cache is
performed through the extension module, but the cache hit detection
and load is performed by the memory controller independently of the
processor core (and therefore concurrently).

The I/O interface contains peripheral devices, such as the system
time and timer interrupt for real-time thread scheduling, a serial
interface and application-specific devices. Read and write to and
from this module are controlled by the memory controller. All
external devices\footnote{The external device can be as simple as a
line driver for the serial interface that forms part of the interface
module, or a complete bus interface, such as the ISA bus used to
connect e.g.\ an Ethernet chip.} are connected to the I/O interface.

The extension module performs two functions: (a) it contains hardware
accelerators (such as the multiplier unit in this example) and (b)
the multiplexer for the read data that is loaded into the
top-of-stack register. The write data from the top-of-stack (TOS) and
next-of-stack (NOS) are connected directly to all modules.

The division of the processor into those modules greatly simplifies
the adaptation of JOP for different application domains or hardware
platforms. Porting JOP to a new FPGA board usually results in changes
in the memory interface alone. Using the same board for different
applications only involves making changes to the I/O interface. JOP
has been ported to several different FPGAs and prototyping boards and
has been used in different real-world applications (see
Chapter~\ref{chap:results}), but it never proved necessary to change
the processor core.

\section{Microcode}
\label{sec:microcode}

The following discussion concerns two different instruction sets:
\emph{bytecode} and \emph{microcode}. Bytecodes are the instructions
that make up a compiled Java program. These instructions are
executed by a Java virtual machine. The JVM does not assume any
particular implementation technology. Microcode is the native
instruction set for JOP. Bytecodes are translated, during their
execution, into JOP microcode. Both instruction sets are designed
for an extended\footnote{An extended stack machine is one in which
there are instructions available to access elements deeper down in
the stack.} stack machine.

\subsection{Translation of Bytecodes to Microcode}

To date, no hardware implementation of the JVM exists that is
capable of executing \emph{all} bytecodes in hardware alone. This is
due to the following: some bytecodes, such as \code{new}, which
creates and initializes a new object, are too complex to implement
in hardware. These bytecodes have to be emulated by software.

To build a self contained JVM without an underlying operating
system, direct access to the memory and I/O devices is necessary.
There are no bytecodes defined for low-level access. These low-level
services are usually implemented in \emph{native} functions, which
means that another language (C) is native to the processor. However,
for a Java processor, bytecode is the \emph{native} language.

One way to solve this problem is to implement simple bytecodes in
hardware and to emulate the more complex and \emph{native} functions
in software with a different instruction set (sometimes called
microcode). However, a processor with two different instruction sets
results in a complex design.

Another common solution, used in Sun's picoJava \cite{pjMicroArch},
is to execute a subset of the bytecode native and to use a software
trap to execute the remainder. This solution entails an overhead (a
minimum of 16 cycles in picoJava, see \ref{subsec:related:picojava})
for the software trap.

In JOP, this problem is solved in a much simpler way. JOP has a
single \emph{native} instruction set, the so-called microcode.
During execution, every Java bytecode is translated to either one,
or a sequence of microcode instructions. This translation merely
adds one pipeline stage to the core processor and results in no
execution overheads (except for a bytecode branch that takes 4
instead of 3 cycles to execute). The area overhead of the
translation stage is 290~LCs, or about 15\% of the LCs of a typical
JOP configuration. With this solution, we are free to define the JOP
instruction set to map smoothly to the stack architecture of the
JVM, and to find an instruction coding that can be implemented with
minimal hardware.

\begin{figure}
    \centering
    \includegraphics{arch/arch_indirection}
    \caption{Data flow from the Java program counter to JOP microcode}
    \label{fig_arch_data_flow}
\end{figure}


Figure~\ref{fig_arch_data_flow} gives an example of the data flow
from the Java program counter to JOP microcode. The figure
represents the two pipeline stages bytecode fetch/translate and
microcode fetch. The fetched bytecode acts as an index for the jump
table. The jump table contains the start addresses for the bytecode
implementation in microcode. This address is loaded into the JOP
program counter for every bytecode executed. JOP executes the
sequence of microcode until the last one. The last one is marked
with \emph{nxt} in microcode assembler. This \emph{nxt} bit in the
microcode ROM triggers a new translation i.e., a new address is
loaded into the JOP program counter. In
Figure~\ref{fig_arch_data_flow} the implementation of bytecode
\code{idiv} is an example of a longer sequence that ends with
microcode instruction \code{ldm c nxt}.

The difference to other forms of instruction translation in hardware
is that this solution is time predictable. The translation takes one
cycle (one pipeline stage) for each bytecode, independent from the
execution history. Instruction folding, e.g., implemented in picoJava
\cite{pJ1,pjMicroArch}, is also a form of instruction translation in
hardware. Folding is used to translate several (stack oriented)
bytecode instructions to a RISC type instruction. This translation
needs an instruction buffer and the fill level of this instruction
buffer depends on the execution history. The length of this history
that has to be considered for analysis is not bounded. Therefore this
form of instruction translation is not exactly time predictable.


\subsection{Compact Microcode}

For the JVM to be implemented efficiently, the microcode has to
\emph{fit} to the Java bytecode. Since the JVM is a stack machine,
the microcode is also stack-oriented. However, the JVM is not a pure
stack machine. Method parameters and local variables are defined as
\emph{locals}. These locals can reside in a stack frame of the
method and are accessed with an offset relative to the start of this
\emph{locals} area.

Additional local variables (16) are available at the microcode level.
These variables serve as scratch variables, like registers in a
conventional CPU. Furthermore, the constant pool pointer (cp), the
method pointer (mp), and pointers to the method tables of
\code{JVM.java} and \code{JVMHelp.java} are stored in these
variables. The 16 variables are located in the on-chip stack memory.
However, arithmetic and logic operations are performed on the stack.

Some bytecodes, such as ALU operations and the short form access to
\emph{locals}, are directly implemented by an equivalent microcode
instruction (with a different encoding). Additional instructions are
available to access internal registers, main memory and I/O devices.
A relative conditional branch (zero/non zero of TOS) performs control
flow decisions at the microcode level. For optimum use of the
available memory resources, all instructions are 8 bits long. There
are no variable-length instructions and every instruction, with the
exception of \code{wait}, is executed in a single cycle. To keep the
instruction set this dense, the following concept is applied:
immediate values and branch offsets are addressed through one
indirection. The instruction just contains an index for the
constants.

Two types of operands, immediate values and branch distances,
normally force an instruction set to be longer than 8 bits. The
instruction set is either expanded to 16 or 32 bits, as in typical
RISC processors, or allowed to be of variable length at byte
boundaries. A first implementation of the JVM with a 16-bit
instruction set showed that only a small number of different
constants are necessary for immediate values and relative branch
distances.

In the current realization of JOP, the different immediate values
are collected while the microcode is being assembled and are put
into the initialization file for the on-chip memory. These constants
are accessed indirectly in the same way as the local variables. They
are similar to initialized variables, apart from the fact that there
are no operations to change their value during runtime, which would
serve no purpose and would waste instruction codes.  The microcode
local variables, the microcode constants and the stack share the
same on-chip memory. Using a single memory block simplifies the
multiplexer in the execution stage.

A similar solution is used for branch distances. The assembler
generates a VHDL file with a table for all found branch constants.
This table is indexed using instruction bits during runtime. These
indirections during runtime make it possible to retain an 8-bit
instruction set, and provide 16 different immediate values and 32
different branch constants. For a general purpose instruction set,
these indirections would impose too many restrictions. As the
microcode only implements the JVM, this solution is a viable option.

To simplify the logic for instruction decoding, the instruction
coding is carefully chosen. For example, one bit in the instruction
specifies whether the instruction will increment or decrement the
stack pointer. The offset to access the \emph{locals} is directly
encoded in the instruction. This is not the case for the original
encoding of the equivalent bytecodes (e.g. \emph{iload\_0} is 0x1a
and \emph{iload\_1} is 0x1b). Whenever a multiplexer depends on an
instruction, the selection is directly encoded in the instruction.

\subsection{Instruction Set}

JOP implements 54 different microcode instructions. These
instructions are encoded in 8 bits. With the addition of the
\emph{nxt} and \emph{opd} bits in every instruction, the effective
instruction length is 10 bits.

\begin{description}
    \item[Bytecode equivalent:]
These instructions are direct implementations of bytecodes and
result in one cycle execution time for the bytecode (except
\code{st} and \code{ld}): \code{pop}, \code{and}, \code{or},
\code{xor}, \code{add}, \code{sub}, \code{st$<$n$>$}, \code{st},
\code{ushr}, \code{shl}, \code{shr}, \code{nop}, \code{ld$<$n$>$},
\code{ld}, \code{dup}

    \item[Local memory access:]
The first 16 words in the internal stack memory are reserved for
internal variables. The next 16 words contain constants. These
memory locations are accessed using the following instructions:
\code{stm}, \code{stmi}, \code{ldm}, \code{ldmi}, \code{ldi}

    \item[Register manipulation:]
The stack pointer, the variable pointer and the Java program counter
are loaded or stored with: \code{stvp}, \code{stjpc}, \code{stsp},
\code{ldvp}, \code{ldjpc}, \code{ldsp}, \code{star}

    \item[Bytecode operand:]
The operand is loaded from the bytecode RAM, converted to a 32-bit
word and pushed on the stack with: \code{ld\_opd\_8s},
\code{ld\_opd\_8u}, \code{ld\_opd\_16s}, \code{ld\_opd\_16u}

    \item[External memory access:] The autonomous memory
        subsystem and the I/O subsystem are accessed by using the
        following instructions: \code{stmra}, \code{stmwa},
        \code{stmwd}, \code{wait}, \code{ldmrd}, \code{stbcrd},
        \code{ldbcstart}, \code{stald}, \code{stast},
        \code{stgf}, \code{stpf}, \code{stcp}

    \item[Multiplier:]
The multiplier is accessed with: \code{stmul}, \code{ldmul}

    \item[Microcode branches:]
Two conditional branches in microcode are available: \code{bz},
\code{bnz}

    \item[Bytecode branch:]
All 17 bytecode branch instructions are mapped to one instruction:
\code{jbr}

\end{description}
%
A detailed description of the microcode instructions can be found in
Appendix~\ref{appx:jop:instr}.

\subsection{Bytecode Example}

The example in Figure~\ref{lst:arch:micro1} shows the implementation
of a single cycle bytecode and an infrequent bytecode as a sequence
of JOP instructions. The suffix \code{nxt} marks the last instruction
of the microcode sequence. In this example, the \code{iadd} bytecode
is mapped to the equivalent \code{add} microcode and executed in a
single cycle, whereas \code{swap} takes four cycles to execute, and
after the last instruction (\code{ldm b nxt}), the first instruction
for the next bytecode is executed. The scratch variables, as shown in
the second example, are stored in the on-chip memory that is shared
with the stack cache.

\begin{figure}
\begin{lstlisting}[caption={Implementation of \code{iadd} and \code{swap}},
label=lst:arch:micro1]
    iadd:   add nxt    // 1 to 1 mapping

    //  a and b are scratch variables for the
    //  JVM code.
    swap:   stm a      // save TOS in variable a
            stm b      // save TOS-1 in variable b
            ldm a      // push a on stack
            ldm b nxt  // push b on stack and fetch next bytecode
\end{lstlisting}
\end{figure}

Some bytecodes are followed by operands of between one and three
bytes in length (except \code{lookupswitch} and \code{tableswitch}).
Due to pipelining, the first operand byte that follows the bytecode
instruction is available when the first microcode instruction enters
the execution stage. If this is a one-byte long operand, it is ready
to be accessed. The increment of the Java program counter after the
read of an operand byte is coded in the JOP instruction (an
\emph{opd} bit similar to the \emph{nxt} bit).

In Listing~\ref{lst:arch:micro2}, the implementation of
\code{sipush} is shown. The bytecode is followed by a two-byte
operand. Since the access to bytecode memory is only
one\footnote{The decision is to avoid buffers that would introduce
time dependencies over bytecode boundaries.} byte per cycle,
\emph{opd} and \emph{nxt} are not allowed at the same time. This
implies a minimum execution time of $n+1$ cycles for a bytecode with
$n$ operand bytes.

\begin{figure}
\begin{lstlisting}[caption={Bytecode operand load},
label=lst:arch:micro2]
    sipush: nop opd        // fetch next byte
            nop opd        // and one more
            ld_opd_16s nxt // load 16 bit operand
\end{lstlisting}
\end{figure}

\subsection{Microcode Branches}

At the microcode level two conditional branches that test the TOS are
available: \code{bz} branch on zero, and \code{bnz} branch on not
zero. The branches are followed by two delay slots, i.e., the
following two instructions are executed independent of the branch
condition outcome. Furthermore, the branch condition is also
pipelined, i.e., it has to be available one cycle earlier.
Listing~\ref{lst:arch:micro3} shows the condition delay and the
branch delay slots.

\begin{figure}
\begin{lstlisting}[caption={Microcode condition delay and branch delay slots},
label=lst:arch:micro3]
        add            // sets the condition for the branch
        nop            // one cycle condition delay slot
        bz label       // a branch on TOS zero
        instr1         // is executed
        instr2         // is executed
        instr3         // executed on fall through
\end{lstlisting}
\end{figure}

\subsection{Flexible Implementation of Bytecodes}
\label{subsec:flex:bc}

As mentioned above, some Java bytecodes are very complex. One
solution already described is to emulate them through a sequence of
microcode instructions. However, some of the more complex bytecodes
are very seldom used. To further reduce the resource implications
for JOP, in this case local memory, bytecodes can even be
implemented by \emph{using} Java bytecodes. That means bytecodes
(e.g., \code{new} or floating point operations) can be implemented
in Java. This feature also allows for the easy configuration of
resource usage versus performance.

During the assembly of the JVM, all labels that represent an entry
point for the bytecode implementation are used to generate the
translation table. For all bytecodes for which no such label is
found, i.e.\ there is no implementation in microcode, a
\emph{not-implemented} address is generated. The instruction sequence
at this address invokes a static method from a system class. This
class contains 256 static methods, one for each possible bytecode,
ordered by the bytecode value. The bytecode is used as the index in
the method table of this system class. A single empty static method
consumes three 32-bit words in memory. Therefore, the overhead of
this special class is 3~KB, which is 9\% of a minimal \emph{hello
world} program (34~KB memory footprint).

\subsection{Summary}

In order to handle the great variation in the complexity of Java
bytecodes, the bytecodes are translated to a different instruction
set, the so-called microcode. This microcode is still an instruction
set for a stack machine, but more RISC-like than the CISC-like JVM
bytecodes.

At the time of this writing 43 of the 201 different bytecodes are
implemented by a single microcode instruction, 92 by a microcode
sequence, and 41 bytecodes are implemented in Java. Furthermore, JOP
contains additional bytecodes that are used to implement low-level
operations, such as direct memory access. Those bytecodes are mapped
to native, static methods in \code{com.jopdesign.sys.Native}. In the
next section we will see how this translation is handled in JOP's
pipeline and how it can simplify interrupt handling.


\section{The Processor Pipeline}
\label{sec:pipeline}

JOP is a fully pipelined architecture with single cycle execution of
microcode instructions and a novel approach of translation from Java
bytecode to these instructions. Figure~\ref{fig_arch_pipeline} shows
the datapath for JOP, representing the pipeline from left to right.
Blocks arranged vertically belong to the same pipeline stage.

\begin{figure}[t]
    \centering
    \includegraphics[scale=\picscale]{arch/arch_pipeline}
    \caption{Datapath of JOP}
    \label{fig_arch_pipeline}
\end{figure}

Three stages form the JOP core pipeline, executing microcode
instructions. An additional stage in the front of the core pipeline
fetches Java bytecodes -- the instructions of the JVM -- and
translates these bytecodes into addresses in microcode. Bytecode
branches are also decoded and executed in this stage. The second
pipeline stage fetches JOP instructions from the internal microcode
memory and executes microcode branches. Besides the usual decode
function, the third pipeline stage also generates addresses for the
stack RAM (the stack cache). As every stack machine microcode
instruction (except \code{nop}, \code{wait}, and \code{jbr}) has
either \emph{pop} or \emph{push} characteristics, it is possible to
generate fill or spill addresses for the \emph{following}
instruction at this stage. The last pipeline stage performs ALU
operations, load, store and stack spill or fill. At the execution
stage, operations are performed with the two topmost elements of the
stack.

The stack architecture allows for a short pipeline, which results in
short branch delays. Two branch delay slots are available after a
conditional microcode branch. A stack machine with two explicit
registers for the two topmost stack elements and automatic
fill/spill to the stack cache needs neither an extra write-back
stage nor any data forwarding. See Section~\ref{sec:stack} for a
detailed description.

The method cache (\emph{Bytecode Cache}), microcode ROM, and stack
RAM are implemented with single cycle access in the FPGA's internal
memories.


\subsection{Java Bytecode Fetch}

In the first pipeline stage, as shown in
Figure~\ref{fig_arch_bc_fetch}, the Java bytecodes are fetched from
the internal memory (\emph{Method cache}). The bytecode is mapped
through the translation table into the address (\emph{jpaddr}) for
the microcode ROM. Interrupts and exceptions are handled by
redirection of the microcode address to the handler code.

\begin{figure}[t]
    \centering
    \includegraphics[scale=\picscale]{arch/arch_bcfetch}
    \caption{Java bytecode fetch and translation}
    \label{fig_arch_bc_fetch}
\end{figure}

The fetched bytecode results in an absolute jump in the microcode
(the second stage). If the bytecode is mapped one-to-one with a JOP
instruction, the following fetched bytecode again results in a jump
in the microcode in the following cycle. If the bytecode is a complex
one, JOP continues to execute microcode. At the end of this
instruction sequence, the next bytecode, and therefore the new jump
address, is requested (signal \emph{nxt}).

The method cache serves as the instruction cache and is filled on
method invoke and return. Details about this time-predictable
instruction cache can be found in Section~\ref{sec:cache}.

The bytecode is also stored in a register for later use as an
operand (requested by signal \emph{opd}). Bytecode branches are also
decoded and executed in this stage. Since \emph{jpc} is also used to
read the operands, the program counter is saved in \emph{jpcbr}
during an instruction fetch. \emph{jinstr} is used to decode the
branch type and \emph{jpcbr} to calculate the branch target address.

\subsection{Microcode Instruction Fetch}

The second pipeline stage, as shown in Figure~\ref{fig_arch_fetch},
fetches microcode instructions from the internal microcode memory and
executes microcode branches.

\begin{figure}[t]
    \centering
    \includegraphics[scale=\picscale]{arch/arch_fetch}
    \caption{Microcode instruction fetch}
    \label{fig_arch_fetch}
\end{figure}

The JOP microcode, which implements the JVM, is stored in the
microcode ROM. The program counter \emph{pc} is incremented during
normal execution. If the instruction is labeled with \emph{nxt} a new
bytecode is requested from the first stage and \emph{pc} is loaded
with \emph{jpaddr}. \emph{jpaddr} is the starting address for the
implementation of that bytecode. The label \emph{nxt} is the flag
that marks the end of the microcode instruction stream for one
bytecode. Another flag, \emph{opd}, indicates that a bytecode operand
needs to be fetched in the first pipeline stage. Both flags are
stored in the microcode ROM.

The register \emph{brdly} contains the target address for a
conditional branch. The same offset is shared by a number of branch
destinations. A table (\emph{branch offset}) is used to store these
relative offsets. This indirection means that only 5 bits need to be
used in the instruction coding for branch targets and thereby allow
greater offsets. The three tables \emph{translation table} (from the
bytecode fetch stage), \emph{microcode ROM}, and \emph{branch offset}
are generated during the assembly of the JVM code. The outputs are
plain VHDL files. For an implementation in an FPGA, recompiling the
design after changing the JVM implementation is a straightforward
operation. For an ASIC with a loadable JVM, it is necessary to
implement a different solution.

FPGAs available to date do not allow asynchronous memory access. They
therefore force us to use the registers in the memory blocks.
However, the output of these registers is not accessible. To avoid
having to create an additional pipeline stage just for a
register-register move, the read address register of the microcode
ROM is fed by the \emph{pc} multiplexer. The memory address register
effectively contains the same value as the \emph{pc}.

\begin{figure}[t]
    \centering
    \includegraphics[scale=\picscale]{arch/arch_decaddr}
    \caption{Decode and address generation}
    \label{fig_arch_decode}
\end{figure}

\subsection{Decode and Address Generation}

Besides the usual decode function, the third pipeline, as shown in
Figure~\ref{fig_arch_decode}, also generates addresses for the stack
RAM.


As we can see in Section~\ref{sec:stack}
Table~\ref{tab_stack_address}, read and write addresses are either
relative to the stack pointer or to the variable pointer. The
selection of the pre-calculated address can be performed in the
decode stage. When an address relative to the stack pointer is used
(either as read or as write address, never for both) the stack
pointer is also decremented or incremented in the decode stage.

Stack machine instructions can be categorized from a stack
manipulation perspective as either \emph{pop} or \emph{push}. This
allows us to generate fill or spill TOS-1 addresses for the
\emph{following} instruction during the decode stage, thereby saving
one extra pipeline stage.


\subsection{Execute}

At the execution stage, as shown in Figure~\ref{fig_arch_exe},
operations are performed using two discrete registers: TOS and
TOS-1, labeled \emph{A} and \emph{B}.

\begin{figure}[t]
    \centering
    \includegraphics[scale=\picscale]{arch/arch_execute}
    \caption{Execution stage}
    \label{fig_arch_exe}
\end{figure}

Each arithmetic/logical operation is performed with registers
\emph{A} and \emph{B} as the source (top-of-stack and next-of-stack),
and register \emph{A} as the destination. All load operations (local
variables, internal register, external memory and periphery) result
in a value being loaded into register \emph{A}. There is no need for
a write-back pipeline stage. Register \emph{A} is also the source for
the store operations. Register \emph{B} is never accessed directly.
It is read as an implicit operand or for stack spill on push
instructions. It is written during the stack spill with the content
of the stack RAM or the stack fill with the content of register
\emph{A}.

Beside the Java stack, the stack RAM also contains microcode
variables and constants. This resource-sharing arrangement not only
reduces the number of memory blocks needed for the processor, but
also the number of data paths to and from the register \nolinebreak
\emph{A}.

The inverted clock on data-in and on the write address register of
the stack RAM is used to perform the RAM write in the same cycle as
the execute operation.

A stack machine with two explicit registers for the two topmost
stack elements and automatic fill/spill needs neither an extra
write-back stage nor any data forwarding. Details of this two-level
stack architecture are described in Section~\ref{sec:stack}.

%\subsection{Pipeline Example}
%
%\emph{\textbf{Show from Java code downto FPGA HW.}}
%
%Is in brown bock.

\subsection{Interrupt Logic}
\label{sec:interrupt}
\index{interrupt}

Interrupts and (precise) exceptions are considered hard to implement
in a pipelined processor \cite{Hennessy02}, meaning implementation
tends to be complex (and therefore resource consuming). In JOP, the
bytecode-microcode translation is used cleverly to avoid having to
handle interrupts and exceptions (e.g., stack overflow) in the core
pipeline.

Interrupts are implemented as special bytecodes. These bytecodes are
inserted by the hardware in the Java instruction stream. When an
interrupt is pending and the next fetched byte from the bytecode
cache is an instruction (as indicated by the \emph{nxt} bit in the
microcode), the associated special bytecode is used instead of the
instruction from the bytecode cache. The result is that interrupts
are accepted at bytecode boundaries. The worst-case preemption delay
is the execution time of the \emph{slowest} bytecode that is
implemented in microcode. Bytecodes that are implemented in Java
(see Section \ref{subsec:flex:bc}) can be interrupted.

The implementation of interrupts at the bytecode-microcode mapping
stage keeps interrupts transparent in the core pipeline and avoids
complex logic. Interrupt handlers can be implemented in the same way
as standard bytecodes are implemented i.e.\ in microcode or Java.

This special bytecode can result in a call of a JVM internal method
in the context of the interrupted thread. This mechanism implicitly
stores almost the complete context of the current active thread on
the stack. This feature is used to implement the preemptive, fixed
priority real-time scheduler in Java \cite{jop:javasched}.

The main source for an interrupt is the $\mu$s accurate timer
interrupt used by the real-time scheduler. I/O device interrupts can
also be connected to the interrupt controller. Hardware generated
exceptions, such as stack overflow or array bounds checks, generate a
system interrupt. The exception reason can be found in a register.

\subsection{Summary}

In this section, we have analyzed JOP's pipeline. The core of the
stack machine constitutes a three-stage pipeline. In the following
section, we will see that this organization is an optimal solution
for the stack access pattern of the JVM.

An additional pipeline stage in front of this core pipeline stage
performs bytecode fetch and the translation to microcode. This
organization has zero overheads for more complex bytecodes and
results in the short pipeline that is necessary for any processor
without branch prediction. This additional translation stage also
presents an elegant way of incorporating interrupts virtually
\emph{for free}.


\clearpage
    \section{An Efficient Stack Machine}
    \label{sec:stack}
    
The concept of a stack has a long tradition, but stack machines no
longer form part of mainstream computers. Although stacks are no
longer used for expression evaluation, they are still used for the
context save on a function call. A niche language, Forth
\cite{Koopman89}, is stack-based and known as an efficient language
for controller applications. Some hardware implementations of the
Forth abstract machine do exist. These Forth processors are stack
machines.

The Java programming language defines not only the language but also
a binary representation of the program and an abstract machine, the
JVM, to execute this binary. The JVM is similar to the Forth
abstract machine in that it is also a stack machine. However, the
usage of the stack differs from Forth in such a way that a Forth
processor is not an ideal hardware platform to execute Java
programs.

In this section, the stack usage in the JVM is analyzed. We will see
that, besides the access to the top elements of the stack, an
additional access path to an arbitrary element of the stack is
necessary for an efficient implementation of the JVM. Two
architectures will be presented for this mixed access mode of the
stack. Both architectures are used in Java processors. However, we
will also show that the JVM does not need a full three-port access
to the stack as implemented in the two architectures. This allows
for a simple and more elegant design of the stack for a Java
processor. This proposed architecture will then be compared with the
other two at the end of this section.

\subsection{Java Computing Model}

The JVM is not a pure stack machine in the sense of, for instance,
the stack model in Forth. The JVM operates on a LIFO stack as its
\emph{operand stack}. The JVM supplies instructions to load values
on the operand stack, and other instructions take their operands
from the stack, operate on them and push the result back onto the
stack. For example, the \code{iadd} instruction pops two values from
the stack and pushes the result back onto the stack. These
instructions are the stack machine's typical zero-address
instructions. The maximum depth of this operand stack is known at
compile time. In typical Java programs, the maximum depth is very
small. To illustrate the operation notation of the JVM,
Table~\ref{tab_stack_not} shows the evaluation of an expression for
a stack machine notation and the JVM bytecodes. Instruction
\code{iload{\_}n} loads an integer value from a local variable at
position \emph{n} and pushes the value on TOS.

\begin{table}[htbp]
    \centering
    \begin{tabular}{ll}
        \toprule
        \multicolumn{2}{c}{\emph{A = B + C * D}}  \\
        \midrule
        Stack & JVM \\
        \midrule
        push B& iload{\_}1 \\
        push C& iload{\_}2 \\
        push D& iload{\_}3 \\
        {*}   & imul \\
        +     & iadd \\
        pop A & istore{\_}0 \\
        \bottomrule
    \end{tabular}
    \caption{Standard stack notation and the corresponding
    JVM instructions}
    \label{tab_stack_not}
\end{table}

The JVM contains another memory area for method local data. This
area is known as \emph{local variables}. Primitive type values, such
as integer and float, and references to objects are stored in these
local variables. Arrays and objects cannot be allocated in a local
variable, as in C/C++. They have to be placed on the heap. Different
instructions transfer data between the operand stack and the local
variables. Access to the first four elements is optimized with
dedicated single byte instructions, while up to 256 local variables
are accessed with a two-byte instruction and, with the \code{wide}
modifier, the area can contain up to 65536 values.

These local variables are very similar to registers and it appears
that some of these locals can be mapped to the registers of a
general purpose CPU or implemented as registers in a Java processor.
On method invocation, local variables could be saved in a frame on a
stack, different from the operand stack, together with the return
address, in much the same way as in C on a typical processor. This
would result in the following memory hierarchy:
%
\begin{itemize}
\item On-chip hardware stack for ALU operations
\item A small register file for frequently-accessed variables
\item A method stack in main memory containing the return address and additional
local variables
\end{itemize}
%
However, the semantics of method invocation suggest a different
model. The arguments of a method are pushed on the operand stack. In
the invoked method, these arguments are not on the operand stack but
are instead accessed as the first variables in the local variable
area. The \emph{real} method local variables are placed at higher
indices. Listing~\ref{lst:stack:param:pass} gives an example of the
argument passing mechanism in the JVM. These arguments could be
copied to the local variable area of the invoked method. To avoid
this memory transfer, the entire variable area (the arguments
\emph{and} the variables of the method) is allocated on the operand
stack. However, in the invoked method, the arguments are buried deep
in the stack.

\begin{lstlisting}[float,caption={Example of parameter passing and access},label={lst:stack:param:pass}]
The Java source:

    int val = foo(1, 2);
    ...
    public int foo(int a, int b) {
        int c = 1;
        return a+b+c;
    }

Compiled bytecode instructions for the JVM:

The invocation sequence:
    aload_0             // Push the object reference
    iconst_1            // and the parameter onto the
    iconst_2            // operand stack.
    invokevirtual   #2  // Invoke method foo:(II)I.
    istore_1            // Store the result in val.

public int foo(int,int):
    iconst_1            // The constant is stored in a method
    istore_3            // local variable (at position 3).
    iload_1             // Arguments are accessed as locals
    iload_2             // and pushed onto the operand stack.
    iadd                // Operation on the operand stack.
    iload_3             // Push c onto the operand stack.
    iadd
    ireturn             // Return value is on top of stack.
\end{lstlisting}

This asymmetry in the argument handling prohibits passing down
parameters through multiple levels of subroutine calls, as in Forth.
Therefore, an extra stack for return addresses is of no use for the
JVM. This single stack now contains the following items in a frame
per method:
%
\begin{itemize}
\item The local variable area
\item Saved context of the caller
\item The operand stack
\end{itemize}
%
A possible implementation of this layout is shown in
Figure~\ref{fig_stack_invoke}. A method with two arguments,
\code{arg{\_}1} and \code{arg{\_}2} (\code{arg{\_}0} is the
\emph{this} pointer), is invoked in this example. The invoked method
\emph{sees} the arguments as \code{var{\_}1} and \code{var{\_}2}.
\code{var{\_}3} is the only local variable of the method. SP is a
pointer to the top of the stack and VP points to the start of the
variable area.

\begin{figure}
    \centering
    \includegraphics[scale=\picscale]{stack/stack_invocation}
    \caption{Stack change on method invocation}
    \label{fig_stack_invoke}
\end{figure}

\subsection{Access Patterns on the Java Stack}
\label{subsec:access}

The pipelined architecture of a Java processor executes basic
instructions in a single cycle. A stack that contains the operand
stack \emph{and} the local variables results in the following access
patterns:
%
\begin{description}
\item[Stack Operation:] Read of the two top elements, operate on them and
push back the result on the top of the stack. The pipeline stages
for this operation are:\newline
\texttt{
    value1 $\leftarrow $ stack[sp], value2 $\leftarrow $ stack[sp-1]\newline
    result $\leftarrow $ value1 op value2, sp $\leftarrow $ sp-1\newline
    stack[sp] $\leftarrow $ result
}

\item[Variable Load:] Read a data element deeper down in the
    stack, relative to a variable base address pointer (VP), and
    push this data on the top of the stack. This operation needs
    two pipeline stages:\newline \texttt{ value $\leftarrow $
    stack[vp+offset], sp $\leftarrow $ sp+1\newline stack[sp]
    $\leftarrow $ value
}

\item[Variable Store:] Pop the top element of the stack and write it in
the variable relative to the variable base address:\newline
\texttt{
    value $\leftarrow $ stack[sp]\newline
    stack[vp+offset] $\leftarrow $ value, sp $\leftarrow $ sp-1
}
\end{description}
%
For pipelined execution of these operations, a three-port memory or
register file (two read ports and one write port) is necessary.

\subsection{Common Realizations of a Stack Cache}

As the stack is a heavily accessed memory region, the stack -- or
part of it -- has to be placed in the upper level of the memory
hierarchy. This part of the stack is referred to as a \emph{stack
cache}. As described in \cite{Hennessy02}, a typical memory hierarchy
contains the following elements, with increasing access time and
size:
%
\begin{itemize}
\item CPU register
\item On-chip cache memory
\item Off-chip cache memory
\item Main memory
\item Magnetic disk for virtual memory
\end{itemize}
%
For a stack cache, a register file is the solution with the shortest access
time. However, in order to store more than a few elements in the cache, an
on-chip memory realization can provide a larger cache. Both variants have
been used and are described below.

\subsubsection{The Register File as a Stack Cache}

An example of a Java processor that uses a register file is Sun's
picoJava \cite{pjMicroArch}. It contains 64 registers, organized as a
circular buffer. To compensate for this \emph{small} stack cache, an
automatic spill and fill circuit needs another read/write port to the
register file. aJile's JEMCore \cite{880720} is a direct-execution
Java processor core that contains 24 registers. Only six of them are
used to cache the top elements of the stack. With this small register
count, local variables are not part of the cache. Ignite
\cite{IGNITE} (formerly known as PSC1000) is a stack processor,
originally designed as a Forth processor and now promoted as a Java
processor, and has an operand stack that contains 18 registers with
automatic spill and fill.

A basic pipeline for a stack processor with a register file contains the
following stages:
%
\begin{enumerate}
\item IF -- instruction fetch
\item ID -- instruction decode
\item EX -- read register file and execute
\item WB -- write result back to register file
\end{enumerate}
%
With this pipeline structure, a single data-forwarding path between
WB and EX is necessary. The ALU with the register file (with a size
of 16, a common size for RISC processors) and the bypass unit are
shown in Figure~\ref{fig_stack_cache_reg}. In
Table~\ref{tab_resource_reg_cache} the hardware resources of this
type of stack cache are approximated, using the values given in
Table~\ref{tab_simp_gate_count} (a MUX not found in this table is
assumed to use combinations of the basic types; e.g.\ two 8:1 and
one 2:1 for a 16:1). An experimental evaluation of this architecture
in an FPGA is described in Section~\ref{subsec:resource}.

\begin{table}[hbtp]
    \centering
    \begin{tabular}{lc}
        \toprule
        Basic function & Gate count \\
        \midrule
        D-Flip-Flop&5 \\
        2:1 MUX&3 \\
        4:1 MUX&5 \\
        8:1 MUX&9 \\
        SRAM Bit&1.5 \\
        \bottomrule
    \end{tabular}
    \caption{Simplified gate count for basic functions}
    \label{tab_simp_gate_count}
\end{table}

\begin{figure*}
    \centering
    \includegraphics[scale=\picscale]{stack/stack_cache_reg}
    \caption{A stack cache with registers}
    \label{fig_stack_cache_reg}

    \vspace{\floatsep}    % zusaetzlicher Abstand zwischen zwei `floats'

    \begin{tabular}{lld{1}}
        \toprule
        Function block& Basic function& \cc{Gate count} \\
        \midrule
        Register File& 512 D-Flip-Flops& 2,560 \\
        Read MUX& 2x32 16:1 MUX&1,344 \\
        Forward MUX& 32 2:1 MUX&96 \\
        ALU buffer& 32 D-Flip-Flops&160 \\
        \midrule
        \textbf{Total}& &4,160 \\
        \bottomrule
    \end{tabular}
    \captionof{table}{Estimated gate count for a register stack cache}
    \label{tab_resource_reg_cache}
\end{figure*}

\subsubsection{On-chip Memory as a Stack Cache}

Using SRAM on the chip provides a \emph{large} stack cache (e.g.\
128 entries). However, as we have seen in
Section~\ref{subsec:access}, a three-port memory is necessary. An
additional pipeline stage performs the cache memory read:
%
\begin{enumerate}
\item IF -- instruction fetch
\item ID -- instruction decode
\item RD -- memory read
\item EX -- execute
\item WB -- write result back to memory
\end{enumerate}
%
With this pipeline structure, two data forwarding paths are
necessary. The resulting architecture is shown in
Figure~\ref{fig_stack_cache_ram} and a gate count estimate is
provided in Table~\ref{tab_resource_sram_cache}. This version needs
70{\%} more resources than the first one, but provides an eight
times larger stack cache.

Example designs that use this kind of stack cache are (i) Komodo
\cite{Zulauf00}, a Java processor intended as a basis for research
on multithreaded real-time scheduling, and (ii) FemtoJava
\cite{Femto01}, a research project to build an application specific
Java processor.

A three-port memory is an expensive option for an ASIC and unusual
in an FPGA. It can be emulated in an FPGA by two memories with a
single read and write port. The write data is written in both memory
blocks and each memory block provides a different read port.
However, this solution also doubles the amount of memory.

Both designs (Komodo and FemtoJava) avoid the memory doubling by
serializing the two reads. This serialization results in a minimum of
two clock cycles execution time for basic instructions or halves the
clock frequency of the whole pipeline.

\begin{figure*}
    \centering
    \includegraphics[scale=\picscale]{stack/stack_cache_ram}
    \caption{A stack cache with on-chip RAM}
    \label{fig_stack_cache_ram}

    \vspace{\floatsep}    % zusaetzlicher Abstand zwischen zwei `floats'

    \begin{tabular}{lld{1}}
        \toprule
        Function block& Basic function& \cc{Gate count} \\
        \midrule
        Stack RAM& e.g.\ 128x32 Bits& 6,144 \\
        Port buffer& 2x32 D-Flip-Flops& 320 \\
        Forward MUX& 32x 2:1 MUX, 3:1 MUX& 288 \\
        ALU buffer& 2x32 D-Flip-Flops& 320 \\
        \midrule
        \textbf{Total}& & 7,072 \\
        \bottomrule
    \end{tabular}
    \captionof{table}{Estimated gate count for a stack cache with RAM}
    \label{tab_resource_sram_cache}
\end{figure*}

\subsection{A Two-Level Stack Cache}

In this section, we will discuss access patterns of the JVM and
their implication on the functional units of the pipeline. A faster
and smaller architecture is proposed for the stack cache of a Java
processor.

\subsubsection{JVM Stack Access Revised}

If we analyze the JVM's access patterns to the stack in more detail,
we can see that a two-port read is only performed with the two top
elements of the stack. All other operations with elements deeper in
the stack, local variables load and store, only need one read port.
If we only implement the two top elements of the stack in registers,
we can use a standard on-chip RAM with one read and one write port.

We will show that all operations can be performed with this
configuration. Let $A$ be the top-of-stack, $B$ the element below
top-of-stack. The memory that serves as the second level cache is
represented by the array $sm$. Two indices in this array are used:
$p$ points to the logical third element of the stack and changes as
the stack grows or shrinks, $v$ points to the base of the local
variables area in the stack and $n$ is the address offset of a
variable. $op$ is a two operand stack operation with a single result
(i.e.\ a typical ALU operation).


\begin{description}
\begin{samepage}
\item[Case 1:]
ALU operation \newline \textit{A $\leftarrow $ A op B
\newline B $\leftarrow $ sm[p] \newline p $\leftarrow $ p -- 1
\newline }The two operands are provided by the two top level
registers. A single read access from $sm$ is necessary to fill $B$
with a new value.
\end{samepage}
%
\begin{samepage}
\item[Case 2:]
    Variable load (\textit{Push}) \newline
    \textit{
    sm[p+1]$\leftarrow $ B \newline
    B $\leftarrow $ A \newline
    A$\leftarrow $ sm[v+n] \newline
    p $\leftarrow $ p + 1 \newline
    }
    One read access from \textit{sm} is necessary for the variable read. The
former TOS value moves down to $B$ and the data previously in $B$ is
written to \textit{sm}.
\end{samepage}
%
\begin{samepage}
\item[Case 3:]
    Variable store (\textit{Pop}) \newline
    \textit{sm[v+n] $\leftarrow $ A \newline
    A $\leftarrow $ B \newline
    B $\leftarrow $ sm[p] \newline
    p $\leftarrow $ p - 1 \newline }
    The TOS value is written to \textit{sm}. $A$ is filled with $B$ and $B$ is filled in an
identical manner to Case 1, needing a single read access from
\textit{sm}.
\end{samepage}
\end{description}
%
We can see that all three basic operations can be performed with a
stack memory with one read and one write port. Assuming a memory is
used that can handle concurrent read and write access, there is no
structural access conflict between $A$, $B$ and \textit{sm}. That
means that all operations can be performed concurrently in a single
cycle.

As we can see in Figure~\ref{fig_stack_invoke} the operand stack and
the local variables area are distinct regions of the stack. A
concurrent read from and write to the stack is only performed on a
variable load or store. When the read is from the local variables
area the write goes to the operand area; a read from the operand
area is concurrent with a write to the local variables area.
Therefore there is no concurrent read and write to the same location
in \textit{sm}. There is no constraint on the read-during-write
behavior of the memory (old data, undefined or new data), which
simplifies the memory design. In a design where read and write-back
are located in different pipeline stages, as in the architectures
described above, either the memory must provide the new data on a
read-during-write, or external forward logic is necessary.

From the three cases described, we can derive the memory addresses
for the read and write port of the memory, as shown in
Table~\ref{tab_stack_address}.

\begin{table}[htbp]
    \centering
    \begin{tabular}{cc}
        \toprule
        Read address&Write address \\
        \midrule p&p+1 \\
        v+n&v+n \\
        \bottomrule
    \end{tabular}
    \caption{Stack memory addresses}
    \label{tab_stack_address}
\end{table}

\subsubsection{The Datapath}

The architecture of the two-level stack cache can be seen in
Figure~\ref{fig_stack_cache_jop}. Register $A$ represents the
top-of-stack and register $B$ the data below the top-of-stack. ALU
operations are performed with these two registers and the result is
placed in $A$. During such an ALU operation, $B$ is filled with new
data from the stack RAM. A new value from the local variable area is
loaded directly from the stack RAM into $A$. The data previously in
$A$ is moved to $B$ and the data from $B$ is spilled to the stack
RAM. $A$ is stored in the stack RAM on a store instruction to the
local variable. The data from $B$ is moved to $A$ and $B$ is filled
with a new value from the stack RAM.
%All these operations are performed concurrently in one cycle.

With this architecture, the pipeline can be reduced to three stages:
%
\begin{enumerate}
\item IF -- instruction fetch
\item ID -- instruction decode
\item EX -- execute, load or store
\end{enumerate}
%
The estimated resource usage of this two-level stack cache
architecture is given in Table~\ref{tab_resource_jop_cache}. It can
be seen that this architecture is roughly as complex as the solution
given above (about 5{\%} less gates). However, the reduced
complexity with the two-port RAM instead of a three-port RAM is not
included in the table. The critical path through the ALU contains
only one 2:1 MUX to register $A$ in this solution, rather than one
3:1 MUX in one ALU path and one 2:1 MUX in the other ALU path. As no
data forwarding logic is necessary, the decoding logic is also
simpler.

\begin{figure*}
    \centering
    \includegraphics[scale=\picscale]{stack/stack_cache_jop}
    \caption{Two-level stack cache}
    \label{fig_stack_cache_jop}

    \vspace{\floatsep}    % zusaetzlicher Abstand zwischen zwei `floats'

    \begin{tabular}{lld{1}}
        \toprule
        Function block& Basic function& \cc{Gate count} \\
        \midrule
        Stack RAM& e.\ g.\ 128x32 Bits&6,144 \\
        TOS, TOS-1 buffer& 2x32 D-Flip-Flops&320 \\
        Three MUX& 3x32 2:1 MUX&288 \\
        \midrule
        \textbf{Total}& &6,752 \\
        \bottomrule
    \end{tabular}
    \captionof{table}{Estimated gate count for a two-level stack cache}
    \label{tab_resource_jop_cache}
\end{figure*}

\subsubsection{Data Forwarding -- A Non-Issue}

Data dependencies in the instruction stream result in the so-called
\emph{data hazards} \cite{Hennessy02} in the pipeline. Data
forwarding is a technique that moves data from a later pipeline
stage back to an earlier one to solve this problem. The term
\emph{forward} is correct in the temporal domain as data is
transferred to an instruction in the future. However, it is
misleading in the structural domain as the forward direction is
towards the \emph{last} pipeline stage for an instruction.

As the probability of data dependency is very high in a stack-based
architecture, one would expect several data forwarding paths to be
necessary. However, in the two-level architecture proposed, with its
resulting three-stage pipeline, no data hazards will occur and no
data forwarding is therefore necessary. This simplifies the decoding
stage and reduces the number of multiplexers in the execution path.
We will show that none of the three data hazard types
\cite{Hennessy02} is an issue in this architecture. With instructions
$i$ and $j$, where $i$ is issued before $j$, the data hazard types
are:

\paragraph{Read after write:} $j$ reads a source before $i$ writes it. This
is the most common type of hazard and, in the architectures
described above, is solved by using the ALU buffers and the
forwarding multiplexer in the ALU datapath. On a stack architecture,
write takes three forms:
%
\begin{itemize}
    \item Implicit write of TOS during an ALU operation
    \item Write to the TOS during a load instruction
    \item Write to an arbitrary entry of the stack with a store instruction
\end{itemize}
%
A read also occurs in three different forms:
\begin{itemize}
    \item Read two top values from the stack for an ALU operation
    \item Read TOS for a store instruction
    \item Read an arbitrary entry of the stack with the load instruction
\end{itemize}
%
With the two top elements of the stack as discrete registers, these
values are read, operated on and written back in the same cycle. No
read that depends on TOS or TOS-1 suffers from a data hazard. Read
and write access to a local variable is also performed in the same
pipeline stage. Thus, the read after write order is not affected.
However, there is also an additional hidden read and write: the fill
and spill of register B:



%
\begin{itemize}
\item \textit{B fill:}
$B$ is written during an ALU operation and on a variable store.
During an ALU operation, the operands are the values from $A$ and
the old value from $B$. The new value for $B$ is read from the stack
memory and does not depend on the new value of $A$. During a
variable store operation, $A$ is written to the stack memory and
does not depend on $B$. The new value for $B$ is also read from the
stack memory and it is not obvious that this value does not depend
on the written value. However, the variable area and the operand
stack are distinct areas in the stack (this changes only on method
invocation and return), guaranteeing that concurrent read/write
access does not produce a data hazard.

\item \textit{B spill:}
$B$ is read on a load operation. The new value of $B$ is the old
value of $A$ and does not therefore depend on the stack memory read.
$B$ is written to the stack. For the read value from the stack
memory that goes to $A$, the argument concerning the distinct stack
areas in the case of \textit{B fill} described above still applies.
\end{itemize}
%
\paragraph{Write after read:} $j$ writes a destination before it is read by
$i$. This cannot take place as all reads and writes are performed in
the same pipeline stage keeping the instruction order.

\paragraph{Write after write:} $j$ writes an operand before it is written by
$i$. This hazard is not present in this architecture as all writes
are performed in the same pipeline stage.



\subsection{Resource Usage Compared}
\label{subsec:resource}

The three architectures described above are implemented in Altera's
EP1C6Q240C6 \cite{AltCyc} FPGA. The three-port memory for the second
solution is emulated with two embedded memory blocks. The ALU for
this comparison is kept simple with the following functions: NOP,
ADD, SUB, POP, AND, OR, XOR and load external data. The load of
external data is necessary in order to prevent the synthesizer from
optimizing away the whole design. A real implementation of an ALU
for a Java processor, as described in Section~\ref{sec:pipeline}, is
a little bit more complex with a barrel shifter and additional load
paths. In order to gain the maximum operating frequency for the
design, the testbed for this architecture contains registers for the
external data, the RAM address buses, and the control and select
signals. Table~\ref{tab_stack_resources} shows the resource usage
and maximum operation frequency of the three different
architectures.

\begin{table*}
    \centering
    \begin{tabular}{lccccccc}
        \toprule
        Design& \multicolumn{2}{c}{Total}&
        \multicolumn{2}{c}{Cache}&
        Memory& fmax & Size\\
         & LCs& Reg.& LCs&
        Reg.& (bit)& (MHz) & (word)\\
        \midrule
%        ALU only& 194& 0& -& -& -&- &-\\
        Testbed w.\ ALU& 261& 166& -& -& -&237 & - \\
        16 register cache& 968& 657& 707& 491& 0&110 & 16 \\
%        16 Register cache& 297& 140& 36& -26& 1024&113 & 16 \\
        SRAM cache& 372& 185& 111& 19& 8,192&153 & 128\\
        Two-level cache& 373& 184& 112& 18& 4,096& 213 & 130\\
        \bottomrule
    \end{tabular}
    \caption{Resource and performance compared}
    \label{tab_stack_resources}
\end{table*}


LC stands for `Logic Cell' and is the basic element in an FPGA: a
4-bit lookup table with a register. The LC count in the table
includes the register count. The ALU alone without any stack cache
needs 194 LCs. In the first line, the testbed is combined with the
ALU without any stack caching, as a reference design. With this
configuration, we can obtain the maximum possible speed of the
registered ALU in this FPGA technology, in this case an operating
frequency of 237~MHz or a 4.2~ns delay. This value is an upper bound
of the system frequency. Every pipelined architecture needs one or
more multiplexer in the ALU path, either for data forwarding or for
operand selection, resulting in a longer delay. The fourth and fifth
columns represent the resource usage of the cache logic without the
testbed and ALU. The last column shows the effective cache size in
data words.

The version with the 16 registers was synthesized with two different
synthesizer settings. In the first setting, the register file is
implemented with discrete registers while, with a different setting,
the register file is automatically implemented in two 32-bits
embedded RAM blocks. Two different RAM blocks are necessary to
provide two read ports and one write port. In both versions, the
delay time to read the register file (delay through the 16:1 MUX of
4.9~ns or RAM access time of 4.6~ns) is in the same order as the
delay time through the ALU, resulting in a system frequency of half
the theoretical frequency of that with the ALU alone. As the
structure of the version with the embedded RAM block is very similar
with the SRAM cache, only the version with the discrete registers is
shown in Table~\ref{tab_stack_resources}.

The stack cache with a RAM and registers on the RAM output (the
additional pipeline stage) performs better than the first solution.
However, the 3:1 MUX in the critical path still adds 2.3~ns to the
delay time. Compared with the proposed solution (in the last line),
we see that double the amount of RAM is needed for the two read
ports.

The two-level stack cache solution performs at 213~MHz, i.e.\ almost
the theoretical system frequency (in practice, about 10{\%} slower).
Only a 2:1 MUX is added to the critical path. The single read port
memory needs half the number of memory bits of the other two
solutions.

\subsection{Summary}

In this section, the stack architecture of the JVM was analyzed. We
have seen that the JVM is different from the classical stack
architecture. The JVM uses the stack both as an operand stack
\textit{and} as the storage place for local variables. Local
variables are placed in the stack at a \textit{deeper} position. To
load and store these variables, an access path to an arbitrary
position in the stack is necessary. As the stack is the most
frequently accessed memory area in the JVM, caching of this memory
is mandatory for a high-performing Java processor.

A common solution, found in a number of different Java processors,
is to implement this stack cache as a standard three-port register
file with additional support to address this register file in a
stack like manner. The architectures presented above differ in the
realization of the register file: as a discrete register or in
on-chip memory. Implementing the stack cache as discrete registers
is very expensive. A three-port memory is also an expensive option
for an ASIC and unusual in an FPGA. It can be emulated by two
memories with a single read and write port. However, this solution
also doubles the amount of memory.

Detailed analysis of the access patterns to the stack showed that
only the two top elements of the stack are accessed in a single
cycle. Given this fact, the proposed architecture uses registers to
cache only the two top elements of the stack. The next level of the
stack cache is provided by a simple on-chip memory. The memory
automatically spills and fills the second register. Implementing the
two top elements of the stack as fixed registers, instead of
elements that are indexed by a stack pointer, also greatly
simplifies the overall pipeline.

The proposed stack architecture has the following advantages: (i)
Simpler cache memory results in having half the memory usage of the
other solutions in an FPGA. (ii) Minimal impact on the raw speed of
the ALU. Operates at almost the theoretical maximum system frequency
of the ALU. (iii) Single read, execute and write-back pipeline stage
results in an overall 3-stage pipeline processor design. (iv) No
data forwarding is necessary, which simplifies instruction decode
logic and reduces the multiplexer count in the critical path.


\clearpage
    
\section{HW/SW Codesign}
\label{sec:hwsw:co}

Using a hardware description language and loading the design in an
FPGA the former strict border between hardware and software gets
blurred. Is configuring an FPGA not more like loading a program for
execution?

This looser distinction makes it possible to move functions easily
between hardware and software resulting in a highly configurable
design. If speed is an issue, more functions are realized in
hardware. If cost is the primary concern these functions are moved
to software and a smaller FPGA can be used. Let us examine these
possibilities on a relatively expensive function:
\emph{multiplication}.

Bytecode \code{imul} performs a 32 bit signed multiplication with a
32 bit result. There are no exceptions on overflow. Since 32 bit
single cycle multiplications are far beyond the possibilities of
current, mainstream FPGAs the first solution is a sequential
multiplier.

\paragraph{Sequential Booth Multiplier in VHDL}

\begin{lstlisting}[float, caption={Booth multiplier in VHDL},
language=VHDL, label=lst:arch:hwsw:vhdl]
    process(clk, wr_a, wr_b)

        variable count  : integer range 0 to width;
        variable pa     : signed(64) downto 0);
        variable a_1    : std_logic;
        alias p         : signed(32 downto 0)
                          is pa(64 downto 32);

    begin
        if rising_edge(clk) then
            if wr_a='1' then
                p := (others => '0');
                pa(width-1 downto 0) := signed(din);

            elsif wr_b='1' then
                b <= din;
                a_1 := '0';
                count := width;
            else
                if count > 0 then
                    case std_ulogic_vector'(pa(0), a_1) is
                        when "01" =>
                            p := p + signed(b);
                        when "10" =>
                            p := p - signed(b);
                        when others =>
                            null;
                    end case;
                    a_1 := pa(0);
                    pa := shift_right(pa, 1);
                    count := count - 1;
                end if;
            end if;
        end if;
        dout <= std_logic_vector(pa(31 downto 0));
    end process;
\end{lstlisting}
%
Listing~\ref{lst:arch:hwsw:vhdl} shows the VHDL code of the
multiplier. Two microcode instructions are used to access this
function: \code{stmul} stores the two operands (from TOS and TOS-1)
and starts the sequential multiplier. After 33 cycles, the result is
loaded with \code{ldmul}. Listing~\ref{lst:arch:hwsw:micro} shows
the microcode for \code{imul}.

\begin{lstlisting}[float, caption={Microcode to access the Booth multiplier},
label=lst:arch:hwsw:micro]
    imul:
            stmul       // store both operands and start
            pop         // pop second operand

            ldi 5       // 6*5+3 cycles wait
imul_loop:              // wait loop
            dup
            nop
            bnz imul_loop
            ldi -1      // decrement in branch slot
            add

            pop         // remove counter

            ldmul   nxt // load result
\end{lstlisting}

\paragraph{Multiplication in Microcode}

If we run out of resources in the FPGA, we can move the function to
microcode. The implementation of \code{imul} is almost identical to
the Java code in Listing~\ref{lst:arch:hwsw:java} and needs 73
microcode instructions.

\paragraph{Bytecode imul in Java}

Microcode is stored in an embedded memory block of the FPGA. This is
also a resource of the FPGA. We can move the code to external memory
by implementing \code{imul} in Java bytecode. Bytecodes not
implemented in microcode result in a static Java method call from a
special class (\code{com.jopdesign.sys.JVM}). This class has
prototypes for each bytecode ordered by the bytecode value. This
allows us to find the right method by indexing the method table with
the value of the bytecode. Listing~\ref{lst:arch:hwsw:java} shows
the Java method for \code{imul}. The additional overhead for this
implementation is a call and return with cache refills.


\begin{lstlisting}[float, caption={Implementation of bytecode \code{imul} in Java},
label=lst:arch:hwsw:java]
    public static int imul(int a, int b) {

        int c, i;
        boolean neg = false;
        if (a<0) {
            neg = true;
            a = -a;
        }
        if (b<0) {
            neg = !neg;
            b = -b;
        }
        c = 0;
        for (i=0; i<32; ++i) {
            c <<= 1;
            if ((a & 0x80000000)!=0) c += b;
            a <<= 1;
        }
        if (neg) c = -c;
        return c;
    }
\end{lstlisting}

\paragraph{Implementations Compared}

\tablename~\ref{tab_arch_hwsw_compared} lists the resource usage and
execution time for the three implementations. Execution time is
measured with both operands negative, the worst-case execution time
for the software implementations. The implementation in Java is
slower than the microcode implementation as the Java method is
loaded from main memory into the bytecode cache.

\begin{table}
    \centering
    \begin{tabular}{ld{2}d{3}d{0}}
    \toprule
    & \cc{Hardware} & \cc{Microcode} & \cc{Time} \\
    & \cc{[LC]} & \cc{[Byte]} & \cc{[Cycle]} \\
    \midrule
    VHDL & 156 & 10 & 35 \\
    Microcode & 0 & 73 & 750 \\
    Java & 0 & 0 & ~2,300 \\
    \bottomrule
    \end{tabular}
    \caption{Different implementations of \code{imul} compared}
    \label{tab_arch_hwsw_compared}
\end{table}

Only a few lines of code have to be changed to select one of the
three implementations. This principle can also be applied to other
expensive bytecodes: e.g.\ \code{idiv}, \code{ishr}, \code{iushr} and
\code{ishl}. As a result, the resource usage of JOP is highly
configurable and can be selected for each application according to
the needs of the application. Treating VHDL as a software language
allows easy movement of function blocks between hardware and
software.

\clearpage
    
\section{Real-Time Predictability}
\label{sec:rtpredict}

%%% part of that intro is found in the cache section

%\subsection{Time Predictable Architecture}
%
%Worst case execution time (WCET) analysis of real-time programs is
%essential for any schedulability analysis. To provide a tight WCET
%value a good model of the processor is necessary. However, the
%architectural advancement in modern processor designs is dominated
%by the rule: '\emph{Make the common case fast}`. This is the
%opposite to: '\emph{Reduce the worst case}` and complicates WCET
%analysis. JOP was designed to provide an architecture that can be
%exactly modeled. Execution time of bytecodes is known cycle
%accurate. It is possible to analyze the timing on the bytecode level
%without the uncertainties of an interpreting JVM or generated native
%code from ahead-of-time compilers for Java.
%
%
%Common architectural components, such as branch prediction and
%branch target buffers enhance average performance, but have usually
%a very pessimistic WCET. In JOP, branch prediction is avoided. This
%results in pressure on the pipeline length. The processor has a
%minimal pipeline length of four stages resulting in a four cycle
%execution time for a bytecode branch.
%
%The stack is a frequently accessed memory area and therefore
%implemented in on-chip memory and serving as data cache. Data
%exchange between internal stack and main memory is under microcode
%control and therefore WCET analyzable.
%
%Instruction cache is mandatory to bridge the growing gap between CPU
%speed and main memory access time. Standard cache organizations
%improve average execution time, but are hard to predict for worst
%case execution time (WCET) analysis. We tackled this problem from
%the architectural side: An instruction cache organization where
%simpler and more accurate WCET analysis is more important than
%average case performance.
%
%JOP has a novel instruction cache: A \emph{method cache}. A complete
%method is loaded in the cache on invocation and on return. This
%cache fill strategy lumps all cache misses together and is very
%simple to analyze. Cache block replacement depends on the call tree
%instead of instruction addresses and is therefore WCET analyzable.
%
%
%Comparing this cache organization quantitative with a benchmark
%derived from a real-time application we have seen that the proposed
%\emph{method cache} performs similar or even better to a traditional
%direct mapped cache with respect to the bytes that have to be filled
%an a cache miss. The number of memory transactions, which result in a
%high miss penalty on memories with high latency, are lower with the
%proposed cache solution than in a traditional cache.
%
%




General-purpose processors are optimized for average throughput, and
non real-time operating systems are responsible for fair and
efficient scheduling of resources. Real-time systems need a processor
with low and known WCET of instructions. Real-time operating systems
have properties, such as fast interrupt response, rapid context
switch, short blocking times and a scheduler that implements a
simple, in most cases strictly priority driven, scheduling algorithm.
This section describes design decisions for JOP to support such
real-time systems.

\subsection{Interrupts}

Interrupts are usually associated with low-level programming of
device drivers. The priorities of interrupts and their handler
functions are above task priorities and yield to an immediate
context switch. In this form, interrupts cannot be integrated in a
schedule with \emph{normal} tasks. The execution time of the
interrupt handler has to be integrated in the schedulability
analysis as additional blocking time. A better solution is to handle
interrupts, which represent external events, as schedulable objects
with priority levels in the range of real-time tasks, as suggested
in the RTSJ.

\paragraph{The Timer Interrupt}

The timer or clock interrupt has a different semantic than other
interrupts. The main purpose of the timer interrupt is
representation of time and release of periodic or time triggered
tasks. One common implementation is a clock tick. The interrupt
occurs at a regular interval (e.g.\ 10 ms) and a decision has to be
taken whether a task has to be released. This approach is simple to
implement, but there are two major drawbacks: The resolution of
timed events is bound by the resolution of the clock tick and clock
ticks without a task switch are a waste of execution time.

A better approach, used in JOP, is to generate timer interrupts at
the release times of the tasks. The scheduler is now responsible for
reprogramming the timer after each occurrence of a timer interrupt.
The list of sleeping threads has to be searched to find the nearest
release time in the future of a higher priority thread than the one
that will be released now. This time is used for the next timer
interrupt.

\paragraph{External Events}

Hardware interrupts, other than the timer interrupt, are represented
as asynchronous events with an associated thread. This means that
the event is a \emph{normal} schedulable object under the control of
the scheduler. With a minimum interarrival time, enforced by
hardware, these events can be incorporated into the priority
assignment and schedulability analysis in the same way as periodic
tasks.

\paragraph{Software Interrupts}

The common software generated interrupts, such as illegal memory
access or divide by zero, are represented by Java runtime exceptions
and need no special handler. They can be detected with a try-catch
block.

Asynchronous notification from the program is supported, in the same
way as an external event, as a schedulable object with an associated
thread. The event is triggered through the call of \code{fire()}.
The thread with the handler is placed in the runnable state and
scheduled according to priority.

\paragraph{Hardware Failures}

Serious hardware failures, such as illegal opcode or parity error
from the memory systems, lead to a shutdown of the system. However,
a \emph{last try} to call a handler that changes the state of the
system to a safe state and inform an upper level system, can improve
the integrity of the overall system.

\subsection{Task Switch}

An important issue in real-time systems is the time for a task
switch. A task switch consists of two actions:
\begin{itemize}
    \item \emph{Scheduling} is the selection of the task order and timing
    \item \emph{Dispatching} is the term for the context switch between tasks
\end{itemize}

\paragraph{Scheduling}

Most real-time systems use a fixed-priority preemptive scheduler.
Tasks with the same priority are usually scheduled in a FIFO order.
Two common ways to assign priorities are rate monotonic or, in a
more general form, deadline monotonic assignment. When two tasks get
the same priority, we can choose one of them and assign a higher
priority to that task and the task set is still schedulable. We get
a strictly monotonic priority order and do not have to deal with
FIFO order. This eliminates queues for each priority level and
results in a single, priority ordered task list.

Strictly fixed priority schedulers suffer from a problem called
\emph{priority inversion} \cite{626613}. The problem where a low
priority task blocks a high priority task on a shared resource is
solved by raising the priority of the low priority task. Two
standard priority inversion avoidance protocols are common:
%
\begin{description}
    \item[Priority Inheritance Protocol:] A lock assigns the priority
of the highest-priority waiting task to the task holding the lock
until that task releases the resource.

    \item[Priority Ceiling Emulation Protocol:] A lock gets a priority
assigned above the priority of the highest-priority task that will
ever acquire the lock. Every task will be immediately assigned the
priority of that lock when acquiring it.
\end{description}
%
The priority inheritance protocol is more complex to implement and
the time when the priority of a task is raised is not so obvious. It
is not raised because the task does anything, but because another
task reaches some point in its execution path.

Using priority ceiling emulation with unique priorities, different
from task priorities, the priority order is still strictly
monotonic. The priority ordered task list is expanded with slots for
each lock. If a task acquires a lock, it is placed in the
corresponding slot. With this extension to the task list, scheduling
is still simple and can be efficiently implemented.

\paragraph{Dispatching}


The time for a context switch depends on the size of the state of
the tasks. For a stack machine it is not so obvious what belongs to
the state of a task. If the stack resides in main memory, only a few
registers (e.g. program counter and stack pointer) need to be saved
and restored. However, the stack is a frequently accessed memory
region of the JVM. The stack can be seen as a data cache and should
be placed near the execution unit (in this case, \emph{near} means
on the chip and not in external memory). However, on-chip memory is
usually too small to hold the stack content for all tasks. This
means that the stack is part of the execution context and has to be
saved and restored on a context switch.

In JOP, the stack is placed in local (on-chip) FPGA memory with
single cycle access time. With this configuration, the next question
is how much of the stack to place there. Either the complete stack
of a thread or only the stack frame of the current method can reside
locally. If the complete stack of a thread is stored in local
memory, the invocation of methods and returns are fast, but the
context is large. For fast context switches, it is preferable to
have only a short stack in local memory. This results in less data
being transferred to and from main memory, but more memory transfers
on method invocation and return. The local stack can be further
divided into small pieces, each holding only one stack frame of one
thread. During the context switch, only the stack pointer needs to
be saved and restored. The outcome of this is a very fast context
switch, although the size of the local memory limits the maximum
number of threads.

Since JOP is a soft-core processor, these different solutions can be
configured for different application requirements. It is even
possible to mix of these policies: some stack slots can be assigned
to \emph{important} threads, while the remaining threads share one
slot. This stack slot only needs to be exchanged with the main
memory when switching \emph{to} a less \emph{important} thread.

\subsection{Architectural Design Decisions}

In hard real-time systems, meeting temporal requirements is of the
same importance as functional correctness. This results in different
architectural constraints than in a design for a non real-time
system. A low upper bound of the execution time is of premium
importance. Good average execution time is useless for a pure hard
real-time system.

Common architectural components, found in general purpose processors
to enhance average performance, are usually problematic for the WCET
analysis. A pragmatic approach to this problem is to ignore these
features for the analysis. With a processor designed for real-time
applications, these features have to be substituted by predictable
architecture enhancements.

\paragraph{Branch Prediction}

As the pipelines of current general-purpose processors get longer to
support higher clock rates, the penalty of branches gets too high.
This is compensated by branch prediction logic with branch target
buffers. However, the upper bound of the branch execution time is the
same as without this feature. In JOP, branch prediction is avoided.
This results in pressure on the pipeline length. The core processor
has a pipeline length of as little as three stages resulting in a
branch execution time of three cycles in microcode. The two slots in
the branch delay can be filled with instructions or \emph{nop}. With
the additional bytecode fetch and translation stage, the overall
pipeline is four stages and results in a four cycle execution time
for a bytecode branch.

\paragraph{Caches and Instruction Prefetch}

To reduce the growing gap between the clock frequency of the
processor and memory access times multi-level cache architectures
are commonly used. Since even a single level cache is problematic
for WCET analysis, more levels in the memory architecture are almost
not analyzable. The additional levels also increase the latency of
memory access on a cache miss.

In a stack machine, the stack is a frequently accessed memory area.
This makes the stack an ideal candidate to be placed near the
execution unit in the memory hierarchy. In JOP the stack is
implemented as internal memory with the two top elements as explicit
registers. This single cycle memory can be seen as a data cache.
However, unlike in picoJava, this limited memory is not automatically
spilled and filled. Automatic spill and fill introduces unpredictable
access to the main memory. Data exchange between internal stack and
main memory is under program control and can be done on method
invocation/return or on a thread switch.

The next most accessed memory area is the code area. A simple
prefetch queue, as it is found in older processors, could increase
instruction throughput after executing a multi-cycle bytecode. For a
stream of single cycle bytecodes, prefetching is useless and the
frequent occurrence of branches and method invocations, about
12--23\% (see Section~\ref{sec:bench:jvm}) in typical Java programs,
reduces the performance gain. The prefetch queue also results in
(probably unbounded) execution time dependencies over a stream of
instructions, which complicates timing analysis.

JOP has a method cache with a novel replace policy. Since typical
methods in Java programs are short and there are only relative
branches in a method, a complete method is loaded in the cache on
invocation and on return. This cache fill strategy lumps all cache
misses together and is very simple to analyze. It also simplifies
the hardware of the cache since no tag memory or address translation
is necessary. The \emph{romizer} tool JavaCodeCompact checks the
maximum allowed method size. Section~\ref{sec:cache} describes the
proposed cache solution in detail. Memory areas for the heap and
class description with the constant pool are not cached in JOP.

\paragraph{Superscalar Processors}

A superscalar processor consists of several execution units and
tries to extract instruction level parallelism (ILP) with out of
order execution. Again, this is a nightmare for timing analysis. The
code for a stack machine has less implicit parallelism than a
register machine.

One form of enhancement, usually implemented in stack machines, is
instruction folding. The instruction stream is scanned to find
frequent patterns like load-load-add-store and substitutes these
four instructions with one, RISC-like, operation. There are two
issues with instruction folding in JOP: The combined instruction
needs two read and one write access to the stack in a single cycle.
This would result in doubling of the internal memory usage in the
FPGA. It also needs, at minimum, four bytes read access to the
method cache. To overcome word boundaries, prefetching has to be
introduced after the method cache. This results in an additional
pipeline stage, time dependency of instructions with a more complex
analysis and more hardware resources for the multiplexers.

Programs for embedded and real-time systems are usually
multi-threaded. In future work, it will be investigated if the
additional hardware resources needed for ILP can be better used with
additional processor cores utilizing this implicit thread-level
parallelism.

\paragraph{Time-Predictable Instructions}

A good model of a processor with accurate timing information is
essential for a tight WCET analysis. The architecture of JOP and the
microcode are designed with this in mind. Execution time of bytecodes
is known cycle accurately (see Chapter~\ref{chap:wcet} and
Appendix~\ref{appx:bytecode}). It is possible to analyze the WCET on
the bytecode level \cite{R:Bernat:2000a} without the uncertainties of
an interpreting JVM \cite{R:Bate:2000a} or generated native code from
ahead-of-time compilers for Java.

\subsection{Summary}

In this section, we argued that, while common techniques in
processor architectures increase average throughput, they are not
feasible for real-time systems. The influence of these architectural
enhancements is at best hardly WCET-analyzable.

The proposed alternatives influence the processor architecture, as
described in earlier sections, as well as the software architecture
that will be described in Section~\ref{sec:rtprof}.

However, the most important architectural enhancement for pipelined
machines is caching, which is necessary even in embedded systems. We
have shown in Section~\ref{sec:stack} how a time-predictable data
cache for a stack machine can be implemented. In the following
section, we will propose a time-predictable cache for instructions.


\clearpage
    \section{A Time-Predictable Instruction Cache}
    \label{sec:cache}
    \label{sec:cache}


Worst-case execution time (WCET) analysis \cite{pusch:maxt:jnl} of
real-time programs is essential for any schedulability analysis. To
provide a low WCET value, a good processor model is necessary.
However, caches for the instructions and data is a classic example of
the paradigm \emph{Make the common case fast}, which complicates WCET
analysis. Avoiding or ignoring this feature in real-time systems, due
to its unpredictable behavior, results in a very pessimistic WCET
value. Plenty of effort has gone into research into integrating the
instruction cache in the timing analysis of tasks \cite{Arnold1994,
Healy1995, 225068} and the influence of the cache on task preemption
\cite{279589, Mataix:1996}. The influence of different cache
architectures on WCET analysis is described in
\cite{Heckmann:IEEE2003}.

We will tackle this problem from the architectural side -- an
instruction cache organization in which simpler and more accurate
WCET analysis is more important than average case performance.

In this section, we will explore the method cache, as it is
implemented in JOP, with a novel replacement policy. In Java bytecode
only relative branches exist, and a method is therefore only left
when a return instruction has been executed.\footnote{An uncaught
exception also results in a method exit.} It has been observed that
methods are typically short (see \cite{jop:thesis}) in Java
applications. These properties are utilized by a cache architecture
that stores complete methods. A complete method is loaded into the
cache on both invocation and return. This cache fill strategy lumps
all cache misses together and is very simple to analyze.

The method cache was first presented in \cite{jop:jtres_cache} and is
now also used by the Java processor SHAP \cite{shap:mcache}.
Furthermore, the idea has been adapted for a processor that executes
compiled C programs \cite{Metzlaff:SPM:2008}.

\subsection{Method Cache}

In this section, we will develop a solution for a time-predictable
instruction cache. Typical Java programs consist of short methods.
There are no branches out of the method and all branches inside are
relative. In the proposed architecture, the full code of a method is
loaded into the cache before execution. The cache is filled on
invocations and returns. This means that all cache fills are lumped
together with a known execution time. The full loaded method and
relative addressing inside a method also result in a simpler cache.
Tag memory and address translation are not necessary.

In the method cache several cache blocks (similar to cache lines) are
used for a method. The main difference from a conventional cache is
that the blocks for a method are all loaded at once and need to be
consecutive.

Choosing the block size is now a major design decision. Smaller
block sizes allow better memory usage, but the search time for a hit
also increases.

With varying block numbers per method, an LRU replacement becomes
impractical. When the method found to be LRU is smaller than the
loaded method, this new method invalidates two cached methods.

For the replacement, we will use a pointer $next$ that indicates the
start of the blocks to be replaced on a cache miss. Two practical
replace policies are:
%
\begin{description}
\item [Next block:]At the very first beginning, $next$ points to the
first block. When a method of length $l$ is loaded into the block
$n$, $next$ is updated to $(n+l)\;mod\;block\;count$.
\item [Stack oriented:]$next$ is updated in the same way as before
on a method load. It is also updated on a method return --
independent of a resulting hit or miss -- to point to the first
block of the leaving method.
\end{description}
%
We will show the operation of these different replacement policies
in an example with three methods: a(), b() and c() of block sizes 2,
2 and 1. The cache consists of 4 blocks and is therefore too small
to hold all the methods during the execution of the code fragment
shown in Listing~\ref{lst:cache:replace}.
%
\begin{samepage}
\begin{lstlisting}[float,caption={Code fragment for the replacement example},
label=lst:cache:replace]
    a() {
        for (;;) {
            b();
            c();
        }
        ...
    }
\end{lstlisting}
\end{samepage}
%
Tables \ref{tab_cache_replace_next} and
\ref{tab_cache_replace_stack} show the cache content during program
execution for both replacement policies. The content of the cache
blocks is shown after the execution of the invoke or return
instruction. An uppercase letter indicates that this block has been
newly loaded. A right arrow depicts the block to be replaced on a
cache miss (the \emph{next} pointer). The last row shows the number
of blocks that are filled during the execution of the program.

\begin{table}
    \centering
\begin{tt}
    \begin{tabular}{lrrrrrrrrrrr}
    \toprule

                &a()    &b()    &ret    &c()    &ret    &b()    &ret    &c()    &ret    &b()    &ret    \\
    \midrule
    \rm{Block 1}&A      &$\to$a &$\to$a &C      &c      &B      &b      &b      &$\to$- &B      &b      \\
    \rm{Block 2}&A      &a      &a      &$\to$- &A      &$\to$a &$\to$a &C      &c      &B      &b      \\
    \rm{Block 3}&$\to$- &B      &b      &b      &A      &a      &a      &$\to$- &A      &$\to$a &$\to$a \\
    \rm{Block 4}&-      &B      &b      &b      &$\to$- &B      &b      &b      &A      &a      &a      \\
    \midrule
    \rm{Fill}      &2   &4      &       &5      &7      &9      &       &11     &13     &15     &       \\
    \bottomrule

    \end{tabular}
\end{tt}
    \caption{Next block replacement policy}
    \label{tab_cache_replace_next}
\end{table}

\begin{table}
    \centering
\begin{tt}
    \begin{tabular}{lrrrrrrrrrrr}
    \toprule

                &a()    &b()    &ret    &c()    &ret    &b()    &ret    &c()    &ret    &b()    &ret    \\
    \midrule
    \rm{Block 1}&A      &$\to$a &a      &a      &a      &$\to$a &a      &a      &a      &$\to$a &a      \\
    \rm{Block 2}&A      &a      &a      &a      &a      &a      &a      &a      &a      &a      &a      \\
    \rm{Block 3}&$\to$- &B      &$\to$b &C      &$\to$c &B      &$\to$b &C      &$\to$c &B      &$\to$b \\
    \rm{Block 4}&-      &B      &b      &$\to$- &-      &B      &b      &$\to$- &-      &B      &b      \\
    \midrule
    \rm{Fill}      &2   &4      &       &5      &       &7      &       &8      &       &10     &       \\
    \bottomrule

    \end{tabular}
\end{tt}
    \caption{Stack oriented replacement policy}
    \label{tab_cache_replace_stack}
\end{table}

In this example, the stack oriented approach needs fewer fills, as
only methods b() and c() are exchanged and method a() stays in the
cache. However, if, for example, method b() is the size of one
block, all methods can be held in the cache using the the \emph{next
block} policy, but b() and c() would be still exchanged using the
\emph{stack} policy. Therefore, the first approach is used in the
proposed cache.


\subsection{WCET Analysis}

\label{sec:cache:wcet}

The proposed instruction cache is designed to simplify WCET
analysis. Due to the fact that all cache misses are only included in
two instructions (\emph{invoke} and \emph{return}), the instruction
cache can be ignored on all other instructions. The time needed to
load a complete method is calculated using the memory properties
(latency and bandwidth) and the length of the method. On an invoke,
the length of the invoked method is used, and on a return, the
method length of the caller is used to calculate the load time.

With a single method cache this calculation can be further
simplified. For every invoke there is a corresponding return. That
means that the time needed for the cache load on return can be
included in the time for the invoke instruction. This is simpler
because both methods, the caller and the callee, are known at the
occurrence of the invoke instruction. The information about which
method was the caller need not be stored for the return instruction
to be analyzed.

With more than one method in the cache, a cache hit detection has to
be performed as part of the WCET analysis. If there are only two
blocks, this is trivial, as (i) a hit on invoke is only possible if
the method is the same as the last invoked (e.g.\ a single method in
a loop) and (ii) a hit on return is only possible when the method is
a leaf in the call tree. In the latter case, it is always a hit.

When the cache contains more blocks (i.e.\ more than two methods can
be cached), a part of the call tree has to be taken into account for
hit detection. The method cache further complicates the analysis, as
the method length also determines the cache content. However, this
analysis is still simpler than a cache modeling of a direct-mapped
instruction cache, as cache block replacement depends on the call
tree instead of instruction addresses.

In traditional caches, data access and instruction cache fill
requests can compete for the main memory bus. For example, a load or
store at the end of the processor pipeline competes with an
instruction fetch that results in a cache miss. One of the two
instructions is stalled for additional cycles by the other
instruction. With a data cache, this situation can be even worse.
The worst-case scenario for the memory stall time for an instruction
fetch or a data load is two miss penalties when both cache reads are
a miss. This unpredictable behavior leads to very pessimistic WCET
bounds.

A \emph{method cache}, with cache fills only on invoke and return,
does not interfere with data access to the main memory. Data in the
main memory is accessed with \emph{getfield} and \emph{putfield},
instructions that never overlap with \emph{invoke} and
\emph{return}. This property removes another uncertainty found in
traditional cache designs.


\subsection{Caches Compared}

In this section, we will compare the different cache architectures
in a quantitative way. Although our primary concern is
predictability, performance remains important. We will therefore
first present the results from a conventional direct-mapped
instruction cache. These measurements will then provide a baseline
for the evaluation of the proposed architecture.

Cache performance varies with different application domains. As the
proposed system is intended for real-time applications, the
benchmark for these tests should reflect this fact. However, there
are no standard benchmarks available for embedded real-time systems.
A real-time application was therefore adapted to create this
benchmark. The application is from one node of a distributed motor
control system \cite{jop:wises03} (see also
Section~\ref{sec:app:kfl}). A simulation of the environment (sensors
and actors) and the communication system (commands from the master
station) forms part of the benchmark for simulating the real-world
workload.

The data for all measurements was captured using a simulation of JOP
and running the application for 500,000 clock cycles. During this
time, the major loop of the application was executed several hundred
times, effectively rendering any misses during the initialization
code irrelevant to the measurements.

\subsubsection{Direct-Mapped Cache}

\tablename~\ref{tab_cache_direct} gives the memory bytes and memory
transactions per instruction byte for a standard direct-mapped
cache. As we can see from the values for a cache size of 4KB, the
kernel of the application is small enough to fit completely into the
4KB cache. The cache performs better (i.e.\ fewer bytes are
transferred) with smaller block sizes. With smaller block sizes, the
chance of unused data being read is reduced and the larger number of
blocks reduces conflict misses. However, reducing the block size
also increases memory transactions (MTIB), which directly relates to
memory latency.


\begin{table}
    \centering
    \begin{tabular}{cd{2.0}cc}
    \toprule

    Cache size & \cc{Block size} & MBIB & MTIB \\

    \midrule

     1 KB &  8 & 0.28 & 0.035 \\
     1 KB & 16 & 0.38 & 0.024 \\
     1 KB & 32 & 0.58 & 0.018 \\
     2 KB &  8 & 0.17 & 0.022 \\
     2 KB & 16 & 0.25 & 0.015 \\
     2 KB & 32 & 0.41 & 0.013 \\
     4 KB &  8 & 0.00 & 0.001 \\
     4 KB & 16 & 0.01 & 0.000 \\
     4 KB & 32 & 0.01 & 0.000 \\

    \bottomrule

    \end{tabular}
    \caption{Direct-mapped cache}
    \label{tab_cache_direct}
\end{table}

Which configuration performs best depends on the relationship between
memory bandwidth and memory latency. Examples of average memory
access times in cycles per instruction byte for different memory
technologies are provided in Table~\ref{tab_cache_direct_mem}. The
third column shows the cache performance for a Static RAM (SRAM) that
is very common in embedded systems. A latency of 1 clock cycle and an
access time of 2 clock cycles per 32-bit word are assumed. For the
synchronous DRAM (SDRAM) in the forth column, a latency of 5 cycles
(3 cycles for the row address and 2 cycles for the CAS latency) is
assumed. The memory delivers one word (4 bytes) per cycle. The Double
Data Rate (DDR) SDRAM in the last column has an enhanced latency of
4.5 cycles and transfers data on both the rising and falling edge of
the clock signal.

%
%       simulate 32-bits (DDR) SDRAM
%
%       SDRAM:  5 cycle latency: 3 cycle row address and 2 cycle CAS latency
%               4 Bytes / cycle (0.25 / Byte)
%
%       DDR:    4.5 cycle latency: 2 cycle row address and 2.5 cycle CAS latency
%               4 Bytes / 0.5 cycle (0.125 / Byte)
%
%       SRAM 15 ns at 100 MHz: 1 cycle latency, 2 cycles / word
%


\begin{table}
    \centering
    \begin{tabular}{cd{2.0}ccc}
    \toprule

    Cache size & \cc{Block size} & SRAM & SDRAM & DDR \\

    \midrule

    1 KB & 8 & \textbf{0.18} & 0.25 & 0.19 \\
    1 KB & 16 & 0.22 & \textbf{0.22} & 0.16 \\
    1 KB & 32 & 0.31 & 0.24 & \textbf{0.15} \\
    2 KB & 8 & \textbf{0.11} & 0.15 & 0.12 \\
    2 KB & 16 & 0.14 & \textbf{0.14} & \textbf{0.10} \\
    2 KB & 32 & 0.22 & 0.17 & 0.11 \\

    \bottomrule

    \end{tabular}
    \caption{Direct-mapped cache, average memory access time}
    \label{tab_cache_direct_mem}
\end{table}

The data in bold give the best block size for different memory
technologies. As expected, memories with a higher latency and
bandwidth perform better with larger block sizes. For small block
sizes, the latency clearly dominates the access time. Although the
SRAM has half the bandwidth of the SDRAM and a quarter of the DDR,
with a block size of 8 bytes, it is faster than the DRAM memories.
In most cases a block size of 16 bytes is the fastest solution and
we will therefore use this configuration for comparison with the
following cache solutions.

\subsubsection{Fixed Block Cache}

Cache performance for single method per block architectures is shown
in Table~\ref{tab_cache_block}. The measurements for a simple 8 byte
prefetch queue are also given, for reference. With prefetching, we
would expect to see an MBIB of about 1. The 37\% overhead results
from the fact that the prefetch queue fetches 4 bytes a time and has
to buffer a minimum of 3 bytes for the instruction fetch stage. On a
branch or return, the queue is flushed and these bytes are lost.

A single block that has to be filled on every invoke and return
requires considerable overheads. More than twice the amount of data
is read from the main memory than is consumed by the processor.
However, the memory transaction count is 16 times lower than with
simple prefetching, which can compensate for the large MBIB for main
memories with high latency.

The solution with two blocks for two methods performs almost twice
as well as the simple one method cache. This is due to the fact
that, for all leaves in the call tree, the caller method can be
found on return. If the block count is doubled again, the number of
misses is reduced by a further 25\%, but the cache size also
doubles. For this measurement, an LRU replacement policy applies for
the two and four block caches.

\begin{table}
    \centering
    \begin{tabular}{lcccc}
    \toprule

    Type & Cache size & MBIB & MTIB \\

    \midrule
        Prefetch & 8 B & 1.37 & 0.342  \\
        Single method & 1 KB & 2.32 & 0.021  \\
        Two blocks & 2 KB & 1.21 & 0.013  \\
        Four blocks & 4 KB & 0.90 & 0.010  \\
    \bottomrule

    \end{tabular}
    \caption{Fixed block cache}
    \label{tab_cache_block}
\end{table}

The same memory parameters as in the previous section are also used
in Table~\ref{tab_cache_block_mem}. With the high latency of the
DRAMs, even the simple one block cache is a faster (and more
accurately predictable) solution than a prefetch queue. As MBIB and
MTBI show the same trend as a function of the number of blocks, this
is reflected in the access time in all three memory examples.

\begin{table}
    \centering
    \begin{tabular}{lcccc}
    \toprule

    Type & Cache size & SRAM & SDRAM & DDR \\

    \midrule
        Prefetch & 8 B & 1.02 & 2.05 & 1.71 \\
        Single Method & 1 KB & 1.18 & 0.69 & 0.39 \\
        Two blocks & 2 KB & 0.62 & 0.37 & 0.21 \\
        Four blocks & 4 KB & 0.46 & 0.27 & 0.16 \\
    \bottomrule

    \end{tabular}
    \caption{Fixed block cache, average memory access time}
    \label{tab_cache_block_mem}
\end{table}

\subsubsection{Variable Block Cache}

\tablename~\ref{tab_cache_var_block} shows the cache performance of
the proposed solution, i.e.\ of a method cache with several blocks
per method, for different cache sizes and number of blocks. For this
measurement, a \emph{next block} replacement policy applies.

\begin{table}
    \centering
    \begin{tabular}{cd{2.0}cc}
    \toprule

    Cache size & \cc{Block count} & MBIB & MTIB \\

    \midrule
        1 KB & 8 & 0.80 & 0.009  \\
        1 KB & 16 & 0.71 & 0.008  \\
        1 KB & 32 & 0.70 & 0.008  \\
        1 KB & 64 & 0.70 & 0.008  \\
        2 KB & 8 & 0.73 & 0.008  \\
        2 KB & 16 & 0.37 & 0.004  \\
        2 KB & 32 & 0.24 & 0.003  \\
        2 KB & 64 & 0.12 & 0.001  \\
        4 KB & 8 & 0.73 & 0.008  \\
        4 KB & 16 & 0.25 & 0.003  \\
        4 KB & 32 & 0.01 & 0.000  \\
        4 KB & 64 & 0.00 & 0.000  \\
    \bottomrule

    \end{tabular}
    \caption{Variable block cache}
    \label{tab_cache_var_block}
\end{table}

In this scenario, as the MBIB is very high at a cache size of 1KB
and almost independent of the block count, the cache capacity is
seen to be clearly dominant. The most interesting cache size with
this benchmark is 2KB. Here, we can see the influence of the number
of blocks on both performance parameters. Both values benefit from
more blocks. However, a higher block count requires more time or
more hardware for the hit detection. With a cache size of 4KB and
enough blocks, the kernel of the application completely fits into
the variable block cache, as we have seen with a 4KB traditional
cache. From the gap between 16 and 32 blocks (within the 4KB cache),
we can say that the application consists of fewer than 32 different
methods.

It can be seen that even the smallest configuration with a cache
size of 1KB and only 8 blocks outperforms fixed block caches with 2
or 4KB in both parameters (MBIB and MTIB). Compared with the fixed
block solutions, MTIB is low in all configurations. This is due to
the better hit rate, as indicated by the lower MBIB.

In most configurations, MBIB is higher than for the direct-mapped
cache. It is very interesting to note that, in all configurations
(even the small 1KB cache), MTIB is lower than in all 1KB and 2KB
configurations of the direct-mapped cache. This is a result of the
complete method transfers when a miss occurs and is clearly an
advantage for main memory systems with high latency.

As in the previous examples, Table~\ref{tab_cache_var_block_mem}
shows the average memory access time per instruction byte for three
different main memories.

\begin{table}
    \centering
    \begin{tabular}{cd{2.0}ccc}
    \toprule

    Cache size & \cc{Block count} & SRAM & SDRAM & DDR \\

    \midrule
        1 KB & 8 & 0.41 & 0.24 & 0.14 \\
        1 KB & 16 & 0.36 & 0.22 & 0.12 \\
        1 KB & 32 & 0.36 & 0.21 & 0.12 \\
        1 KB & 64 & 0.36 & 0.21 & 0.12 \\
        2 KB & 8 & 0.37 & 0.22 & 0.13 \\
        2 KB & 16 & 0.19 & 0.11 & 0.06 \\
        2 KB & 32 & 0.12 & 0.08 & 0.04 \\
        2 KB & 64 & 0.06 & 0.04 & 0.02 \\
    \bottomrule

    \end{tabular}
    \caption{Variable block cache, average memory access time}
    \label{tab_cache_var_block_mem}
\end{table}

In the DRAM configurations, the variable block cache directly
benefits from the low MTBI. When comparing the values between SDRAM
and DDR, we can see that the bandwidth affects the memory access
time in a way that is approximately linear. The high latency of
these memories is completely hidden. The configuration with 16 or
more blocks and dynamic RAMs outperforms the direct-mapped cache of
the same size. As expected, a memory with low latency (the SRAM in
this example) depends on the MBIB values. The variable block cache
is slower than the direct-mapped cache in the 1KB configuration
because of the higher MBIB (0.7 compared to 0.3-0.6), and performs
very similarly at a cache size of 2KB.

In Table~\ref{tab_cache_compared}, the different cache solutions with
a size of 2KB are summarized.
%
\begin{table}
    \centering
    \begin{tabular}{lcc}
    \toprule

    Cache type & MBIB & MTIB \\

    \midrule
        Single method        & 2.32 & 0.021  \\
        Two blocks           & 1.21 & 0.013  \\
        Variable block (16)  & 0.37 & 0.004  \\
        Variable block (32)  & 0.24 & 0.003  \\
        Direct-mapped        & 0.25 & 0.015 \\
    \bottomrule

    \end{tabular}
    \caption{Caches compared}
    \label{tab_cache_compared}
\end{table}
%
All full method caches with two or more blocks have a lower MTIB
than a conventional cache solution. This becomes more significant
with increasing latency in main memories. The MBIB value is only
quite high for one or two methods in the cache. However, the most
surprising result is that the variable block cache with 32 blocks
outperforms a direct-mapped cache of the same size at both values.

We can see that predictability is indirectly related to performance
-- a trend we had anticipated. The most predictable solution with a
single method cache performs very poorly compared to a conventional
direct-mapped cache. If we accept a slightly more complex WCET
analysis (taking a small part of the call tree into account), we can
use the two-block cache that is about two times better.

With the variable block cache, it could be argued that the WCET
analysis becomes too complex, but it is nevertheless simpler than
that with the direct-mapped cache. However, every hit in the
two-block cache will also be a hit in a variable block cache (of the
same size). A tradeoff might be to analyze the program by assuming a
two-block cache but using a version of the variable block cache. The
additional performance gain can than be used by non- or soft
real-time parts of an application.


\subsection{Summary}

In this section, we have extended the single cache performance
measurement \emph{miss rate} to a two value set, memory read and
transaction rate, in order to perform a more detailed evaluation of
different cache architectures. From the properties of the Java
language -- usually small methods and relative branches -- we
derived the novel idea of a \emph{method cache}, i.e.\ a cache
organization in which whole methods are loaded into the cache on
method invocation and the return from a method. This cache
organization is time-predictable, as all cache misses are lumped
together in these two instructions. Using only one block for a
single method introduces considerable overheads in comparison with a
conventional cache, but is very simple to analyze. We extended this
cache to hold more methods, with one block per method and several
smaller blocks per method.

Comparing these organizations quantitatively with a benchmark
derived from a real-time application, we have seen that the
variable block cache performs similarly to (and in one configuration
even better than) a direct-mapped cache, in respect of the bytes
that have to be filled on a cache miss. In all configurations and
sizes of the variable block cache, the number of memory
transactions, which relates to memory latency, is lower than in a
traditional cache.

Only filling the cache on method invocation and return simplifies
WCET analysis and removes another source of uncertainty, as there is
no competition for the main memory access between instruction cache
and data cache.

%\subsubsection{Future Work}
%\begin{itemize}
%    \item Variation of direct-mapped with link addresses
%    Varies between version of paper and now
%    \item cache graphs: cycles per loop
%    \item more benchmarks
%\end{itemize}

%\bibliographystyle{plain}
%\bibliography{../bib/mybib}
%
%\begin{table*}
%    \centering
%    \begin{tabular}{lcccccc}
%    \toprule
%    & & & & \multicolumn{3}{c}{Memory access time} \\
%    \cmidrule{5-7}
%    Type & Size & MBIB & MTIB & SRAM & SDRAM & DDR SDRAM \\
%    \midrule
%
%    Prefetch buffer & 8 B & 1.37 & 0.342                 & 1.02 & 2.05 & 1.71 \\
%    Single method cache & 1 KB & 2.32 & 0.021            & 1.18 & 0.69 & 0.39 \\
%    Two block cache & 2 KB & 1.21 & 0.013                & 0.62 & 0.37 & 0.21 \\
%    Four block cache & 4 KB & 0.90 & 0.010               & 0.46 & 0.27 & 0.16 \\
%    Direct mapped 8 bytes & 1 KB & 0.28 & 0.035          & 0.18 & 0.25 & 0.19 \\
%    Direct mapped 16 bytes & 1 KB & 0.38 & 0.024         & 0.22 & 0.22 & 0.16 \\
%    Direct mapped 32 bytes & 1 KB & 0.58 & 0.018         & 0.31 & 0.24 & 0.15 \\
%    Direct mapped 8 bytes & 2 KB & 0.17 & 0.022          & 0.11 & 0.15 & 0.12 \\
%    Direct mapped 16 bytes & 2 KB & 0.25 & 0.015         & 0.14 & 0.14 & 0.10 \\
%    Direct mapped 32 bytes & 2 KB & 0.41 & 0.013         & 0.22 & 0.17 & 0.11 \\
%    Direct mapped 8 bytes & 4 KB & 0.00 & 0.001          & 0.00 & 0.00 & 0.00 \\
%    Direct mapped 16 bytes & 4 KB & 0.01 & 0.000         & 0.00 & 0.00 & 0.00 \\
%    Direct mapped 32 bytes & 4 KB & 0.01 & 0.000         & 0.00 & 0.00 & 0.00 \\
%    Variable block cache 8 blocks & 1 KB & 0.80 & 0.009  & 0.41 & 0.24 & 0.14 \\
%    Variable block cache 16 blocks & 1 KB & 0.71 & 0.008 & 0.36 & 0.22 & 0.12 \\
%    Variable block cache 32 blocks & 1 KB & 0.70 & 0.008 & 0.36 & 0.21 & 0.12 \\
%    Variable block cache 64 blocks & 1 KB & 0.70 & 0.008 & 0.36 & 0.21 & 0.12 \\
%    Variable block cache 8 blocks & 2 KB & 0.73 & 0.008  & 0.37 & 0.22 & 0.13 \\
%    Variable block cache 16 blocks & 2 KB & 0.37 & 0.004 & 0.19 & 0.11 & 0.06 \\
%    Variable block cache 32 blocks & 2 KB & 0.24 & 0.003 & 0.12 & 0.08 & 0.04 \\
%    Variable block cache 64 blocks & 2 KB & 0.12 & 0.001 & 0.06 & 0.04 & 0.02 \\
%    Variable block cache 8 blocks & 4 KB & 0.73 & 0.008  & 0.37 & 0.22 & 0.13 \\
%    Variable block cache 16 blocks & 4 KB & 0.25 & 0.003 & 0.13 & 0.08 & 0.05 \\
%    Variable block cache 32 blocks & 4 KB & 0.01 & 0.000 & 0.00 & 0.00 & 0.00 \\
%    Variable block cache 64 blocks & 4 KB & 0.00 & 0.000 & 0.00 & 0.00 & 0.00 \\
%
%    \bottomrule
%
%    \end{tabular}
%    \caption[Cache performance compared]{Cache performance in MBIB and MTIB of all variations of
%    the method cache and a conventional direct mapped cache. Average
%    memory access time per instruction byte for three different main
%    memory technologies. All numbers are in clock cycles.
%    }
%    \label{tab_cache_all}
%\end{table*}
%
%
%
%\end{document}


\chapter{JOP Runtime System}
\label{chap:runtime}

    A Java processor alone is not a complete JVM. This chapter describes
the definition of a real-time profile for Java and a framework for a
user-defined scheduler in Java. It concludes with the description of
the JVM internal data structures to represent classes and objects.

    
\section{A Real-Time Profile for Embedded Java}
\label{sec:rtprof}

As standard Java is under-specified for real-time systems and the
RTSJ does not fit for small embedded systems a new and simpler
real-time profile is defined in this section and implemented on JOP.
The guidelines of the specification are:

\begin{itemize}
\item High-integrity profile
\item Easy syntax
\item Easy to implement
\item Low runtime overhead
\item No syntactic extension of Java
\item Minimum change of Java semantics
\item Support for time measurement if a WCET analysis tool is not available
\item Known overheads (documentation of runtime behavior and memory
requirements of every JVM operation and all methods have to be
provided)
\end{itemize}

The real-time profile under discussion is inspired by the restricted
versions of the RTSJ described in \cite{Pusch01} and
\cite{ravenscar:java} (see Section~\ref{subsec:restr:rtsj}). It is
intended for high-integrity real-time applications and as a test case
to evaluate the architecture of JOP as a Java processor for real-time
systems.

The proposed definition is not compatible with the RTSJ. Since the
application domain for the RTSJ is different from high-integrity
systems, it makes sense for it \emph{not} to be compatible with the
RTSJ. Restrictions can be enforced by defining new classes (e.g.\
setting thread priority in the constructor of a real-time thread
alone, enforcing minimum interarrival times for sporadic events).

%Hardware interrupts, that are usually handled by device drivers, are
%part of this profile. These interrupts are mapped to events and
%scheduled in the same way as application threads. This feature
%allows priority assignment to the device drivers and the execution
%time can be incorporated in the schedulability analysis with normal
%tasks. This solution also avoids problems with preemption latency
%caused by device drivers.

All hardware interrupts are represented by threads under the control
of the scheduler. With this solution, a priority is assigned to the
device drivers and the execution time can be incorporated in the
schedulability analysis with normal tasks. This solution also avoids
problems with preemption latency provoked by device drivers. One
example of this problem is the \emph{caps-lock} issue in Linux
\cite{REDLinux2003}: A device driver performs a spinlock wait for
keyboard acknowledgement and produces preemption latency up to
9166$\mu$s. With the proposed concept of hardware interrupts under
scheduler control, a lower assigned priority to such a device driver
avoids preemption delays of \emph{more important} real-time threads
and events.

%This specification is functional compatible with Ravenscar-Java (RJ)
%\cite{ravenscar:java}, but avoids inheritance of complex RTSJ classes. In
%fact, it is possible (and has been done) to implement RJ with the
%additional necessary RTSJ classes on top of it.

To verify that this specification is expressive enough for
high-integrity real-time applications, Ravenscar-Java (RJ)
\cite{ravenscar:java} (see Section~\ref{subsec:rj}), with the
additional necessary RTSJ classes, has been implemented on top of it.
However, RJ inherits some of the complexity of the RTSJ. Therefore,
the implementation of RJ has a larger memory and runtime overhead
than this simple specification.

\subsection{Application Structure}

The application is divided in two different phases:
\emph{initialization} and \emph{mission}. All non time-critical
initialization, global object allocations, thread creation and
startup are performed in the initialization phase. All classes need
to be loaded and initialized in this phase. The mission phase starts
after invocation of \code{startMission()}. The number of threads is
fixed and the assigned priorities remain unchanged. The following
restrictions apply to the application:

\begin{itemize}
\item Initialization and mission phase
\item Fixed number of threads
\item Threads are created at initialization phase
\item All shared objects are allocated at initialization
\end{itemize}

\subsection{Threads}

Concurrency is expressed with two types of \emph{schedulable
objects}:
\begin{description}
    \item[Periodic activities] are represented by threads that execute
in an infinite loop invoking \code{waitForNextPeriod()} to get
rescheduled in predefined time intervals.

    \item[Asynchronous sporadic activities] are represented by event
handlers. Each event handler is in fact a thread, which is released
by an hardware interrupt or a software generated event (invocation
of \code{fire()}). Minimum interarrival time has to be specified on
creation of the event handler.

\end{description}
%
The classes that implement the \emph{schedulable objects} are:
%
\begin{description}
    \item[RtThread] represents a periodic task. As usual task
work is coded in \code{run()}, which gets invoked on
\code{missionStart()}. A scoped memory object can be attached to an
\code{RtThread} at creation.

    \item[HwEvent] represents an interrupt with a minimum
interarrival time. If the hardware generates more interrupts, they
get lost.

    \item[SwEvent] represents a software-generated event.
It is triggered by \code{fire()} and needs to override
\code{handle()}.

\end{description}
%
Listing~\ref{lst:arch:rt:profile:schobj} shows the definition of the
basic classes.

\begin{lstlisting}[float,caption={Schedulable objects},
label=lst:arch:rt:profile:schobj,{emph=RtThread,enterMemory,
exitMemory,run,waitForNextPeriod,startMission,HwEvent,handle,
SwEvent,fire,handle}]

public class RtThread {

    public RtThread(int priority, int period)
    public RtThread(int priority, int period, int offset)
    public RtThread(int priority, int period, Memory mem)
    public RtThread(int priority, int period, int offset,
                    Memory mem)

    public void enterMemory()
    public void exitMemory()

    public void run()
    public boolean waitForNextPeriod()

    public static void startMission()
}

public class HwEvent extends RtThread {

    public HwEvent(int priority, int minTime, int number)
    public HwEvent(int priority, int minTime, Memory mem,
                   int number)

    public void handle()
}

public class SwEvent extends RtThread {

    public SwEvent(int priority, int minTime)
    public SwEvent(int priority, int minTime, Memory mem)

    public final void fire()
    public void handle()
}
\end{lstlisting}

Listing~\ref{lst:arch:rt:profile:example} shows the principle coding
of a worker thread. An example for creation of two real-time threads
and an event handler can be seen in
Listing~\ref{lst:arch:rt:profile:usage}.


\subsection{Scheduling}


The scheduler is a preemptive, priority-based scheduler with
unlimited priority levels and a unique priority value for each
schedulable object. No real-time threads or events are scheduled
during the initialization phase.

The design decision to use unique priority levels, instead of FIFO
within priorities, is based on following facts: Two common ways to
assign priorities are rate monotonic and, in a more general form,
deadline monotonic assignment. When two tasks are given the same
priority, we can choose one of them and assign a higher priority to
that task and the task set will still be schedulable. This results
in a strictly monotonic priority order and we do not need to deal
with FIFO order. This eliminates queues for each priority level and
results in a single, priority ordered task list with unlimited
priority levels.

Synchronized blocks are executed with priority ceiling emulation
protocol. An object, used for synchronization, for which the
priority is not set, top priority is assumed. This avoids priority
inversions on objects that are not accessible from the application
(e.g. objects inside a library).

In addition, the scheduler contains methods for worst-case time
measurement for both the periodic work and handler methods. These
measured execution times can be used during development when no WCET
analysis tool is available.

\subsection{Memory}

The profile does not support a garbage collector. All memory should
be allocated at the initialization phase. Without a garbage
collector, the heap implicitly becomes immortal memory (as defined by
the RTSJ). For objects created during the mission phase, a scoped
memory is provided.\footnote{As we now consider real-time GC as the
better solution, scopes are not supported in the current
implementation of the profile.} Each scoped memory area is assigned
to one \code{RtThread}. A scoped memory area cannot be shared between
threads. No references are allowed from the heap to scoped memory.
Scoped memory is explicitly entered and left using invocations from
the application logic. Memory areas are cleared both on creation and
when leaving the scope (invocation of \code{exitMemory()}), leading
to a memory area with constant allocation time, as opposed to memory
with linear allocation time (as the memory type \code{LTMemory} in
the RTSJ) \cite{Corsaro:2003:DPR}.


\subsection{Restrictions on Java}

A list of some of the language features that should be avoided for
WCET analyzable real-time threads and bound memory usage:

\begin{description}
    \item[WCET:] Only analyzable language constructs are allowed
        (see \cite{pusch:maxt:jnl}).

    \item[Static class initialization:] Since the definition when
        to call the static class initializer is problematic (see
        Section~\ref{para:restrict:clinit}), they are disallowed.
        Move this code to a static method (e.g. \code{init()})
        and invoke it explicitly in the initialization phase.

    \item[Inheritance:] Reduce usage of interfaces and overridden methods.

    \item[String concatenation:] In the immortal memory scope,
        only string concatenation with string literals is
        allowed.

    \item[Finalization:] \code{finalize()} has a weak definition
in Java. Because real-time systems run \emph{forever}, objects in
the heap, which is immortal in this specification, will never be
finalized. Objects in scoped memory are released on
\code{exitMemory()}. However, finalizations on these objects
complicate WCET analysis of \code{exitMemory()}.

    \item[Dynamic Class Loading:] Due to the implementation and WCET analysis
complexity dynamic class loading is avoided.

\end{description}
%
A program analysis tool can greatly help in enforcing these
restrictions.


\begin{lstlisting}[float,caption={A periodic real-time thread},
label=lst:arch:rt:profile:example]

public class Worker extends RtThread {

    private SwEvent event;

    public Worker(int p, int t,
                    SwEvent ev) {

        super(p, t,
            // create a scoped memory area
            new Memory(10000)
        );
        event = ev;
        init();
    }

    private void init() {
        // all initialzation stuff
        // has to be placed here
    }

    public void run() {

        for (;;) {
            work();       // do some work
            event.fire(); // and fire an event

            // some work in scoped memory
            enterMemory();
            workWithMem();
            exitMemory();

            // wait for next period
            if (!waitForNextPeriod()) {
                missedDeadline();
            }
        }
        // should never reach this point
    }
}
\end{lstlisting}

\begin{lstlisting}[float,caption={Start of the application},
label=lst:arch:rt:profile:usage]
    // create an Event
    Handler h = new Handler(3, 1000);

    // create two worker threads with
    // priorities according to their periods
    FastWorker fw = new FastWorker(2, 2000);
    Worker w = new Worker(1, 10000, h);

    // change to mission phase for all
    // periodic threads and event handler
    RtThread.startMission();

    // do some non real-time work
    // and invoke sleep() or yield()
    for (;;) {
        watchdogBlink();
        Thread.sleep(500);
    }
\end{lstlisting}

\subsection{Implementation Results}

The initial idea was to implement scheduling and dispatching in
microcode. However, many Java bytecodes have a one to one mapping to
a microcode instruction, resulting in a single cycle execution. The
performance gain of an algorithm coded in microcode is therefore
negligible. As a result, almost all of the scheduling is implemented
in Java. Only a small part of the dispatcher, a memory copy, is
implemented in microcode and exposed with a special bytecode.

Experimental results of basic scheduling benchmarks, such as periodic
thread jitter, context switch time for threads and asynchronous
events, can be found in \cite{jop:rtjava}.

To implement system functions, such as scheduling, in Java, access
to JVM and processor internal data structures have to be available.
However, Java does not allow memory access or access to hardware
devices. In JOP, this access is provided by way of additional
bytecodes. In the Java environment, these bytecodes are represented
as static native methods. The compiled invoke instruction for these
methods (\code{invokestatic}) is replaced by these additional
bytecodes in the class file. This solution provides a very efficient
way to incorporate low-level functions into a pure Java system. The
translation can be performed during class loading to avoid
non-standard class files.

A pure Java system, without an underlying RTOS, is an unusual system
with some interesting new properties. Java is a safer execution
environment than C (e.g.\ no pointers) and the boundary between
\emph{kernel} and \emph{user space} can become quite loose.
Scheduling, usually part of the operating system or the JVM, is
implemented in Java and executed in the same context as the
application. This property provides an easy path to a framework for
user-defined scheduling.

    \section{Real-Time Scheduler}

The following section describes the preemptive, priority based
scheduler on JOP.

\subsection{Interrupts}

Interrupts are usually associated with low-level programming of
device drivers. The priorities of interrupts and their handler
functions are above task priorities and yield to an immediate context
switch. In this form, interrupts cannot be integrated in a schedule
with \emph{normal} tasks. The execution time of the interrupt handler
has to be integrated in the schedulability analysis as additional
blocking time. A better solution is to handle interrupts, which
represent external events, as schedulable objects with priority
levels in the range of real-time tasks, as suggested in the RTSJ.

\paragraph{The Timer Interrupt}

The timer or clock interrupt has a different semantic than other
interrupts. The main purpose of the timer interrupt is representation
of time and release of periodic or time triggered tasks. One common
implementation is a clock tick. The interrupt occurs at a regular
interval (e.g.\ 10 ms) and a decision has to be taken whether a task
has to be released. This approach is simple to implement, but there
are two major drawbacks: The resolution of timed events is bound by
the resolution of the clock tick and clock ticks without a task
switch are a waste of execution time.

A better approach, used in JOP, is to generate timer interrupts at
the release times of the tasks. The scheduler is now responsible for
reprogramming the timer after each occurrence of a timer interrupt.
The list of sleeping threads has to be searched to find the nearest
release time in the future of a higher priority thread than the one
that will be released now. This time is used for the next timer
interrupt.

\paragraph{External Events}

Hardware interrupts, other than the timer interrupt, are represented
as instances of \code{Runnable}. The interrupt handler runs at top
priority. To incorporate external event handling as a \emph{normal}
schedulable object under the control of the scheduler the interrupt
handler can simply fire software event. In that case the handler code
executes in the associated thread. When the minimum interarrival time
of the hardware events is known, these events can be incorporated
into the priority assignment and schedulability analysis in the same
way as periodic tasks.

\paragraph{Software Interrupts}

The common software generated interrupts, such as illegal memory
access or divide by zero, are represented by Java runtime exceptions
and need no special handler. They can be detected with a try-catch
block.

Asynchronous notification from the program is supported, in the same
way as an external event, as a schedulable object with an associated
thread. The event is triggered through the call of \code{fire()}. The
thread with the handler is placed in the runnable state and scheduled
according to its priority.


\subsection{Task Switch}

An important issue in real-time systems is the time for a task
switch. A task switch consists of two actions:
\begin{itemize}
    \item \emph{Scheduling} is the selection of the task order
        and timing
    \item \emph{Dispatching} is the term for the context switch
        between tasks
\end{itemize}

\paragraph{Scheduling}

Most real-time systems use a fixed-priority preemptive scheduler.
Tasks with the same priority are usually scheduled in a FIFO order.
Two common ways to assign priorities are rate monotonic or, in a more
general form, deadline monotonic assignment. When two tasks get the
same priority, we can choose one of them and assign a higher priority
to that task and the task set is still schedulable. We get a strictly
monotonic priority order and do not have to deal with FIFO order.
This eliminates queues for each priority level and results in a
single, priority ordered task list.

Strictly fixed priority schedulers suffer from a problem called
\emph{priority inversion} \cite{626613}. The problem where a low
priority task blocks a high priority task on a shared resource is
solved by raising the priority of the low priority task. Two standard
priority inversion avoidance protocols are common:
%
\begin{description}
    \item[Priority Inheritance Protocol:] A lock assigns the
        priority of the highest-priority waiting task to the task
        holding the lock until that task releases the resource.

    \item[Priority Ceiling Emulation Protocol:] A lock gets a
        priority assigned above the priority of the
        highest-priority task that will ever acquire the lock.
        Every task will be immediately assigned the priority of
        that lock when acquiring it.
\end{description}
%
The priority inheritance protocol is more complex to implement and
the time when the priority of a task is raised is not so obvious. It
is not raised because the task does anything, but because another
task reaches some point in its execution path.

Using priority ceiling emulation with unique priorities, different
from task priorities, the priority order is still strictly monotonic.
The priority ordered task list is expanded with slots for each lock.
If a task acquires a lock, it is placed in the corresponding slot.
With this extension to the task list, scheduling is still simple and
can be efficiently implemented.

In the current implementation of JOP locks have implicitly top
priority. On a uniprocessor system this top priority is enforced by
switching off the interrupts. In the CMP version of JOP a single
global lock is acquired at \code{monitorenter}. 

\paragraph{Dispatching}


The time for a context switch depends on the size of the state of the
tasks. For a stack machine it is not so obvious what belongs to the
state of a task. If the stack resides in main memory, only a few
registers (e.g. program counter and stack pointer) need to be saved
and restored. However, the stack is a frequently accessed memory
region of the JVM. The stack can be seen as a data cache and should
be placed near the execution unit (in this case, \emph{near} means on
the chip and not in external memory). However, on-chip memory is
usually too small to hold the stack content for all tasks. This means
that the stack is part of the execution context and has to be saved
and restored on a context switch.

In JOP, the stack is placed in local (on-chip) FPGA memory with
single cycle access time. With this configuration, the next question
is how much of the stack to place there. Either the complete stack of
a thread or only the stack frame of the current method can reside
locally. If the complete stack of a thread is stored in local memory,
the invocation of methods and returns are fast, but the context is
large. For fast context switches, it is preferable to have only a
short stack in local memory. This results in less data being
transferred to and from main memory, but more memory transfers on
method invocation and return. The local stack can be further divided
into small pieces, each holding only one stack frame of one thread.
During the context switch, only the stack pointer needs to be saved
and restored. The outcome of this is a very fast context switch,
although the size of the local memory limits the maximum number of
threads.

Since JOP is a soft-core processor, these different solutions can be
configured for different application requirements. It is even
possible to mix of these policies: some stack slots can be assigned
to \emph{important} threads, while the remaining threads share one
slot. This stack slot only needs to be exchanged with the main memory
when switching \emph{to} a less \emph{important} thread.


\section{User-Defined Scheduler}
\label{sec:usersched}

The novel approach to implement a real-time scheduler in Java opens
up new possibilities. An obvious next step is to extend this system
to provide a framework for user-defined scheduling in Java. New
applications, such as multimedia streaming, result in \emph{soft}
real-time systems that need a more flexible scheduler than the
traditional fixed priority based ones. This section provides a
simple-to-use framework to evaluate new scheduling concepts for
these applications in real-time Java.

The following section analyzes which events are exposed to the
scheduler and which functions from the JVM need to be available in
the user space. It is followed by the definition of the framework
and examples of how to implement a scheduler using this framework.

\subsection{Schedule Events}

The most important element of the user-defined scheduler is to
define which events result in the scheduling of a new task. When
such an event occurs, the user-defined scheduler is invoked. It can
update its task list and decide which task is dispatched.

\begin{description}

\item[Timer interrupt:] For timed scheduling decisions, a programmable
timer generates exact timed interrupts. The scheduler controls the
time interval for the next interrupt.

\item[HW interrupt:] Each hardware-generated interrupt can be associated
with an asynchronous event. This allows the execution of a device
driver under the control of the scheduler. Latencies of the device
driver can be controlled by assigning the right priority in a
priority scheduler.

\item[Monitor:] To allow different implementations of priority inversion
protocols, hooks for \code{monitorenter} and \code{monitorexit} are
provided.

\item[Thread block:] Each thread can cease execution via a call of the
scheduler. This function is used to implement methods such as
\code{waitForNextPeriod()} or \code{sleep()}. The reason for
blocking (e.g. end of periodic work) has to be communicated to the
scheduler (e.g. next time to be unblocked for a periodic task).

\item[SW event:] Invoking \code{fire()} on an event provides support for
signaling. \code{wait()}, \code{notify()} or \code{notifyAll()} are
not necessary. However, this mechanism is not part of the scheduling
framework. It can be implemented with the user-defined scheduler and
an associated thread class.

\end{description}

\subsection{Data Structures}

To implement a scheduler in Java, some internal JVM data structures
need to be accessible.

\begin{description}

\item[Object:] In Java, any object (including an object from the class
\code{Class} for static methods) can be used for synchronization.
Different priority inversion protocols require different data
structures to be associated with an object. Each object provides a
field, accessed through a \code{Scheduler} method, in which these
data structures can be attached.

\item[Thread:] A list of all threads is provided to the scheduler. The
scheduler is also notified when a new thread object is created or a
thread terminates. The scheduler controls the start of threads.

\end{description}

\subsection{Services for the Scheduler}

The real-time JVM and the hardware platform have to provide some
minimum services. These services are exposed through
\code{Scheduler}:

\begin{description}

\item[Dispatch:] The current active thread is interrupted and a new
thread is placed in the run state.

\item[Time:] System time with high resolution (microseconds, if the
hardware can provide it) is used for time derived scheduling
decisions.

\item[Timer:] A programmable timer interrupt (not a timer tick) is
necessary for accurate time triggered scheduling.

\item[Interrupts:] To protect the data structures of the scheduler all
interrupts can be disabled and enabled.

\end{description}

\subsection{Class Scheduler}

The class \code{Scheduler} has to be extended to implement a
user-defined scheduler. The class \code{Task} represents
\emph{schedulable objects}. For non-trivial scheduling algorithms,
\code{Task} is also extended. The scheduler lives in normal thread
space. There is no special context such as kernel space. The methods
of \code{Scheduler} are categorized by the caller module and
described in detail below.

\paragraph{Application}

To use a scheduler in an application, the application only has to
create one instance of the scheduler class and has to decide when
scheduling starts.

\begin{lstlisting}[emph=Scheduler]
public Scheduler()
\end{lstlisting}
A single instance of the scheduler is created by the application.

\begin{lstlisting}[emph=start]
public void start()
\end{lstlisting}
This method initiates the transition to the mission phase of the
application. All created tasks are started and scheduled under the
control of the user scheduler.

\paragraph{Task}

A user-defined scheduler usually needs an associated user-defined
thread class (an extension of \code{Task}). This class interacts
with the scheduler by invoking following methods from
\code{Scheduler}:

\begin{lstlisting}[emph=addTask]
void addTask(Task t)
\end{lstlisting}
The scheduler has access to the list of created tasks to use at the
start of scheduling. For dynamic task creation after the start of
the scheduler, this method is called by the constructor of Task, to
notify the scheduler to update its list.

\begin{lstlisting}[emph=isDead]
void isDead(Task t)
\end{lstlisting}
The scheduler is notified when a Task returns from the \code{run()}
method. The scheduler removes this Task from the list of schedulable
objects.

\begin{lstlisting}[emph=block]
void block()
\end{lstlisting}
Every \code{Task} can cease execution via a call of the scheduler.
This method is used to implement methods such as
\code{waitForNextPeriod()} or \code{sleep()} in a user defined
thread class.

\paragraph{Java Virtual Machine}

The methods listed below provide the essential points of
communication between the JVM and the scheduler. As a response to an
interrupt (hardware or timer), entrance or exit of a synchronized
method/block the JVM invokes a method from the scheduler.

\begin{lstlisting}[emph=schedule]
abstract void schedule()
\end{lstlisting}
This is the main entry point for the scheduler. This method has to
be overridden to implement the scheduling algorithm. It is called
from the JVM on a timed event or a software interrupt (see
\code{genInt()}) is issued (e.g. when a \code{Task} gives up
execution).

\begin{lstlisting}[emph=interrupt]
void interrupt(int nr)
\end{lstlisting}
The scheduler is notified on a hardware event. It can directly call
an associated device driver or use this information to unblock a
waiting task.

\begin{lstlisting}[emph={monitorEnter,monitorExit}]
 void monitorEnter(Object o)
 void monitorExit(Object o)
\end{lstlisting}
These methods are invoked by the JVM on synchronized methods and
blocks (JVM bytecodes \code{monitorenter} and \code{monitorexit}).
They provide hooks for executing dynamic priority changes in the
scheduler.

\paragraph{Scheduler}

Services of the JVM needed to implement a scheduler are provided
through static methods.

\begin{lstlisting}[emph=genInt]
static final void genInt()
\end{lstlisting}
This service from the JVM schedules a software interrupt. As a
result, \code{schedule()} is called. This method is the standard way
of switching control to the scheduler. It is e.g. invoked by
\code{block()}.

\begin{lstlisting}[emph={enableInt,disableInt}]
 static final void enableInt()
 static final void disableInt()
\end{lstlisting}
The scheduler cannot use monitors to protect its data structures as
the scheduler itself is in charge of handling monitors. To protect
the data structures of the scheduler, it can globally enable and
disable interrupts.

\begin{lstlisting}[emph=dispatch]
static final void dispatch(Task nextTask, int nextTim)
\end{lstlisting}
This method dispatches a \code{Task} and schedules a timer interrupt
at \code{nextTim}.

\begin{lstlisting}[emph={attachData,getAttachedData}]
 static final void attachData(Object obj, Object data)
 static final Object getAttachedData(Object obj)
\end{lstlisting}
The behavior of the priority inversion avoidance protocol is defined
by the user scheduler. The root of the Java class hierarchy
(\code{java.lang.Object}) contains a JVM internal reference of
generic type Object that can be used by the scheduler to attach data
structures for monitors. The first argument of these methods is the
object to be used as a monitor.

\paragraph{Scheduler or Task}

The following two methods are utility functions useful for the
scheduler and the thread implementation.

\begin{lstlisting}[emph=getNow]
static final int getNow()
\end{lstlisting}
To support time-triggered scheduling, the system provides access to
a high-resolution time or counter. The returned value is the time
since startup in microseconds. The exact resolution is
implementation-dependent.

\begin{lstlisting}[emph=getRunningTask]
static final Task getRunningTask()
\end{lstlisting}
The current running \code{Task} (in which context the scheduler is
called) is returned by this method.

\subsection{Class Task}

A basic structure for schedulable objects is shown in
Listing~\ref{lst:arch:rt:user:task}. This class is usually extended
to provide a thread implementation that fits in the user-defined
scheduler. The class \code{Task} is intended to be minimal. To avoid
inheriting methods that do not fit for some applications, it does not
extend \code{java.lang.Thread}. However, \code{Task} can be used to
\emph{implement} \code{java.lang.Thread}.

\begin{lstlisting}[float,caption=A basic schedulable object,
label=lst:arch:rt:user:task, emph={Task, start, enterMemory,
exitMemory,run,getFirstTask,getNextTask}]
    public class Task {

        public Task()
        public Task(Memory mem)
        void start()

        public void enterMemory()
        public void exitMemory()

        public void run()

        static Task getFirstTask()
        static Task getNextTask()
    }
\end{lstlisting}

The methods \code{enterMemory} and \code{exitMemory} are used by the
application to provide scoped memory for temporary allocated
objects. \code{Task} provides a list of active tasks for the
scheduler.

One issue, raised by the implementation of the framework, is the way
in which access rights to methods need to be defined in Java. All
methods, except \code{start()}, should be \code{private} or
\code{protected}. However, some methods, such as \code{schedule()},
are invoked by a part of the JVM, which is also written in Java but
resides in a different package. This results in defining the methods
as public and \emph{hoping} that they are not invoked by the
application code. The C++ concept of friends would greatly help in
sharing information over package boundaries without making this
information public.

\subsection{A Simple Example Scheduler}

Listing~\ref{lst:arch:rt:user:example} shows a full example of using
this framework to implement a simple round robin scheduler.

The only method that needs to be supplied is \code{schedule()}. For
a more advanced scheduler, it is necessary to provide a combination
of a user defined thread class and a scheduler class. These two
classes have to be tightly integrated, as the scheduler uses
information provided by the thread objects for its scheduling
decisions.

\pagebreak
\begin{lstlisting}[caption=A very simple scheduler,
label=lst:arch:rt:user:example]
    public class RoundRobin extends Scheduler {

        //
        //   test threads
        //
        static class Work extends Task {

            private int c;

            Work(int ch) {
                c = ch;
            }

            public void run() {

                for (;;) {
                    Dbg.wr(c); // debug output

                    // busy wait to simulate
                    // 3 ms workload in Work.
                    int ts = Scheduler.getNow();
                    ts += 3000;
                    while (ts-Scheduler.getNow()>0)
                        ;
                }
            }
        }

        //
        //    user scheduler starts here
        //

        public void addTask(Task t) {
            // we do not allow tasks to be
            // added after start().
        }

        //
        //    called by the JVM
        //
        public void schedule() {
            Task t = getRunningTask().getNextTask();
            if (t==null) t = Task.getFirstTask();
            dispatch(t, getNow()+10000);
        }


        public static void main(String[] args) {

            new Work('a');
            new Work('b');
            new Work('c');

            RoundRobin rr = new RoundRobin();

            rr.start();
        }
    }
\end{lstlisting}


\subsection{Interaction of Task, Scheduler and the JVM}

The framework is used to re-implement the scheduler described in
Section~\ref{sec:rtprof}. In the original implementation, the
interaction between scheduling and threads was simple, as the
scheduling was part of the thread class. Using the framework, these
functions have to be split to two classes, extending \code{Task} and
\code{Scheduler}. Both classes are placed in the same package to
provide simpler information sharing with some protection from the
rest of the application. For performance reasons, data structures are
directly exposed from one class to the other.

The resulting implementation is compatible with the first
definition, with the exception that \code{RtThread} now extends
\code{Task}. However, no changes in the application code are
necessary.

\figurename~\ref{fig_arch_rt_user_interaction} is an interaction
example of this scheduler within the framework. The interaction
diagram shows the message sequences between two application tasks,
the scheduler, the JVM and the hardware. The hardware represents
interrupt and timer logic. The corresponding code fragments of the
application, \code{RtThread} and \code{PriorityScheduler} are shown
in Listing~\ref{lst:arch:rt:user:app}, \ref{lst:arch:rt:user:rtthr}
and \ref{lst:arch:rt:user:prsched}. Task 2 is a periodic task with a
higher priority than Task 1.

\begin{figure*}
    \centering
    \includegraphics[scale=\picscale]{runtime/rt_user_interaction}
    \caption[Interaction diagram of the user scheduler framework]
    {Interaction and message exchange between the application,
the scheduler, the JVM and the hardware}
    \label{fig_arch_rt_user_interaction}
\end{figure*}

The first event is a timer event to unblock Task 2 for a new period.
The generated timer event results in a call of the user defined
scheduler. The scheduler performs its scheduling decision and issues
a context switch to Task 2. With every context switch the timer is
reprogrammed to generate an interrupt at the next time triggered
event for a higher priority task. Task 2 performs the periodic work
and ceases execution by invocation of \code{waitForNextPeriod()}.
The scheduler is called and requests an interrupt from the hardware
resulting in the same call sequence as with a timer or other
hardware interrupt. The software generated interrupt imposes
negligible overhead and results in a single entry point for the
scheduler. Task 1 is the only ready task in this example and is
resumed by the scheduler.

Using a general scheduling framework for a real-time scheduler is
not without its costs. Additional methods are invoked from a
scheduling event until the actual dispatch takes place. The context
switch is about 20\% slower than in the original implementation. It
is the opinion of the author that the additional cost is outweighed
by the flexibility of the framework.

\begin{lstlisting}[float,caption={Code fragment oft the application},
label=lst:arch:rt:user:app]
        for (;;) {
            doPeriodicWork();
            waitForNextPeriod();
        }
\end{lstlisting}

\begin{lstlisting}[float,caption={Implementation in RtThread},
label=lst:arch:rt:user:rtthr]
    public boolean waitForNextPeriod() {

        synchronized(monitor) {

            // ps is the instance of
            // the PriorityScheduler
            int nxt = ps.next[nr] + period;

            int now = Scheduler.getNow()
            if (nxt-now < 0) {
                // missed deadline
                doMissAction();
                return false;
            } else {
                // time for the next unblock
                ps.next[nr] = nxt;
            }
            // just schedule an interrupt
            // schedule() gets called.
            ps.block();
        }
        return true;
    }
\end{lstlisting}

\begin{lstlisting}[float,caption={Implementation of the PriorityScheduler},
label=lst:arch:rt:user:prsched]
    public void schedule() {

        // Find the ready thread with
        // the highest priority.
        int nr = getReady();

        // Search the list of sleeping threads
        // to find the nearest release time
        // in the future of a higher priority
        // thread than the one that will be
        // released now.
        int time = getNextTimer(nr);

        // This time is used for the next
        // timer interrupt.
        // Perform the context switch.
        dispatch(task[nr], time);
        // No access to locals after this point.
        // We are running in the NEW context!
    }
\end{lstlisting}


\subsection{Predictability}

The architecture of JOP is designed to simplify WCET analysis. Every
JVM bytecode maps to one ore more microcode instructions. Every
microcode instruction takes exactly one cycle to execute. Thus, the
execution time at the bytecode level is known cycle accurately. The
microcode contains no data dependent or unbound loops that would
compromise the WCET analysis (see Chapter~\ref{chap:wcet}).

The worst-case time for dispatching is known cycle accurately on
this architecture. Only the time behavior of the user scheduler
needs to be analyzed. With the known WCET of every bytecode, as
listed in Appendix~\ref{appx:bytecode}, the WCET of the scheduler
can be obtained by examining it at the bytecode level. This can be
done manually or with a WCET analysis tool.

\subsection{Related Work}

Several implementations of user-level schedulers in standard
operating systems have been proposed. In \cite{REDLinux2003}, the
Linux scheduling mechanism is enhanced. It is divided into a
dispatcher and an allocator. The dispatcher remains in kernel space;
while the allocator is implemented as a user space function. The
allocator transforms four basic scheduling parameters (priority,
start time, finish time and budget) into scheduling attributes to be
used by the dispatcher. Many existing schedulers can be supported
with this parameter set, but others that are based on different
parameters cannot be implemented. This solution does not address the
implementation of protocols for shared resources.

A different approach defines a new API to enable applications to use
application-defined scheduling in a way compatible with the
scheduling model defined in POSIX \cite{787339}. It is implemented
in the MaRTE OS, a minimal real-time kernel that provides the C and
Ada language POSIX interface. This interface has been submitted to
the Real-Time POSIX Working Group for consideration.

One approach to user-level scheduling in Java can be found in
\cite{Feizabadi:2003:UAS}. A thread \emph{multiplexor}, as part of
the FLEX ahead-of-time compiler system for Java, is used for utility
accrual scheduling. However, the underlying operating system -- in
this case Linux -- can still be seen through the framework and there
is no support for Java synchronization.

\subsection{Summary}

This section and Section~\ref{sec:rtprof} consider the
implementation of real-time scheduling on a Java processor. The
novelty of the described approach is in implementing functions
usually associated with an RTOS in Java. That means that real-time
Java is not based on an RTOS, and therefore not restricted to the
functionality provided by the RTOS. With JOP, a self-contained
real-time system in pure Java becomes possible. This system is
augmented with a framework to provide scheduling functions at the
application level. The implementation of the specification,
described in Section~\ref{sec:rtprof}, is successfully used as the
basis for a commercial real-time application in the railway
industry. Future work will extend this framework to support multiple
schedulers. A useful combination of schedulers would be: one for
standard \code{java.lang.Thread} (optimized for throughput), one for
soft real-time tasks and one for hard real-time tasks.

    \section{JVM Architecture}

\index{JVM}

This section presents the details of the implementation of the JVM on
JOP. The representation of objects and the stack frame is chosen to
support JOP as a processor for real-time systems. However, since the
data structures are realized through microcode they can be easily
changed for a system with different needs.

\subsection{Runtime Data Structures}

\index{JVM!data structures}

Memory is addressed as 32-bit data, which means that memory pointers
are incremented for every four bytes. No single byte or 16-bit access
is necessary. The abstract type reference is a pointer (via a handle
indirection) to memory that represents the object or an array. The
reference is pushed on the stack before an instruction can operate on
it. A null reference is represented by the value 0.

\subsubsection{Stack Frame}

On invocation of a method, the invoker's context is saved in a newly
allocated frame on the stack. It is restored when the method returns.
The saved context consists of the following registers:

\begin{description}

\item[SP:] Immediately before invocation, the stack pointer
    points to the last argument for the called function. This
    value is reduced by the argument count (i.e. the arguments
    are consumed) and saved in the new stack frame.

\item[PC:] The pointer to the next bytecode instruction after the invoke
instruction.

\item[VP:] The pointer to the memory area on the stack that contains
the locals.

\item[CP:] The pointer to the constant pool of the class from the invoking
method.

\item[MP:] The pointer to the method structure of the invoking method.

\end{description}

SP, PC and VP are registers in JOP while CP and MP are local
variables of the JVM. \figurename~\ref{fig_jvm_stack_invoke} provides
an example of the stack before and after invoking a method. In this
example, the called method has two arguments and contains two local
variables. If the method is a virtual one, the first argument is the
reference to the object (the \emph{this}-pointer). The arguments
implicitly become locals in the called method and are accessed in the
same way as local variables. The start of the stack frame
(\emph{Frame} in the figure) needs not to be saved. It is not needed
during execution of the method or on return. To access the starting
address of the frame (e.g. for an exception) it can be calculated
with information from the method structure:

\[Frame = VP + arg\_cnt + locals\_cnt\]

\begin{figure}
    \centering
    \includegraphics[scale=\picscale]{jvm/jvm_stack_invocation}
    \caption{Stack change on method invocation}
    \label{fig_jvm_stack_invoke}
\end{figure}

\subsubsection{Object Layout}

\index{object layout}

\figurename~\ref{fig_jvm_object} shows the representation of an
object in memory. The object reference points to a handle area and
the first element in the handle area points to the first instance
variable of the object. At the offset $1$ in the handle area, a
pointer is located to access class information. To speed-up method
invocation, it points directly to the method table of the objects
class instead of the beginning of the class data. The handle
indirection simplifies the implementation of a compacting GC (see
Chapter~\ref{chap:rtgc}).


\begin{figure}
    \centering
    \includegraphics[scale=\picscale]{jvm/jvm_object}
    \caption{Object format}
    \label{fig_jvm_object}
\end{figure}

\subsubsection{Array Layout}

\index{array layout}

\figurename~\ref{fig_jvm_array} shows the representation of an array
in memory. The array reference points to a handle area and the first
element in the handle area points to the first element of the array.
At the offset $1$ in the handle area, the length of the array can be
found.

\begin{figure}
    \centering
    \includegraphics[scale=\picscale]{jvm/jvm_array}
    \caption{Array format}
    \label{fig_jvm_array}
\end{figure}


\subsubsection{Class Structure}

\index{class structure}


Runtime class information, as shown in Figure~\ref{fig_jvm_class},
consists of the instance size, GC info, the dispatch table for the
methods, a reference to the super class, the constant pool, and an
optional interface table. The class variables (static fields) are
located at the start of the memory to speedup access to the fields
(the constant pool index of \code{getstatic} and \code{putstatic} is
converted at class link time to the address in memory). Furthermore
all reference static fields are located in one continuous region for
simple GC root scanning. A pointer to the static primitive fields of
the class is needed for the implementation of hardware objects.

\begin{figure}
    \centering
    \includegraphics[scale=\picscale]{jvm/jvm_class}
    \caption{Runtime class structure}
    \label{fig_jvm_class}
\end{figure}


The class reference is obtained from the constant pool when a new
object is created. The method vector base pointer is a reference from
an object to its class (see Figure~\ref{fig_jvm_object}). It is used
on \code{invokevirtual}. The index is the argument of the bytecode to
avoid an additional memory access in invoke (the original index into
the constant pool is overwritten by the direct index at class link
time). A pointer to the method structure of the current method is
saved in the JVM variable MP. The method structure, as shown in
Figure~\ref{fig_jvm_method}, contains the starting address and length
of the method (in 32-bit words), argument and local variable count
and a pointer to the constant pool of the class. Since the constant
pool is an often accessed memory area, a pointer to it is cached in
the JVM variable CP.

\begin{figure}
    \centering
    \includegraphics[scale=\picscale]{jvm/jvm_method}
    \caption{Method structure}
    \label{fig_jvm_method}
\end{figure}


The interface table contains references to the method structures of
the implementation. Only classes that implement an interface contain
this table. To avoid searching the class hierarchy on
\code{invokeinterface}, each interface method is assigned a unique
index. This arrangement provides constant execution time, but can
lead to large interface tables.

The constant pool contains various constants of a class. The entry at
index 0 is the length of the pool. All constants, which are symbolic
in the class files, are resolved during class linking. The different
constant types and their values after resolving are listed in
\tablename~\ref{tab_jvm_const_pool}. The names for the types are the
same as in the JVM specification \cite{jvm}.


\begin{table}[t]
    \centering
    \begin{tabular}{ll}
        \toprule
        Constant type &  Description \\
        \midrule
        Class &  A pointer to a class (class reference) \\
        Fieldref &   For static fields: a direct pointer to the field \\
                &   For object fields: the position relative to the object \\
                & reference \\
        Methodref &  For static methods: a direct pointer to the method structure \\
                & For virtual methods: the offset in the method table \\
                & (= index*2) and the number of arguments \\
        InterfaceMethodref &  A system wide unique index into the interface table \\
        String  & A pointer to the string object that represents the string \\
                & constant \\
        Integer & The constant value \\
        Float   & The constant value \\
        Long    & This constant value spans two entries in the constant pool \\
        Double  & Same as for long constants \\
        NameAndType & Not used \\
        Utf8    & Not used \\
        \bottomrule
    \end{tabular}
    \caption{Constant pool entries}
    \label{tab_jvm_const_pool}
\end{table}

\subsection{Class Initialization}
\label{para:restrict:clinit}

\index{JVM!class initialization}


According to \cite{jvm} the static initializers of a class C are
executed immediately before one of the following occurs: (i) an
instance of C is created; (ii) a static method of C is invoked or
(iii) a static field of C is used or assigned. The issue with this
definition is that it is not allowed to invoke the static
initializers at JVM startup and it is not so obvious when it gets
invoked.

It follows that the bytecodes \code{getstatic}, \code{putstatic},
\code{invokestatic} and \code{new} can lead to class initialization
and the possibility of high WCET values. In the JVM, it is necessary
to check every execution of these bytecodes if the class is already
initialized. This leads to a loss of performance and is violated in
some existing implementations of the JVM. For example, the first
version of CACAO \cite{cacao} invokes the static initializer of a
class at compilation time. Listing~\ref{lst:retrict:clinit} shows an
example of this problem.

\cmd{JOPizer} tries to find a correct order of the class initializers
and puts this list into the application file. If a circular
dependency is detected the application will not be built. The class
initializers are invoked at JVM startup.

\begin{lstlisting}[float=t,caption={Class initialization can occur very late},
label=lst:retrict:clinit]
    public class Problem {

        private static Abc a;
        public static int cnt; // implicitly set to 0

        static {
            // do some class initializaion
            a = new Abc();  //even this is ok.
        }

        public Problem() {
            ++cnt;
        }
    }

    // anywhere in some other class, in situation,
    // when no instance of Problem has been created
    // the following code can lead to
    // the execution of the initializer
    int nrOfProblems = Problem.cnt;
\end{lstlisting}

\subsection{Synchronization}

Synchronization is possible with methods and on code blocks. Each
object has a monitor associated with it and there are two different
ways to gain and release ownership of a monitor. Bytecodes
\code{monitorenter} and \code{monitorexit} explicitly handle
synchronization. In other cases, synchronized methods are marked in
the class file with the access flags. This means that all bytecodes
for method invocation and return must check this access flag. This
results in an unnecessary overhead on methods without
synchronization. It would be preferable to encapsulate the bytecode
of synchronized methods with bytecodes \code{monitorenter} and
\code{monitorexit}. This solution is used in Suns picoJava-II
\cite{pjProgRef}. The code is manipulated in the class loader. Two
different ways of coding synchronization, in the bytecode stream and
as access flags, are inconsistent. With \cmd{JOPizer} the same
manipulation of the methods is performed to wrap the method code in a
synchronized block when the method is defined synchronized.

    \subsection{Booting the JVM}

\index{JVM!boot-up}

One interesting issue for an embedded system is how the boot-up is
performed. On power-up, the FPGA starts the configuration state
machine to read the FPGA configuration data either from a Flash or
via a download cable (for development). When the configuration has
finished, an internal reset is generated. After that reset, microcode
instructions are executed starting from address 0. At this stage, we
have not yet loaded any application program (Java bytecode). The
first sequence in microcode performs this task. The Java application
can be loaded from an external Flash or via a serial line (or an USB
port) from a PC. The microcode assembly configured the mode.
Consequently, the Java application is loaded into the main memory. To
simplify the startup code we perform the rest of the startup in Java
itself, even when some parts of the JVM are not yet setup.

In the next step, a minimal stack frame is generated and the special
method \code{Startup.boot()} is invoked. From now on JOP runs in
Java mode. The method \code{boot()} performs the following steps:
\begin{samepage}
\begin{enumerate}
    \item Send a greeting message to \emph{stdout}
    \item Detect the size of the main memory
    \item Initialize the data structures for the garbage collector
    \item Initialize \code{java.lang.System}
    \item Print out JOP's version number, detected clock speed, and
    memory size
    \item Invoke the static class initializers in a predefined order
    \item Invoke the \code{main} method of the application class
\end{enumerate}
\end{samepage}

    \label{sec:gc}
    Automatic memory management or garbage collection greatly simplifies
development of large systems. However, garbage collection is usually
not used in real-time systems due to the unpredictable temporal
behavior of current implementations of a garbage collector. In this
chapter we describe a concurrent collector that is scheduled
periodically in the same way as ordinary application threads. We
provide an upper bound for the collector period so that the
application threads will never run out of memory. This chapter is
based on following papers:
\cite{jop:rtgc_sched,jop:scjgc,jop:nbobjcopy:jtres2008}.


\section{Introduction}

Garbage Collection (GC) is an essential part of the Java runtime
system. GC enables automatic dynamic memory management which is
essential to build large applications. Automatic memory management
frees the programmer from complex and error prone explicit memory
management (\code{malloc} and \code{free}).

However, garbage collection is considered unsuitable for real-time
systems due to the unpredictable blocking times introduced by the GC
work. One solution, used in the Real-Time Specification for Java
(RTSJ) \cite{rtsj}, introduces new thread types with program-managed,
scoped memory for dynamic memory requirements. This scoped memory
(and static memory called \emph{immortal} memory) is not managed by
the GC, and strict assignment rules between different memory areas
have to be checked at runtime. This programming model differs largely
from standard Java and is difficult to use \cite{Niessner03,
conf/isorc/PizloFHV04}.

We believe that for the acceptance of Java for real-time systems, the
restrictions imposed by the RTSJ are too strong. To simplify creation
of possible large real-time applications, most of the code should be
able to use the GC managed heap. For a collector to be used in
real-time systems two points are essential:
\begin{itemize}
    \item The GC has to be incremental with a short maximum blocking time
    that has to be known
    \item The GC has to keep up with the garbage generated by the
    application threads to avoid out-of-memory stalls
\end{itemize}
The first point is necessary to limit interference between the GC
thread and high-priority threads. It is also essential to minimize
the overhead introduced by read- and write-barriers, which are
necessary for synchronization between the GC thread and the
application threads. The design of a GC within these constraints is
the topic of this chapter.

The second issue that has to be considered is scheduling the GC so
that the GC collects enough garbage. The memory demands (static and
dynamic) by the application threads have to be analyzed. These
requirements, together with the properties of the GC, result in
scheduling parameters for the GC thread. We will provide a solution
to calculate the maximum period of the GC thread that will collect
enough memory in each collector cycle so we will never run out of
memory. The collector cycle depends on the heap size and the
allocation rate of the application threads.

To distinguish between other garbage collectors and a collector for
(hard) real-time systems we define a real-time collector as follows:

\begin{quote}
    A real-time garbage collector provides time predictable
    automatic memory management for tasks with a bounded memory
    allocation rate with minimal temporal interference to tasks
    that use only static memory.
\end{quote}


The collector presented in this chapter is based on the work by
Steele \cite{gc:steele75},  Dijkstra \cite{gc:dijkstra78} and Baker
\cite{gc:baker78}. However, the copying collector is changed to
perform the copy of an object concurrently by the collector and not
as part of the mutator work. Therefore we name it
\emph{concurrent-copy} collector.

We will use the terms first introduced by Dijkstra with his
\emph{On-the-Fly} concurrent collector \cite{gc:dijkstra78}. The
application is called the \emph{mutator} to reinforce that the
application changes (mutates) the object graph while the GC does the
collection work. The GC process is simply called \emph{collector}. In
the following discussion we will use the color scheme of white, gray,
and black objects:

\begin{description}
    \item[Black] indicates that the object and all immediate
    descendants have been visited by the collector.
    \item[Grey] objects have been visited, but the descendants may
    not have been visited by the collector, or the mutator has
    changed the object.
    \item[White] objects are unvisited. At the beginning of a GC
        cycle all objects are white. At the end of the tracing,
        all white objects are garbage.
\end{description}

At the end of a collection cycle all black objects are live (or
floating garbage) and all white objects are garbage.

\subsection{Incremental Collection}

An incremental collector can be realized in two ways: either by doing
part of the work on each allocation of a new object or by running the
collector as an independent process. For a single-threaded
application, the first method is simpler as less synchronization
between the application and the collector is necessary. For a
multi-threaded environment there is no advantage by interleaving
collector work with object allocation. In this case we need
synchronization between the collector work done by one thread and the
manipulation of the object graph performed by the other mutator
thread. Therefore we will consider a concurrent solution where the
collector runs in its own thread or processor. It is even possible to
realize the collector as dedicated hardware \cite{gc:flavius}.

\subsection{Conservatism}

Incremental collector algorithms are conservative, meaning that
objects becoming unreachable during collection are not collected by
the collector --- they are floating garbage. Many approaches exist to
reduce this conservatism in the general case. However, algorithms
that completely avoid floating garbage are impractical. For different
conservative collectors the worst-case bounds are all the same (i.e.,
all objects that become unreachable during collection remain floating
garbage). Therefore the level of conservatism is not an issue for
real-time collectors.

\subsection{Safety Critical Java}

In \cite{jop:scjava} a profile for safety critical Java (SCJ) is
defined. SCJ has two interesting properties that may simplify the
implementation of a real-time collector. Firstly, the split between
initialization and mission phase, and secondly the simplified
threading model (which also mandates that self-blocking operations
are illegal in the mission phase).  During initialization of the
application a SCJ virtual machine does not have to meet any real-time
constraints (other than possibly a worst case bound on the entire
initialization phase). It is perfectly acceptable to use a
non-real-time GC implementation during this phase -- even a
stop-the-world GC. As the change from initialization to mission phase
is explicit, it is clear when the virtual machine must initiate
real-time collection and which code runs during the mission phase.


Simplifying the threading model has the following advantage, if the
collector thread runs at a lower priority than all other threads in
the system, it is the case that when it runs \emph{all} other threads
have returned from their calls to \code{run()}. This is trivially
true due to the priority preemptive scheduling
discipline.\footnote{If we would allow blocking in the application
threads, we would also need to block the GC thread.} Any thread that
has not returned from its \code{run()} method will preempt the GC
until it returns. This has the side effect that the GC will never see
a root in the call stack of another thread. Therefore, the usually
atomic operation of scanning call stacks can be omitted in the
mission phase. We will elaborate on this property in
Section~\ref{sec:scj:simple}.


\section{Scheduling of the Collector Thread}
\label{sec:gcsched}

The collector work can be scheduled either \emph{work} based or
\emph{time} based. On a work based scheduling, as performed in
\cite{gc:siebert:phd}, an incremental part of the collector work is
performed at object allocation. This approach sounds quite natural as
threads that allocate more objects have to pay for the collector
work. Furthermore, no additional collector thread is necessary. The
main issue with this approach is to determine how much work has to be
done on each allocation -- a non trivial question as collection work
consists of different phases. A more subtle question is: Why should a
high frequency (and high priority) thread increase its WCET by
performing collector work that does not have to be done at that
period? Leaving the collector work to a thread with a longer period
allows higher utilization of the system.

On a time based scheduling of the collector work, the collector runs
in its own thread. Scheduling this thread as a \emph{normal}
real-time thread is quite natural for a hard real-time system. The
question is: which priority to assign to the collector thread? The
Metronome collector \cite{gc:bacon03} uses the highest priority for
the collector. Robertz and Henriksson \cite{780745} and Schoeberl
\cite{jop:rtgc_sched} argue for the lowest priority. When building
hard real-time systems the answer must take scheduling theory into
consideration: the priority is assigned according to the period,
either rate monotonic \cite{321743} or more general deadline
monotonic \cite{Audsley-etal91}. Assuming that the period of the
collector is the longest in the system and the deadline equals the
period the collector gets the lowest priority.

In this section we provide an upper bound for the collector period
so that the application threads will never run out of memory. The
collector period, besides the WCET of the collector, is the single
parameter of the collector that can be incorporated in standard
schedulability analysis.

The following symbols are used in this section: heap size for a
mark-compact collector ($H_{MC}$) and for a concurrent-copying
collector ($H_{CC}$) containing both semi-spaces, period of the GC
thread ($T_{GC}$), period of a single mutator thread ($T_M$), period
of mutator thread $i$ ($T_i$) from a set of threads, and memory
amount allocated by a single mutator ($a$) or by mutator $i$ ($a_i$)
from a set of threads.

We assume that the mutator allocates all memory at the start of the
period and the memory becomes garbage at the end. In other words the
memory is live for one period. This is the worst-case,\footnote{See
Section~\ref{sec:wc:live} for an example where the worst-case
lifetime is two periods.} but very common as we can see in the
following code fragment.
\begin{samepage}
\begin{lstlisting}
    for (;;) {
        Node n = new Node();
        work(n);
        waitForNextPeriod();
    }
\end{lstlisting}
\end{samepage}
The object \code{Node} is allocated at the start of the period and
\code{n} will reference it until the next period when a new
\code{Node} is created and assigned to \code{n}. In this example we
assume that no reference to \code{Node} is stored (inside
\code{work}) to an object with a longer lifetime.


\subsection{An Example} \label{sec:example}

\begin{figure}
\begin{center}
    \input{jvm/exmc.latex}
    \caption{Heap usage during a mark-compact collection cycle}
\label{fig:exmc}
\end{center}
\end{figure}

\begin{figure}
\begin{center}
    \input{jvm/excc2.latex}
    \caption{Heap usage during a concurrent-copy collection cycle}
\label{fig:excc}
\end{center}
\end{figure}


We start our discussion with a simple example\footnote{The relation
between the heap size and the mutator/collector proportion is an
arbitrary value in this example. We will provide the exact values in
the next sections.} where the collector period is 3 times the mutator
period ($T_{GC} = 3 T_M$) and a heap size of 8 objects ($8a$). We
show the heap during one GC cycle for a mark-compact and a
concurrent-copy collector. The following letters are used to show the
status of a memory cell (that contains one object from the mutator in
this example) in the heap: $g_i$ is garbage from mutator cycle $i$,
$l$ is the live memory, and $f$ is floating garbage. We assume that
all objects that become unreachable during the collection remain
floating garbage.

Figure~\ref{fig:exmc} shows the changes in the heap during one
collection cycle. At the start there are three objects ($g_1$,
$g_2$, and $g_3$) left over from the last cycle (floating garbage)
which are collected by the current cycle and one live object $l_4$.
During the collection the live objects become unreachable and are
now floating garbage (e.g. $f_4$ in the second sub-figure). At the
end of the cycle, just before compacting, we have three garbage
cells ($g_1$-$g_3$), three floating garbage cells ($f_4$-$f_6$) and
one live cell $l_7$. Compaction moves the floating garbage and the
live cell to the start of the heap and we end up with four free
cells. The floating garbage will become garbage in the next
collection cycle and we start over with the first sub-figure with
three garbage cells and one live cell.

Figure~\ref{fig:excc} shows one collection cycle of the
concurrent-copy collector. We have two memory spaces: the
\emph{from-space} and the \emph{to-space}. Again we start the
collection cycle with one live cell and three garbage cells left over
from the last cycle. Note that the order of the cells is different
from the previous example. New cells are allocated in the to-space
from the top of the heap, whereas moved cells are allocated from the
bottom of the heap. The second sub-figure shows a snapshot of the
heap during the collection: formerly live object $l_4$ is already
floating garbage $f_4$ and copied into to-space. A new cell $l_5$ is
allocated in the to-space. Before the flip of the two semi-spaces the
from-space contains the three garbage cells ($g_1$-$g_3$) and the
to-space the three floating garbage cells ($f_4$-$f_6$) and one live
cell $l_7$. The last sub-figure shows the heap after the flip: The
from-space contains the three floating cells which will be garbage
cells in the next cycle and the one live cell. The to-space is now
empty.

From this example we see that the necessary heap size for a
mark-compact collector is similar to the heap size for a copying
collector. We also see that the compacting collector has to move
more cells (all floating garbage cells and the live cell) than the
copying collector (just the one cell that is live at the beginning
of the collection).


\subsection{Minimum Heap Size} \label{sec:min:heap}

In this section we show the memory bounds for a mark-compact
collector with a single heap memory and a concurrent-copying
collector with the two spaces \emph{from-space} and \emph{to-space}.

\subsubsection{Mark-Compact} \label{sec:gcsched:mc}

For the mark-compact collector, the heap $H_{MC}$ can be divided into
allocated memory $M$ and free memory $F$
%
\begin{equation}\label{equ:mcheap}
    H_{MC} = M + F = G + \overline{G} + L + F
\end{equation}
%
where $G$ is garbage at the start of the collector cycle that will
be reclaimed by the collector. Objects that become unreachable
during the collection cycle and will not be reclaimed are floating
garbage $\overline{G}$. These objects will be detected in the next
collection cycle. We assume the worst case that all objects that die
during the collection cycle will not be detected and therefore are
floating garbage. $L$ denotes all live,i.e.\  reachable, objects.
$F$ is the remaining free space.

We have to show that we will never run out of memory during a
collection cycle ($F\ge0$). The amount of allocated memory $M$ has to
be less than or equal to the heap size $H_{MC}$

\begin{equation}\label{equ:mcheapmin}
    H_{MC} \ge M = G + \overline{G} + L
\end{equation}


In the following proof the superscript $n$ denotes the collection
cycle. The subscript letters $S$ and $E$ denote the value at the
start and the end of the cycle, respectively.

\begin{lemma}

For a collection cycle the amount of allocated memory $M$ is bounded
by the maximum live data $L_{max}$ at the start of the collection
cycle and two times $A_{max}$, the maximum data allocated by the
mutator during the collection cycle.

\begin{equation}\label{equ:mc:lemma}
    M \le L_{max} + 2 A_{max}
\end{equation}

\end{lemma}

\begin{proof}

During a collection cycle $G$ remains constant. All live data that
becomes unreachable will be floating garbage. Floating garbage
$\overline{G}_E$ at the end of cycle $n$ will be detected (as
garbage $G$) in cycle $n+1$.
%
\begin{equation}\label{equ:flg}
    G^{n+1} = \overline{G}_E^n
\end{equation}
%
The mutator allocates $A$ memory and transforms part of this memory
and part of the live data at the start $L_S$ to floating garbage
$\overline{G}_E$ at the end of the cycle. $L_E$ is the data that is
still reachable at the end of the cycle.
%
\begin{equation}\label{equ:trans}
    L_S + A = L_E + \overline{G}_E
\end{equation}
%
with $A \le A_{max}$ and $L_S \le L_{max}$. A new collection-cycle
start immediately follows the end of the former cycle. Therefore the
live data remains unchanged.
%
\begin{equation}\label{equ:ldata}
    L_S^{n+1} = L_E^{n}
\end{equation}

We will show that (\ref{equ:mc:lemma}) is true for cycle 1. At the
start of the first cycle we have no garbage ($G=0$) and no live data
($L_S=0$). The heap contains only free memory.
%
\begin{equation}\label{equ:mc:start}
    M_S^1 = 0
\end{equation}
%
During the collection cycle the mutator allocates $A^1$ memory. Part
of this memory will be live at the end and the remaining will be
floating garbage.
%
\begin{equation}\label{equ:mc:cyc1:A}
    A^1 = L_E^1 + \overline{G}_E^1
\end{equation}
%
Therefore at the end of the first cycle
%
\begin{align}
\nonumber
    M_E^1 & = L_E^1 + \overline{G}_E^1\\
    M^1   & = A^1
\end{align}
%
As $A^1 \le A_{max}$ (\ref{equ:mc:lemma}) is fulfilled for cycle 1.

Under the assumption that (\ref{equ:mc:lemma}) is true for cycle
$n$, we have to show that (\ref{equ:mc:lemma}) holds for cycle
$n+1$.

\begin{equation}
    M^{n+1} \le L_{max} + 2 A_{max}
\end{equation}

\begin{align}
    M^n & = G^n + \overline{G}_E^n + L_E^n\\
    M^{n+1} & = G^{n+1} + \overline{G}_E^{n+1} + L_E^{n+1}\\
\nonumber
            & = \overline{G}_E^n + L_S^{n+1} + A^{n+1}
                & \mbox{apply (\ref{equ:flg}) and (\ref{equ:trans})}\\
\nonumber
            & = \overline{G}_E^n + L_E^{n} + A^{n+1}
                & \mbox{apply (\ref{equ:ldata})}\\
            & = L_S^{n} + A^n + A^{n+1}
                & \mbox{apply (\ref{equ:trans})}
\end{align}

As $L_S \le L_{max}$, $A^n \le A_{max}$ and $A^{n+1} \le A_{max}$

\begin{equation}
    M^{n+1} \le L_{max} + 2 A_{max}
\end{equation}


\end{proof}

\subsubsection{Concurrent-Copy}

In the following section we will show the memory bounds for a
concurrent-copying collector with the two spaces \emph{from-space}
and \emph{to-space}. We will use the same symbols as in
Section~\ref{sec:gcsched:mc} and denote the maximum allocated memory
in the from-space as $M_{From}$ and the maximum allocated memory in
the to-space as $M_{To}$.

For a copying-collector the heap $H_{CC}$ is divided in two equal
sized spaces $H_{From}$ and $H_{To}$. The amount of allocated memory
$M$ in each semi-space has to be less than or equal to
$\frac{H_{CC}}{2}$
%
\begin{equation}\label{equ:ccheapmin}
    H_{CC} = H_{From} + H_{To} \ge 2M
\end{equation}
%

\begin{lemma}

For a collection cycle, the amount of allocated memory $M$ in each
semi-space is bounded by the maximum live data $L_{max}$ at the start
of the collection cycle and $A_{max}$, the maximum data allocated by
the mutator during the collection cycle.

\begin{equation}\label{equ:cc:lemma}
    M \le L_{max} + A_{max}
\end{equation}

\end{lemma}

\begin{proof}

Floating garbage at the end of cycle $n$ will be detectable garbage
in cycle $n+1$
%
\begin{equation}\label{equ:cc:flg}
    G^{n+1} = \overline{G}_E^n
\end{equation}
%
Live data at the end of cycle $n$ will be the live data at the start
of cycle $n+1$

\begin{equation}\label{equ:cc:ldata}
    L_S^{n+1} = L_E^{n}
\end{equation}


The allocated memory $M_{From}$ in the from-space contains garbage
$G$ and the live data at the start $L_s$.
%
\begin{equation}
    M_{From} = G + L_S
\end{equation}
%
All new objects are allocated in the to-space. Therefore the memory
requirement for the from-space does not change during the collection
cycle. All garbage $G$ remains in the from-space and the to-space
only contains floating garbage and live data.
%
\begin{equation}
    M_{To} = \overline{G} + L
\end{equation}
%
At the start of the collection cycle, the to-space is completely
empty.
%
\begin{equation}
    M_{To\_S} = 0
\end{equation}
%
During the collection cycle all live data is copied into the
to-space and new objects are allocated in the to-space.

\begin{equation}
    M_{To\_E} = L_S + A
\end{equation}

At the end of the collector cycle, the live data from the start $L_S$
and new allocated data $A$ stays either live at the end $L_E$ or
becomes floating garbage $\overline{G}_E$.

\begin{equation}
    L_S + A = L_E + \overline{G}_E
\end{equation}

For the first collection cycle there is no garbage ($G=0$) and no
live data at the start ($L_S=0$), i.e.\ the from-space is empty
($M_{From}^1=0$). The to-space will only contain all allocated data
$A^1$, with $A^1 \le A_{max}$, and therefore (\ref{equ:cc:lemma}) is
true for cycle 1.

Under the assumption that (\ref{equ:cc:lemma}) is true for cycle
$n$, we have to show that (\ref{equ:cc:lemma}) holds for cycle
$n+1$.

\begin{align}
\nonumber
    M_{From}^{n+1} & \le L_{max} + A_{max}\\
    M_{To}^{n+1}   & \le L_{max} + A_{max}
\end{align}



At the start of a collection cycle, the spaces are flipped and the
new to-space is cleared.
%
\begin{align}
\nonumber
    H_{From}^{n+1} & \Leftarrow H_{To}^n\\
    H_{To}^{n+1}   & \Leftarrow \emptyset
\end{align}
%
The from-space:
%
\begin{align}
    M_{From}^{n}  & = G^n + L_S^n\\
    M_{From}^{n+1} & = G^{n+1} + L_S^{n+1}\\
\nonumber
                   & = \overline{G}_E^n + L_E^n\\
                   & = L_S^n + A^n
\end{align}
%
As $L_S \le L_{max}$ and $A^n \le A_{max}$
%
\begin{equation}
    M_{From}^{n+1} \le L_{max} + A_{max}
\end{equation}
%
The to-space:
%
\begin{align}
    M_{To}^{n}     & = \overline{G}_E^n + L_E^n\\
    M_{To}^{n+1}   & = \overline{G}_E^{n+1} + L_E^{n+1}\\
\nonumber
                   & = L_S^{n+1} + A^{n+1}\\
                   & = L_E^{n} + A^{n+1}
\end{align}
%
% end
%
As $L_E \le L_{max}$ and $A^{n+1} \le A_{max}$
%
\begin{equation}
    M_{To}^{n+1} \le L_{max} + A_{max}
\end{equation}
%
\end{proof}

From this result we can see that the dynamic memory consumption for a
mark-compact collector is similar to a concurrent-copy collector.
This is contrary to the common belief that a copy collector needs the
double amount of memory.

We have seen that the double-memory argument against a copying
collector does not hold for an incremental real-time collector. We
need double the memory of the allocated data during a collection
cycle in either case. The advantage of the copying collector over a
compacting one is that newly allocated data are placed in the
to-space and does not need to be copied. The compacting collector
moves all newly created data (that is mostly floating garbage) at the
compaction phase.

\subsection{Garbage Collection Period}

GC work is inherently periodic. After finishing one round of
collection the GC starts over. The important question is which is
the \emph{maximum} period the GC can be run so that the application
will never run out of memory. Scheduling the GC at a shorter period
does not hurt but decreases utilization.

In the following, we derive the maximum collector period that
guarantees that we will not run out of memory. The maximum period
$T_{GC}$ of the collector depends on $L_{max}$ and $A_{max}$ for
which safe estimates are needed.

We assume that the mutator allocates all memory at the start of the
period and the memory becomes garbage at the end. In other words the
memory is live for one period. This is the worst case, but very
common.

%\subsubsection{Single Mutator Thread}
%
%First we give an upper bound for the collector cycle time for a
%single mutator thread.
%
%\begin{lemma}
%
%For a single mutator thread with period $T_M$ that allocates memory
%``$a$" each period, the maximum collector period $T_{GC}$ that
%guarantees that we will not run out of memory is
%
%\begin{align}\label{sth:mc:lemma}
%    T_{GC} & \le T_M\left\lfloor\frac{H_{MC}-a}{2a}\right\rfloor\\
%    \label{sth:cc:lemma}
%    T_{GC} & \le T_M\left\lfloor\frac{H_{CC}-2a}{2a}\right\rfloor
%\end{align}
%
%\end{lemma}
%
%\begin{proof}
%The maximum live data referenced by a single mutator is the maximum
%data allocated by the mutator in a single cycle.
%\begin{equation}
%    L_{max} = a
%\end{equation}
%A single mutator allocates $a$ memory during the period $T_M$.
%Therefore the maximum allocation during the collector period
%$T_{GC}$ is
%%
%\begin{equation}
%    A_{max} = a \left\lceil\frac{T_{GC}}{T_{M}}\right\rceil
%\end{equation}
%%
%Using equations (\ref{equ:mcheapmin}) and (\ref{equ:mc:lemma}) we
%get the minimum heap size $H_{MC}$ for a mark-compact collector
%%
%\begin{align}
%\nonumber
%    H_{MC} & \ge L_{max} + 2 A_{max}\\
%    H_{MC} & \ge a \left(1 + 2
%    \left\lceil\frac{T_{GC}}{T_{M}}\right\rceil\right)
%\end{align}
%%
%Equations (\ref{equ:ccheapmin}) and (\ref{equ:cc:lemma}) result in
%the minimum heap size $H_{CC}$, containing both semi-spaces, for the
%concurrent-copy collector
%%
%\begin{align}
%\nonumber
%    H_{CC} & \ge 2(L_{max} + A_{max})\\
%    H_{CC} & \ge 2a \left(1 +
%    \left\lceil\frac{T_{GC}}{T_{M}}\right\rceil\right)
%\end{align}
%%
%The ceiling function covers the worst-case schedule between the
%collector thread and the mutator thread. We are interested in the
%maximum collector period $T_{GC}$ with a given heap size $H_{MC}$ or
%$H_{CC}$
%%
%\begin{align}
%    \label{equ:mc:ceil}
%    \left\lceil\frac{T_{GC}}{T_{M}}\right\rceil &
%    \le \frac{H_{MC}-a}{2a}
%\end{align}
%%
%\begin{align}
%    \label{equ:cc:ceil}
%    \left\lceil\frac{T_{GC}}{T_{M}}\right\rceil &
%    \le \frac{H_{CC}-2a}{2a}
%\end{align}
%%
%The maximum quotient ($\frac{T_{GC}}{T_{M}}$) that fulfills
%(\ref{equ:mc:ceil}) or (\ref{equ:cc:ceil}) is an integer $n$. $n$ is
%the largest integer that is less than or equal the right side of
%(\ref{equ:mc:ceil}) or (\ref{equ:cc:ceil}). Therefore we get for the
%mark-compact collector
%%
%\begin{align}
%    \label{equ:mc:floor}
%    \frac{T_{GC}}{T_{M}} &
%    \le \left\lfloor\frac{H_{MC}-a}{2a}\right\rfloor
%\end{align}
%%
%\begin{equation}
%    \Rightarrow T_{GC} \le T_M \left\lfloor\frac{H_{MC}-a}{2a}\right\rfloor
%\end{equation}
%%
%and for the concurrent-copy collector
%%
%\begin{align}
%    \label{equ:cc:floor}
%    \frac{T_{GC}}{T_{M}} &
%    \le \left\lfloor\frac{H_{CC}-2a}{2a}\right\rfloor
%\end{align}
%%
%\begin{equation}
%    \Rightarrow T_{GC} \le T_M \left\lfloor\frac{H_{CC}-2a}{2a}\right\rfloor
%\end{equation}
%
%\end{proof}
%
%\subsubsection{Several Mutator Threads}

In this section the upper bound of the period for the collector
thread is given for $n$ independent mutator threads.

\begin{theorem}
\label{sch:theorem}

For ``$n$" mutator threads with period $T_i$ where each thread
allocates $a_i$ memory each period, the maximum collector period
$T_{GC}$ that guarantees that we will not run out of memory is

\begin{align}\label{nth:mc:theorem}
    T_{GC} & \le \frac{H_{MC}-3\sum_{i=1}^{n} a_i}{2\sum_{i=1}^{n} \frac{a_i}{Ti}}\\
    \label{nth:cc:theorem}
    T_{GC} & \le \frac{H_{CC}-4\sum_{i=1}^{n} a_i}{2\sum_{i=1}^{n}
    \frac{a_i}{Ti}}
\end{align}

\end{theorem}

\begin{proof}

For $n$ mutator threads with periods $T_i$ and allocations $a_i$
during each period the values for $L_{max}$ and $A_{max}$ are
%
\begin{align}\label{nth:lmax}
    L_{max} & = \sum_{i=1}^{n} a_i\\
    A_{max} & = \sum_{i=1}^{n}
    \left\lceil\frac{T_{GC}}{T_i}\right\rceil a_i
\end{align}
%
The ceiling function for $A_{max}$ covers the individual worst cases
for the thread schedule and cannot be solved analytically. Therefore
we use a conservative estimation $A^{'}_{max}$ instead of $A_{max}$.
%
\begin{equation}
    A^{'}_{max} = \sum_{i=1}^{n} \left(\frac{T_{GC}}{T_i}+1\right)
    a_i
    \ge \sum_{i=1}^{n} \left\lceil\frac{T_{GC}}{T_i}\right\rceil a_i
\end{equation}
%
From (\ref{equ:mcheapmin}) and (\ref{equ:mc:lemma}) we get the
minimum heap size for a mark-compact collector
\begin{align}
\nonumber
    \label{equ:mc:mthreads:exact}
    H_{MC} & \ge L_{max} + 2 A_{max}\\
           & \ge \sum_{i=1}^{n} a_i + 2 \sum_{i=1}^{n}
             \left\lceil\frac{T_{GC}}{T_i}\right\rceil a_i
\end{align}
%
For a given heap size $H_{MC}$ we get the conservative upper bound
of the maximum collector period $T_{GC}$
%
\footnote{It has to be noted that this is a conservative value for
the maximum collector period $T_{GC}$. The maximum value
$T_{GC_{max}}$ that fulfills (\ref{equ:mc:mthreads:exact}) is in the
interval
%
\begin{equation}
\nonumber
    \left(\frac{H_{MC} - 3\sum_{i=1}^{n} a_i} {2\sum_{i=1}^{n}
        \frac{a_i}{T_i}},
        \frac{H_{MC} - \sum_{i=1}^{n} a_i} {2\sum_{i=1}^{n}
        \frac{a_i}{T_i}}\right)
\end{equation}
%
and can be found by an iterative search.}
%
\begin{align}
\nonumber
        2 A^{'}_{max} & \le H_{MC}-L_{max}
\\
        2\sum_{i=1}^{n} \left(\frac{T_{GC}}{T_i}+1\right) a_i
        & \le H_{MC}-L_{max}
\\
        T_{GC}
        & \le \frac{H_{MC}-L_{max} - 2\sum_{i=1}^{n} a_i}
        {2\sum_{i=1}^{n} \frac{a_i}{T_i}}
\end{align}
%
\begin{equation}
    \Rightarrow T_{GC} \le \frac{H_{MC}-3\sum_{i=1}^{n} a_i}{2\sum_{i=1}^{n} \frac{a_i}{Ti}}
\end{equation}
%
Equations (\ref{equ:ccheapmin}) and (\ref{equ:cc:lemma}) result in
the minimum heap size $H_{CC}$, containing both semi-spaces, for the
concurrent-copy collector
\begin{align} \nonumber
    H_{CC} & \ge 2 L_{max} + 2 A_{max}\\
           & \ge 2\sum_{i=1}^{n} a_i + 2\sum_{i=1}^{n}
              \left\lceil\frac{T_{GC}}{T_i}\right\rceil a_i
\end{align}
%
For a given heap size $H_{CC}$ we get the conservative upper bound
of the maximum collector period $T_{GC}$
\begin{align}
\nonumber
        2 A^{'}_{max} & \le H_{CC} - 2L_{max}
\\
        2\sum_{i=1}^{n} \left(\frac{T_{GC}}{T_i}+1\right) a_i
        & \le H_{CC} - 2L_{max}
\\
        T_{GC}
        & \le \frac{H_{CC} - 2L_{max} - 2\sum_{i=1}^{n} a_i}
        {2\sum_{i=1}^{n} \frac{a_i}{T_i}}
\end{align}
%
\begin{equation}
    \Rightarrow T_{GC} \le \frac{H_{CC}-4\sum_{i=1}^{n} a_i}{2\sum_{i=1}^{n} \frac{a_i}{Ti}}
\end{equation}
\end{proof}

\subsubsection{Producer/Consumer Threads} \label{sec:prod:cons}

So far we have only considered threads that do not share objects for
communication. This execution model is even more restrictive than
the RTSJ scoped memories that can be shared between threads. In this
section we discuss how our GC scheduling can be extended to account
for threads that share objects.

Object sharing is usually done by a producer and a consumer thread.
I.e.,\ one thread allocates the objects and stores references to
those objects in a way that they can be accessed by the other
thread. This other thread, the consumer, is in charge to \emph{free}
those objects after use.

An example of this sharing is a device driver thread that
periodically collects data and puts them into a list for further
processing. The consumer thread, with a longer period, takes all
available data from the list at the start of the period, processes
the data, and removes them from the list. During the data processing,
new data can be added by the producer. Note that in this case the
list will probably never be completely empty. This typical case
cannot be implemented by an RTSJ shared scoped memory. There would be
no point in the execution where the shared memory will be empty and
can get recycled.

The question now is how much data will be alive in the worst case.
We denote $T_p$ as the period of the producer thread $\tau_p$ and
$T_c$ as the period of the consumer thread $\tau_c$. $\tau_p$
allocates $a_p$ memory each period. During one period of the
consumer $\tau_c$ the producer $\tau_p$ allocates
\begin{equation*}
    \left\lceil\frac{T_c}{T_p}\right\rceil a_p
\end{equation*}
memory. The worst case is that $\tau_c$ takes over all objects at
the start of the period and frees them at the end. Therefore the
maximum amount of live data for this producer/consumer combination
is
\begin{equation*}
    \left\lceil\frac{2 T_c}{T_p}\right\rceil a_p
\end{equation*}
To incorporate this extended lifetime of objects we introduce a
lifetime factor $l_i$ which is
\begin{equation}\label{equ:liv:fac}
    l_i = \left\{
    \begin{array}{ll}
    1 & :\ \mbox{for normal threads}\\
    \left\lceil\frac{2 T_c}{T_i}\right\rceil & :
    \ \mbox{for producer}\ \tau_i\ \mbox{and associated consumer}\ \tau_c
    \end{array}
    \right.
\end{equation}
and extend $L_{max}$ from (\ref{nth:lmax}) to
\begin{equation}
    L_{max} = \sum_{i=1}^{n} a_i l_i
\end{equation}
The maximum amount of memory $A_{max}$ that is allocated during one
collection cycle is not changed due to the \emph{freeing} in a
different thread and therefore remains unchanged.

The resulting equations for the maximum collector period are
\begin{equation}
    T_{GC} \le \frac{H_{MC}-\sum_{i=1}^{n} a_i l_i - 2\sum_{i=1}^{n} a_i}{2\sum_{i=1}^{n} \frac{a_i}{Ti}}
\end{equation}
and
\begin{equation}
    T_{GC} \le \frac{H_{CC}-2\sum_{i=1}^{n} a_i l_i - 2\sum_{i=1}^{n} a_i}{2\sum_{i=1}^{n}
    \frac{a_i}{Ti}}
\end{equation}



% to conservative stuff:

%During $T_C$ $\tau_p$ allocates maximal
%$\left\lceil\frac{T_C}{T_P}\right\rceil$ objects. This amount is
%also the maximum amount of data $\tau_c$ will process in one period.
%Therefore the maximum amount of live objects during $T_c$ is:
%\begin{equation}\label{equ:prod:consume}
%    a_c = 2\left\lceil\frac{T_C}{T_P}\right\rceil a_p
%\end{equation}
%For a safe estimate of $T_{GC}$ we will assign the shared data to
%$\tau_c$ and use $a_c$ as the amount of allocated memory in
%Theorem~\ref{sch:theorem}. In this case the allocated memory for
%$\tau_p$ will be set to zero for the calculation of $T_{GC}$.
%



\subsubsection{Static Objects}

The discussion about the collector cycle time assumes that all live
data is produced by the periodic application threads and the maximum
lifetime is one period. However, in the general case we have also
live data that is allocated in the initialization phase of the
real-time application and stays alive until the application ends. We
incorporate this value by including this static live memory $L_s$ in
$L_{max}$
\begin{equation}
    L_{max} = L_s + \sum_{i=1}^{n} a_i l_i
\end{equation}


A mark-compact collector moves all static data to the bottom of the
heap in the first and second\footnote{A second cycle is necessary as
this static data can get intermixed by floating garbage from the
first collector cycle.} collection cycle after the allocation. It
does not have to compact these data during the following collection
cycles in the mission phase. The concurrent-copy collector moves
these static data in each collection cycle. Furthermore, the memory
demand for the concurrent copy is increased by the double amount of
the static data (compared to the single amount in the mark-compact
collector)\footnote{Or the collector period gets shortened.}.

The SCJ application model with an initialization and a mission phase
can reduce the amount of live data that needs to be copied (see
Section~\ref{sec:scj:simple}).


\subsubsection{Object Lifetime} \label{sec:wc:live}

Listing~\ref{lst:bcnode} shows an example of a periodic thread that
allocates an object in the main loop and the resulting bytecodes.

\begin{lstlisting}[float=t, caption={Example periodic thread and the corresponding Java bytecodes},
label=lst:bcnode]
    public void run() {

        for (;;) {
            Node n = new Node();
            work(n);
            waitForNextPeriod();
        }
    }

public void run();
  Code:
   0:   new #20; //class Node
   3:   dup
   4:   invokespecial   #22; //"<init>":()V
   7:   astore_1
   8:   aload_1
   9:   invokestatic    #26; //work:(Node)V
   12:  aload_0
   13:  invokevirtual   #30; //wFNP:()Z
   16:  pop
   17:  goto    0
\end{lstlisting}

There is a time between allocation of \code{Node} and the assignment
to \code{n} where a reference to the former \code{Node} (from the
former cycle) and the new \code{Node} (on the operand stack) is live.
To handle this issue we can either change the values of $L_{max}$ and
$A_{max}$ to accommodate this additional object or change the
top-level code of the periodic work to explicitly assign a
null-pointer to the local variable \code{n} as it can be seen in
Listing~\ref{lst:excode} from the evaluation section. Programming
against the SCJ profile avoids this issues (see
Section~\ref{sec:scj:simple}).

However, this null pointer assignment is only necessary at the
top-level method that invokes \code{waitForNextPeriod} and is
therefore not as complex as explicit freeing of objects. Objects
that are created inside \code{work} in our example do not need to be
\emph{freed} in this way as the reference to the object gets
\emph{lost} on return from the method.

\section{SCJ Simplifications} \label{sec:scj:simple}

The restrictions of the computational model for safety critical Java
allow for optimizations of the GC. We can avoid atomic stack
scanning for roots and do not have to deal with exact pointer
finding. Static objects, which would belong into immortal memory in
the RTSJ, can be detected by a special GC cycle at transition to the
mission phase. We can treat those objects specially and do not need
to collect them during the mission phase. This static memory area is
automatically sized.

It has to be noted that our proposal is extending JSR 302. Clearly,
adding RTGC to SCJ reduces the importance of scopes and would likely
relegate them to the small subset of applications where fast
deallocation is crucial. Discussing the interaction between scoped
memory  and RTGC is beyond the scope of this chapter.

\subsection{Simple Root Scanning}

Thread stack scanning is usually performed atomically. Scanning of
the thread stacks with a snapshot-at-beginning write barrier
\cite{gc:yuasa90} allows optimization of the write barriers to
consider only field access (\code{putfield} and \code{putstatic}) and
array access. Reference manipulation in locals and on the operand
stack can be ignored for a write barrier. However, this optimization
comes at the cost of a possible large blocking time due to the
atomicity of stack scanning.

A subtle difference between the RTSJ and the SCJ definition is the
possibility to use local variables within \code{run()} (see example
in Figure~\ref{lst:rtsj:per}). Although handy for the programmer to
preserve state information in locals,\footnote{Using multiple
\code{wFNP()} invocations for local mode changes can also come handy.
The author has used this fact heavily in the implementation of a
modem/PPP protocol stack.}
%(see Section~\ref{chap:ejip}) in footnote .... will there be such a section?
GC implementation can greatly benefit from \emph{not} having
reference values on the thread stack when the thread suspenses
execution.

If the GC thread has the lowest priority and there is no blocking
library function that can suspend a real-time thread, then the GC
thread will only run when all real-time threads are waiting for
their next period -- and this waiting is performed after the return
from the \code{run()} method.  In that case the other thread stacks
are completely \emph{empty}. We do not need to scan them for roots
as the only roots are the references in static (class) variables.

For a real-time GC root scanning has to be exact. With conservative
stack scanning, where a primitive value is treated as a pointer,
possible large data structures can be kept alive artificially. To
implement exact stack scanning we need the information of the stack
layout for each possible GC preemption point. For a high-priority GC
this point can be at each bytecode (or at each machine instruction
for compiling Java). The auxiliary data structure to capture the
stack layout (and information which machine register will hold a
reference for compiled Java) can get quite large~\cite{jop:gcroots}
or require additional effort to compute.

With a low-priority GC and the RTSJ model of periodic thread coding
with \code{wFNP()} the number of GC preemption points is decreased
dramatically. When the GC runs all threads will be in \code{wFNP()}.
Only the stack information for those places in the code have to be
available. It is also assumed that \code{wFNP()} is not invoked very
deep in the call hierarchy. Therefore, the stack high will be low
and the resulting blocking time short.

As mentioned before, the SCJ style periodic thread model results in
an empty stack at GC runtime. As a consequence we do not have to
deal with exact stack scanning and need no additional information
about the stack layout.

\subsection{Static Memory} \label{sec:static:mem}

A SCJ copying collector will perform best when all live data is
produced by periodic threads and the maximum lifetime of a newly
allocated object is one period.  However, some data structures
allocated in the initialization phase stay alive for the whole
application lifetime.  In an RTSJ application this data would be
allocated in immortal memory.  With a real-time GC there is no notion
of {immortal} memory, instead we will use the term \emph{static}
memory. Without special treatment, a copying collector will move this
data at each GC cycle. Furthermore, the memory demand for the
collector increases by the amount of the static data.


As those static objects (mostly) live \emph{forever}, we propose a
solution similar to the immortal memory of the RTSJ. All data
allocated during the initialization phase (where no application
threads are scheduled) is considered potentially static. As part of
the transition to the mission phase a \emph{special} collection cycle
in a stop-the-world fashion is performed. Objects that are still
alive after this cycle are assumed to live forever and make up the
\emph{static} memory area. The remaining memory is used for the
garbage collected heap.

%% Jan's text
%The initialization phase and the transition to the mission phase are
%usually not time critical. However, there are classes of
%applications for which startup is critical, for example in avionics
%systems it is essential for the system to come up promptly after a
%momentary power failure. There are two potential solutions, one
%could trade initialization time GC against more copy work during the
%mission phase, or, as an alternative, one could push most of the
%initialization time work to virtual machine build-time as is done in
%Ovm~\cite{ovm:tecs:07}.

This static memory will still be scanned by the collector to find
references into the heap but it is not collected. The main
differences between our static memory and the immortal memory of the
RTSJ are:

This static live data will still be scanned by the collector to find
references into the heap but it is not collected. The main
differences between our immortal memory and the memory areas of the
RTSJ are:
\begin{itemize}
    \item The choice of allocation context is implicit. There is
        no need to specify where an object must be allocated. We
        do not have to state explicitly which data belongs to the
        application life-time data. This information is
        implicitly gathered by the start-mission transition.
    \item References from the static memory to the garbage
        collected heap are allowed contrary to the fact in the
        RTSJ that references to scoped memories, that have to be
        used for dynamic memory management without a GC, are not
        allowed from immortal memory.
\end{itemize}

The second fact greatly simplifies communication between threads. For
a typical producer/consumer configuration the container for the
shared data is allocated in immortal memory and the actual data in
the garbage collected heap. With this \emph{immortal} memory solution
the actual $L_{max}$ only contains allocated memory from the periodic
threads.


\section{Implementation}

%% rewrite in RTS version

The collector for JOP is an incremental collector
\cite{jop:rtgc_sched, jop:scjgc} based on the copy collector by
Cheney \cite{gc:cheney70} and the incremental version by Baker
\cite{gc:baker78}. To avoid the expensive read barrier in Baker's
collector all object copies are performed concurrently by the
collector. The collector is concurrent and resembles the collectors
presented by Steele~\cite{gc:steele75} and
Dijkstra~et~al.~\cite{gc:dijkstra78}. Therefore we call it the
\emph{concurrent-copy} collector.

The collector and the mutator are synchronized by two barriers. A
Brooks-style~\cite{gc:broo84} forwarding directs the access to the
object either into tospace or fromspace. The forwarding pointer is
kept in a separate handle area as proposed in \cite{gc:nort87}. The
separate handle area reduces the space overheads as only one pointer
is needed for both object copies. Furthermore, the indirection
pointer does not need to be copied. The handle also contains other
object related data, such as type information, and the mark list. The
objects in the heap only contain the fields and no object header.

The second synchronization barrier is a \emph{snapshot-at-beginning}
write-barrier \cite{gc:yuasa90}. A snapshot-at-beginning
write-barrier synchronizes the mutator with the collector on a
reference store into a static field, an object field, or an array.


% We have implemented the concurrent-copy GC on the Java processor JOP
% \cite{jop:thesis, jop:jnl:jsa2007}.
The whole collector, the \code{new} operation, and the write barriers
are implemented in Java (with the help of native functions for direct
memory access). The object copy operation is implemented in hardware
and can be interrupted by mutator threads after each word copied
\cite{jop:nbobjcopy:jtres2008}. The copy unit redirects the access to
the object under copy, depending on the accessed field, either to the
original or the new version of the object.

Although we show the implementation on a Java processor, the GC is
not JOP specific and can also be implemented on a conventional
processor.

\subsection{Heap Layout}

Figure~\ref{fig:handles} shows a symbolic representation of the heap
layout with the handle area and two semi-spaces, \emph{fromspace} and
\emph{tospace}. Not shown in this figure is the memory region for
runtime constants, such as class information or string constants.
This memory region, although logically part of the heap, is neither
scanned, nor copied by the GC. This constant area contains its own
handles and all references into this area are ignored by the GC.

\begin{figure*}[t]
  \centering
  \includegraphics{jvm/handles}
  \caption{Heap layout with the handle area}\label{fig:handles}
\end{figure*}

To simplify object move by the collector, all objects are accessed
with one indirection, called the handle. The handle also contains
auxiliary object data structures, such as a pointer to the method
table or the array length. Instead of Baker's read barrier we have an
additional mark stack which is a threaded list within the handle
structure. An additional field (as shown in Figure~\ref{fig:handles})
in the handle structure is used for a free list and a use list of
handles.

The indirection through a handle, although a very light-weight read
barrier, is usually still considered as a high overhead. Metronome
\cite{gc:bacon03} uses a forwarding pointer as part of the object and
performs forwarding \emph{eagerly}. Once the pointer is forwarded,
subsequent uses of the reference can be performed on the direct
pointer until a GC preemption point. This optimization is performed
by the compiler.

JOP uses a hardware based optimization for this indirection
\cite{jop:oohw:jtres2007}. The indirection is unconditionally
performed in the memory access unit. Furthermore, null pointer checks
and array bounds checks are done in parallel to this indirection.

There are two additional benefits from an explicit handle area instead of a
forwarding pointer: (a) access to the method table or array size needs no
indirection, and (b) the forwarding pointer and the auxiliary data
structures do not need to be copied by the GC.

The fixed handle area is not subject to fragmentation as all handles
have the same size and are recycled at a sweep phase with a simple
free list. However, the reserved space has to be sized (or the GC
period adapted) for the maximum number of objects that are live or
are floating garbage.


\subsection{The Collector}

The collector can operate in two modes: (1) as stop-the-world
collector triggered on allocation when the heap is full, or (2) as
concurrent real-time collector running in its own thread.

The real-time collector is scheduled periodically at the lowest
priority and within each period it performs the following steps:
\begin{description}
    \item[Flip] An atomic flip exchanges the roles of tospace and
    fromspace.
    \item[Mark roots] All static references are pushed onto the mark
    stack. Only a single push operation needs to be atomic. As the
    thread stacks are empty we do not need an atomic scan of thread
    stacks.
    \item[Mark and copy] An object is popped from the mark stack,
    all referenced objects, which are still white, are pushed on the
    mark stack, the object is copied to tospace and the handle
    pointer is updated.
    \item[Sweep handles] All handles in the use list are checked if
    they still point into tospace (black objects) or can be added to
    the handle free list.
    \item[Clear fromspace] At the end of the collector work the
    fromspace that contains only white objects is initialized with
    zero. Objects allocated in that space (after the next flip) are
    already initialized and allocation can be performed in constant
    time.
\end{description}
%
To reduce blocking time, a hardware unit performs copies of objects
and arrays in an interruptible fashion, and records the copy position
on an interrupt. On an object or array access the hardware knows
whether the access should go to the already copied part in the
tospace or in the not yet copied part in the fromspace. It has to be
noted that splitting larger arrays into smaller chunks, as done in
Metronome~\cite{gc:bacon03} and in the GC for the
JamaicaVM~~\cite{gc:siebert:phd}, is a software option to reduce the
blocking time.

The collector has two modes of operation: one for the initialization
phase and one for the mission phase. At the initialization phase it
operates in a stop-the-world fashion and gets invoked when a memory
request cannot be satisfied. In this mode the collector scans the
stack of the single thread conservatively. It has to be noted that
each reference points into the handle area and not to an arbitrary
position in the heap. This information is considered by the GC to
distinguish pointers from primitives. Therefore the chance to keep an
object artificially alive is low.

%% This is RTS paper stuff and should go away in the handbook
As part of the mission start one stop-the-world cycle is performed to
clean up the heap from garbage generated at initialization. From that
point on the GC runs in concurrent mode in its own thread and omits
scanning of the thread stacks.

\subsubsection{Implementation Code Snippets}

This sections shows the important code fragments of the
implementation. As can be seen, the implementation is quite short.

\paragraph{Flip} involves manipulation of a few pointers and changes
the meaning of black (\code{toSpace}) and white.
%
\begin{lstlisting}
    synchronized (mutex) {
        useA = !useA;
        if (useA) {
            copyPtr = heapStartA;
            fromSpace = heapStartB;
            toSpace = heapStartA;
        } else {
            copyPtr = heapStartB;
            fromSpace = heapStartA;
            toSpace = heapStartB;
        }
        allocPtr = copyPtr+semi_size;
    }
\end{lstlisting}


\paragraph{Root Marking} When the GC runs in concurrent mode only
the static reference fields form the root set and are scanned. The
stop-the-world mode of the GC also scans all stacks from all
threads.
%
\begin{lstlisting}
    int addr = Native.rdMem(addrStaticRefs);
    int cnt = Native.rdMem(addrStaticRefs+1);
    for (i=0; i<cnt; ++i) {
        push(Native.rdMem(addr+i));
    }
\end{lstlisting}

\paragraph{Push} All gray objects are pushed on a gray stack. The
gray stack is a list threaded within the handle structure.
%
\begin{lstlisting}
    if (Native.rdMem(ref+OFF_GREY)!=0) {
        return;
    }
    if (Native.rdMem(ref+OFF_GREY)==0) {
        // pointer to former gray list head
        Native.wrMem(grayList, ref+OFF_GREY);
        grayList = ref;
    }
\end{lstlisting}

\paragraph{Mark and Copy} The following code snippet shows the
central GC loop.
%
\begin{lstlisting}
    for (;;) {

        // pop one object from the gray list
        synchronized (mutex) {
            ref = grayList;
            if (ref==GREY_END) {
                break;
            }
            grayList = Native.rdMem(ref+OFF_GREY);
            // mark as not in list
            Native.wrMem(0, ref+OFF_GREY);
        }

        // push all childs
        // get pointer to object
        int addr = Native.rdMem(ref);
        int flags = Native.rdMem(ref+OFF_TYPE);
        if (flags==IS_REFARR) {
            // is an array of references
            int size = Native.rdMem(ref+OFF_MTAB_ALEN);
            for (i=0; i<size; ++i) {
                push(Native.rdMem(addr+i));
            }
        } else if (flags==IS_OBJ){
            // its a plain object
            // get pointer to method table
            flags = Native.rdMem(ref+OFF_MTAB_ALEN);
            // get real flags
            flags = Native.rdMem(flags+MTAB2GC_INFO);
            for (i=0; flags!=0; ++i) {
                if ((flags&1)!=0) {
                    push(Native.rdMem(addr+i));
                }
                flags >>>= 1;
            }
        }

        // now copy it - color it BLACK
        int size = Native.rdMem(ref+OFF_SIZE);
        synchronized (mutex) {
            // update object pointer to the new location
            Native.wrMem(copyPtr, ref+OFF_PTR);
            // set it BLACK
            Native.wrMem(toSpace, ref+OFF_SPACE);
            // copy it
            for (i=0; i<size; ++i) {
                Native.wrMem(Native.rdMem(addr+i), copyPtr+i);
            }
            copyPtr += size;
        }
    }
\end{lstlisting}

\paragraph{Sweep Handles} At the end of the mark and copy phase the
handle area is swept to find all unused handles (the one that still
point into \code{fromSpace}) and add them to the free list.
%
\begin{lstlisting}
    synchronized (mutex) {
        ref = useList;      // get start of the list
        useList = 0;        // new uselist starts empty
    }

    while (ref!=0) {

        int next = Native.rdMem(ref+OFF_NEXT);
        // a BLACK one
        if (Native.rdMem(ref+OFF_SPACE)==toSpace) {
            // add to used list
            synchronized (mutex) {
                Native.wrMem(useList, ref+OFF_NEXT);
                useList = ref;
            }
        // a WHITE one
        } else {
            // add to free list
            synchronized (mutex) {
                Native.wrMem(freeList, ref+OFF_NEXT);
                freeList = ref;
                Native.wrMem(0, ref+OFF_PTR);
            }
        }
        ref = next;
    }
\end{lstlisting}

\paragraph{Clear Fromspace} The last step of the GC clears the
fromspace to provide a constant time allocation after the next flip.
%
\begin{lstlisting}
        for (int i=fromSpace; i<fromSpace+semi_size; ++i) {
            Native.wrMem(0, i);
        }
\end{lstlisting}

\subsection{The Mutator}

The coordination between the mutator and the collector is performed
within the \code{new} and \code{newarray} bytecodes and within write
barriers for JVM bytecodes \code{putfield} and \code{putstatic} for
reference fields, and bytecode \code{aastore}. The field access
bytecodes are substituted at application link time (run of
\code{JOPizer}). Only write accesses to reference fields are
substituted by special versions of the bytecodes
(\code{putfield\_ref} and \code{putstatic\_ref}). Therefore, the
write barrier code is only executed on reference write access.

\subsubsection{Allocation}

Objects are allocated black (in tospace). In non real-time collectors
it is more common to allocate objects white. It is argued
\cite{gc:dijkstra78} that objects die young and the chances are high
that the GC never needs to touch them. However, in the worst case no
object that is created and becomes garbage during the GC cycle can be
reclaimed. Those floating garbage will be reclaimed in the next GC
cycle. Therefore, we do not benefit from the white allocation
optimization in a real-time GC. Allocating a new object black has the
benefit that those objects do not need to be copied. The same
argument applies to the chosen write barrier. The code in
Listing~\ref{lst:new} shows the simple implementation of bytecode
\code{new}:

\begin{lstlisting}[float=t, caption={Implementation of bytecode \code{new} in JOP�s JVM},
label=lst:new]

synchronized (GC.mutex) {
    // we allocate from the upper part
    allocPtr -= size;
    ref = getHandle(size);
    // mark as object
    Native.wrMem(IS_OBJ, ref+OFF_TYPE);
    // pointer to method table in the handle
    Native.wrMem(cons+CLASS_HEADR, ref+OFF_MTAB_ALEN);
}
\end{lstlisting}

As the old fromspace is cleared by the GC, the new object is already
initialized and \code{new} executes in constant time. The methods
\code{Native.rdMem()} and \code{Native.wrMem()} provide direct access
to the main memory. Only those two native methods are necessary for
an implementation of a GC in pure Java.

\subsubsection{Write Barriers}

For a concurrent (incremental) GC some coordination between the
collector and the mutator are necessary. The usual solution is a
write barrier in the mutator to not foil the collector. According to
\cite{gc:wils94} GC concurrent algorithms can be categorized into:

\begin{description}
    \item[Snapshot-at-beginning] Keep the object graph as it was at
    the the GC start
    \begin{itemize}
        \item Save the to-be-overwritten reference
        \item More conservative -- not an issue for RTs as worst case
        counts
        \item Allocate black
        \item New objects (e.g.\ new stack frames) do not need a
        write barrier
        \item Optimization: with atomic root scan of the thread
        stacks no write barrier is necessary for locals and the JVM
        stack
    \end{itemize}
    \item[Incremental update] \emph{Help} the GC by doing some collection
    work in the mutator
    \begin{itemize}
        \item Preserve strong tri-color invariant (no pointer from
        black to white objects)
        \item On black to white shade the white object (shade the
        black is unusual)
        \item Allocate black (in contrast to \cite{gc:dijkstra78})
        \item Needs write barriers for locals and manipulation on
        the stack
        \item Less conservative than snapshot-at-beginning
    \end{itemize}
\end{description}

The usual choice is snapshot-at-beginning with atomic root scan of
all thread stacks to avoid write barriers on locals. Assume the
following assignment of a reference:
\begin{lstlisting}
    o.r = ref;
\end{lstlisting}
There are three references involved that can be manipulated:
\begin{itemize}
    \item The old value of \code{o.r}
    \item The new value \code{ref}
    \item The object \code{o}
\end{itemize}
The three possible write barriers are:
\begin{enumerate}
    \item
Snapshot-at-beginning/weak tri-color invariant:
\begin{lstlisting}
    if (white(o.r)) markGrey(o.r);
    o.r = ref;
\end{lstlisting}
    \item
Incremental/strong tri-color invariant with push forward
\begin{lstlisting}
    if (black(o) && white(ref)) markGrey(ref);
    o.r = ref;
\end{lstlisting}
This barrier can be optimized to only check if \code{ref} is white.

    \item
Incremental/strong tri-color invariant with push back
\begin{lstlisting}
    if (black(o) && white(ref)) markGrey(o);
    o.r = ref;
\end{lstlisting}

\end{enumerate}

We have no stack roots when the collector runs. Therefore we could
use the incremental write barrier for object fields only. However,
for the worst case all floating garbage will not be found by the GC
in the current cycle. Therefore, we use the snapshot-at-begin write
barrier in our implementation.

A snapshot-at-beginning write-barrier synchronizes the mutator with
the collector on a reference store into a static field, an object
field, or an array. The \emph{to be overwritten} field is shaded gray
as shown in Listing~\ref{lst:barrier}. An object is shaded gray by
pushing the reference of the object onto the mark
stack.\footnote{Although the GC is a copying collector a mark stack
is needed to perform the object copy in the GC thread and not by the
mutator.} Further scanning and copying into tospace -- coloring it
black -- is left to the GC thread. One field in the handle area is
used to implement the mark stack as a simple linked list.
Listing~\ref{lst:barrier} shows the implementation of \code{putfield}
for reference fields.

\begin{lstlisting}[float=t, caption={Snapshot-at-beginning write-barrier in JOP�s JVM},
label=lst:barrier]

private static void f_putfield_ref(int ref, int value, int index) {

    synchronized (GC.mutex) {

        // snapshot-at-beginning barrier
        int oldVal = Native.getField(ref, index);
        // Is it white?
        if (oldVal != 0
            && Native.rdMem(oldVal+GC.OFF_SPACE) != GC.toSpace) {
            // Mark grey
            GC.push(oldVal)
        }
        Native.putField(ref, index, value);
    }
}
\end{lstlisting}

%All \code{putfield} bytecodes are replaced by quick variants on class
%linking. During this step also \code{putfield} instructions for
%references and double-word length fields (\code{double} and
%\code{long}) are replaced by special bytecodes. Therefore, the code
%shows the special bytecode \code{putfield\_ref}.

Note that field and array access is implemented in hardware on JOP.
Only write accesses to reference fields need to be protected by the
write-barrier, which is implemented in software. During class linking
all write operations to reference fields (\code{putfield} and
\code{putstatic} when accessing reference fields) are replaced by a
JVM internal bytecodes (e.g., \code{putfield\_ref}) to execute the
write-barrier code as shown before.
% RTS version
% in Figure~\ref{fig:barrier}.
The shown code is part of a special class
(\code{com.jopdesign.sys.JVM}) where Java bytecodes that are not
directly implemented by JOP can be implemented in Java
\cite{jop:thesis}. %% add a reference to a Section in the handbook.

The methods of class \code{Native} are JVM internal methods needed to
implement part of the JVM in Java. The methods are replaced by
regular or JVM internal bytecodes during class linking. Methods
\code{getField(ref, index)} and \code{putField(ref, value, index)}
map to the JVM bytecodes \code{getfield} and \code{putfield}. The
method \code{rdMem()} is an example of an internal JVM bytecode and
performs a memory read. The null pointer check for
\code{putfield\_ref} is implicitly performed by the hardware
implementation of \code{getfield} that is executed by
\code{Native.getField()}. The hardware implementation of
\code{getfield} triggers an exception interrupt when the reference is
null. The implementation of the write-barrier shows how a bytecode is
substituted by a special version (\code{pufield\_ref}), but uses in
the software implementation the hardware implementation of that
bytecode (\code{Naitve.putfield()}).

In principle this write-barrier could also be implemented in
microcode to avoid the expensive invoke of a Java method. However,
the interaction with the GC, which is written in Java, is simplified
by the Java implementation. As a future optimization we intend to
inline the write-barrier code.

The collector runs in its own thread and the priority is assigned
according to the deadline, which equals the period of the GC cycle.
As the GC period is usually longer than the mutator task deadlines,
the GC runs at the lowest priority. When a high priority task becomes
ready, the GC thread will be preempted. Atomic operations of the GC
are protected simply by turning the timer interrupt off.\footnote{If
interrupt handlers are allowed to change the object graph those
interrupts also need to be disabled.} Those atomic sections lead to
release jitter of the real-time tasks and shall be minimized. It has
to be noted that the GC protection with interrupt disabling is not an
option for multiprocessor systems.

\section{Evaluation}

To evaluate the proposed real-time GC we execute a simple test
application on JOP and measure heap usage and the release time jitter
of high priority threads. The test setup consists of JOP implemented
in an Altera Cyclone FPGA clocked at 100~MHz. The main memory is a
1~MB SRAM with an access time of two clock cycles. JOP is configured
with a 4~KB method cache (a special form of instruction cache) and a
128 entry stack cache. No additional data cache is used.

\subsection{Scheduling Experiments}

In this section we test an implementation of the concurrent-copy
garbage collector on JOP. The tests are intended to get some
confidence that the formulas for the collector periods are correct.
Furthermore we visualize the actual heap usage of a running system.

The examples are synthetic benchmarks that emulate worst-case
execution time (WCET) by executing a busy loop after allocation of
the data. The WCET of the collector was measured to be 10.4~ms when
executing it with scheduling disabled during one collection cycle for
example 1 and 11.2~ms for example 2. We use 11~ms and 12~ms
respectively as the WCET of the collector for the following
examples\footnote{It has to be noted that measuring execution time is
not a safe method to estimate WCET values.}.


Listing~\ref{lst:excode} shows our worker thread with the busy loop.
The data is allocated at the start of the period and freed after the
simulated execution. \code{waitForNextPeriod} blocks until the next
release time for the periodic thread.

\begin{lstlisting}[float, caption={Example periodic thread with a busy loop},
label=lst:excode]
    public void run() {

        for (;;) {
            int[] n = new int[cnt];
            // simulate work load
            busy(wcet);
            n = null;
            waitForNextPeriod();
        }
    }

    final static int MIN_US = 10;

    static void busy(int us) {

        int t1, t2, t3;
        int cnt;

        cnt = 0;
        // get the current time in us
        t1 = Native.rd(Const.IO_US_CNT);

        for (;;) {
            t2 = Native.rd(Const.IO_US_CNT);
            t3 = t2-t1;
            t1 = t2;
            if (t3<MIN_US) {
                cnt += t3;
            }
            if (cnt>=us) {
                return;
            }
        }
    }
\end{lstlisting}

For the busy loop to simulate \emph{real} execution time, and not
elapsed time, the constant \code{MIN\_US} has to be less than the
time for two context switches, but larger than the execution time of
one iteration of the busy loop. In this case only cycles executed by
the busy loop are counted for the execution time and interruption
due to a higher priority thread is not part of the execution time
measurement.

In our example we use a concurrent-copy collector with a heap size
(for both semi-spaces) of 100~KB. At startup the JVM allocates about
3.5~KB data. We incorporate\footnote{The suggested handling of static
data to be moved to \emph{immortal} memory at mission start is not
yet implemented.} these 3.5~KB as static live data $L_s$.



\subsubsection{Independent Threads}

The first example consists of two threads with the properties listed
in Table~\ref{fig:ex1}. $T_i$ is the period, $C_i$ the WCET, and
$a_i$ the maximum amount of memory allocated each period. Note that
the period for the collector thread is also listed in the table
although it is a result of the worker thread properties and the heap
size.

\begin{table}[tb]
\begin{center}
\begin{tabular}{lrrr}
    \toprule
    & \multicolumn{1}{c}{$T_i$} & \multicolumn{1}{c}{$C_i$} & \multicolumn{1}{c}{$a_i$} \\
    \midrule
    $\tau_1$ & 5 ms & 1 ms & 1 KB \\
    $\tau_2$ & 10 ms & 3 ms & 3 KB \\
    $\tau_{GC}$ & 77 ms & 11 ms & \\
    \bottomrule
\end{tabular}
    \caption{Thread properties for experiment 1}
\label{fig:ex1}
\end{center}
\end{table}

With the periods $T_i$ and the memory consumption $a_i$ for the two
worker threads we calculate the maximum period $T_{GC}$ for the
collector thread $\tau_{GC}$ by using Theorem~\ref{sch:theorem}
\begin{align*}
    T_{GC} & \le \frac{H_{CC}-2\left(L_s+\sum_{i=1}^{n} a_i\right)-2\sum_{i=1}^{n} a_i}
        {2\sum_{i=1}^{n} \frac{a_i}{Ti}} \\
           & \le \frac{100-2(3.5+4)-2\cdot4}
           {2\left(\frac{1}{5}+\frac{3}{10}\right)}\mbox{ms}\\
           & \le 77\mbox{ms}
\end{align*}

The priorities are assigned rate-monotonic \cite{321743} and we
perform a quick schedulability check with the periods $T_i$ and the
WCETs $C_i$ by calculation of the processor utilization $U$ for all
three threads
\begin{align*}
    U & = \sum_{i=1}^{3}\left(\frac{C_i}{T_i}\right)\\
      & = \frac{1}{5} + \frac{3}{10} + \frac{11}{77}\\
      & = 0.643
\end{align*}
which is less than the maximum utilization for three tasks
\begin{align*}
    U_{max} & = m*(2^{\frac{1}{m}}-1)\\
      & = 3*(2^{\frac{1}{3}}-1)\\
      & \approx 0.78
\end{align*}

In Figure~\ref{fig:ex1:mem} the memory trace for this system is
shown.
\begin{figure}
\begin{center}
    \includegraphics[width=\excelwidth]{jvm/gc_ex1}
    \caption{Free memory in experiment 1}
\label{fig:ex1:mem}
\end{center}
\end{figure}
The graph shows the free memory in one semi-space (the to-space,
which is 50~KB) during the execution of the application. The
individual points are recorded with time-stamps at the end of each
allocation request.

In the first milliseconds we see allocation requests that are part
of the JVM startup (most of it is static data). The change to the
mission phase is delayed 100~ms and the first allocation from a
periodic thread is at 105~ms. The collector thread also starts at
the same time and the first semi-space flip can be seen at 110~ms
(after one allocation from each worker thread). We see the 77~ms
period of the collector in the jumps in the free memory graph after
the flip. The different memory requests of two times 1~KB from
thread $\tau_1$ and one time 3~KB from thread $\tau_2$ can be seen
every 10~ms.

In this example the heap is used until it is almost full, but the
application never runs out of memory and no thread misses a deadline.
From the regular allocation pattern we also see that this collector
runs concurrently. With a stop-the-world collector we would notice
gaps of 10~ms (the measured execution time of the collector) in the
graph.

\subsubsection{Producer/Consumer Threads}

For the second experiment we split our thread $\tau_1$ to a producer
thread $\tau_1$ and a consumer thread $\tau_3$ with a period of
30~ms. We assume after the split that the producer's WCET is halved
to 500~us. The consumer thread is assumed to be more efficient when
working on lager blocks of data than in the former example
($C_3$=2~ms instead of 6*500~$\mu$s). The rest of the setting
remains the same (the worker thread $\tau_2$). Table~\ref{fig:ex2}
shows the thread properties for the second experiment.

\begin{table}[tb]
\begin{center}
\begin{tabular}{lrrr}
    \toprule
    & \multicolumn{1}{c}{$T_i$} & \multicolumn{1}{c}{$C_i$} &\multicolumn{1}{c}{$a_i$} \\
    \midrule
    $\tau_1$ & 5 ms & 0.5 ms & 1 KB \\
    $\tau_2$ & 10 ms & 3 ms & 3 KB \\
    $\tau_3$ & 30 ms & 2 ms & \\
    $\tau_{GC}$ & 55 ms & 12 ms & \\
    \bottomrule
\end{tabular}
    \caption{Thread properties for experiment 2}
\label{fig:ex2}
\end{center}
\end{table}

As explained in Section~\ref{sec:prod:cons} we calculate the
lifetime factor $l_1$ for memory allocated by the producer $\tau_1$
with the corresponding consumer $\tau_3$ with period $T_3$.
\begin{equation*}
    l_1 = \left\lceil\frac{2 T_3}{T_1}\right\rceil
        = \left\lceil\frac{2 \times 30}{5}\right\rceil
        = 12
\end{equation*}
%
The maximum collector period $T_{GC}$ is
\begin{align*}
    T_{GC} & \le \frac{H_{CC}-2\left(L_s+\sum_{i=1}^{n} a_i l_i\right)-2\sum_{i=1}^{n} a_i}
        {2\sum_{i=1}^{n} \frac{a_i}{Ti}} \\
           & \le \frac{100-2(3.5+1\cdot12+3+0)-2\cdot4}
           {2\left(\frac{1}{5}+\frac{3}{10}+\frac{0}{30}\right)}\mbox{ms}\\
           & \le 55\mbox{ms}
\end{align*}
We check the maximum processor utilization:
\begin{align*}
    U & = \sum_{i=1}^{4}\left(\frac{C_i}{T_i}\right)\\
      & = \frac{0.5}{5} + \frac{3}{10} + \frac{2}{30} + \frac{12}{55}\\
      & = 0.685 \le 4*(2^{\frac{1}{4}}-1) \approx 0.76
\end{align*}

In Figure~\ref{fig:ex2:mem} the memory trace for the system with one
producer, one consumer, and one independent thread is shown.
\begin{figure}
\begin{center}
    \includegraphics[width=\excelwidth]{jvm/gc_ex2}
    \caption{Free memory in experiment 2}
\label{fig:ex2:mem}
\end{center}
\end{figure}
%
Again, we see the 100~ms delayed mission start after the startup and
initialization phase, in this example at about 106~ms. Similar to
the former example the first collector cycle performs the flip a few
milliseconds after the mission start. We see the shorter collection
period of 55~ms. The allocation pattern (two times 1~KB and one time
3~KB per 10~ms) is the same as in the former example as the threads
that allocate the memory are still the same.

We have also run this experiment for a longer time than shown in
Figure~\ref{fig:ex2:mem} to see if we find a point in the execution
trace where the remaining free memory is less than the value at
217~ms. The pattern repeats and the observed value at 217~ms is the
minimum.




\subsection{Measuring Release Jitter}

%% TODO use text and new results form JTRES/TECS paper.

Our main concern on garbage collection in real-time systems is the
blocking time introduced  by the GC due to atomic code sections.
This blocking time will be seen as release time jitter on the
real-time threads. Therefore we want to measure this jitter.

\begin{lstlisting}[float=t, caption={Measuring release time jitter},
label=lst:measure]

public boolean run() {

    int t = Native.rdMem(Const.IO_US_CNT);
    if (!notFirst) {
        expected = t+period;
        notFirst = true;
    } else {
        int diff = t-expected;
        if (diff>max) max = diff;
        if (diff<min) min = diff;
        expected += period;
    }
    work();

    return true;
}
\end{lstlisting}

Listing~\ref{lst:measure} shows how we measure the jitter. Method
\code{run()} is the main method of the real-time thread and executed
on each periodic release. Within the real-time thread we have no
notion about the start time of the thread. As a solution we measure
the actual time on the first iteration and use this time as first
release time. In each iteration the expected time, stored in the
variable \code{expected}, is incremented by the \code{period}. In
each iteration (except the first one) the actual time is compared
with the expected time and the maximum value of the difference is
recorded.

As noted before, we have no notion about the \emph{correct} release
times. We measure only relative to the first release. When the first
release is delayed (due to some startup code or interference with a
higher priority thread) we have a positive offset in
\code{expected}. On an exact release in a later iteration the time
difference will be negative (in \code{diff}). Therefore, we also
record the minimum value for the difference between the actual time
and the expected time. The maximum measured release jitter is the
difference between \code{max} and \code{min}.

To provide a baseline we measure the release time jitter of a single
real-time thread (plus an endless loop in the \code{main} method as
an idle non-real-time background thread). No GC thread is scheduled.
The code is similar to the code in Listing~\ref{lst:measure}. A stop
condition is inserted that prints out the minimum and maximum time
differences measured after 1 million iterations.

\begin{table}
    \centering
    \begin{tabular}{rr}
    \toprule
    Period & Jitter \\
    \midrule
    200 $\mu$s & 0 $\mu$s \\
    100 $\mu$s & 0 $\mu$s \\
    50 $\mu$s & 17 $\mu$s \\
    \bottomrule
    \end{tabular}
    \caption{Release jitter for a single thread}
    \label{tab:single}
\end{table}

Table~\ref{tab:single} shows the measured jitter for different
thread periods. We observed no jitter for periods of 100~$\mu$s and
longer. At a period of 50~$\mu$s the scheduler introduces a
considerable amount of jitter. From this measurement we conclude
that 100~$\mu$s is the practical shortest period we can handle with
our system. We will use this period for the high-priority real-time
thread in the following measurement with an enabled GC.

\subsection{Measurements}

%% TODO use new results form JTRES/TECS paper.

The test application consisting of three real-time threads
($\tau_{hf}$, $\tau_{p}$, and $\tau_{c}$), one logging thread
$\tau_{log}$, and the GC thread $\tau_{gc}$. All three real-time
threads measure the difference between the expected release time and
the actual release time (as shown in Figure~\ref{lst:measure}). The
minimum and maximum values are recorded and regularly printed to the
console by the logging thread $\tau_{log}$. Table~\ref{tab:exp}
shows the release parameters for the five threads. Priority is
assigned deadline monotonic. Note that the GC thread has a shorter
period than the logger thread, but a longer deadline. For our
approach to work correctly the GC thread \emph{must} have the lowest
priority. Therefore all other threads with a longer period than the
GC thread must be assigned a shorter deadline.

\begin{table}
    \centering
    \begin{tabular}{lrrr}
    \toprule
    Thread & Period & Deadline & Priority \\
    \midrule
    $\tau_{hf}$ & 100 $\mu$s & 100 $\mu$s & 5 \\
    $\tau_{p}$ &  1 ms & 1 ms & 4 \\
    $\tau_{c}$ & 10 ms & 10 ms & 3 \\
    $\tau_{log}$ & 1000 ms & 100 ms & 2 \\
    $\tau_{gc}$ & 200 ms & 200 ms & 1 \\
    \bottomrule
    \end{tabular}
    \caption{Thread properties of the test program}
    \label{tab:exp}
\end{table}

Thread $\tau_{hf}$ represents a high-frequency thread without
dynamic memory allocation. This thread should observe minimal
disturbance by the GC thread.

The threads $\tau_{p}$ and $\tau_{c}$ represent a producer/consumer
pair that uses dynamically allocated memory for communication. The
producer appends the data at a frequency of 1~kHz to a simple list.
The consumer thread runs at 100~Hz and processes all currently
available data in the list and removes them from the list. The
consumer will process between 9 and 11 elements (depending on the
execution time of the consumer and the thread phasing).

It has to be noted that this simple and common communication pattern
cannot be implemented with the scoped memory model of the RTSJ.
First, to use a scope for communication, we have to keep the scope
alive with a \emph{wedge} thread \cite{conf/isorc/PizloFHV04} when
data is added by the producer. We would need to notify this wedge
thread by the consumer when all data is consumed. However, there is
no single instant available where we can \emph{guarantee} that the
list is empty. A possible solution for this problem is described in
\cite{conf/isorc/PizloFHV04} as \emph{handoff} pattern. The pattern
is similar to double buffering, but with an explicit copy of the
data. The elegance of a simple list as buffer queue between the
producer and the consumer is lost.

Thread $\tau_{log}$ is not part of the real-time systems simulated
application code. Its purpose is to print the minimum and maximum
differences between the measured and expected release times (see
former section) of threads $\tau_{hf}$ and $\tau_{p}$ to the console
periodically.

Thread $\tau_{gc}$ is a standard periodic real-time thread executing
the GC logic. The GC thread period was chosen quite short in that
example. A period in the range of seconds would be enough for the
memory allocation by $\tau_{p}$. However, to stress the interference
between the GC thread and the application threads we artificially
shortened the period.

\begin{table}
    \centering
    \begin{tabular}{lr}
    \toprule
    Threads & Jitter \\
    \midrule
    $\tau_{hf}$ & 0 $\mu$s \\
    $\tau_{hf}$, $\tau_{log}$ & 7 $\mu$s \\
    $\tau_{hf}$, $\tau_{log}$,$\tau_{p}$,$\tau_{c}$ & 14 $\mu$s \\
    $\tau_{hf}$, $\tau_{log}$,$\tau_{p}$,$\tau_{c}$,$\tau_{gc}$ & 54 $\mu$s \\
    \bottomrule
    \end{tabular}
    \caption{Jitter measured on a 100~MHz processor for the high priority thread in different configurations}
    \label{tab:jitter}
\end{table}
As a first experiment we run only $\tau_{hf}$ and the logging thread
$\tau_{log}$ to measure jitter introduced by the scheduler. The
maximum jitter observed for $\tau_{hf}$ is 7 $\mu$s -- the blocking
time of the scheduler.

In the second experiment we run all threads except the GC thread.
For the first 4 seconds we measure a maximum jitter of 14~$\mu$s for
thread $\tau_{hf}$. After those 4 seconds the heap is full and GC is
necessary. In that case the GC behaves in a stop-the-world fashion.
When a new object request cannot be fulfilled the GC logic is
executed in the context of the allocating thread. As the bytecode
\code{new} is itself in an atomic region the application is blocked
until the GC finishes. Furthermore, the GC performs a conservative
scan of all thread stacks. We measure a release delay of 63~ms for
all threads due to the blocking during the full collection cycle.
From that measurement we can conclude for the sample application and
the available main memory: (a) the measured maximum period of the GC
thread is in the range of 4 seconds; (b) the estimated execution
time for one GC cycle is 63~ms. It has to be noted that measurement
is not a substitution for static timing analysis. Providing WCET
estimates for a GC cycle is a challenge for future work.

In our final experiment we enabled all threads. The GC is scheduled
periodically at 200~ms as the lowest priority thread -- the scenario
we argue for. The GC logic is set into the concurrent mode on mission
start. In this mode the thread stacks are not scanned for roots.
Furthermore when an allocation request cannot be fulfilled the
application is stopped. This radical stop is intended for testing. In
a more tolerant implementation either an out-of-memory exception can
be thrown or the requesting thread has to be blocked, its thread
stack scanned and released when the GC has finished its cycle.

We ran the experiment for several hours and recorded the maximum
release jitter of the real-time threads. For this test we used
slightly different periods (prime numbers) to avoid the regular
phasing of the threads. The harmonic relation of the original
periods can lead to too optimistic measurements. The applications
never ran out of memory. The maximum jitter observed for the high
priority task $\tau_{hf}$ was 54~$\mu$s. The maximum jitter for task
$\tau_{p}$ was 108~$\mu$s. This higher value on $\tau_{p}$ is
expected as the execution interferes with the execution of the
higher priority task $\tau_{hf}$.

\subsection{Discussion}


With our measurements we have shown that quite short blocking times
are achievable. Scheduling introduces a blocking time of about
7--14~$\mu$s and the GC adds another 40~$\mu$s resulting in a
maximum jitter of the highest priority thread of 54~$\mu$s. In our
first implementation we performed the object copy in pure Java,
resulting in blocking times around 200~$\mu$s. To speedup the copy
we moved this function to microcode. However, the microcoded
\emph{memcpy} still needs 18 cycles per 32-bit word copy. Direct
support in hardware can lead to a copy time of 4--5 clock cycles per
word.

%% Rewrite/redo with copy unit (for RTS) and probably also
%% the handbook. Or use JTRES/TECS results.

The maximum blocking time of 54~$\mu$s on a 100 MHz processor is less than
blocking times reported for other solutions.

Blocking time for Metronome (called pause times in the papers) is
reported to be 6~ms \cite{gc:jtres:metronome} on a 500~MHz PowerPC
at 50\% CPU utilization. Those large blocking times are due to the
scheduling of the GC at the highest priority with a polling based
yield within the GC thread. A fairer comparison is against the
\emph{jitter} of the pause time. In \cite{gc:bacon05} the variation
of the pause time is given between 500~$\mu$s and 2.4~ms on a 1~Ghz
machine. It should be noted that Metronome is a GC intended for
mixed real-time systems whereas we aim only for hard real-time
systems.

Robertz performed a similar measurement as we did for his thesis
\cite{gc:robertz:thesis} with a time-triggered GC on a 350~MHz
PowerPC. He measured a maximum jitter of 20~$\mu$s ($\pm10$~$\mu$s)
for a high priority task with a period of 500~$\mu$s.

It has to be noted that our experiment is a small one and we need
more advanced real-time applications for the evaluation of real-time
GC. The problem is that it is hard to find even static based
real-time application benchmarks (at least applications written for
safety critical Java). Running standard benchmarks that measure
average case performance (e.g., SPEC jvm98) is not an option to
evaluate a real-time collector.

%SUN RTS jitter +- 10us, but heap schedulables 100us
%lund release time jitter 40us - from where do I have this number
%\url{http://www.robot.lth.se/java/} compiler
%read \url{http://www.ulb.ac.be/di/ssd/goossens/RTS05.pdf}


\section{Analysis}

To integrate GC into the WCET and scheduling analysis we need to
know the worst-case memory consumption including the maximum
lifetime of objects and the WCET of the collector itself.

\subsection{Worst Case Memory Consumption}

Similar to the necessary WCET analysis of the tasks that make up the
real-time system, a worst case memory allocation analysis of the
tasks is necessary. For objects that are not shared between tasks
this analysis can be performed by the same methods known from the
WCET analysis. We have to find the worst-case program path that
allocates the maximum amount of memory.

The analysis of memory consumption by objects that are shared
between tasks for communication is more complex as an inter-task
analysis is necessary.

\subsection{WCET of the Collector}

For the schedulability analysis the WCET of the collector has to be
known. The collector performs following steps\footnote{These steps
can be distinct steps as in the mark-compact collector or
interleaved as in the concurrent-copy collector.}:
\begin{enumerate}
    \item Traverse the live object graph
    \item Copy the live data
    \item Initialize the free memory
\end{enumerate}

The execution time of the first step depends on the maximum amount
of live data and the number of references in each object. The second
step depends on the size of the live objects. The last step depends
on the size of the memory that gets freed during the collection
cycle. For a concurrent-copy collector this time is constant as a
complete semi-space gets initialized to zero. It has to be noted
that this initialization could also be done at the allocation of the
objects (as the \code{LTMemory} from the RTSJ implies). However,
initialization in the collector is more efficient and the necessary
time is easier to predict.

The maximum allocated memory and the type of the allocated objects
determine the control flow (the flow facts) of the collector.
Therefore, this information has to be to incorporate into WCET
analysis of the collector thread.

\section{Summary} \label{sec:gc:summery}

In this chapter we have presented a real-time garbage collector in
order to benefit from a more dynamic programming model for real-time
applications. The collector is incremental and scheduled as a normal
real-time thread and, according to its deadline, assigned the lowest
priority in the system. The restrictions from the SCJ programming
model and the low priority result in two advantages: (a) avoidance of
stack root scanning and (b) short blocking time of high priority
threads. At 100~MHz we measured 40~$\mu$s maximum blocking time
introduced by the GC thread.

To guarantee that the applications will not run out of memory, the
period of the collector thread has to be short enough. We provided
the maximum collector periods for a mark-compact collector type and a
concurrent-copy collector. We have also shown how a longer lifetime
due to object sharing between threads can be incorporated into the
collector period analysis.

A critical operation for a concurrent, compacting GC is the atomic
copy of large arrays. JOP has been extended by a copy unit that can
be interrupted. This unit is integrated with the memory access unit
and redirects the access to either fromspace or tospace depending on
the array/field index and the value of the copy pointer.

\section{Further Reading}

Garbage collection was first introduced for list processing systems
(LISP) in the 1960s. The first collectors were \emph{stop-the-world}
collectors that are called when a request for a new element can not
be fulfilled. The collector, starting from pointers known as the root
set, scans the whole graph of reachable objects and marks these
objects live. In a second phase the collector \emph{sweeps} the heap
and adds unmarked objects to the free list. On the marked objects,
which are live, the mark is reset in preparation for the next cycle.

However, this simple sweep results in a fragmented heap which is an
issue for objects of different sizes. An extension, called
\emph{mark-compact}, moves the objects to compact the heap instead
of the sweep. During this compaction all references to the moved
objects are updated to point to the new location.


Both collectors need a stack during the marking phase that can grow
in the worst-case up to the number of live objects. Cheney
\cite{gc:cheney70} presents an elegant way how this mark stack can
be avoided. His GC is called \emph{copying-collector} and divides
the heap into two spaces: the \emph{to-space} and the
\emph{from-space}. Objects are moved from one space to the other as
part of the scan of the object graph.

However, all the described collectors are still stop-the-world
collectors. The pause time of up to seconds in large interactive
LISP applications triggered the research on incremental collectors
that distribute collection work more evenly \cite{gc:steele75,
gc:dijkstra78, gc:baker78}. These collectors were sometimes called
\emph{real-time} although they do not fulfill hard real-time
properties that we need today. A good overview of GC techniques can
be found in \cite{gc:jone96} and in the GC survey by Wilson
\cite{gc:wils94}.

Baker \cite{gc:baker78} extends Cheneys \cite{gc:cheney70} copying
collector for incremental GC. However, it uses an expensive read
barrier that moves the object to the to-space as part of the mutator
work. Baker proposes the \emph{Treadmill} \cite{gc:baker92} to avoid
copying. However, this collector works only with objects of equal
size and still needs an expensive read barrier.

In \cite{gc:hwgc94} a garbage-collected memory module is suggested to
provide a real-time collector. A worst-case delay time of 1$\mu$s is
claimed without giving the processor speed.

Metronome is a collector intended for soft real-time systems
\cite{gc:bacon03}. Non real-time applications are used (SPECjvm98) in
the experiments. They propose a collector with constant utilization
to meet real-time requirements. However, utilization is \emph{not} a
real-time measure per se; it should be schedulability or response
time instead. In contrast to our proposal the GC thread is scheduled
at the highest priority in short periods. To ensure that, despite the
high priority of the GC thread, mutator threads will be scheduled,
the GC thread runs only for a fraction of time within a time window.
This fraction and the size of the time window can be adjusted for
different work loads.
%Pause times are in the range of 12~ms.

Although not mandated, all commercial and academic implementations of
the RTSJ \cite{gc:siebert:phd,Mackinac,rtsj:ibm:2007,ovm:tecs:07} and
related real-time Java systems \cite{perc:pico:um} also contain a
real-time garbage collector.


In \cite{gc:pfeffer04} two collectors based on \cite{gc:dijkstra78}
and \cite{gc:baker92} are implemented on a multithreaded
microcontroller.  Higuera suggests in \cite{gc:higu02} the use of
hardware features from picoJava to speed up RTSJ memory region
protection and garbage collection.

The work closest to our scheduling analysis is presented in
\cite{780745}. The authors provide an upper bound of the GC cycle
as\footnote{We use our symbols in the equation for easier comparison
to our finding.}
%
\begin{equation}
\nonumber
    T_{GC} \le \frac{\frac{H-L_{max}}{2}-\sum_{i=1}^{n} a_i}{\sum_{i=1}^{n} \frac{a_i}{Ti}}
\end{equation}
%
Although stated that this bound ``is thus not dependent of any
particular GC algorithm", the result applies only for single heap GC
algorithms (e.g.\ mark-compact) and not for a copying collector. A
value for $L_{max}$ is not given in the paper. If we use our value
of $L_{max} =\sum_{i=1}^{n} a_i$ the result is
%
\begin{equation}
\nonumber
    T_{GC} \le \frac{H-3\sum_{i=1}^{n} a_i}{2\sum_{i=1}^{n} \frac{a_i}{Ti}}
\end{equation}
%
This result is the same as in our finding (see
Theorem~\ref{sch:theorem}) for the mark-compact collector. No
analysis is given how objects with longer lifetime and static
objects can be incorporated.

%% TODO: add JamicaVM, Sun RTS


\chapter{The SimpCon Interconnect}
\label{chap:simpcon}
\newcommand{\scgrsc}{.65}
\newcommand{\scgrp}{simpcon}
\hyphenation{SimpCon}
SimpCon \cite{simpcon} is the main interconnection interface used
for JOP. The IO modules and the main memory are connected via this
standard. In the following chapter an introduction to SimpCon is
presented.


\section{Introduction}

The intention of the following SoC interconnect standard is to be
simple and efficient with respect to implementation resources and
transaction latency.

SimpCon is a fully synchronous standard for on-chip
interconnections. It is a point-to-point connection between a master
and a slave. The master starts either a read or write transaction.
Master commands are single cycle to free the master to continue on
internal operations during an outstanding transaction. The slave has
to register the address when needed for more than one cycle. The
slave also registers the data on a read and provides it to the
master for more than a single cycle. This property allows the master
to delay the actual read if it is busy with internal operations.

The slave signals the end of the transaction through a novel
\emph{ready counter} to provide an early notification. This early
notification simplifies the integration of peripherals into
pipelined masters.

Slaves can also provide several levels of pipelining. This feature
is announced by two static output ports (one for read and one write
pipeline levels).

Off-chip connections (e.g.\ main memory) are device specific and
need a slave to perform the translation. Peripheral interrupts are
not covered by this specification.

\subsection{Feature}

\begin{itemize}
    \item Master/slave point-to-point connection
    \item Synchronous operation
    \item Read and write transactions
    \item Early pipeline release for the master
    \item Pipelined transactions
    \item Open-source specification
    \item Low implementation overheads
\end{itemize}

\subsection{Basic Read Transaction}

Figure~\ref{fig:sc:basic:rd} shows a basic read transaction for a
slave with one cycle latency. The acknowledge signals are omitted
from the figure. In the first cycle, the address phase, the
\sign{rd} signals the slave to start the read transaction. The
address is registered by the slave. During the following cycle, the
read phase, the slave performs the read and registers the data. Due
to the register in the slave the data is available in the third
cycle, the result phase. To simplify the master, the read data stays
valid till the next read request response.

\begin{figure}
    \centering
    \includegraphics[scale=\scgrsc]{\scgrp/sc_basic_rd}
    \caption{Basic read transaction}
    \label{fig:sc:basic:rd}
\end{figure}

\subsection{Basic Write Transaction}

A write transaction consists of a single cycle address/command phase
started by assertion of \sign{wr} where the address and the write
data are valid. \sign{address} and \sign{wr\_data} are usually
registered by the slave. The end of the write cycle is signalled to
the master by the slave with \sign{rdy\_cnt}. See section
\ref{sec:ack} and an example in Figure~\ref{fig:sc:wr:ws}.

\section{SimpCon Signals}

This sections defines the signals used by the SimpCon connection.
Some of the signals are optional and may not be present on a
peripheral device.

All signals are a single direction point-to-point connection between
a master and a slave. The signal details are described by the device
that drives the signal. Table~\ref{tab:sc:signals} lists the signals
that define the SimpCon interface. The column Direction indicates
wether the signal is driven by the master or the slave.

\begin{table}
    \centering

    \begin{tabular}{lrlll}
        \toprule
        Signal & Width & Direction & Required & Description \\
        \midrule
        \sign{address} & 1--32 & Master & No & Address lines from the
        master\\
        & & & & to the slave port\\
        \sign{wr\_data} & 32 & Master & No & Data lines from the
        master\\
        & & & & to the slave port\\
        \sign{rd} & 1 & Master & No & Start of a read transaction \\
        \sign{wr} & 1 & Master & No & Start of a write transaction \\
        \sign{rd\_data} & 32 & Slave & No & Data lines from the
        slave\\
        & & & & to the master port\\
        \sign{rdy\_cnt} & 2 & Slave & Yes & Transaction end signalling \\
        \sign{rd\_pipeline\_level} & 2 & Slave & No & Maximum pipeline
        level\\
        & & & & for read transactions \\
        \sign{wr\_pipeline\_level} & 2 & Slave & No & Maximum pipeline
        level\\
        & & & & for write transactions \\
        \bottomrule

    \end{tabular}
    \caption{SimpCon port signals}
    \label{tab:sc:signals}

\end{table}



\subsection{Master Signal Details}

This section describes the signals that are driven by the master to
initiate a transaction.

\subsubsection{address}

Master addresses represent word addresses as offsets in the slaves
address range. \sign{address} is valid a single cycle either with
\sign{rd} for a read transaction or with \sign{wr} and
\sign{wr\_data} for a write transaction.

The number of bits for \sign{address} depend on the slaves address
range. For a single port slave \sign{address} can be omitted.

\subsubsection{wr\_data}

The \sign{wr\_data} signals carry the data for a write transaction.
It is valid for a single cycle together with \sign{address} and
\sign{wr}. The signal is typically 32 bits wide. Slaves can ignore
upper bits when the slave port is less than 32 bits.

\subsubsection{rd}

The \sign{rd} signal is asserted a single clock cycle to start a
read transaction. \sign{address} has to be valid in the same cycle.

\subsubsection{wr}

The \sign{wr} signal is asserted a single clock cycle to start a
write transaction. \sign{address} and \sign{wr\_data} have to be
valid in the same cycle.

\subsubsection{sel\_byte}

The \sign{sel\_byte} signal is reserved for future versions of the
SimpCon specification to add individual byte enables.

\subsection{Slave Signal Details}

This section describes the signals that are driven by the slave as a
response to transaction initiated by the master.

\subsubsection{rd\_data}

The \sign{wr\_data} signals carry the result for a read transaction.
The data is valid when \sign{rdy\_cnt} reaches 0 and stays valid
till a new read result is available. The signal is typically 32 bits
wide. Slaves that provide less than 32 bits should pad the upper
bits with 0.

\subsubsection{rdy\_cnt}

The \sign{rdy\_cnt} signal provides the number of cycles till the
pending transaction will finish. A 0 means that either read data is
available or a write transaction has been finished. Values of 1 and
2 mean the the transaction will finish in at least 1 or 2 cycles.
The maximum value is 3 and means the the transaction will finish in
3 or \emph{more} cycles. Note that not all values have to be used in
a transaction. Each monotonic sequence of \sign{rdy\_cnt} values is
legal.

\subsubsection{rd\_pipeline\_level}

The static \sign{rd\_pipeline\_level} provides the master with the
read pipeline level of the slave. The signal has to be constant to
enable the synthesizer to optimize the pipeline level dependent
state machine in the master.


\subsubsection{wr\_pipeline\_level}

The static \sign{wr\_pipeline\_level} provides the master with the
write pipeline level of the slave. The signal has to be constant to
enable the synthesizer to optimize the pipeline level dependent
state machine in the master.

\section{Slave Acknowledge}
\label{sec:ack}

Flow control between the slave and the master is usually done by a
single signal in the form of \emph{wait} or \emph{acknowledge}. The
\sign{ack} signal, e.g.\ in the Wishbone specification, is set when
the data is available or the write operation has finished. However,
for a pipelined master it can be of interest to know it
\emph{earlier} when a transaction will finish.

For a lot of slaves, e.g.\ a SRAM interface with fixed wait states,
this information is available inside the slave. In the SimpCon
interface this information is communicated to the master through the
two bit signal \sign{rdy\_cnt}. \sign{rdy\_cnt} signals the number
of cycles till the read data will be available or the write
transaction will be finished. Value 0 is equivalent to an \emph{ack}
signal and 1, 2, and 3 are equivalent to a wait request with the
distinction that the master knows how long the wait request will
last.

To avoid too many signals at the interconnect \sign{rdy\_cnt} has a
width of two bits. Therefore, the maximum value of 3 has the special
meaning that the transaction will finish in 3 or \emph{more} cycles.
As a result the master can only use the values 0, 1, and 2 to
release actions in it's pipeline.

Idle slaves will keep the former value of 0 for \sign{rdy\_cnt}.
Slaves, that don't know in advance how many wait states are need for
the transaction can produce sequences that omit any of the numbers
3, 2, and 1. The master has to handle this situations.

Figure~\ref{fig:sc:rd:ws} shows an example of a slave that needs
three cycles for the read to be processed. In cycle 1 the read
command and the address are set by the master. The slave registers
the address and sets \sign{rdy\_cnt} to 3 in cycle 2. The read takes
three cycles (2--4) during which \sign{rdy\_cnt} is decremented. In
cycle 4 the data is available inside the slave and gets registered.
It is available in cycle 5 for the master and \sign{rdy\_cnt} is
finally 0. Both, the \sign{rd\_data} and \sign{rdy\_cnt} will keep
their value till a new transaction is requested.

\begin{figure}
    \centering
    \includegraphics[scale=\scgrsc]{\scgrp/sc_rd_ws}
    \caption{Read transaction with wait states}
    \label{fig:sc:rd:ws}
\end{figure}


Figure~\ref{fig:sc:wr:ws} shows an example of a slave that needs
three cycles for the write to be processed. The address, the data to
be written and the write command are valid during cycle 1. The slave
registers the address and write data during cycle 1 and performs the
write operation during cycles 2--4. The \sign{rdy\_cnt} is
decremented and a non-pipelined slave can accept a new command after
cycle 4.

\begin{figure}
    \centering
    \includegraphics[scale=\scgrsc]{\scgrp/sc_wr_ws}
    \caption{Write transaction with wait states}
    \label{fig:sc:wr:ws}
\end{figure}



\section{Pipelining}

Figure~\ref{fig:sc:pipe:level} shows a read transaction for a slave
with four cycles latency. Without any pipelining the next read
transaction will start in cycle 7 after the data from the former
read transaction is read by the master. The three bottom lines show
when new read transactions will be started for different pipeline
levels. With pipeline level 1 a new transaction can start in the
same cycle when the former read data is available (in this example
in cycle 6). Higher levels mean that the next read will start
earlier as shown for level 2 and 3.

\begin{figure}
    \centering
    \includegraphics[scale=\scgrsc]{\scgrp/sc_pipe_level}
    \caption{Different pipeline levels for a read transaction}
    \label{fig:sc:pipe:level}
\end{figure}

Implementation of level 1 in the slave is trivial (just two more
transitions in the state machine). It is recommended to provide
level 1 at least for read transactions. Level 2 is a little bit more
complex but usually no additional address or data registers are
needed.

To implement level 3 pipelining in the slave at least an additional
address register is needed. However, to use level 3 the master has
to issue the request in the same cycle as \sign{rdy\_cnt} goes to 2.
That means this transition is combinatorial. We see in
Figure~\ref{fig:sc:pipe:level} that \sign{rdy\_cnt} value of 3 means
three or more cycles till the data is available and can therefore
not be used to trigger a new transaction.

\section{Multiple Master}

SimpCon defines no signals for the communication between a master
and an arbiter. However, it is possible to build a multi master
system with SimpCon. The SimpCon interface can be used as
interconnect between the masters and the arbiter and the arbiter and
the slaves. In this case the arbiter acts as slave for the master
and as master for the peripheral devices.

The missing arbitration protocol in SimpCon results in the need to
queue $n-1$ requests in an arbiter for $n$ masters. However, for
this additional HW we get zero overheads for the bus request. The
master, which gets the bus will will start the slave transaction in
the same cycle.
\\
\\
TODO: add a timing diagram to explain this concept.


\section{Examples}

This section provides some examples for the application of the
SimpCon definition.

\subsection{IO Port}

TODO: Show how simple an IO port can be with SimpCon. We need no
addresses and can tie \sign{bsy\_cnt} to 0. We only need the
\sign{rd} or \sign{wr} signal to enable the port.

\subsection{SRAM interface}

The following example is taken from an implementation of SimpCon for
a Java processor. The processor is clocked with 100MHz and the main
memory consists of 15ns static RAMs. Therefore the minimum access
time for the RAM is two cycles. The slack time of 5ns forces us to
use output registers for the RAM address and write data and input
registers for the read data in the IO cells of the FPGA. These
registers fit nice with the intention of SimpCon to use registers
inside the slave.

Figure~\ref{fig:sc:sram} shows the interface for a non-pipelined
read access followed by a write access. Four signals are driven by
the master and two signal by the slave. The lower half of the figure
shows the signals at the FPGA pins where the RAM is connected.

\begin{figure}
    \centering
    \includegraphics[scale=\scgrsc]{\scgrp/sc_sram}
    \caption{Static RAM interface without pipelining}
    \label{fig:sc:sram}
\end{figure}

In cycle 1 the read transaction is started by the master and the
slave registers the address. The slave also sets the registered
control signals \sign{ncs} and \sign{noe} during cycle1. Due to the
IO cell registers, the address and control signals are valid at the
FPGA pins very early in cycle 2. At the end of cycle 3 (15ns after
\sign{address}, \sign{ncs} and \sign{noe} are stable) the data from
the RAM is available and can be sampled with the rising edge for
cycle 4.

The master reads the data in cycle 4 and starts a write transaction
in cycle 5. Address and data are again registered from the slave and
are available for the RAM at the beginning of cycle 6. To perform a
write in two cycles the nwr signal is registered by a negative
triggered flip-flop.

In figure~\ref{fig:sc:sram:prd} we see a pipelined read from the RAM
with pipeline level 2. With this pipeline level and the two cycles
read access time of the RAM we get the maximum bandwidth possible.

\begin{figure}
    \centering
    \includegraphics[scale=\scgrsc]{\scgrp/sc_sram_prd}
    \caption{Pipelined read from a static RAM}
    \label{fig:sc:sram:prd}
\end{figure}

We can see the start of the second read transaction in cycle 3
during the read of the first data from the RAM. The new address is
registered in the same cycle and available for the RAM in the
following cycle 4. Although we have a pipeline level of 2 we need no
additional address or data register. The read data is available for
two cycles (\sign{rdy\_cnt} 2 or 1 for the next read) and the master
is free to select one of the two cycles to read the data.

\subsection{Master Multiplexing}

To add several slaves to a single master the \sign{rd\_data} and
\sign{bsy\_cnt} have to be multiplexed. Due to the fact that all
\sign{rd\_data} signals are registered by the slaves a single
pipeline stage will be enough for a large multiplexer. The selection
of the multiplexer is also known at the transaction start but needed
at most in the next cycle. Therefore it can be registered to further
speed up the multiplexer.
\\
\\
TODO: add a schematic for the master \sign{rd\_data} multiplexer.

\section{Status}

\begin{itemize}
    \item First timing diagrams drawn
    \item SimpCon SRAM interface for JOP on Cyclone and Spartan-3 is
    available
    \item Project at opencores.org accepted
    \item Simple UART as SimpCon example
    \item IO in JOP changed to SimpCon (uart, cnt, usb)
\end{itemize}
%
Next steps:
%
\begin{itemize}
    \item Continue this document
    \item Provide Wishbone bridges
\end{itemize}
%
to clarify:
\begin{itemize}
    \item Use transaction or transfer in this document?
    \item Use address phase or better command cycle?
\end{itemize}


\chapter{The Embedded Java TCP/IP Stack}
\label{chap:ejip}
\newcommand{\grpath}{ejip}
The following section describes \emph{ejip}, a small TCP/IP stack
written entirely in Java for small, resource constraint devices.

%\section{Introduction}
%
%About ejip....

\section{TCP/IP Decisions}

\begin{itemize}
    \item Ignore IP options (TODO copy the data)
    \item Ignore Fragmentation
    \begin{itemize}
        \item Can we do this?
        \item Use and accept TCP max. length
        \item Restrict UDP/TCP to 512 bytes
        \item What is than the point for 1500 bytes Ethernet
        packages?
    \end{itemize}
    \item Should we go for simple routing?
    \item TCP/IP as stop \& go (like TFTP) -- but do a window probe
    (p 329)
    \begin{itemize}
        \item Sending: send segment and wait for ack
        \item Ignore out-of-order packets -- drop packets with not
        expected sequence \#
        \item Use the notion of TCPConnection
    \end{itemize}
    \item Don't bother with half close
    \item Do an event based API for TCP/IP and hide it by an
    optional Layer (Stream)
    \item Zero copy design: use the packet up to the application
    (e.g.\ TCP server)
    \item MSS at SYN (for MTU) -- use 512 byte data and check the
    link layer MTU (e.g.\ Slip, Ppp)
    \item Do the buffering at application level
    \begin{itemize}
        \item Really decided?
        \item Get data with sequence \#!
        \item Can be hidden by a stream as in jtcpip
    \end{itemize}
    \item There is no such thing as UDPConnection. UDP is
    connectionless!
\end{itemize}
\section{Notes}

\begin{itemize}
    \item Use smaller (and more) buffers and hold the buffer until
    ack or timeout.
    \item Keep the small web server (Html.java) as stateless (single) web
    page (probably with applet in code byte array)
    \item Interaction diagram would be helpful.
    \item Do timeouts on a simple 500~ms timer. Invoke
    System.currentTimeMillis() only once for the timer tick
    \item Check for minimal size (get rid of JDK code)
    \item Each part of the stack should implement Runable for the
    loop
    \item Event driven mode does not support blocking wait
    abstraction
    \item Prepare for performance measurements
    \begin{itemize}
        \item UDP (small/large) packages, multithreaded on PC
        \item TCP (small/large)
        \item Compare with jtcpip
        \item Where is the bottleneck (checksum, copy in jtcpip)
    \end{itemize}
    \item protothread switch expansion is very interesting for
    non-blocking (no wFNP()) profiles
\end{itemize}

Further (requested) properties of ejip:
\begin{itemize}
    \item Can run as single thread application -- don't use blocking
    (stop-and-wait) semantics of sockets -- fits even for SJC level
    0
    \item Invoking the application on a packet is event driven
    handling
    \item WCET analyzable! Check it with WCA and Volta
\end{itemize}

\subsection{Buffer Handling}

Should we use lists instead of linear search? Linear
    search is WCET predictable.

\begin{itemize}
    \item Application send:
    \begin{enumerate}
        \item Request a packet from the pool
        \item Fill it with data
        \item Return it to the pool for processing
    \end{enumerate}
    \item Interface receive:
    \begin{enumerate}
        \item Request a packet
        \item Fill it
        \item Return it to the pool for stack handling
    \end{enumerate}
    \item Interface send ready:
    \begin{enumerate}
        \item Grab a send read packet
        \item Send it
        \item Free the packet
    \end{enumerate}
    \item Grab a received packet:
    \begin{enumerate}
        \item Check IP header
        \item Invoke UDP/TCP/ICMP method
        \item Question: who frees the packet?
    \end{enumerate}
    \item Application receive invoked by the network code
\end{itemize}

\section{TCP/IP}

Two possibilities:
\begin{enumerate}
    \item TCP packets are kept in retransmission state till ack
    arrives
    \item The application has to reproduce the data (uip)
\end{enumerate}

The control is static (singleton), bit the data is OO.

\subsection{To Check}

\begin{itemize}
    \item Implement simple HTTP server with jtcpip
    \item Check uip -- done
    \item Check original TCP/IP source
\end{itemize}

\subsection{Start TCP/IP}

\begin{itemize}
    \item Start ejip documentation + do Javadoc
    \item Start with simple telnet/HTTP requests and get connection
    established working
    \begin{itemize}
        \item Basic done
        \item Retransmit a lost SYN missing
        \item Active open missing
    \end{itemize}
    \item Get a simple notion of time
    (Util/System.currentTimeMillis())
    \item on retransmission: add a new state for packet - sent, but
    not acked -- goes down to driver that send the data!
\end{itemize}

TCP/IP has to act on:
\begin{enumerate}
    \item Rcv package
    \item Write data from app
    \item Timeout (check connections)
\end{enumerate}

\subsection{TCP/IP Handler}

Event based and provides hooks for:
\begin{itemize}
    \item Connect
    \item Rcv data
    \item TX free?
    \item Close
    \item timeout?
\end{itemize}

Questions:
\begin{itemize}
    \item Is the application allowed to send data only when called
    form ejip or also independent?
    \item Application state in TCPConnection? or retransmission
    states?
    \item Stop/restart TCP/IP machine for application flow control?
\end{itemize}

\section{TODO}

\begin{itemize}
    \item Is checksum generation correct? (setting to 0xffff when 0,
    only in UDP?)
\end{itemize}


\chapter{Ongoing Work}
\label{chap:ongoing}
    \section{Interrupts}
\label{sec:interrupt}

This is a working document for changes in the interrupt system. That
means extending the simple timer interrupt and exception interrupt
system to a full interrupt controller including inter-processor
interrupts.

see page 8-41 of Intel document for interrupt handling with priority
and EOI signalling.

Questions:
\begin{itemize}
    \item Do we need interrupts within interrupts for our RT based
    interrupt system? I don't think so -- than interrupt priority
    classes and enabling is not an issue
    \item Shall we try to avoid a new interrupt shortly after EOI
    signalling (stack issue)?
    \item Do we need an ack signal to the peripheral device? I don't
    think so
\end{itemize}

BTW: restricting interarrival time in HW is not that new --
Wikipedia talks about it ;-)

Also IA-32e mode can specify task priorities that disable classes of
interrupts.

Check also:
\begin{itemize}
    \item LEON/SPARC docu
    \item NIOS docu
    \item MicroBlaze? ARM?
\end{itemize}

\subsection{Current State}

describe what is function of the current interrupt system (including
VHDL and Java source)


\chapter{Results}
\emph{Copy some stuff from the thesis (and JSA). Change this chapter
to new form... Update with actual numbers!}

\label{chap:results}
    
In this chapter, we present the evaluation results for JOP. In the
following section, the hardware platform that is used for
benchmarking is described. This is followed by a comparison of JOP's
resource usage with other soft-core processors. In
Section~\ref{sec:performance} the performance of a number of
different solutions for embedded Java is compared with embedded
application benchmarks. Comparison at bytecode level can be found in
\cite{jop:austrochip05}. This chapter concludes with a description of
real-world applications based on JOP.

\section{Hardware Platforms}

During the development of JOP and its predecessors, several
different FPGA boards were developed. The first experiments involved
using Altera FPGAs EPF8282,\linebreak[4] EPF8452, EPF10K10 and ACEX
1K30 on boards that were connected to the printer port of a PC for
configuration, download and communication. The next step was the
development of a stand-alone board with FLASH memory and static RAM.
This board was developed in two variants, one with an ACEX 1K50 and
the other with a Cyclone EP1C6 or EP1C12. Both boards are
pin-compatible and are used in commercial applications of JOP. The
Cyclone board is the hardware that is used for the following
evaluations.

This board is an ideal development platform for JOP. Static RAM and
FLASH are connected via independent buses to the FPGA. All unused
FPGA pins and the serial line are available via four connectors. The
FLASH can be used to store configuration data for the FPGA and
application program/data. The FPGA can be configured with a
ByteBlasterMV download cable or loaded from the FLASH (with a small
CPLD on board). As the FLASH is also connected to the FPGA, it can be
programmed from the FPGA. This allows for upgrades of the Java
program and even the processor core itself in the field. The board is
slightly different from other FPGA prototyping boards, in that its
connectors are on the bottom side. Therefore, it can be used as a
module (60~mm x 48~mm), i.e.\ as part of a larger board that contains
the periphery. The Cyclone board contains:
%
%\begin{samepage}
\begin{itemize}
\item Altera Cyclone EP1C6Q240 or EP1C12Q240
\item Step Down voltage regulator (1V5)
\item Crystal clock (20~MHz) at the PLL input (up to 640~MHz
    internal)
\item 512~KB FLASH (for FPGA configuration and program code)
\item 1~MB fast asynchronous RAM (15 ns)
\item Up to 128~MB NAND FLASH
\item ByteBlasterMV port
\item Watchdog with a LED
\item EPM7064 PLD to configure the FPGA from the FLASH on watchdog reset
\item Serial interface driver (MAX3232)
\item 56 general-purpose I/O pins
\end{itemize}
%\end{samepage}
%
The RAM consists of two independent 16-bit banks (with their own
address and control lines). Both RAM chips are on the bottom side of
the PCB, directly under the FPGA pins. As the traces are very short
(under 10~mm), it is possible to use the RAMs at full speed without
reflection problems. The two banks can be combined to form 32-bit
RAM or support two independent CPU cores. Pictures and the schematic
of the board can be found in Appendix~\ref{appx:cycore}.

\index{Baseio}

The expansion board Baseio hosts the CPU module and provides a
complete Java processor system with Internet connection. A step down
switching regulator with a large AC/DC input range supplies the core
board. All input and output pins are EMC/ESD-protected and routed to
large connectors (5.08~mm Phoenix). Analog comparators can be used to
build sigma-delta ADCs. For FPGA projects with a network connection,
a CS8900 Ethernet controller with an RJ45 connector is included on
the expansion board. Pictures and the schematic of the board can be
found in Appendix~\ref{appx:baseio}.


\section{Chip Area and Clock Frequency}

Cost is an important issue for embedded systems. The cost of a chip
is directly related to the die size (the cost per die is roughly
proportional to the square of the die area \cite{Hennessy02}).
Processors for embedded systems are therefore optimized for minimum
chip size. In this section, we will compare JOP with different
processors in terms of size. One major design objective in the
development of JOP was to create a small system that can be
implemented in a low-cost FPGA.


Table~\ref{tab:soft-cores} compares the resource consumption and
maximum clock frequency of a time-predictable processor (JOP), a
standard MIPS architecture (YARI), the LEON SPARC processor, and a
complex Java processor (picoJava), when implemented in the same FPGA
(Altera EP1C6/12 FPGA \cite{AltCyc}). For the resource comparison we
compare the consumption of the two basic structures of an FPGA; Logic
cells (LC) and embedded memory blocks. The maximum frequency for all
soft-core processors is in the same technology.

\index{Java processor!picoJava}

\begin{table}
  \begin{center}
    \begin{tabular}[t]{lrrr}
        \toprule
      Soft-core    & Logic Cells & Memory  & Frequency \\
        \midrule
      JOP          & 3,300       &  7.6 KB & 100~MHz    \\
      YARI         & 6,668       & 18.9 KB & 75~MHz     \\
      LEON3        & 7,978       & 10.9 KB & 35~MHz \\
      picoJava     & 27,560      & 47.6 KB & 40~MHz     \\
        \bottomrule
    \end{tabular}
  \end{center}
    \caption{Resource consumption and maximum operating frequency of JOP, YARI, LEON3, and picoJava.}
    \label{tab:soft-cores}
\end{table}

%        Lightfoot\footnotemark \cite{Lightfoot} & 3400 & 4 & 40 \\
%        NIOS A \cite{NIOS} & 1828 & 6.2 & 120 \\
%        NIOS B \cite{NIOS} & 2923 & 5.5 & 119 \\
%        SPEAR\footnotemark \cite{Delvai:ECRTS2003} & 1700 & 8 & 80 \\

JOP is configured with a 1~KB stack cache, 2~KB microcode ROM, and
4~KB method cache with 16 blocks. YARI is a MIPS compatible soft-core
\cite{cacao:yari:techrep}, optimized for FPGA technology. YARI is
configured with a 4-way set-associative instruction cache and a 4-way
set-associative write-through data cache. Both caches are 8~KB. LEON3
\cite{LEON}, the open-source implementation of the SPARC V8
architecture, has been ported to the exact same hardware that was
used for the JOP numbers. LEON3 is representative for a RISC
processor that is used in embedded real-time systems (e.g., by ESA
for space missions). The size a frequency numbers of picoJava-II
\cite{pJ1} are taken from an implementation in a Altera Cyclone-II
FPGA \cite{master:puffitsch}.



The streamlined architecture of JOP results in a small design: JOP is
half the size of the MIPS core YARI or the SPARC core LEON. Compared
with picoJava, JOP consumes about 12\% of the resources. JOP's size
allows implementing a CMP version of JOP even in a low-cost FPGA. The
simple pipeline of JOP achieves the highest clock frequency of the
three designs. From the frequency comparison we can estimate that the
maximum clock frequency of JOP in an ASIC will also be higher than a
standard RISC pipeline in an ASIC.

%The vendor independent and open-source RISC processor LEON can be
%clocked only with 35\% of JOP's frequency.

To prove that the VHDL code for JOP is as portable as possible, JOP
was also implemented in a Xilinx Spartan-3 FPGA \cite{Spartan3}. Only
the instantiation and initialization code for the on-chip memories is
vendor-specific, whilst the rest of the VHDL code can be shared for
the different targets. JOP consumes about the same LC count in the
Spartan device, but has a slower clock frequency (83~MHz).


%All configurations of JOP contain the on-chip microcode memory, the
%1~KB stack cache, a 1~KB method cache, a memory interface to a
%32-bit static RAM, and an 8-bit FLASH interface for the Java program
%and the FPGA configuration data. The minimum configuration
%implements multiplication and the shift operations in microcode. In
%the typical configuration, these operations are implemented as a
%sequential Booth multiplier and a single-cycle barrel shifter. The
%typical configuration also contains some useful I/O devices such as
%an UART and a timer with interrupt logic for multi-threading. The
%typical configuration of JOP consumes about 30\% of the LCs in a
%Cyclone EP1C6, thus leaving enough resources free for
%application-specific logic.

%As a reference, NIOS \cite{NIOS}, Altera's popular RISC soft-core, is
%also included in \tablename~\ref{tab_results_compare}. NIOS has a
%16-bit instruction set, a 5-stage pipeline and can be configured with
%a 16 or 32-bit datapath. Version A is the minimum configuration of
%NIOS. Version B adds an external memory interface, multiplication
%support, and a timer. Version A is comparable with the minimal
%configuration of JOP, and Version B with its typical configuration.



Table~\ref{tab:results:gate:count} provides gate count estimates for
JOP, picoJava, the aJile processor, and, as a reference, an old Intel
Pentium MMX processor. Equivalent gate count for an LC\footnote{The
factors are derived from the data provided for various processors in
Chapter~\ref{chap:related} and from the resource estimates in
\cite{jop:stack}.} varies between 5.5 and 7.4 -- we chose a factor of
6 gates per LC and 1.5 gates per memory bit for the estimated gate
count for JOP in the table. JOP is listed in the typical
configuration that consumes 3300 LCs. The Pentium MMX contains 4.5M
transistors \cite{pentium:mmx} that are equivalent to 1125K gates.

%Different LC/gate values:
%    from tow-level stack: 1LC = 5.4 gates
%    from ram stack: 1LC = 5.5 gates
%    from register stack: 1LC = 5.9 gates
%    Moon: 3660LCs - 27K gates + 3KB ROM + 1KB RAM: 1LC = 7.4 gates
%    Lightfoot: 3400LCs - 25K gates: 1LC = 7.4 gates
%    picoJava: ROM/RAM - 1Bit = 1 gate
%
%    JOP: 1831 LCs, 3.25KB: 1831*6 + 26624*1.5 = 11K + 39k
%    JOP: 2050*6 + 3.5*1024*8*1.5 = 12K + 43K
%    JOP: 3300*6 + 7.6*1024*8*1.5 = 20K + 93K
%    Pentium MMX: 4.5 Mio. Transistors -> 1100K

\begin{table}
    \centering
    \begin{tabular}{lrrr}
        \toprule
        Processor & \multicolumn{1}{c}{Core} & \multicolumn{1}{c}{Memory} & \multicolumn{1}{c}{Sum.} \\
        & \multicolumn{1}{c}{(gate)} & \multicolumn{1}{c}{(gate)} & \multicolumn{1}{c}{(gate)}\\
        \midrule
        JOP & 20K & 93K & 113K\\
        picoJava & 128K & 314K & 442K\\
        aJile & 25K & 912K & 937K\\
        Pentium MMX & & & 1125K\\
        \bottomrule
    \end{tabular}
    \caption{Gate count estimates for various processors}
    \label{tab:results:gate:count}
\end{table}

We can see from the table that the on-chip memory dominates the
overall gate count of JOP, and to an even greater extent, of the
aJile processor. The aJile processor is roughly the same size as the
Pentium MMX, and both are about 10 times larger than JOP.


\section{Performance} \label{sec:performance}

One important question remains: is a time-predictable processor slow?
We evaluate the average case performance of JOP by comparing it with
other embedded Java systems: Java processors from industry and
academia and two just-in-time (JIT) compiler based systems. For the
comparison we use \code{JavaBenchEmbedded},\footnote{Available at
\url{http://www.jopwiki.com/JavaBenchEmbedded}.} a set of open-source
Java benchmarks for embedded systems. \code{Kfl} and \code{Lift} are
two real-world applications, described in
Section~\ref{sec:applications}, adapted with a simulation of the
environment to run as stand-alone benchmarks. \code{UdpIp} is a
simple client/server test program that uses a TCP/IP stack written in
Java.

\begin{table}
    \centering
    \begin{tabular}{lD{.}{.}{2}D{.}{.}{2}D{.}{.}{2}}
        \toprule

 & \multicolumn{1}{c}{Kfl}
    & \multicolumn{1}{c}{UdpIp} & \multicolumn{1}{c}{Lift}\\
        \midrule
Cjip & 176 & 91 & \\
jamuth & 3400 & 1500 & \\
EJC & 9893 &  2882 & \\
SHAP & 11570 & 5764 & 12226 \\
aJ100 & 14148 & 6415 & \\
JOP & 19907 & 8837 & 18930 \\
picoJava & 23813 & 11950 & 25444 \\
CACAO/YARI & 39742 & 17702 & 38437 \\
        \bottomrule
    \end{tabular}
    \caption{Application benchmark performance on different Java systems.
    The table shows the benchmark results in
    iterations per second -- a higher
    value means higher performance.
    }
    \label{tab:results}
\end{table}


Table~\ref{tab:results} shows the raw data of the performance
measurements of different embedded Java systems for the three
benchmarks. The numbers are iterations per second whereby a higher
value represents better performance. Figure~\ref{fig:bench} shows the
results scaled to the performance of JOP.

The numbers for JOP are taken from an implementation in the Altera
Cyclone FPGA \cite{AltCyc}, running at 100~MHz. JOP is configured
with a 4~KB method cache and a 1~KB stack cache.

\begin{figure}[t]
    \centering
    \includegraphics[width=\excelwidth]{results/perf}
    \caption{Performance comparison of different Java systems with
    embedded application benchmarks. The results are scaled to the performance of JOP}
    \label{fig:bench}
\end{figure}

\index{Java processor!aJile} \index{Java processor!Cjip} \index{Java
processor!jamuth} \index{Java processor!SHAP}

Cjip \cite{Cjip} and aJ100 \cite{aJile} are commercial Java
processors, which are implemented in an ASIC and clocked at 80 and
100~Mhz, respectively. Both cores do not cache instructions. The
aj100 contains a 32~KB on-chip stack memory. jamuth
\cite{jamuth:jtres07} and SHAP \cite{shap} are Java processors that
are implemented in an FPGA. jamuth is the commercial version of the
Java processor Komodo \cite{komodo2003}, a research project for
real-time chip multithreading. jamuth is configured with a 4~KB
direct-mapped instruction cache for the measurements. The
architecture of SHAP is based on JOP and enhanced with a hardware
object manager. SHAP also implements the method cache
\cite{shap:mcache}. The benchmark results for SHAP are taken from the
SHAP website.\footnote{\url{http://shap.inf.tu-dresden.de/}, accessed
December, 2008} SHAP is configured with a 2~KB method cache and 2~KB
stack cache.

\index{Java processor!picoJava}

picoJava \cite{pJ1} is a Java processor developed by Sun. picoJava is
no longer produced and the second version (picoJava-II) was available
as open-source Verilog code. Puffitsch implemented picoJava-II in an
FPGA (Altera Cyclone-II) and the performance numbers are obtained
from that implementation \cite{master:puffitsch}. picoJava is
configured with a direct-mapped instruction cache and a 2-way
set-associative data cache. Both caches are 16~KB.

EJC \cite{EJC} is an example of a JIT system on a RISC processor
(32-bit ARM720T at 74~MHz). The ARM720T contains an 8~KB unified
cache. To compare JOP with a JIT based system in exactly the same
hardware we use the research JVM CACAO \cite{cacao} on top of the
MIPS compatible soft-core YARI \cite{cacao:yari}. YARI is configured
with a 4-way set-associative instruction cache and a 4-way
set-associative write-through data cache. Both caches are 8~KB.

The measurements do not provide a clear answer to the question of
whether a time-predictable architecture is slow. JOP is about 40\%
faster than the commercial Java processor aJ100, but picoJava is 30\%
faster than JOP and the JIT/RISC combination (CACAO/YARI) is about
2.7 times faster than JOP. We conclude that a time-predictable
solution will never be as fast in the average case as a solution
optimized for the average case.

%%\subsection{Notes for targets}
%%
%%\subsubsection{JStamp}
%%
%%\begin{verbatim}
%%    aJile project in \usr2\ajile\bench
%%    Sources from JOP target
%%    LowLevel.java from directory aJile
%%    Remove .class in ...\dist\classes
%%    Generate .class with \bat\ajc.bat in source directory
%%        destination is ...\dist\classes
%%        e.g. in ...\src\bench: ajc jbe/DoAll
%%    aJile ChemBuilder (bench.ajp) use COM1 for System.out
%%    Terminal on Serial A, 9600 baud
%%        Did not get the System.out in Charade (but it worked some
%%        time ago)
%%    Charade: Reset, File-Load \usr2\ajile\bench\build.bin
%%\end{verbatim}


\section{Applications}
\label{sec:applications}


Since the start of the development of JOP in late 2000 it has been
successfully deployed in several embedded control and automation
systems. The following section highlights three different industrial
real-time applications that are based on JOP. This section is based
on \cite{jop:app}; the first application is also described in
\cite{jop:wises03}.

Implementation of a processor in an FPGA is a little bit more
expensive than using an ASIC processor. However, additional
application logic, such as a communication controller or an AD
converter, can also be integrated into the FPGA. Integration of the
processor and the surrounding logic in the same reprogrammable chip
is a flexible solution: one can even produce the PCB before all logic
components are developed as the interconnection is programmed on-chip
and not routed on the PCB. For low-volume projects, as those
presented in this section, this flexibility reduces development cost
and therefore outweighs the cost of the FPGA device. It has to be
noted that low-cost FPGAs, that are big enough for JOP, are available
at \$11 for a single unit.

Furthermore, most embedded systems are implemented as distributed
systems and even very small and memory constraint devices need to
communicate. In control applications this communication has to be
performed under real-time constraints. We show in this section
different communication systems that are all based on simple
communication patterns.

\subsection{The Kippfahrleitung}
\label{sec:app:kfl}

The first commercial project where JOP had to prove that a Java
processor is a valuable option for embedded real-time systems was a
distributed motor control system.

In rail cargo, a large amount of time is spent on loading and
unloading of goods wagons. The contact wire above the wagons is the
main obstacle. Balfour Beatty Austria developed and patented a
technical solution, the so-called \emph{Kippfahrleitung}, to tilt up
the contact wire. \figurename~\ref{fig:results:kfl:mast} shows the
construction of the mechanical tilt system driven by an asynchronous
motor (just below the black tube). The little box mounted on the mast
contains the control system. The black cable is the network
interconnection of all control systems. In
\figurename~\ref{fig:results:kfl:mast2} the same mast is shown with
the contact wire tilted up.

\begin{figure}
    \centering
    \includegraphics[width=7cm]{results/results_kfl_mast1}
    \caption{A \emph{Kippfahrleitung} mast in down position}
    \label{fig:results:kfl:mast}
\end{figure}


The contact wire is tilted up on a distance of up to one kilometer.
For a maximum distance of 1~km the whole system consists of 12~masts.
Each mast is tilted by an asynchronous motor. However, the individual
motors have to be synchronized so the tilt is performed in a smooth
way. The maximum difference of the position of the contact wire is
10~cm. Therefore, a control algorithm has to slow down the faster
motors.

\begin{figure}
    \centering
    \includegraphics[width=4cm]{results/results_kfl_mast2}
    \caption{The mast in the up position with the tilted contact wire}
    \label{fig:results:kfl:mast2}
\end{figure}

\subsubsection{Hardware}

Each motor is controlled by its own embedded system (as seen in
\figurename~\ref{fig:results:kfl:mast}) by silicon switches. The
system measures the position of the arm with two end sensors and a
revolving sensor. It also supervises the supply voltage and the
amount of current through the motor. Those values are transmitted to
the base station.

The base station, shown in Figure~\ref{fig:results:base}, provides
the user interface for the operator via a simple display and a
keyboard. It is usually located at one end of the line. The base
station acts as master and controls the deviation of individual
positions during the tilt. In technical terms, this is a distributed,
embedded real-time control system, communicating over a shared
network. The communication bus (up to one kilometer) is attached via
an isolated RS485 data interface.

\begin{figure}
    \centering
    \includegraphics[width=5cm]{results/base}
    \caption{The base station with the operator interface}
    \label{fig:results:base}
\end{figure}


Although this system is not a mass product, there are nevertheless
cost constraints. Even a small FPGA is more expensive than a general
purpose CPU. To compensate for this, additional chips for the memory
and the FPGA configuration were optimized for cost. One standard
128~KB Flash is used to hold FPGA configuration data, the Java
program and a logbook. External main memory is reduced to 128~KB with
an 8-bit data bus. Furthermore, all peripheral components, such as
two UARTS, four sigma delta ADCs, and I/O ports are integrated in the
FPGA.

Five silicon switches in the power line are controlled by the
application program. A wrong setting of the switches due to a
software error could result in a short circuit. Simple logic in the
FPGA (coded in VHDL) can enforce the proper conditions for the
switches. The sigma-delta ADCs are used to measure the temperature of
the silicon switches and the current through the motor.


\subsubsection{Software Architecture}

The main task of the program is to measure the position using the
revolving sensor and to communicate with the base station under
real-time constraints. The conservative style of a cyclic executive
was chosen for the application. At application start all data
structures are allocated and initialized. In the mission phase no
allocation takes place and the cyclic executive loop is entered and
never exited. The simple infinite loop, unblocked at constant time
intervals, is shown in Listing~\ref{lst:results:main}. At the time
the application was developed no static WCET analysis tool for Java
was available. The actual execution time was measured and the maximum
values have been recorded regularly. The loop and communication
periods have been chosen to leave slack fur unexpected execution time
variations. However, the application code and the Java processor are
fully WCET analyzable, as shown later \cite{jop:wcet:jtres06}. The
application is used in Chapter~\ref{chap:wcet} as a test case for the
WCET analysis tool.

No interrupts or direct memory access (DMA) devices that can
influence the execution time are used in the simple system. All
sensors and the communication port are polled in the cyclic
executive.

\begin{lstlisting}[float,caption=The cyclic executive (simplified version),
label=lst:results:main]

private static void forever() {

    for (;;) {
        Msg.loop();
        Triac.loop();
        if (Msg.available) {
            handleMsg();
        } else {
            chkMsgTimeout();
        }
        handleWatchDog();
        Timer.waitForNextInterval();
    }
}
\end{lstlisting}

\subsubsection{Communication}

Communication is based on a master/slave model. Only the base station
(the master) is allowed to send a request to a single mast station.
This station is then required to reply within bounded time. The
master handles timeout and retry. If an irrecoverable error occurs,
the base station switches off the power for all mast stations,
including the power supplies to the motors. This is the safe state of
the whole system.

In a master/slave protocol no media access protocol is needed. In the
case of a failure in the slave that delays a message collision can
occur. The collision is detected by a violation of the message CRC.
Spurious collisions are tolerated due to the retry of the base
station. If the RS485 link is broken and only a subset of the mast
stations reply the base station, the base station switches of the
power supply for the whole system.

On the other hand the mast stations supervise the base station. The
base station is required to send the requests on a regular basis. If
this requirement is violated, each mast station switches off its
motor. The local clocks are not synchronized. The mast stations
measure the time elapsed since the last request from the base station
and locally switch off based on a timeout.

The maximum distance of 1~km determines the maximum baud rate of the
RS485 communication network. The resulting 12 masts on such a long
line determine the number of packets that have to be sent in one
master/slave round. Therefore, the pressure is high on the packet
length. The data is exchanged in small packets of four bytes,
including a one-byte CRC. To simplify the development, commands to
reprogram the Flash in the mast stations and to force a reset are
included. Therefore, it is possible to update the program, or even
change the FPGA configuration, over the network.

\subsection{The SCADA Device TeleAlarm}

TeleAlarm (TAL) is a typical remote terminal unit of a supervisory
control and data acquisition (SCADA) system. It is used by the Lower
Austria's energy provider EVN (electricity, gas, and heating) to
monitor the distribution plant. TeleAlarm also includes output ports
for remote control of gas valves.

\subsubsection{Hardware}

The TAL device consists of a CPU FPGA module and an I/O board. The
FPGA module contains an Altera Cyclone device, 1~MB static memory,
512~KB Flash, and 32~MB NAND Flash. The I/O board contains several
EMC protected digital input and output ports, two 20~mA input ports,
Ethernet connection, and a serial interface. Furthermore, the device
performs loading of a rechargeable battery to survive power down
failures. On power down, an important event for a energy provider, an
alarm is sent. The rechargeable battery is also monitored and the
device switches itself off when the minimal voltage threshold is
reached. This event is sent to the SCADA system before the power is
switched off.

The same hardware is also used for a different project: a lift
control in an automation factory in Turkey. The simple lift control
software is now used as a test case for WCET tool development (see
Chapter~\ref{chap:wcet}).

\subsubsection{Communication}

The communication between the TAL and the main supervisory control
system is performed with a proprietary protocol. On a value change,
the TAL sends the new data to the central system. Furthermore, the
remote units are polled by the central system at a regular base. The
TAL itself also sends the actual state regularly. TAL can communicate
via Internet/Ethernet, a modem, and via SMS to a mobile phone.

EVN uses a mixture of dial-up network and leased lines for the plant
communication. The dial-up modems are hosted by EVN itself. For
safety and security reason there is no connection between the control
network and the office network or the Internet.

\begin{figure}[t]
    \centering
    \includegraphics[scale=0.75]{results/tal}
    \caption{EVN SCADA system with the modem pool and TALs as remote terminal units}
    \label{fig:tal}
\end{figure}

\figurename~\ref{fig:tal} shows the SCADA system setup at EVN.
Several TALs are connected via modems to the central modem pool. The
modem pool itself is connected to the central server. It has to be
noted that there are many more TALs in the field than modems in the
pool. The communication is usually very short (several seconds) and
performed on demand and on a long regular interval. Not shown in the
figure are additional SCADA stations and other remote terminal units
from other manufacturers.

\subsection{Support for Single Track Railway Control}

Another application of JOP is in a communication device with soft
real-time properties -- Austrian Railways' (\"OBB) new support system
for single-track lines. The system helps the superintendent at the
railway station to keep track of all trains on the track. He can
submit commands to the engine drivers of the individual trains.
Furthermore, the device checks the current position of the train and
generates an alarm when the train enters a track segment without a
clearance.

At the central station all track segments are administered and
controlled. When a train enters a non-allowed segment all trains
nearby are warned automatically. This warning generates an alarm at
the locomotive and the engine driver has to perform an emergency
stop.

\figurename~\ref{fig:zlb} gives an overview of the system. The
display and command terminal at the railway station is connected to
the Intranet of the railway company. On the right side of the figure
a picture of the terminal that is connected to the Internet via GPRS
and to a GPS receiver is shown. Each locomotive that enters the track
is equipped with either one or two of those terminals.

\begin{figure}
    \centering
    \includegraphics[scale=0.75]{results/zlb}
    \caption{Support system for single track railway control for the Austrian railway company}
    \label{fig:zlb}
\end{figure}

It has to be noted that this system is not a safety-critical system.
The communication over a public mobile phone network is not reliable
and the system is not certified for safety. The intension is just to
\emph{support} the superintendent and the engine drivers.

\subsubsection{Hardware}

Each locomotive is equipped with a GPS receiver, a GPRS modem, and
the communication device (terminal). The terminal is a custom made
device. The FPGA module is the same as in TAL, only the I/O board is
adapted for this application. The I/O board contains several serial
line interfaces for the GPS receiver, the GPRS modem, debug and
download, and display connection. Auxiliary I/O ports connected to
relays are reserved for future use. A possible extension is to stop
the train automatically.

\subsubsection{Communication}

The current position of the train is measured with GPS and the
current track segment is calculated. The number of this segment is
regularly sent to the central station. To increase the accuracy of
the position, differential GPS correction data is transmitted to the
terminal. The differential GPS data is generated by a ground base
reference located at the central station.

The exchange of positions, commands, and alarm messages is performed
via a public mobile phone network (via GPRS). The connection is
secured via a virtual private network that is routed by the mobile
network provider to the railway company's Intranet. The application
protocol is command/response and uses UDP/IP as transport layer. Both
systems (the central server and the terminal) can initiate a command.
The system that sends the command is responsible for retries when no
response arrives. The deadline for the communication of important
messages is in the range of several seconds. After several
non-successful retries the operator is informed about the
communication error. He is than in charge to perform the necessary
actions.

Besides the application specific protocol a TFTP server is
implemented in the terminal. It is used to update the track data for
the position detection and to upload a new version of the software.
The flexibility of the FPGA and an Internet connection to the
embedded system allows to upgrade the software and even the processor
in the field.


\subsection{Communication and Common Design Patterns} \label{sec:comm}

Although we described embedded systems from quite different
application domains we have been facing similar challenges. All
systems are distributed systems and therefore need to communicate.
Furthermore, they are real-time systems (at least with soft
deadlines) and need to trust the communication and perform regular
checks. The issues in the design of embedded real-time systems are
quite similar in the three described projects. We found that several
design patterns are used over and over and describe three of them in
this section.


\subsubsection{Master/Slave Designs}

Developing safe embedded systems is an exercise in reducing
complexity. One paradigm to simplify embedded software development is
the master/slave pattern. Usually a single master is responsible to
initiate commands to the slaves. The single master is also
responsible to handle reliable communication. The master/slave
pattern also fits very well with the command/response pattern for the
communication.

\subsubsection{Dealing with Communication Errors}

Communication is not per se reliable. The RS485 link at the
Kippfahrleitung operates in a rough environment and electromagnetic
influences can lead to packet loss. The TAL system can suffer from
broken phone lines. The single track control system operates on a
public mobile phone network -- a network without any guarantees for
the GPRS data traffic. Therefore, we have to find solutions to
operate in a safe and controlled manner the distributed system
despite the chance of communication errors and failures.

Reliable communication is usually provided by the transport layer,
TCP/IP in the case of the Internet. However, the timeouts in TCP/IP
are way longer than the communication deadlines within control
systems. The approach in all three presented projects is to use a
datagram oriented protocol and perform the timeout and retransmission
at the application level. To simplify the timeout handling a simple
command and response pattern is used. One partner sends a command and
expects the response within a specific time bound. The command
initiator is responsible for retransmission after the timeout. The
response partner just needs to reply to the command and does not need
to remember the state of the communication. After several timeouts
the communication error is handled by an upper layer. Either the
operator is informed (in the SCADA and the railway control system) or
the whole system is brought into a safe state (in the motor control
project).

Communication errors are either transient or longer lasting.
Transient communication errors are lost packets due to network
overload or external electromagnetic influences. In a
command/response system the lost packets (either the command or the
response) is detected by a timeout on the response. A simple
retransmission of the command can correct those transient errors.

A longer network failure, e.g.\ caused by a wire break, can be
detected by too many transmission retries. In such a case the system
has to enter some form of safe state. Either the power is switched
off or a human operator has to be informed. The individual timeout
values and the number of retries depend, similar to thread periods,
on the controlled environment. In the Kippfahrleitung the maximum
timeout is in the millisecond range, whereas in the SCADA system the
timeout is several minutes.

\subsubsection{Software Update}

Correction of implementation bugs during development can be very
costly when physical access to the embedded system is necessary for a
software update. Furthermore, a system is usually never really
finished. When the system is in use the customer often finds new ways
to enhance the system or requests additional features.

Therefore, an important feature of a networked embedded system is a
software and parameter update in the field. In the first project the
software update is performed via a home-made protocol. The other
projects use the Internet protocol to some extent and therefore TFTP
is a natural choice. TFTP is a very simple protocol that can be
implemented within about 100 lines of code. It is applicable even in
very small and resource constraint embedded devices.

\subsection{Discussion} \label{sec:lessions}

Writing embedded control software in Java is still not very common
due to the lack of small and efficient implementations of the JVM.
Our Java processor JOP is a solution for some embedded systems.

Using Java as the implementation language was a pleasure during
programming and debugging. We did not waste many hours to hunt for
pointer related bugs. The stricter (compared to C) type system of
Java also catches many more programming errors at compile time.
However, when using Java in a small embedded system one should not
expect that a full blown Java library is available. Almost all of the
code had to be written without library support. Embedded C
programmers are aware of that fact, but Java programmers are new in
the embedded domain and have to learn the difference between a PC and
a 1~MB memory embedded system.

Up to date FPGAs in embedded control systems are only used for
auxiliary functions or to implement high-performance DPS algorithm
directly in hardware. Using the FPGA as the main processor is still
not very common. However, combining the main processor with some
peripheral devices in the same chip can simplify the PCB layout and
also reduce the production cost. Furthermore, a field-reprogrammable
hardware device offers a great deal of flexibility: When some part of
the software becomes the bottleneck, an implementation of that
function in hardware can be a solution. Leaving some headroom in the
logic resources can extend the lifetime of the product.

For a prototype, JOP has been attached to a time-triggered
network-on-chip \cite{jop:ttnoc}. It would be an interesting exercise
to implement a JOP based node in a time-triggered distributed system
as proposed by \cite{journals/pieee/KopetzB03}. The combination of a
real-time Java processor and a real-time network can ensure real-time
characteristics for the whole system.


\section{Summary}

In this chapter, we presented an evaluation of JOP. We have seen
that JOP is the smallest hardware realization of the JVM available
to date. Due to the efficient implementation of the stack
architecture, JOP is also smaller than a \emph{comparable} RISC
processor in an FPGA. Implemented in an FPGA, JOP has the highest
clock frequency of all known Java processors.

We compared JOP against several embedded Java systems. JOP is about
40\% faster than the commercial Java processor aJ100, but picoJava
and a JIT/RISC combination are faster than JOP. These results show
that a time-predictable architecture does not need to be slow, but
will never be as fast as an architecture optimized for average case
performance.

Furthermore, we have presented three industrial applications
implemented in Java on an embedded, real-time Java processor. All
projects included custom designed hardware (digital functions) and
the central computation unit implemented in a single FPGA. The
applications are written in pure Java without the need for native
methods in C. Java proved to be a productive implementation language
for embedded systems. Usage of JOP in four real-world applications
showed that the processor is mature enough to be used in commercial
projects.


\chapter{Related Work}
\label{chap:related}

\emph{Update with JSA writing and remove old, not used stuff (or
shorten it)}
    
Two different approaches can be found to improve Java bytecode
execution by hardware. The first type operates as a Java coprocessor
in conjunction with a general-purpose microprocessor. This
coprocessor is placed in the instruction fetch path of the main
processor and translates Java bytecodes to sequences of instructions
for the host CPU or directly executes basic Java bytecodes. The
complex instructions are emulated by the main processor. Java chips
in the second category replace the general-purpose CPU. All
applications therefore have to be written in Java. While the first
type enables systems with mixed code capabilities, the additional
component significantly raises costs.
\tablename~\ref{tab_related_proc} provides an overview of the
described Java hardware.

Blank fields in the table indicate that the information is not
available or not applicable (e.g. for simulation-only projects).
Minimum CPI is the number of clock cycles for a simple instruction
such as \code{nop}. One entry, the TINI system, is not a real Java
hardware, but is included in the table since it is often
incorrectly\footnote{TINI is a standard interpreting JVM running on
an enhanced 8051 processor.} cited as an embedded Java processor.


%\begin{table}
%    \centering
%{\footnotesize
%\begin{tabular}
%    {|>{\bfseries}p{1.4cm}|m{1.3cm}|>{\raggedright}m{1.3cm}|>{\raggedright}m{1.3cm}
%    |r|>{\raggedright}m{1.35cm}|r|m{1.6cm}|}
%
%    \hline
%         & Type & Target  & Size & Speed & Java     & Min. & Remarks \\
%         &      & technology &      & [MHz] & standard & CPI  &         \\
%    \hline
%    Hard-Int & Translation & Simulation only &  &  &  &  &  \\
%    \hline
%    DELFT & Translation & Simulation only &  &  &  &  &  \\
%    \hline
%    JIFFY & Translation & Xilinx FPGA & 3800 LCs, 1KB RAM &  &  &  &  \\
%    \hline
%    Jazelle & Co-processor & ASIC 0.18$\mu$ & 12K gates & 200 &  &  & Integration with ARM \\
%    \hline
%    JSTAR & Co-processor & ASIC 0.18$\mu$ Softcore & 30K gates + 7KB & 104 & J2ME CLDC\footnotemark[2] &  &  \\
%    \hline
%    TINI & Software JVM & Enhanced 8051 clone &  &  & Java 1.1 subset &  & A small Java system for embedded applications. \\
%    \hline
%    picoJava & Processor & No realization & 128K gates + memory &  & Full & 1 &  \\
%    \hline
%    aJile & Processor & ASIC 0.25$\mu$ & 25K gates + ROM & 100 & J2ME CLDC\footnotemark[2] &  &  \\
%    \hline
%    Cjip & Processor & ASIC 0.35$\mu$ & 70K gates + ROM, RAM & 67 & J2ME CLDC\footnotemark[2] & 6 & Rewriteable microcode \\
%    \hline
%    Ignite & Stack processor & Xilinx FPGA & 9700 LCs &  &  &  &  \\
%    \hline
%    Moon & Processor & Altera FPGA & 3660 LCs, 4KB RAM &  &  &  &  \\
%    \hline
%    Lightfoot & Processor & Xilinx FPGA & 3400 LCs & 40 &  &  &  \\
%    \hline
%    LavaCORE & Processor & Xilinx FPGA & 3800 LCs 30K gates & 20 &  &  &  \\
%    \hline
%    Komodo & Processor & Xilinx FPGA & 2600 LCs & 20 & Subset: 50 bytecodes & 4 &  \\
%    \hline
%    FemtoJava & Processor & Altera Flex 10K & 2000 LCs & 4 & Subset: 69 bytecodes, 16-bit ALU & 3 & Application specific Java processor. \\
%    \hline
%    JSM & Processor & Xilinx FPGA &  & 3.5 & Java Card &  & \cite{JSM01} \\
%    \hline
%%    \hline
%%    JOP & Processor & Altera, Xilinx FPGA & 2100 LCs + 3KB RAM & 100 & J2ME CLDC & 1 & Typical configuration on a Cyclone FPGA \\
%
%\end{tabular}
%}
%    \caption{Java hardware}
%    \label{tab_related_proc}
%\end{table}


\begin{table}
    \centering
{\footnotesize
\begin{tabular}
    {|>{\bfseries}p{1.6cm}|m{1.5cm}|>{\raggedright}m{1.6cm}|>{\raggedright}m{1.6cm}
    |r|>{\raggedright}m{1.5cm}|r|}

    \hline
         & Type & Target  & Size & Speed & Java     & Min. \\
         &      & technology &      & [MHz] & standard & CPI  \\
    \hline
    Hard-Int & Translation & Simulation only &  &  &  &  \\
    \hline
    DELFT & Translation & Simulation only &  &  &  &    \\
    \hline
    JIFFY & Translation & Xilinx FPGA & 3800 LCs, 1KB RAM &  &  &   \\
    \hline
    Jazelle & Co-processor & ASIC 0.18$\mu$ & 12K gates & 200 &  &  \\
    \hline
    JSTAR & Co-processor & ASIC 0.18$\mu$ Softcore & 30K gates + 7KB & 104 & J2ME CLDC\footnotemark[2] &  \\
    \hline
    TINI & Software JVM & Enhanced 8051 clone &  &  & Java 1.1 subset &   \\
    \hline
    picoJava & Processor & No realization & 128K gates + memory &  & Full & 1  \\
    \hline
    aJile & Processor & ASIC 0.25$\mu$ & 25K gates + ROM & 100 & J2ME CLDC\footnotemark[2] &   \\
    \hline
    Cjip & Processor & ASIC 0.35$\mu$ & 70K gates + ROM, RAM & 67 & J2ME CLDC\footnotemark[2] & 6 \\
    \hline
    Ignite & Stack processor & Xilinx FPGA & 9700 LCs &  &  &   \\
    \hline
    Moon & Processor & Altera FPGA & 3660 LCs, 4KB RAM &  &  &   \\
    \hline
    Lightfoot & Processor & Xilinx FPGA & 3400 LCs & 40 &  &   \\
    \hline
    LavaCORE & Processor & Xilinx FPGA & 3800 LCs 30K gates & 20 &  &   \\
    \hline
    Komodo & Processor & Xilinx FPGA & 2600 LCs & 20 & Subset: 50 bytecodes & 4  \\
    \hline
    FemtoJava & Processor & Altera Flex 10K & 2000 LCs & 4 & Subset: 69 bytecodes, 16-bit ALU & 3 \\
    \hline
    JSM \cite{JSM01} & Processor & Xilinx FPGA &  & 3.5 & Java Card &   \\
    \hline
%    \hline
%    JOP & Processor & Altera, Xilinx FPGA & 2100 LCs + 3KB RAM & 100 & J2ME CLDC & 1 & Typical configuration on a Cyclone FPGA \\

\end{tabular}
}
    \caption{Java hardware}
    \label{tab_related_proc}
\end{table}

\footnotetext[2]{J2ME CLDC stands for Java2 Micro Edition, Connected
Limited Device Configuration, which is described in
Section~\ref{subsec:cldc}.}


% Change this: \emph{JOP is included with a typical configuration as a
% reference. Further details of the resource usage of JOP is described
% in Section~xxx.}


\section{Hardware Translation and Coprocessors}

The simplest enhancement for Java is a translation unit, which
substitutes the switch statement of an interpreter JVM (bytecode
decoding) through hardware and/or translates simple bytecodes
to a sequence of RISC instructions on the fly.

A standard JVM interpreter contains a loop with a large switch
statement that decodes the bytecode (see
Listing~\ref{lst:intro:java:intprt}). This switch statement is
compiled to an indirect branch. The destinations of these indirect
branches change frequently and do not benefit from branch-prediction
logic. This is the main overhead for simple bytecodes on modern
processors. The following approaches enhance the execution of Java
programs on a standard processor through the substitution of the
memory read and switch statement with bytecode fetch and decode
through hardware.


\subsection{Hard-Int}

Radhakrichnan \cite{HardInt} proposes an additional architecture for
a standard RISC processor to speed up a JVM interpreter. The
architecture, called Hard-Int, is placed between the cache and
instruction fetch of the RISC processor. Simple Java bytecodes are
translated to a sequence of RISC instructions. For native RISC code,
the unit is bypassed. This architecture implements the expensive
switch statement of a typical interpreter in hardware. A simulation
of a SPARC processor with four execution units shows a speedup by a
factor of 2.6 over JDK 1.2 JIT with SPECjvm98. Since the architecture
is only evaluated in a software simulation, the impact of the
inserted hardware on the clock frequency of the RISC processor is
unknown. No estimation of the additional hardware cost for the
translation unit is given.


\subsection{DELFT-JAVA Engine}

In his thesis \cite{DELFT}, Glossner describes a processor for
multimedia applications in Java. A RISC processor is extended with
DSP capabilities and Java specific instructions. This combination
results in a very complex processor. Simple JVM instructions are
dynamically translated to the DELFT instruction set. However, no
explanation is given as to how this is done. A new
register-addressing mode, indirect register addressing with auto
increment or decrement, provides support for stack caching in the
register file. The translation of JVM bytecode to the DELFT
instruction set maps stack-based dependencies into pipeline
dependencies. The author expects that these dependencies can be
resolved with standard techniques such as register renaming and
out-of-order execution. To accelerate dynamic linking, a link
translation buffer cache resolves entries from the constant pool.


The processor is validated through a C++ model. An experiment with a
synthetic benchmark (vector multiplication) compared a stack machine
with an ideal register machine. The ideal register machine performs
register renaming and out-of-order execution on multiple execution
units. The achieved speedup in this experiment was 2.7. The
high-level simulation model is more a proof of concept and no
estimation is given for the resources needed to implement this
complex processor. Since only a restricted subset of the JVM was
simulated, no Java applications could be used to estimate the
expected speedup.


\subsection{JIFFY}

An interesting approach to enhance Java execution in embedded
systems is presented in Acher's thesis \cite{JIFFY}. He states that
JIT-compilation in software is not possible on most embedded devices
because of resource constraints. JIFFY, a JIT in an FPGA, is
proposed as a solution to this problem. The compilation is done in
the following steps:

The Java bytecode is translated into an intermediate language with
three registers and a stack. The reduction to three registers is due
to the fact that bytecodes are using a maximum of three stack
operands, and it simplifies translation to CISC-architectures with a
low register count. In the next step, this instruction sequence,
which is still stack-based, is optimized. The main effect of this
optimization is to transform stack-based operations into
register-based operations. These optimized instructions in the
intermediate language are translated to native instructions of the
target architecture in the last step.

The quality of the generated code was tested with software versions
of JIFFY for a CISC (80586) and a RISC (Alpha 21164) architecture.
The resulting code is about 1.1 to 7.5 times faster than interpreting
Java bytecode on the x86 architecture. The speedup is similar to
Sun's first JIT compiler (sunwjit in JDK 1.1). The compilation time
is estimated to be 50 to 70 clock cycles for one bytecode. This is 10
times faster than the efficient CACAO JIT \cite{Krall98}. A first
prototype implementation in an FPGA used 3800 LCs and 8KBits RAM (80
\% of a Xilinx XC2S200).


\subsection{Jazelle}

Jazelle \cite{Jazelle} is an extension of the ARM 32-bit RISC
processor, similar to the Thumb state (a 16-bit mode for reduced
memory consumption). The Jazelle coprocessor is integrated into the
same chip as the ARM processor. The hardware bytecode decoder logic
is implemented in less than 12K gates. It accelerates, according to
ARM, some 95\% of the executed bytecodes. 140 bytecodes are executed
directly in hardware, while the remaining 94 are emulated by
sequences of ARM instructions. This solution also uses code
modification with \textit{quick} instructions to substitute certain
object-related instructions after link resolution. All Java
bytecodes, including the emulated sequences, are re-startable to
enable a fast interrupt response time.


A new ARM instruction puts the processor into the Java state.
Bytecodes are fetched and decoded in two stages, compared to a single
stage in ARM state. Four registers of the ARM core are used to cache
the top stack elements. Stack spill and fill is handled automatically
by the hardware. Additional registers are reused for the Java stack
pointer, the variable pointer, the constant pool pointer and locale
variable 0 (the \textit{this} pointer in methods). Keeping the
complete state of the Java mode in ARM registers simplifies its
integration into existing operating systems.

\subsection{JSTAR, JA108}

Nozomi's JA108 \cite{JSTAR}, previously known as JSTAR, is a Java
coprocessor that sits between the native processor and the memory
subsystem. JA108 fetches Java bytecodes from memory and translates
them into native microprocessor instructions. JA108 acts as a
pass-through when the core processor's native instructions are being
executed. The JA108 is targeted for use in mobile phones to increase
performance of Java multimedia applications. The coprocessor is
available as standalone package or with included memory and can be
operated up to 104~MHz. The resource usage for the JSTAR is known to
be about 30K gates plus 45~Kbits for the microcode.

\subsection{A Co-Designed Virtual Machine}

In his thesis \cite{KentPhD}, Kent proposes an interesting new form
of Java coprocessor. He investigates hardware/software co-design for
a JVM within the context of a desktop workstation. The execution of
the JVM is partitioned between an FPGA and the host processor. An
FPGA board with local memory is connected via the PCI bus to the
host. This solution provides an add-on accelerator without changing
the system. Moreover, as the FPGA can be configured for a different
task, the add-on hardware can be used for non-Java applications.

The critical issue in this approach is the partitioning of the JVM
and the memory regions between hardware and software. Not all Java
bytecodes can be executed in hardware. All object-oriented bytecodes
are performed in software. However, once these bytecodes are replaced
by their \textit{quick} variants, some of them can then be executed
in hardware. The most accessed data structures, i.e.\ the method's
bytecode, execution stack, and local variables, are placed in the
FPGA board memory. The constant pool and the heap reside in the PC's
main memory. The software part of the JVM decides during runtime
which instruction sequences can be executed by the hardware. Due to
the high cost of a context switch, this is a critical decision. Kent
explored various algorithms with different block sizes to find the
optimum partitioning of the instructions between the host processor
and the FPGA. Tests with small benchmarks on a simulation showed
performance gains by a factor of 6 to 11, when compared with an
interpreting JVM. Kent is now working on the concurrent use of the
FPGA and the host system to execute Java applications. Additional
performance increases are expected for multi-threaded applications.

In our view, there are two potential problems with this approach.
Firstly, the execution context for the hardware is too small. As
\code{invokevirtual} and the quick version are implemented in the
software partition, the maximum context is one method body. As shown
in Section~\ref{sec:bench:jvm:methods}, Java methods are usually
small (about 30\% are less than 9 bytes long), resulting in many
context switches. The second issue is the raw speedup, without
communication overhead, of the FPGA solution. This speedup is stated
to be around of 10 times greater, with the same clock frequency.
However, FPGA clock rate will never reach the clock rate of a
general-purpose processor. With a meaningful design, such as a CPU,
the clock rate of an FPGA is about 20 to 50 times lower. However,
everyone who uses an FPGA as target technology for a processor
design faces this problem. It is better not to try to compete
against mainstream PC technology.

\section{Java Processors}

Java Processors are primarily used in an embedded system. In such a
system, Java is the native programming language and all operating
system related code, such as device drivers, are implemented in
Java. Java processors are simple or extended stack architectures
with an instruction set that resembles more or less the bytecodes
from the JVM.

\subsection{picoJava}
\label{subsec:related:picojava}

Sun's picoJava is the Java processor most often cited in research
papers. It is used as a reference for new Java processors and as the
basis for research into improving various aspects of a Java
processor. Ironically, this processor was never released as a
product by Sun. After Sun decided to not produce picoJava in
silicon, Sun licensed picoJava to Fujitsu, IBM, LG Semicon and NEC.
However, these companies also did not produce a chip and Sun finally
provided the full Verilog code under an open-source license.

Sun introduced the first version of picoJava \cite{pJ1} in 1997. The
processor was targeted at the embedded systems market as a pure Java
processor with restricted support of C. picoJava-I contains four
pipeline stages. A redesign followed in 1999, known as picoJava-II.
This is the version described below. picoJava-II is now freely
available with a rich set of documentation \cite{pjMicroArch,
pjProgRef}.

Simple Java bytecodes are directly implemented in hardware; most of
them execute in one to three cycles. Other performance critical
instructions, for instance invoking a method, are implemented in
microcode. picoJava traps on the remaining complex instructions, such
as creation of an object, and emulates the instruction. To access
memory, internal registers, and for cache management, picoJava
implements 115 extended instructions with 2-byte opcodes. These
instructions are necessary to write system-level code to support the
JVM.

Traps are generated on interrupts, exceptions, and for instruction
emulation. A trap is rather expensive and has a minimum overhead of
16 clock cycles:

\begin{verbatim}
    6 clocks trap execution
    n clocks trap code
    2 clocks set VARS register
    8 clocks return from trap
\end{verbatim}


This minimum value can only be achieved if the trap table entry is
in the data cache and the first instruction of the trap routine is
in the instruction cache. The worst-case interrupt latency is 926
clock cycles \cite{pjProgRef}.

\begin{figure*}
    \centering
%    \includegraphics[scale=\picscale]{related/related_picojava}
    \includegraphics[scale=0.85]{related/related_picojava}
    \caption[Block diagram of picoJava-II]
    {Block diagram of picoJava-II (from \cite{pjMicroArch})}
    \label{fig_related_picojava}
\end{figure*}

\figurename~\ref{fig_related_picojava} shows the major function
units of picoJava. The integer unit decodes and executes picoJava
instructions. The instruction cache is direct-mapped, while the data
cache is two-way set-associative, both with a line size of 16 bytes.
The caches can be configured between 0 and 16 Kbytes. An instruction
buffer decouples the instruction cache from the decode unit. The FPU
is organized as a microcode engine with a 32-bit datapath supporting
single- and double-precision operations. Most single-precision
operations require four cycles. Double-precision operations require
four times the number of cycles as single-precision operations. For
low-cost designs, the FPU can be removed and the core traps on
floating-point instructions to a software routine to emulate these
instructions. picoJava provides a 64-entry stack cache as a register
file. The core manages this register file as a circular buffer, with
a pointer to the top of stack. The stack management unit
automatically performs spill to and fill from the data cache to
avoid overflow and underflow of the stack buffer. To provide this
functionality the register file contains five memory ports.
Computation needs two read ports and one write port, the concurrent
spill and fill operations the two additional read and write ports.
The processor core consists of following six pipeline stages:
%
\begin{description}

\item[Fetch:]
Fetch 8 bytes from the instruction cache or 4 bytes from the bus
interface to the 16-byte-deep prefetch buffer.

\item[Decode:]
Group and precode instructions (up to 7 bytes) from the prefetch
buffer. Instruction folding is performed on up to four bytecodes.

\item[Register:]
Read up to two operands from the register file (stack cache).

\item[Execute:]
Execute simple instructions in one cycle or microcode for
multi-cycle instructions.

\item[Cache:]
Access the data cache.

\item[Writeback:]
Write the result back into the register file.

\end{description}
%
The integer unit together with the stack unit provides a mechanism,
called instruction folding, to speed up common code patterns found
in stack architectures, as shown in
\figurename~\ref{fig_related_folding}.
%
\begin{figure}
A Java instruction
    \begin{verbatim}
    c = a + b;
    \end{verbatim}
translates to the following bytecodes:
    \begin{verbatim}
    iload_1
    iload_2
    iadd
    istore_3
    \end{verbatim}
    \caption{A common folding pattern that is executed in a single cycle}
    \label{fig_related_folding}
\end{figure}
%
When all entries are contained in the stack cache, the picoJava core
can fold these four instructions into one RISC-style single cycle
operation.

picoJava contains a simple mechanism to speed-up the common case for
monitor enter and exit. The two low order bits of an object
reference are used to indicate the lock holding or a request to a
lock held by another thread. These bits are examined by
\code{monitorenter} and \code{monitorexit}. For all other operations
on the reference, these two bits are masked out by the hardware.
Hardware registers cache up to two locks held by a single thread.

To efficiently implement a generational or an incremental garbage
collector, picoJava offers hardware support for write barriers
through memory segments. The hardware checks all stores of an object
reference if this reference points to a different segment (compared
to the store address). In this case, a trap is generated and the
garbage collector can take the appropriate action. Additional two
reserved bits in the object reference can be used for a write barrier
trap.

The architecture of picoJava is a stack-based CISC processor
implementing 341 different instructions \cite{pJ1} and is the most
complex Java processor available. The processor can be implemented
\cite{Sekar2000} in about 440K gates (128K for the logic and 314K
for the memory components: 284x80 bits microcode ROM, 2x192x64 bits
FPU ROM and 2x16~KB caches). We have implemented picoJava-II in a
Cyclone-II FPGA \cite{pjfpga} and the design consumed 27500~LCs and
48~KB on-chip memory.

\subsection{aJile JEMCore}

aJile's JEMCore is a direct-execution Java processor that is
available as both an IP core and a stand alone processor
\cite{aJile, 880720}. It is based on the 32-bit JEM2 Java chip
developed by Rockwell-Collins. JEM2 is an enhanced version of JEM1,
created in 1997 by the Rockwell-Collins Advanced Architecture
Microprocessor group. Rockwell-Collins originally developed JEM for
avionics applications by adapting an existing design for a
stack-based embedded processor. Rockwell-Collins decided not to sell
the chip on the open market. Instead, it licensed the design
exclusively to aJile Systems Inc., which was founded in 1999 by
engineers from Rockwell-Collins, Centaur Technologies, Sun
Microsystems, and IDT.


The core contains 24 32-bit wide registers. Six of them are used to
cache the top elements of the stack. The datapath consists of a
32-bit ALU, a 32-bit barrel shifter and the support for floating
point operations (disassembly/assembly, overflow and NaN detection).
The control store is a 4K by 56 ROM to hold the microcode that
implements the Java bytecode. An additional RAM control store can be
used for custom instructions. This feature is used to implement the
basic synchronization and thread scheduling routines in microcode. It
results in low execution overhead with a thread-to-thread yield in
less than one $\mu$s (at 100~MHz). An optional Multiple JVM Manager
(MJM) supports two independent, memory protected JVMs. The two JVMs
execute time-sliced on the processor. According to aJile, the
processor can be implemented in 25K gates (without the microcode
ROM). The MJM needs additional 10K gates.

Two silicon versions of JEM exist today: the aJ-80 and the aJ-100.
Both versions comprise a JEM2 core, the MJM, 48KB zero wait state
RAM and peripheral components, such as timer and UART. 16KB of the
RAM is used for the writable control store. The remaining 32KB is
used for storage of the processor stack. The aJ-100 provides a
generic 8-bit, 16-bit or 32-bit external bus interface, while the
aJ-80 only provides an 8-bit interface. The aJ-100 can be clocked up
to 100MHz and the aJ-80 up to 66MHz. The power consumption is about
1mW per MHz.

Since aJile was a member of the Real-Time for Java Expert Group, the
complete RTSJ will be available in the near future. One nice feature
of this processor is its availability. A relatively cheap
development system, the JStamp \cite{JStamp}, was used to compare
this processor with JOP.

\subsection{Cjip}

The Cjip processor \cite{Imsys, Cjip} supports multiple instruction
sets, allowing Java, C, C++, and assembler to coexist. Internally,
the Cjip uses 72 bit wide microcode instructions, to support the
different instruction sets. At its core, Cjip is a 16-bit CISC
architecture with on-chip 36KB ROM and 18KB RAM for fixed and
loadable microcode. Another 1KB RAM is used for eight independent
register banks, string buffer and two stack caches. Cjip is
implemented in 0.35-micron technology and can be clocked up to 66MHz.
The logic core consumes about 20\% of the 1.4-million-transistor
chip. The Cjip has 40 program controlled I/O pins, a high-speed 8 bit
I/O bus with hardware DMA and an 8/16 bit DRAM interface.


The JVM is implemented largely in microcode (about 88\% of the Java
bytecodes). Java thread scheduling and garbage collection are
implemented as processes in microcode. Microcode is also used to
implement virtual peripherals such as watchdog timers, display and
keyboard interfaces, sound generators, and multimedia codecs.

Microcode instructions execute in two or three cycles. A JVM bytecode
requires several microcode instructions. The Cjip Java instruction
set and the extensions are described in detail in \cite{CjipRef}. For
example: a bytecode \code{nop} executes in 6 cycles while an
\code{iadd} takes 12 cycles. Conditional bytecode branches are
executed in 33 to 36 cycles. Object oriented instructions, such
\code{getfield}, \code{putfield}, or \code{invokevirtual} are not
part of the instruction set.


\subsection{Ignite, PSC1000}

The PSC1000 \cite{IGNITE} is a stack processor, based on ShBoom
(originally designed by Chuck Moore \cite{ShBoom}), designed for high
speed Forth applications. The PSC1000 was later renamed to Ignite and
promoted as a Java processor, though it has roots in Forth. The
instruction set, called ROSC (Removed Operand Set Computer), is
different from Java bytecodes. A small JVM driver converts Java
bytecode into the stack instruction set of the processor.


The processor contains two on-chip stacks, as usual in Forth
processors \cite{Koopman89}, and additional 16 global registers. The
first elements of the stacks are directly accessible. The bottleneck
of instruction fetching without a cache is avoided by fetching up to
four 8-bit instructions from a 32-bit memory. To simplify instruction
decoding, immediate values and branch offsets are placed
right-aligned in such an instruction group. The PSC1000 is available
as an ASIC at 80~MHz and as a soft-core for Xilinx FPGAs (9700 LCs).


\subsection{Moon}

Vulcan ASIC's Moon processor is an implementation of the JVM to run
in an FPGA. The execution model is the often-used mix of direct,
microcode and trapped execution. As described in \cite{Vulcan2000},
a simple stack folding is implemented in order to reduce five memory
cycles to three for instruction sequences like
\textit{push-push-add}. The first version of Moon uses 3.840 LCs and
10 embedded memory blocks in an Altera FPGA. The Moon2 processor
\cite{Vulcan2003} is available as an encrypted HDL source for Altera
FPGAs (22\% of an APEX 20K400E equates to 3660 LCs) or as VHDL or
Verilog source code. The minimum silicon cost is given as 27K gates
plus 3KB ROM and 1KB single port RAM. The single port RAM is used to
implement 256 entries of the stack.


\subsection{Lightfoot}

The Lightfoot 32-bit core \cite{Lightfoot} is a hybrid 8/32-bit
processor based on the Harvard architecture. Program memory is 8 bits
wide and data memory is 32 bits wide. The core contains a 3-stage
pipeline with an integer ALU, a barrel shifter, and a 2-bit multiply
step unit. There are two different stacks with the top elements
implemented as registers and memory extension. The data stack is used
to hold temporary data -- it is not used to implement the JVM stack
frame. As the name implies, the return stack holds return addresses
for subroutines and it can be used as an auxiliary stack. The TOS
element is also used to access memory. The processor architecture
specifies three different instruction formats: soft bytecodes,
non-returnable instructions, and single-byte instructions that can be
folded with a return instruction. Soft bytecode instructions cause
the processor to branch to one of 128 locations in low program
memory, where the implementation of the soft bytecodes resides. This
operation has a single cycle overhead and the address of the
following instruction is pushed onto the return stack. The
instruction set implies that it is optimized to write an efficient
interpreted JVM.


The core is available in VHDL and can be implemented in less than
30K gates. According to DCT, the performance is typically 8 times
better than RISC interpreters running at the same clock speed. The
core is also provided as an EDIF netlist for dedicated Xilinx
devices. It needs 1710 CLBs (= 3400 LCs) and 2 Block RAMs. In a
Vertex-II (2V1000-5), it can be clocked up to 40MHz.


\subsection{LavaCORE}

LavaCORE \cite{LavaCORE} is another Java processor targeted at
Xilinx FPGA architectures. It implements a set of instructions in
hardware and firmware. Floating-point operations are not
implemented. A 32x32-bit dual-ported RAM implements a register-file.
For specialized embedded applications, a tool is provided to analyze
which subset of the JVM instructions is used. The unused
instructions can be omitted from the design. The core can be
implemented in 1926 CLBs (= 3800 LCs) in a Virtex-II (2V1000-5) and
runs at 20MHz.

\subsection{Komodo}
\label{subsec:related:komodo}

Komodo \cite{Zulauf00} is a multithreaded Java processor with a
four-stage pipeline. It is intended as a basis for research on
real-time scheduling on a multithreaded microcontroller
\cite{komodo2003}. Simple bytecodes are directly implemented, while
more complex bytecodes, such as \code{iaload}, are implemented as a
microcode sequence. The unique feature of Komodo is the instruction
fetch unit with four independent program counters and status flags
for four threads. A priority manager is responsible for hardware
real-time scheduling and can select a new thread after each bytecode
instruction.


The first version of Komodo in an FPGA implements a very restricted
subset of the JVM (only 50 bytecodes). The design can be clocked at
20MHz. However, the pipeline runs at 5MHz for single cycle external
memory access and three-port access of stack memory in one pipeline
stage. The resource usage is 1300 CLBs (= 2600 LCs) in a Xilinix XC
4036 XL.

\subsection{FemtoJava}

FemtoJava \cite{Femto01} is a research project to build an
application specific Java processor. The bytecode usage of the
embedded application is analyzed and a customized version of
FemtoJava is generated. FemtoJava implements up to 69 bytecode
instructions for an 8 or 16 bit datapath. These instructions take 3,
4, 7 or 14 cycles to execute. Analysis of small applications (50 to
280 byte code) showed that between 22 and 69 distinct bytecodes are
used. The resulting resource usage of the FPGA varies between 1000
and 2000 LCs. With the reduction of the datapath to 16 bits the
processor is not Java conformant.

\subsection{jHISC}

The jHISC project \cite{jHISC:jnl2006} proposes a high-level
instruction set architecture for Java. This project is closely
related to picoJava. The processor consumes 15500~LCs in an FPGA and
the maximum frequency in a Xilinx Virtex FPGA is 30~MHz.
%
%However, the resulting design is probably not very well balanced.
%The processor consumes 15500 LCs compared to about 3000 LCs for JOP.
%The maximum frequency in a Xilinx Virtex FPGA is 30 MHz compared to
%100 MHz for JOP.
%
According to \cite{jHISC:jnl2006} the prototype can only run simple
programs and the performance is estimated with a simulation. In
\cite{jHISC2006} the clocks per instruction (CPI) values for jHISC
are compared against picoJava and JOP. However, it is not explained
with which application the CPI values are collected. We assume that
the CPI values for picoJava and JOP are derived from the manual and
do not include any effects of pipeline stalls or cache misses.

\section{Additional Comments}

The two classes of hardware accelerators for Java can be further
subdivided as shown in \figurename~\ref{fig_related_tree}. Many of
the Java processors are stack machines that have been derived from
Forth processors. Two different stacks in these so-called Java
processors (Cjip, Ignite and Lightfoot) do not fit very well for the
JVM. Although stack based, Forth is different from Java bytecode.
Instruction mix in Forth shows about 25\% call and returns
\cite{Koopman89}, so Forth processors are optimized for fast call
and return. In Java, the percentage of call/return is only about 6\%
(see Section~\ref{sec:bench:jvm}). With subroutine exits so common,
it is no wonder that most of the Forth stack machines have a
mechanism for combining subroutine exits with other instructions and
provide two stacks to avoid the mixture of parameters and return
addresses. However, a JVM stack frame is more complex than in Forth
(see Section~\ref{sec:stack}) and there is no use for such a
mechanism. An additional return stack provides no advantage for the
JVM.

In Forth only the top elements can be accessed, which results in a
simple stack design with only one access port. In the JVM, parameters
for a method are explicitly pushed onto the stack before invocation.
These parameters are then accessed in the method relative to a
variable pointer. This mechanism needs a dual ported memory with
simultaneous read and write access. These basic differences between
Forth and the JVM lead to a sub-optimal implementation of the JVM on
a Forth based processor.

\begin{figure*}
    \centering
    \includegraphics[scale=\picscale]{related/related_tree}
    \caption{Java hardware}
    \label{fig_related_tree}
\end{figure*}


There are problems in getting information about commercial products.
When new companies started developing Java processors, a lot of
information was available. This information was usually more of a
presentation of the concept; nevertheless it gave some insights into
how they approached the different design problems. However, at the
point at which the projects reached production quality, this
information quietly disappeared from their websites. It was replaced
with colorful marketing prospectuses about the wonderful world of the
new Java-enabled mobile phones. Only one company, aJile Ltd.,
presented information about their product in a refereed conference
paper.

Many research projects for a Java processor in an FPGA exists.
Examples can be found in \cite{Femto01}, \cite{Kim2000} and
\cite{368445}. These projects have much in common -- the basic
implementation of a stack machine with integer instructions is easy.
However, the realization of the complete JVM is the hard part and
therefore beyond the scope of these projects.

Other than the aJile processor and the Komodo project, no solution
addresses the problem of real-time predictability. For this reason,
as well as its availability, the aJile processor is used for
comparison with JOP.

\section{Summary}
\label{sec:related:summary}

In Table~\ref{tab:related:plus:minus}, features of selected Java
processors are compared. The category `Predictability' means how well
the processor is time-predictable. In the category `Size', the chip
size is estimated, and the category `Performance' means average
performance. The category `JVM conformance' lists how complete the
implementation of the JVM specification \cite{jvm} is. The
`Flexibility' parameter indicates how well the processor can be
adapted to different application domains.

The assessment of the various parameters is, however, somewhat
subjective as the information is mainly derived from written
documentation. In Section~\ref{sec:performance}, the overall
performance of various Java systems, including the aJile processor,
is compared with JOP.

The last column of the table shows the features required for JOP.
This is, therefore, our research objective in a nutshell.

\begin{table}[htp]
    \centering
    \begin{tabular}{lccccc}
        \toprule
                        & picoJava & aJile   & Komodo  & FemtoJava & JOP     \\
        \midrule
        Predictability  & $--$     & $\cdot$ & $-$     & $\cdot$   & $++$    \\
        Size            & $--$     & $-$     & $+$     & $-$       & $++$    \\
        Performance     & $++$     & $+$     & $-$     & $--$      & $+$     \\
        JVM conformance & $++$     & $+$     & $-$     & $--$      & $\cdot$ \\
        Flexibility     & $--$     & $--$    & $+$     & $++$      & $++$    \\
        \bottomrule
    \end{tabular}
    \caption{Feature comparison of selected Java processors}
    \label{tab:related:plus:minus}
\end{table}

Due to the great variation in execution times for a trap, picoJava
is given a double minus in the `Predictability' category. picoJava
is also the largest processor in the list. However, its performance
and JVM compatibility are expected to be superior to those of other
processors.

The aJile processor is intended as a solution for real-time systems.
However, no information is available about bytecode execution times.
As this processor is a commercial product and has been on the market
for some time, it is expected that its JVM implementation would
conform to Java standards, as defined by Sun.

Komodo's multithreading is similar to hyper-threading in modern
processors that are trying to hide latencies in instruction fetching.
However, this feature leads to very pessimistic WCET values (in
effect rendering the performance gain useless). The fact that the
pipeline clock is only a quarter of the system clock also wastes a
considerable amount of potential performance.

FemtoJava is given a double plus for flexibility, due to the
application-dependent generation of the processor. However,
FemtoJava is only a 16-bit processor and therefore not JVM
compliant. The resource usage is also very high, compared to the
minimal Java subset implemented and the low performance of the
processor.

So far, all processors in the list perform weakly in the area of
time-predictable execution of Java bytecodes. However, a low-level
analysis of execution times is of primary importance for WCET
analysis. Therefore, the main objective of JOP is to define and
implement a processor architecture that is as predictable as
possible. However, it is equally important that this does not result
in a low performance solution. Performance shall not suffer as a
result of the time-predictable architecture.

The second main aim of this work is to design a small processor.
Size and the resulting energy consumption are a main concern in
embedded systems. The proposed Java processor needs to be small
enough to be implemented in a low-cost FPGA device. With this
constraint, an implementation in an ASIC will also result in a very
small core that can be part of a larger system-on-a-chip.

The embedded market is diverse and one size does not fit all. A
configurable processor in which we can trade size for performance
provides the flexibility for a variety of application domains. The
aim of the architecture of JOP is to support this flexibility.

%\section{Derived Work}
%\label{sec:derived}
%
%Quite common for open-source projects are derived projects.
%Especially the research community appreciates open-source projects.
%Following list describes projects that are either completely based
%on JOP or influenced to a great extent.
%
%JOP triggered research on implementation of the JVM in hardware for
%real-time systems. The publications on JOP and also the fact that
%JOP is open-source made the project and ideas easy accessible for
%other researchers. Several research projects are directly or
%indirectly based on the research project JOP:
%
%\begin{itemize}
%    \item Lund -- Flavius
%    \item Dresden
%    \item Graz
%    \item Albertos MS thesis
%    \item \cite{conf/iscas/KoT07} JOP based dual-issue Javaprocessor
%    \item WCET work by Rasmus, Trevor, Elena, and upcoming CISS
%\end{itemize}



\chapter{Notes - to be moved}

\section{Testing JOP}


Several different test cases are used for a regression test of the
processor, system code, the JVM, and the JDK implementation. Most
test cases can be found in \dirent{target/src/test/jvm} and
\dirent{target/src/test/jdk}. Package \code{testrt} contains exmples
to test the real-time scheduler of JOP. The description covers only
newer test cases. Additional test cases can be found in the
according packages.

\section{Garbage Collection}

GC test cases can be found in package \code{gctest} and
\code{gcinc}.

\subsubsection{SimpGC1}

\begin{description}
    \item[Purpose:]
Check if the GC triggers the allocation of memory.
Check if the GC uses all the available heap memory.
    \item[Expected Results:]
The tester should see a message reporting that
the garbage collector was triggered. The tester should see a message reporting
the memory size that should match the heap memory size reported at initialization of the
GC.
    \item[Description:]
The application consists of just one thread which continuously
allocates memory for new objects in the heap. The objects contain
only two integer fields and a constructor that initializes them. The
references to objects are not stored in any variable. The objective
is to have an idea of how many objects of a certain class are
necessary to fill the heap. This measure should be useful for
inducing GC allocation in the next test cases.
    \item[Results Observed:]
9728 objects were created before GC was triggered. GC seems to free
the same amount of memory every time it runs (it always triggers
after the allocation of 9728 objects).
\end{description}

\subsubsection{SimpGC2}

\begin{description}
    \item[Purpose:]
Check if the GC triggers the allocation of memory. Check if
unreferenced local (method scope) objects are garbage collected.
    \item[Expected Results:]
The tester should a set of messages reporting that the garbage
collector was triggered.
    \item[Description:]
Similar to the previous test case. The difference here is that the
allocation of memory is performed through several method calls.
After each method call the objects created are unreferenced. After
GC is activated it should be possible to create the same number of
objects as before allocating any object.
    \item[Results Observed:]
29190 objects were created, 9730 objects before GC triggers, each
time.

\emph{Run out of handles in new Object!}

\emph{GC allocation triggered}

Thus, all the unreferenced objects local to the method were garbage
collected each time GC run.

\end{description}

\subsubsection{SimpGC3}

\begin{description}
    \item[Purpose:]
Test concurrent request of memory from different threads (now using
inheritance instead of interfaces).
    \item[Expected Results:]
The tester should see no exceptions and but several GC messages as
explained below.
    \item[Description:]
We have two threads each creating a linked list. Both threads
add(concurrently) a certain amount of elements to the linked list.
Each element in the linked list has a field with a number. Each
thread keeps an increasing order (with no duplication) of those
fields.  GC is triggered. The Linked lists are checked to see if
they remain correct. We alternate among creation and verification of
the elements in the list, so eventually we run out of memory and GC
is triggered.
    \item[Results Observed:]
No exceptions were thrown during execution, thus object integrity is
fine.
\end{description}

\subsubsection{SimpGC4}

\begin{description}
    \item[Purpose:]
Check the behaviour of the GC when several references to an object
are present.
    \item[Expected Results:]
The tester should see no exceptions.
    \item[Description:]
An object is created and then a reference to the object is "taken"
by one thread each time. A reference to the object is always present
but stored in different places each time. At first the reference is
stored in a static field, then it is stored in a thread's class
variable. The static variable is set to null when a thread acquires
the object reference. In the same way when a thread releases the
object it sets its reference to null. Every time that the reference
is changed, GC is triggered and the object integrity is verified.
    \item[Results Observed:]
GC seems to work fine with this test case. Anyway I think that a
finer granularity in the interaction of threads could make a better
scenario for obtaining a better test (I believe that granularity
here is fixed by the length of the run() method of each runnable
thread).
\end{description}

\subsubsection{SimpGC5}

\begin{description}
    \item[Purpose:]
Check if the GC correctly detects garbage objects.
    \item[Expected Results:]
The tester should see a message verifying that non-garbage objects
are alive.
    \item[Description:]
A data structure similar to a tree is created then references to
certain levels of the tree are stored in class variables. GC is
triggered. The code checks that the objects below the level of the
node for which a reference is held are still alive(and correct).
    \item[Results Observed:]
No issues arose when the test case was run.

Note: It's easy to run out of memory when initializing the tree
(i.e. making the tree "grow" ). I used a binary tree with a depth of
8.

\end{description}

\subsubsection{SimpGC6}

\begin{description}
    \item[Purpose:]
Check if the GC detects cycles.
    \item[Expected Results:]
The tester should verify that the same amount of memory is available
before and after the creation of a set of chained objects.
    \item[Description:]
A cycle is formed by a closed chain of objects, one holding a
reference to the next. To verify that the unreferenced chained
objects are garbage collected, we measure the maximum allocable
object size before and after the allocation of the chained objects.
They should match.
    \item[Results Observed:]
The amount of objects created before GC allocation is triggered was
measured before and after the creation of a set of chained objects.
Through a log file it was verified that the number was the same. For
the Cyclone board and JOP version 01/08/07, the number was 9723.
Note: To obtain a logfile, make japp > myLogfile.
\end{description}

\section{JVM Functions}

The following sections list results from additional test cases:

\subsubsection{Ifacmp}
\begin{description}
   \item[Purpose:]
Test if\_acmp
   \item[Description:]
The test case works by testing both an equal and a non
equal(ifacmpeq = 165 and 0xa5 ifacmpne) with true and false
conditions(like a branch and condition coverage test). This leads to
4 if branches.

Following, we have an extract of the .class :

36:aload\_2 \newline 37:aload\_3 \newline 38:if\_acmpne       43 $
\longleftarrow $ \newline 41:iconst\_0\newline 42:istore\_1\newline
43:aload\_3 \newline 44:astore\_2\newline 45:aload\_2\newline
46:aload\_3        \newline 47:if\_acmpeq       52 $ \longleftarrow
$ \newline 50:iconst\_0        \newline 51:istore\_1        \newline
52:iload\_1         \newline 53:ireturn         \newline

We see there that both eq and ne conditions are produced.
   \item[Results Observed:]
The test run Ok.
\end{description}


\subsubsection{BranchTest1}
\begin{description}
   \item[Purpose:]
Test Branchs on reference comparison.
   \item[Description:]
The test case works by testing the following bytecodes:
\newline If\_acmp\textless cond\textgreater, Ifnonnull, Ifnull. All
conditions are provided(both true and false) to every variant of the
bytecodes (i.e., eq or neq). In this way, every possibile execution
is covered.
  \item[Results Observed:]
The test run Ok.
\end{description}

\subsubsection{BranchTest2}
\begin{description}
   \item[Purpose:]
Test Branch on int comparison.
   \item[Description:]
The test case works by testing the following bytecode: \newline
If\_icmp\textless cond\textgreater. All conditions are provided(both
true and false) to every variant of the bytecode (i.e., eq, neq, gt,
lt, etc). In this way, every possibile execution is covered.
  \item[Results Observed:]
The test run Ok.
\end{description}


\subsubsection{BranchTest3}
\begin{description}
   \item[Purpose:]
Test Branch on int comparison with zero(if\textless
cond\textgreater).
   \item[Description:]
The test case works by testing the following bytecode: \newline
If\textless cond\textgreater. All conditions are provided(both true
and false) to every variant of the bytecode (i.e., eq, neq, gt, lt,
etc). In this way, every possibile execution is covered.
  \item[Results Observed:]
The test run Ok.
\end{description}




\subsubsection{ArrayTest2}
\begin{description}
   \item[Purpose:]
Test arrays of references. Test array initialization, length,
storing and loading reference values to and from arrays of
references. Test array related features of the following bytecodes:
arraylength, aastore.
   \item[Description:]
A set of arrays of related types is created, then operations are
performed on them. By storing elements in the arrays, all cases of
aastore are produced. As a means of verification, the elements
stored in the array are checked in the code.

   \item[Results Observed:]
The test run Ok. Issue related to CheckCast.
\end{description}

\subsubsection{ArrayTest3}
\begin{description}
   \item[Purpose:]
Test arrays of primitives. Test array initialization, length,
storing and loading reference values to and from arrays of
references. Test array related features of the following bytecodes:
baload, caload, iaload, laload, saload,
bastore,castore,iastore,lastore.
   \item[Description:]
A set of arrays of primitive types is created, then load and store
operations are performed on them. By storing and loading elements
into and from the arrays, all supported Xastore and Xaload
instruction are exercised. As a means of verification, the elements
stored in the array are checked in the code.
  \item[Note:]
The throw of NullPointerException still needs to be tested.
   \item[Results Observed:]
The test run Ok.
\end{description}

\subsubsection{Conversion}
\begin{description}
   \item[Purpose:]
Test primitive type conversion. Bytecodes exercised: i2b, i2c, i2l,
i2s, l2i.
   \item[Description:]
This testcase extends the existent Conversion testcase to include
conversion between all the supported primitive types.
   \item[Results Observed:]
The test run Ok.
\end{description}


\subsubsection{Logic1}
\begin{description}
   \item[Purpose:]
Test shift operators. Bytecodes exercised: ishl, ishr, iushr.
   \item[Description:]
   \item[Results Observed:]
The test run Ok.
\end{description}


\subsubsection{Logic2}
\begin{description}
   \item[Purpose:]
Test arithmetic and logic operations. Bytecodes exercised: ineg,
ior, irem, isub, ixor, iand.
   \item[Description:]
Basic bit levels operations are performed.
   \item[Results Observed:]
The test run Ok. The throw of the proper exceptions was manually
verified.
\end{description}


\subsubsection{Logic3}
\begin{description}
   \item[Purpose:]
Test arithmetic and logic operations.
   \item[Description:]
Jump based on int comparison is tested.
   \item[Results Observed:]
Issue arrised: Some comparisons failed due to a failure handling the
overflow.
\end{description}


\subsubsection{Switch2}
\begin{description}
   \item[Purpose:]
Test switch statement. Bytecodes exercised: tableswitch,
lookupswitch.
   \item[Description:]
Two switch expression are used to test switch related bytecodes.
   \item[Results Observed:]
The test run Ok.
\end{description}

\section{JDK Test Cases}

\begin{description}
    \item[Note on name convention]

: to avoid collision among the name of the test case and the name of
the class being tested, we just use the first capital letter of the
first word of the class under test. So the test case for
ByteArrayInputStream is named BArrayInputStream.
    \item[Package: java.io]

    \item[Missing Classes]
The following is a list of missing classes as for the Jdk 1.1.8
version.
Package java.io\\
BufferedInputStream, BufferedOutputStream, BufferedReader, BufferedWriter.\\
\large \bf \begin{center} Package java.io \end{center}

\end{description}

\subsubsection{BArrayInputStream}
\begin{description}
   \item[Purpose:]
Test the ByteArrayInputStream class of JDK.
    \item[Description:]
Two objects of the ByteArrayInputStream class are created and
initialized. All the method are excercised under multiple conditions
and verified.
    \item[Results Observed:]
    The test run ok.
    \item[Coverage:]
    method: 78\%(7/9), block: 86\%(122/142), line: 80\%(23,9/30).
    \item[Note:]
The behaviour of the mark(int) method is different from that in the
JDK 1.1.8 specification. The API specification states that the
method should update the current mark in the stream to the provided
parameter, but the result observed is that the mark is set to the
current position.

\end{description}

\subsubsection{BArrayOutputStream}
\begin{description}
   \item[Purpose:]
Test the ByteArrayInputStream class of JDK.
    \item[Description:]
Two objects of the ByteArrayInputStream class are created and
initialized. All the method are excercised under multiple conditions
and verified to produce the proper output.
    \item[Results Observed:]
    The test run ok. \\ The following methods were not found:\\
    \textbullet toString(String)\\
    \textbullet writeTo(OutputStream)
    It was necessary to remove the try-catch for UnsupportedEncodingException in the
    String toString() method in ByteArrayInputStream in order for the testcase to compile.
    \item[Coverage:]
    method: 100\%(9/9), block: 75\%(93/124), line: 81\%(25/31).
    \item[Note:]
Accordding to JDK 1.1.8 specification, the toString(int) method is
deprecated. This method was
not tested.\\
Accordding to JDK 1.1.8 specification, the constructor String(byte
bytes[],int offset,int length) in class String should not throw the
UnsupportedEncodingException exception.


\end{description}



\subsubsection{DInputStream}
\begin{description}
   \item[Purpose:]
Test the DataInputStream class of JDK.
    \item[Description:]
An object of the myInputStream class is created. The myInputStream
class extends the InputStream class and is used to simulate an
InputStream. Methods in the myInputStream
class are included to produce necessary testing conditions as follows:\\
\textbullet setError() : simulates an I/O error
and causes an exception to be thrown when trying to read from the Stream.\\
\textbullet SetCorrect() : disables the effect of the previous method.\\
\textbullet setEndReached(): simulates an end of
stream. The call to this methods results in a -1 returned from the read() method.\\
\textbullet SetEndNotReached() : disables the effect of the previous method.\\

All the method are excercised under multiple conditions and verified
to produce the proper output.
    \item[Results Observed:]
    The test run ok.
    \item[Coverage:]
    method: 96\%(25/26), block: 69\%(345/498), line: 72\%(65/90).
    \item[Note:]
    Comment and uncomment the methods above to manually test the Exception behaviour.
\end{description}

\subsubsection{DOutputStream}
\begin{description}
   \item[Purpose:]
Test the DataOutputStream class of JDK.
    \item[Description:]
An object of the myOutputStream class is created. The myOutputStream
class extends the OutputStream class and is used to simulate an
OutputStream. Methods in the myOutputStream
class are included to produce necessary testing conditions as follows:\\
\textbullet setError() : simulates an I/O error
and causes an exception to be thrown when trying to write to the Stream.\\
\textbullet SetCorrect() : disables the effect of the previous method.\\
The following methods are used to verify contents written to the OutputStream:\\
\textbullet checkData(byte data,int position): returns a boolean value of true\\
if the provided data is found at that position in the stream.\\
\textbullet checkData(byte data[],int offset, int len): returns true if the data\\
of the provided array is found in the OutputStream starting at the
provided offset. All the method are excercised under several
conditions and verified to produce the proper output.
    \item[Results Observed:]
    The test run ok. \\ The following methods were not found:\\
    \textbullet int size()\\
    \item[Coverage:]
    method: 93\%(13/14), block: 70\%(284/406), line: 78\%(62/80).
    \item[Note:]
Comment and uncomment the methods above to manually test the
Exception behaviour.

\large \bf \begin{center} Package java.lang \end{center}
\end{description}

\subsubsection{PrimitiveClasses}
\begin{description}
   \item[Purpose:]
Test the primitive types wrapper classes of JDK.
    \item[Description:]
We create objects of the primitive wrapper classes and perform operations on them.\\
Class methods are excercised under several conditions and verified
to produce the proper output.
    \item[Results Observed:]
    Possible issue in constructor Boolean(String): according to JVM specification
    1.1.8 when the argument to the method is something different from the string "true"
    a Boolean representing false is allocated.\\
    The problem arises when the argument is the null pointer. An exception is thrown in
    JOP in that case. According to the specification, this constructor should not throw
    any exception, it just must create a Boolean representing false.\\
    When running on Sun's JVM the constructor creates the false Boolean as expected.\\
    The constructor also fails to create a proper true object when given a string as an argument.
    It should ignore case, but "tRue" gives a different result from "true", when they should both
    produce a Boolean true object.\\
    Also the static method valueOf should receive a string instead of a boolean as its argument.\\
    The constructor Byte(String) was not found.
    \item[Coverage:]
    Found in table 1.
    \item[Note:]
The method Boolean.valueOf(boolean) was not tested.
\end{description}

\begin{table}
\begin{center}
\caption{\label{MyTable} Testing Coverage Results}
\begin{tabular}[t]{|l|l|l|l|l|}
\hline
Package & Class & Method & Block & Line \\
\hline
java.lang &  &  &  &  \\
\hline
  & Boolean & 88\% (7/8) & 91\% (64/70)  & 93\% (13/14) \\
\hline
  & Byte & 88\% (7/8) & 91\% (64/70)  & 93\% (13/14) \\
\hline
\end{tabular}
\end{center}
\end{table}


\section{Dependencies}

\subsection{Changing the Class Format}

\begin{itemize}
    \item JOPizer: CLS\_HEAD, dump()
    \item GC.java uses CLASS\_HEADR
    \item JMV.java uses CLASS\_HEADR + offset (checkcast, instanceof)
\end{itemize}

\subsection{Stack Size}

\begin{itemize}
    \item \code{ram\_width} in \code{jop\_config\_xx.vhd}
    \item \code{STACK\_SIZE} in \code{com.jopdesign.sys.Const}
    \item \code{RAM\_LEN} in \code{com.jopdesign.sys.Jopa}
\end{itemize}

\chapter{Summary}
\label{chap:conclusions}

    
In this chapter we will undertake a short review of the project and
summarize the contributions. Java for real-time systems is a new and
active research area. This chapter offers suggestions for future
research, based on the described Java processor.

The research contributions made by this work are related to two
areas: real-time Java and resource-constrained embedded systems.

\section{A Real-Time Java Processor}

The goal of time-predictable execution of Java programs was a
first-class guiding principle throughout the development of JOP:

\begin{itemize}

        \item
The execution time for Java bytecodes can be exactly predicted in
terms of the number of clock cycles.
% The execution time for Java bytecodes is known cycle-accurate.
JOP is therefore a straightforward target for low-level WCET
analysis. There is no mutual dependency between consecutive
bytecodes that could result in unbounded timing effects.

    \item
In order to provide time-predictable execution of Java bytecodes,
the processor pipeline is designed without any prefetching or
queuing. This fact avoids hard-to-analyze and possibly unbounded
pipeline dependencies. There are no pipeline stalls, caused by
interrupts or the memory subsystem, to complicate the WCET analysis.

    \item
A pipelined processor architecture calls for higher memory
bandwidth. A standard technique to avoid processing bottlenecks due
to the higher memory bandwidth is caching.
%In order to fill the gap between processor speed and the memory
%access time, caches are mandatory, even in embedded systems.
However, standard cache organizations improve the average
execution time but are difficult to predict for WCET analysis.
Two time-predictable caches are implemented in JOP: a \emph{stack
cache} as a substitution for the data cache and a \emph{method
cache} to cache the instructions.

As the stack is a heavily accessed memory region, the stack -- or
part of it -- is placed in local memory. This part of the stack is
referred to as the \emph{stack cache} and described in
Section~\ref{sec:stack}. Fill and spill of the stack cache is
subjected to microcode control and therefore time-predictable.

In Section~\ref{sec:cache}, a novel way to organize an
instruction cache, as \emph{method cache}, is given. The cache
stores complete methods, and cache misses only occur on method
invocation and return. Cache block replacement depends on the
call tree, instead of instruction addresses. This \emph{method
cache} is easy to analyze with respect to worst-case behavior and
still provides substantial performance gain when compared to a
solution without an instruction cache.

    \item The time-predictable processor described above provides
        the basis for real-time Java. To enable real-time Java to
        operate on resource-constrained devices, a simple
        real-time profile was defined in Section~\ref{sec:rtprof}
        and implemented in Java on JOP. The beauty of this
        approach is in implementing functions usually associated
        with an RTOS in Java. This means that real-time Java is
        not based on an RTOS, and therefore not restricted to the
        functionality provided by the RTOS. With JOP, a
        self-contained real-time system in pure Java becomes
        possible.

The tight integration of the scheduler and the hardware that
generates schedule events results in low latency and low jitter of
the task dispatch.

    \item
%The timer interrupt in JOP generates interrupts at the release times
%of the tasks. The scheduler is responsible for reprogramming the
%timer after each occurrence of a timer interrupt. Controlling the
%timer interrupt as part of the scheduling results in the low jitter
%of periodic tasks.
%
The defined real-time profile suggests a new way to handle hardware
interrupts to avoid interference between blocking device drivers and
application tasks. Hardware interrupts other than the timer
interrupt are represented as asynchronous events with an associated
thread. These events are \emph{normal} schedulable objects and
subject to the control of the scheduler. With a minimum interarrival
time, these events, and the associated device drivers, can be
incorporated into the priority assignment and schedulability
analysis in the same way as normal application tasks.

\end{itemize}

The contributions described above result in a time-predictable
execution environment for real-time applications written in Java,
without the resource implications and unpredictability of a
JIT-compiler. The described processor architecture is a
straightforward target for low-level WCET analysis.

%New applications, such as multimedia streaming, result in
%\emph{soft} real-time systems that need a more flexible scheduler
%than the traditional fixed priority-based ones.

Implementing a real-time scheduler in Java opens up new
possibilities. The scheduler is extended to provide a framework for
user-defined scheduling in Java. In Section~\ref{sec:usersched}, we
analyzed which events are exposed to the scheduler and which
functions from the JVM need to be available in the user space. A
simple-to-use framework to evaluate new scheduling concepts is
given.



\section{A Resource-Constrained Processor}

Embedded systems are usually very resource-constrained. Using a
low-cost FPGA as the main target technology forced the design to be
small. The following architectural features address this issue:

\begin{itemize}

    \item
The architecture of JOP is best described as:
\begin{quote}
    The JVM is a CISC stack architecture, whereas JOP is a RISC stack
    architecture.
\end{quote}
JOP contains its own instruction set, called microcode in this
handbook, with a novel way of mapping bytecodes to microcode
addresses. This mapping has zero overheads as described in
Section~\ref{sec:microcode}. Basic bytecode instructions have a
one-to-one mapping to microcode instructions and therefore
execute in a single cycle. The stack architecture allows compact
encoding of microinstructions in 8 bits to save internal memory.

This approach allows flexible implementation of Java bytecodes in
hardware, as a microcode sequence, or even in Java itself.

    \item
The analysis of the JVM stack usage pattern in
Section~\ref{sec:stack} led to the design of a resource-efficient
two-level stack cache. This two-level stack cache fits to the
embedded memory technologies of current FPGAs and ASICs and ensures
fast execution of basic instructions.

Part of the stack cache, which is implemented in an on-chip
memory, is also used for microcode variables and constants. This
resource sharing not only reduces the number of memory blocks
needed for the processor, but also the number of data paths to
and from the execution unit.

    \item
Interrupts are considered hard to handle in a pipelined processor,
resulting in a complex (and therefore resource consuming)
implementation. In JOP, the above mentioned bytecode-microcode
mapping is used in a clever way to avoid interrupt handling in the
core pipeline.
%
%An implementation of interrupts at the bytecode-microcode mapping
%keeps interrupts transparent in the core pipeline and avoids complex
%logic. Therefore,
%
Interrupts generate special bytecodes that are inserted in a
transparent way in the bytecode stream. Interrupt handlers can be
implemented in the same way as bytecodes are implemented: in
microcode or in Java.


\end{itemize}

The above design decisions where chosen to keep the size of the
processor small without sacrificing performance. JOP is the smallest
Java processor available to date that provides the basis for an
implementation of the CLDC specification (see
Section~\ref{subsec:cldc}). JOP is a fast execution environment for
Java, without the resource implications and unpredictability of a
JIT-compiler. The average performance of JOP is similar to that of
mainstream, non real-time Java systems.

JOP is a flexible architecture that allows different configurations
for different application domains. Therefore, size can be traded
against performance. As an example, resource intensive instructions,
such as floating point operations, can be implemented in Java. The
flexibility of an FPGA implementation also allows adding
application-specific hardware accelerators to JOP.

The small size of the processor allows the use of low-cost FPGAs in
embedded systems that can compete against standard microcontroller.
JOP has been implemented in several different FPGA families and is
used in different real-world applications.

Programs for embedded and real-time systems are usually
multi-threaded and a small design provides a path to a
multi-processor system in a mid-sized FPGA or in an ASIC.

A tiny architecture also opens new application fields when
implemented in an ASIC. Smart sensors and actuators, for example,
are very sensitive to cost, which is proportional to the die area.


\section{Future Work}

JOP provides a basis for various directions for future research.
Some suggestions are given below:
%
\begin{description}
    \item[Real-time garbage collector:]
In Section~\ref{chap:rtgc}, a real-time garbage collector was
presented. Hardware support of a real-time GC would be an
interesting topic for further research.

Another question that remains with a real-time GC is the analysis of
the worst-case memory consumptions of tasks (similar to the WCET
values), and scheduling the GC so that it can keep up with the
allocation rate.

    \item[Hardware accelerator:] The flexibility of an FPGA
        implementation of a processor opens up new possibilities
        for hardware accelerators. A further step would be to
        generate an application specific-system in which part of
        the application code is moved to hardware. Ideally, the
        hardware description should be extracted automatically
        from the Java source. Preliminary work in this area,
        using JOP as its basis, can be found in \cite{jop:sac05,
        jop:hwmethods}.

    \item[Hardware scheduler:]
In JOP, scheduling and dispatch is done in Java (with some microcode
support). For tasks with very short periods, the scheduling
overheads can prove to be too high. A scheduler implemented in
hardware can shorten this time, due to the parallel nature of the
algorithm.



    \item[Instruction cache:] The cache solution, described in
        Section~\ref{sec:cache}, provides predictable instruction
        cache behavior while, in the average case, still
        performing in a similar way to a direct-mapped cache.
        However, an analysis tool for the worst-case behavior is
        still needed. With this tool, and a more complex analysis
        tool for traditional instruction caches, we also need to
        verify that the worst-case miss penalty is lower than
        with a traditional instruction cache.

A second interesting aspect of the method cache is the fact that
the replacement decision on a cache miss only occurs on method
invoke and return. The infrequency of this decision means that
more time is available for more advanced replacement algorithms.


    \item[Real-time Java:] Although there is already a definition
        for real-time Java, i.e.\ the RTSJ \cite{rtsj}, this
        definition is not necessarily adequate. There is ongoing
        research on how memory should be managed for real-time
        Java applications: scoped memory, as suggested by the
        RTSJ, usage of a real-time GC, or application managed
        memory through memory pools. However, almost no research
        has been done into how the Java library, which is major
        part of Java's success, can be used in real-time systems
        or how it can be adapted to do so. The question of what
        the best memory management is for the Java standard
        library remains unanswered.

    \item[Java computer:] How would a processor architecture and
        operating system architecture look in a `Java only'
        system? Here, we need to rethink our approach to
        processes, protection, kernel- and user-space, and
        virtual memory. The standard approach of using memory
        protection between different processes is necessary for
        applications that are programmed in languages that use
        memory addresses as data, i.e.\ pointer usage and pointer
        manipulation. In Java, no memory addresses are visible
        and pointer manipulation is not possible. This very
        important feature of Java makes it a \emph{safe}
        language. Therefore, an error-free JVM means we do not
        need memory protection between processes and we do not
        need to make a distinction between kernel and user space
        (with all the overhead) in a Java system. Another reason
        for using virtual addresses is link addresses. However,
        in Java this issue does not exist, as all classes are
        linked dynamically and the code itself (i.e.\ the
        bytecodes) only uses relative addressing.

Another issue here is the paging mechanism in a virtual memory
system, which has to be redesigned for a Java computer. For this,
we need to merge the virtual memory management with the GC. It
does not make sense to have a virtual memory manager that works
with plain (e.g.\ 4~KB) memory pages without knowledge about
object lifetime. We therefore need to incorporate the virtual
memory paging with a generational GC. The GC knows which objects
have not been accessed for a long time and can be swapped out to
the hard disk. Handling paging as part of the GC process also
avoids page fault exceptions and thereby simplifies the processor
architecture.

Another question is whether we can substitute the process
notation with threads, or whether we need several JVMs on a Java
only system. It depends. If we can live with the concept of
shared static class members, we can substitute heavyweight
processes with lightweight threads. It is also possible that we
would have to define some further thread local data structures in
the system.

\end{description}
%
It is the opinion of the author that Java is a promising language for
future real-time systems. However, a number of issues remain to be
solved. JOP, with its time-predictable execution of Java bytecodes,
is an important part of a real-time Java system.



%\printindex
%\pagestyle{empty}
%\chaptermark{} % remove chapter mark in the heading, but it does it on the
               % last page too!
\bibliographystyle{plain}
%\bibliographystyle{plnnolow} % changed plain.bst to keep upper case in titles
\bibliography{../bib/all}



%\def\rightmark{} % remove section part in the heading
\appendix
 \ihead{\leftmark} % use chapter (=\leftmark) on both pages in the appendix

\chapter{Publications}
    % This is a copy from /usr/doc

\begin{itemize}

\subsubsection*{2003}

\item Martin Schoeberl.
 Using a {J}ava Optimized Processor in a Real World Application.
 In {\em Proceedings of the First Workshop on Intelligent Solutions in
  Embedded Systems (WISES 2003)}, pages 165--176, Austria, Vienna, June 2003.

\item Martin Schoeberl.
 Design Decisions for a {J}ava Processor.
 In {\em Tagungsband Austrochip 2003}, pages 115--118, Linz, Austria,
  October 2003.

\item Martin Schoeberl.
 {JOP}: {A} {J}ava Optimized Processor.
 In R.~Meersman, Z.~Tari, and D.~Schmidt, editors, {\em On the Move to
  Meaningful Internet Systems 2003: Workshop on {J}ava Technologies for
  Real-Time and Embedded Systems (JTRES 2003)}, volume 2889 of {\em Lecture
  Notes in Computer Science}, pages 346--359, Catania, Italy, November 2003.
  Springer.

\subsubsection*{2004}

\item Martin Schoeberl.
 Restrictions of {J}ava for Embedded Real-Time Systems.
 In {\em Proceedings of the 7th IEEE International Symposium on
  Object-Oriented Real-Time Distributed Computing (ISORC 2004)}, pages
  93--100, Vienna, Austria, May 2004.

\item Martin Schoeberl.
 Design Rationale of a Processor Architecture for Predictable
  Real-Time Execution of {J}ava Programs.
 In {\em Proceedings of the 10th International Conference on
  Real-Time and Embedded Computing Systems and Applications (RTCSA 2004)},
  Gothenburg, Sweden, August 2004.

\item Martin Schoeberl.
 Real-Time Scheduling on a {J}ava Processor.
 In {\em Proceedings of the 10th International Conference on
  Real-Time and Embedded Computing Systems and Applications (RTCSA 2004)},
  Gothenburg, Sweden, August 2004.

\item Martin Schoeberl.
 {J}ava Technology in an {FPGA}.
 In {\em Proceedings of the International Conference on
  Field-Programmable Logic and its applications (FPL 2004)}, Antwerp, Belgium,
  August 2004.

\item Martin Schoeberl.
 A Time Predictable Instruction Cache for a Java Processor.
 In Robert Meersman, Zahir Tari, and Angelo Corsario, editors, {\em On
  the Move to Meaningful Internet Systems 2004: Workshop on {J}ava Technologies
  for Real-Time and Embedded Systems (JTRES 2004)}, volume 3292 of {\em
  Lecture Notes in Computer Science}, pages 371--382, Agia Napa, Cyprus,
  October 2004. Springer.

\subsubsection*{2005}

\item Flavius Gruian, Per Andersson, Krzysztof Kuchcinski, and Martin Schoeberl.
 Automatic generation of application-specific systems based on a
  micro-programmed java core.
 In {\em Proceedings of the 20th ACM Symposium on Applied Computing,
  Embedded Systems track}, Santa Fee, New Mexico, March 2005.

\item Martin Schoeberl.
 Design and implementation of an efficient stack machine.
 In {\em Proceedings of the 12th IEEE Reconfigurable Architecture
  Workshop (RAW2005)}, Denver, Colorado, USA, April 2005. IEEE.

\item Martin Schoeberl.
 {\em JOP: A Java Optimized Processor for Embedded Real-Time Systems}.
 PhD thesis, Vienna University of Technology, 2005.

\item Martin Schoeberl.
 Evaluation of a {J}ava processor.
 In {\em Tagungsband Austrochip 2005}, pages 127--134, Vienna,
  Austria, October 2005.

\subsubsection*{2006}

\item Martin Schoeberl.
 A time predictable {J}ava processor.
 In {\em Proceedings of the Design, Automation and Test in Europe
  Conference (DATE 2006)}, pages 800--805, Munich, Germany, March 2006.

\item Martin Schoeberl.
 Real-time garbage collection for {J}ava.
 In {\em Proceedings of the 9th IEEE International Symposium on Object
  and Component-Oriented Real-Time Distributed Computing (ISORC 2006)}, pages
  424--432, Gyeongju, Korea, April 2006.

\item Martin Schoeberl. Instruction Cache f\"ur Echtzeitsysteme,
    April 2006. Austrian patent AT 500.858.

\item Rasmus Pedersen and Martin Schoeberl.
 An embedded support vector machine.
 In {\em Proceedings of the Fourth Workshop on Intelligent Solutions
  in Embedded Systems (WISES 2006)}, pages 79--89, Jun. 2006.

\item Rasmus Pedersen and Martin Schoeberl.
 Exact roots for a real-time garbage collector.
 In {\em Proceedings of the Workshop on {J}ava Technologies for
  Real-Time and Embedded Systems (JTRES 2006)}, Paris, France, October 2006.

\item Martin Schoeberl and Rasmus Pedersen.
 {WCET} analysis for a {Java} processor.
 In {\em Proceedings of the Workshop on {J}ava Technologies for
  Real-Time and Embedded Systems (JTRES 2006)}, Paris, France, October 2006.

\subsubsection*{2007}

\item Martin Schoeberl, Hans Sondergaard, Bent Thomsen, and Anders~P. Ravn.
A profile for safety critical java.
In {\em 10th IEEE International Symposium on Object and
  Component-Oriented Real-Time Distributed Computing (ISORC'07)}, pages
  94--101, Santorini Island, Greece, May 2007. IEEE Computer Society.

\item Martin Schoeberl.
Mission modes for safety critical java.
In {\em 5th IFIP Workshop on Software Technologies for Future
  Embedded \& Ubiquitous Systems}, May 2007.

\item Raimund Kirner and Martin Schoeberl.
Modeling the function cache for worst-case execution time analysis.
In {\em Proceedings of the 44rd Design Automation Conference, DAC
  2007}, San Diego, CA, USA, June 2007. ACM.

\item Martin Schoeberl.
A time-triggered network-on-chip.
In {\em International Conference on Field-Programmable Logic and its
  Applications (FPL 2007)}, Amsterdam, Netherlands, August 2007.

\item Christof Pitter and Martin Schoeberl.
Time predictable {CPU} and {DMA} shared memory access.
In {\em International Conference on Field-Programmable Logic and its
  Applications (FPL 2007)}, Amsterdam, Netherlands, August 2007.


\item Wolfgang Puffitsch and Martin Schoeberl.
{picoJava-II} in an {FPGA}.
In {\em Proceedings of the 5th international workshop on Java
  technologies for real-time and embedded systems (JTRES 2007)}, Vienna,
  Austria, September 2007. ACM Press.

\item Martin Schoeberl.
Architecture for object oriented programming languages.
In {\em Proceedings of the 5th international workshop on Java
  technologies for real-time and embedded systems (JTRES 2007)}, Vienna,
  Austria, September 2007. ACM Press.

\item Christof Pitter and Martin Schoeberl.
Towards a {Java} multiprocessor.
In {\em Proceedings of the 5th international workshop on Java
  technologies for real-time and embedded systems (JTRES 2007)}, Vienna,
  Austria, September 2007. ACM Press.

\item Martin Schoeberl and Jan Vitek.
Garbage collection for safety critical {Java}.
In {\em Proceedings of the 5th international workshop on Java
  technologies for real-time and embedded systems (JTRES 2007)}, Vienna,
  Austria, September 2007. ACM Press.

\item Martin Schoeberl. {SimpCon} - a simple and efficient {SoC}
    interconnect. In {\em Proceedings of the 15th Austrian
    Workhop on Microelectronics, Austrochip 2007}, Graz, Austria,
  October 2007.

\subsubsection*{2008}

\item Martin Schoeberl. A Java processor architecture for
    embedded real-time systems. {\em Journal of Systems
    Architecture}, 54/1--2:265--286, 2008.

\item Trevor Harmon, Martin Schoeberl, Raimund Kirner, and
    Raymond Klefstad. A modular worst-case execution time
    analysis tool for Java
  processors. In {\em Proceedings of the 14th IEEE Real-Time and
 Embedded Technology and Applications Symposium (RTAS 2008)}, St.
 Louis, MO, United States, April 2008.

\item Martin Schoeberl, Stephan Korsholm, Christian Thalinger,
    and Anders~P. Ravn. Hardware objects for {Java}. In {\em
    Proceedings of the 11th IEEE International Symposium on
  Object/component/service-oriented Real-time distributed
  Computing (ISORC 2008)}, Orlando, Florida, USA, May 2008. IEEE
  Computer Society.

\item Stephan Korsholm, Martin Schoeberl, and Anders~P. Ravn.
    Interrupt Handlers in {Java}.
 In {\em Proceedings of the 11th IEEE International Symposium on
  Object/component/service-oriented Real-time distributed
  Computing (ISORC 2008)}, Orlando, Florida, USA, May 2008. IEEE
  Computer Society.

\item Trevor Harmon, Martin Schoeberl, Raimund Kirner, and
    Raymond Klefstad. Toward libraries for real-time Java. In
    {\em Proceedings of the 11th IEEE International Symposium on
    Object/component/service-oriented Real-time distributed
  Computing (ISORC 2008)}, Orlando, Florida, USA, May 2008. IEEE
  Computer Society.

\item Christof Pitter and Martin Schoeberl. Performance
    evaluation of a {Java} chip-multiprocessor. In {\em
    Proceedings of the 3rd IEEE Symposium on Industrial Embedded
    Systems (SIES 2008)}, Jun. 2008.

\item Martin Schoeberl. Application experiences with a real-time
    {J}ava processor. In {\em Proceedings of the 17th IFAC World
    Congress}, Seoul, Korea, July 2008.

\item Peter Puschner and Martin Schoeberl. On composable system
    timing, task timing, and WCET analysis. In {\em Proceedings
    of the 8th International Workshop on Worst-Case Execution
    Time (WCET) Analysis}, Prague, Czech
  Republic, July 2008.

\item Martin Schoeberl. {\em JOP: A Java Optimized Processor for
    Embedded Real-Time Systems}. Number ISBN 978-3-8364-8086-4.
    VDM Verlag Dr. M{\"u}ller, July 2008.

\item Martin Schoeberl and Wolfgang Puffitsch. Non-blocking
    object copy for real-time garbage collection. In {\em
    Proceedings of the 6th International Workshop on Java
  Technologies for Real-time and Embedded Systems (JTRES 2008)},
  September 2008.

\item Wolfgang Puffitsch and Martin Schoeberl. Non-blocking root
    scanning for real-time garbage collection. In {\em
    Proceedings of the 6th International Workshop on Java
  Technologies for Real-time and Embedded Systems (JTRES 2008)},
  September 2008.

\item Walter Binder, Martin Schoeberl, Philippe Moret, and Alex
    Villazon. Cross-profiling for embedded Java processors. In
    {\em Proceedings of the 5th International Conference on the
    Quantitative Evaluation of SysTems (QEST 2008)}, St Malo,
  France, September 2008.

\item Walter Binder, Alex Villazon, Martin Schoeberl, and
    Philippe Moret. Cache-aware cross-profiling for Java
    processors. In {\em Proceedings of the 2008 international
    conference on Compilers, architecture, and synthesis
    forembedded systems (CASES 2008)}, Atlanta, Georgia, October
    2008. ACM.

\subsubsection*{2009}

\item Martin Schoeberl. Time-predictable computer architecture.
    {\em EURASIP Journal on Embedded Systems}, vol. 2009, Article
    ID 758480:17 pages, 2009.

\item Martin Schoeberl. Time-predictable cache organization. In
    {\em Proceedings of the First International Workshop on
    Software Technologies for Future Dependable Distributed
    Systems (STFSSD 2009)}, Tokyo, Japan, March
  2009. IEEE Computer Society.

\item Andy Wellings and Martin Schoeberl. Thread-local scope
    caching for real-time {J}ava. In {\em Proceedings of the 12th
    IEEE International Symposium on
  Object/component/service-oriented Real-time distributed
  Computing (ISORC 2009)}, Tokyo, Japan, March 2009. IEEE
  Computer Society.

\item Florian Brandner, Tommy Thorn, and Martin Schoeberl.
    Embedded {JIT} compilation with {CACAO} on {YARI}. In {\em
    Proceedings of the 12th IEEE International Symposium on
    Object/component/service-oriented Real-time
  distributed Computing (ISORC 2009)}, Tokyo, Japan, March 2009.
  IEEE Computer Society.

\item Thomas Henties, James~J. Hunt, Doug Locke, Kelvin Nilsen,
    Martin Schoeberl, and Jan Vitek. Java for safety-critical
 applications. In {\em 2nd International Workshop on the
 Certification of Safety-Critical Software Controlled Systems
 (SafeCert 2009)}, Mar. 2009.

\item Martin Schoeberl and Peter Puschner. Is
    chip-multiprocessing the end of real-time scheduling? In {\em
    Proceedings of the 9th International Workshop on Worst-Case
  Execution Time (WCET) Analysis}, Dublin, Ireland, July 2009.
  OCG.

\item Benedikt Huber and Martin Schoeberl. Comparison of implicit
    path enumeration and model checking based WCET
  analysis. In {\em Proceedings of the 9th International Workshop
on Worst-Case
  Execution Time (WCET) Analysis}, Dublin, Ireland, July 2009.
  OCG.

\item Philippe Moret, Walter Binder, Martin Schoeberl, Alex
    Villazon, and Danilo Ansaloni. Analyzing performance and
dynamic behavior of embedded Java software with calling-context
  cross-profiling. In {\em Proceedings of
the 7th International Conference on the Principles and Practice
  of Programming in Java (PPPJ 2009)}, Calgary, Alberta, Canada,
  August 2009. ACM.

\item Martin Schoeberl, Walter Binder, Philippe Moret, and Alex
    Villazon. Design space exploration
for Java processors with cross-profiling. In {\em Proceedings of
the 6th International Conference on the Quantitative Evaluation
  of SysTems (QEST 2009)}, Budapest, Hungary, September 2009.
  IEEE Computer Society.

\item Philippe Moret, Walter Binder, Alex Villazon, Danilo
    Ansaloni, and Martin Schoeberl. Locating performance
    bottlenecks in embedded Java software with
  calling-context cross-profiling. In {\em Proceedings of the 6th
International Conference on the
  Quantitative Evaluation of SysTems (QEST 2009)}, Budapest,
  Hungary, September 2009. IEEE Computer Society.

\item Jack Whitham, Neil Audsley, and Martin Schoeberl. Using
    hardware methods to improve time-predictable performance in
  real-time java systems. In {\em Proceedings of the 7th
International Workshop on Java
  Technologies for Real-time and Embedded Systems (JTRES 2009)},
  Madrid, Spain, September 2009. ACM Press.


\subsubsection*{2010}

\item Martin Schoeberl and Wolfgang Puffitsch. Non-blocking
    real-time garbage collection. {\em Trans. on Embedded
    Computing Sys.}, accepted, 2010.

\item Christof Pitter and Martin Schoeberl. A real-time {Java}
    chip-multiprocessor. {\em Trans. on Embedded Computing Sys.},
    accepted, 2010.

\item Walter Binder, Martin Schoeberl, Philippe Moret, and Alex
    Villazon. Cross-profiling for Java processors. {\em Software:
Practice and Experience}, accepted, 2010.

\end{itemize}


%\chapter{Glossary}
\chapter{Acronyms}
 \label{appx:acro}
\newcommand{\gloss}[3]{
    \textbf{#1} & #2\\
%    \item[#1] {#2}\\{#3}
%    \textbf{#1} & #2\\ & \mbox{#3}\\
%    \textbf{#1} \> \> #2\\ \> \parbox{10cm}{#3}\\ \\
}
\begin{longtable}[l]{ll}
%    abc\=WCET\= \kill % for tabbing
    \gloss{ADC}{Analog to Digital Converter}{}
    \gloss{ALU}{Arithmetic and Logic Unit}{The part of a processor that performs
    arithmetic, logical, and related operations.}
    \gloss{ASIC}{Application-Specific Integrated Circuit}{}
    \gloss{BCET}{Best Case Execution Time}{}
    \gloss{CFG}{Control Flow Graph}{}
    \gloss{CISC}{Complex Instruction Set Computer}{}
    \gloss{CLDC}{Connected Limited Device Configuration}{}
    \gloss{CPI}{average Clock cycles Per Instruction}{}
    \gloss{CRC}{Cyclic Redundancy Check}{}
    \gloss{DMA}{Direct Memory Access}{}
    \gloss{DRAM}{Dynamic Random Access Memory}{}
    \gloss{EDF}{Earliest Deadline First}{}
    \gloss{EMC}{Electromagnetic Compatibility}{}
    \gloss{ESD}{Electrostatic Discharge}{}
    \gloss{FIFO}{Fist In, First Out}{}
    \gloss{FPGA}{Field Programmable Gate Array}{FPGAs are a class of programmable
    logic devices. They contain a matrix of
    LCs, embedded memory blocks, and sophisticated I/O cells.}
    \gloss{GC}{Garbage Collect(ion/or)}{}
    \gloss{IC}{Instruction Count}{}
    \gloss{ILP}{Instruction Level Parallelism}{}
    \gloss{JOP}{Java Optimized Processor}{A processor that implements the JVM
    in hardware with architectural features for time-predictable execution of
    Java applications for real-time systems.}
    \gloss{J2ME}{Java2 Micro Edition}{}
    \gloss{J2SE}{Java2 Standard Edition}{}
    \gloss{JDK}{Java Development Kit}{}
    \gloss{JIT}{Just-In-Time}{}
    \gloss{JVM}{Java Virtual Machine}{}
    \gloss{LC}{Logic Cell}{The basic element in an FPGA: a 4-bit lookup table with
a register.}
    \gloss{LRU}{Least-Recently Used}{}
    \gloss{MBIB}{Memory Bytes read per Instruction Byte}{}
    \gloss{MCIB}{Memory Cycles per Instruction Byte}{}
    \gloss{MP}{Miss Penalty}{}
    \gloss{MTIB}{Memory Transactions per Instruction Byte}{}
    \gloss{MUX}{Multiplexer}{}
    \gloss{OO}{Object Oriented}{}
    \gloss{OS}{Operating System}{}
    \gloss{RISC}{Reduced Instruction Set Computer}{}
    \gloss{RT}{Real-Time}{}
    \gloss{RTOS}{Real-Time Operating System}{}
    \gloss{RTSJ}{Real-Time Specification for Java}{}
    \gloss{SCADA}{Supervisory Control And Data Acquisition}{}
    \gloss{SDRAM}{Synchronous DRAM}{}
    \gloss{SRAM}{Static Random Access Memory}{}
    \gloss{TOS}{Top Of Stack}{}
    \gloss{UART}{Universal Asynchronous Receiver/Transmitter}{}
    \gloss{VHDL}{Very High Speed Integrated Circuit (VHSIC)}{}
    \gloss{}{Hardware Description Language}{}
    \gloss{WCET}{Worst-Case Execution Time}{}
%    \gloss{xxx}{yyy}{}
%    \gloss{xxx}{yyy}{}
%    \gloss{xxx}{yyy}{}
\end{longtable}



\chapter{JOP Instruction Set} \label{appx:jop:instr}
%\documentclass[%draft,
    11pt, % use explicit paper size
    headinclude, footexclude,
    twoside, % this produces strange margins!
    openright, % for new chapters
    notitlepage,
    cleardoubleempty,
    headsepline,
    pointlessnumbers,
    bibtotoc, idxtotoc,
    ]{scrbook}


% that's for virtualbookworm trim size
%\setlength{\paperwidth}{6in} \setlength{\paperheight}{9in}
% that's for CreateSpace trim size
\setlength{\paperwidth}{7.5in} \setlength{\paperheight}{9.25in}
\usepackage{pslatex} % -- times instead of computer modern
% pslatex should be replaced by this:
%\usepackage{mathptmx}
%\usepackage[scaled=.70]{helvet}
%\usepackage{courier}
% pslatex does not work with T1 encoding. <> Problem?

% typeare{calc} without BCOR results to a DIV of 8 for 11pt
%\typearea[0.465in]{15}
% the decision is 14
\typearea[0.5in]{14}
%\typearea[0.535in]{13}
%\typearea[0.57in]{12}

% headings
\usepackage{scrpage2} % for headers
 \setkomafont{pagehead}{\scshape\small}
 \setkomafont{pagenumber}{\scshape\small}
 \automark[section]{chapter}
 \ohead[]{\pagemark}
 \chead[]{}
 \ihead[]{\headmark}
 \ofoot[]{} \cfoot[]{} \ifoot[]{}

%\tolerance=500 % to avoid lines sticking out into the margin
               % needed for 'high-performance' in Intro - contributions
\emergencystretch=2em
% or tol. to 500 and emerg. to 1em?
% pagebreak was ok with 500 and 1em
\interfootnotelinepenalty=10000


% use BCOR = (paperwidth-textwidth)/4
% A4: 210mm x 297mm
% B5: 176mm x 250mm
% Java book: 185mm x 232mm
% Engblom: 120x188 (without head)
% Java: 127x187 (without head)
% 1pt = 1/72.27 in = 0.351 mm
% Thesis is 232mm x 185mm (9,1in x 7,28in) with 11pt
% Springer is 9,2in x 6in with 10pt
% Altera Quaruts handbook is 9in x 7in with 9pt


% for book
% 'Java-format' 526pt x 660pt (Ghostscript)
%\setlength{\paperwidth}{185mm} \setlength{\paperheight}{232mm}
% use that BCOR setting with twoside to compensate the margin
%\areaset[13.75mm]{130mm}{200mm} % Java book format



% use 10pt for code instead of 11pt - but I still would prefer Lucida Typewriter
%\newfont{\myttfont}{cmss10 scaled 1000}
%\newfont{\myttbfont}{cmssdc10 scaled 1000}
%
% This IS Lucida Typewriter
%\newfont{\myttfont}{plsr8r scaled 950}
%\newfont{\myttbfont}{plsb8r scaled 950}
%\newfont{\myttifont}{plsro8r scaled 950}
%%\newfont{\mytttextfont}{plsr8r}

% Lucida is perhaps available in the new Tex installation!!!!
% does not really work!!!
%\newfont{\myttfont}{hlsrt8r scaled 950}
%\newfont{\myttbfont}{hlsbt8r scaled 950}
%\newfont{\myttifont}{hlsrot8r scaled 950}

% I used these .ttf for the official Thesis
%..\ttf2pt1 -e -b LucidaTypewriterRegular.ttf plsr8a
%..\ttf2pt1 -e -b LucidaTypewriterBold.ttf plsb8a
%..\ttf2pt1 -e -b LucidaTypewriterOblique.ttf plsro8a
%..\ttf2pt1 -e -b LucidaTypewriterBoldOblique.ttf plsbo8a

%\newcommand{\javatt}{\myttfont}
%\newcommand{\javattb}{\myttbfont}
%\newcommand{\javatti}{\myttifont}
%\newcommand{\javatext}{\myttfont}
%
%\newcommand{\picscale}{0.909}
%\newcommand{\excelwidth}{11cm}

% end book

% for B5
%\newfont{\javatt}{cmss10}
%\newfont{\javattb}{cmssdc10}
%\newcommand{\picscale}{0.833}
%\newcommand{\excelwidth}{10cm}


%% was set to 9 for the green book
%% should be ok with 11 for the orange book
%\newfont{\javatt}{cmss11}
%\newfont{\javattb}{cmssdc11}
%% TODO find an italic
%\newfont{\javatti}{cmss11}
%\newcommand{\javatext}{\javatt}

\newcommand{\picscale}{1}
\newcommand{\excelwidth}{12cm}

% for SimpCon description
\newcommand{\sign}[1]{{\code{#1}}}

% for chapter head without a number
% \renewcommand{\chaptermark}[1]{\def\myleftmark{#1}}
% \ihead{\myleftmark} \chead{} \ohead{{\rightmark}}

\setkomafont{captionlabel}{\sffamily\bfseries}

\usepackage{latexsym}
\usepackage{graphicx}
\usepackage{amsmath}
\usepackage{amsthm}
\usepackage{longtable}
\usepackage{booktabs}

% I would need Lucida Console!!!
%
%\newfont{\javatt}{pltt12} % lucida teletype, better than normal but with serifs
%\newfont{\javatt}{plss12} % lucida no serifes, but variable spacing
%\newfont{\javatt}{plss10 scaled 1200}
%\newfont{\javattb}{plssdc10 scaled 1200}
% cmss is NOT a tt font....

\usepackage{listings}
%\lstset{language=Java,keywordstyle=,
%basicstyle=\small\javatt,emphstyle=\small\javattb,commentstyle=\small\javatti,
%%basicstyle=\small,
%%
%showstringspaces=false,captionpos=b,columns=flexible}
%%\lstset{basicstyle=\small,language=Java,columns=flexible}

\lstset{basicstyle=\sffamily\small,keywordstyle=\sffamily\small,language=Java,captionpos=b,columns=flexible,showstringspaces=false}

\usepackage{array}
\usepackage{dcolumn}
\newcommand{\cc}[1]{\multicolumn{1}{c}{#1}}
\newcolumntype{d}[1]{D{.}{.}{#1}}

\usepackage{capt-of}
\usepackage[colorlinks=true,urlcolor=black,linkcolor=black,citecolor=black]{hyperref}

% ----------------------

\usepackage{makeidx}
\makeindex


\usepackage{import} % for subimport text and graphics from subdirectory
% does not work with latex2html!


%\newcommand{\codefoot}{\textsf}
%\newcommand{\code}[1]{{\javatext#1}}
%\newcommand{\codeb}[1]{{\javattb#1}}

\newcommand{\codefoot}{\sffamily}
\newcommand{\code}[1]{{\small\sffamily{#1}}}


%\newcommand{\cmd}[1]{{\texttt{#1}}}
%\newcommand{\dirent}[1]{{\texttt{#1}}}
%\newcommand{\menuitem}[1]{\textsf{\textbf{#1}}}
\newcommand{\cmd}[1]{{\code{#1}}\index{#1}}
\newcommand{\cmdb}[1]{{\textsf{\textbf{#1}}}\index{#1}}
\newcommand{\dirent}[1]{{\code{#1}}}
\newcommand{\menuitem}[1]{\textsf{\textsl{#1}}}

\newcommand{\idx}[1]{#1\index{#1}}

% for flow.tex - part of index helper
\newcommand{\eei}[1]{%
\index{extension!\texttt{#1}}\texttt{#1}}

% JVs et al
%\newcommand{\ea}{et al.\xspace}
\newcommand{\ea}{et al.\ }

% used in GC period calculation
\newtheorem{lemma}{Lemma}
\newtheorem{theorem}{Theorem}

%\begin{htmlonly}
%\renewcommand{\code}[1]{{\texttt{#1}}} % for html2LaTeX
%\newcommand{\toprule}{\hline}
%\newcommand{\midrule}{\hline}
%\newcommand{\bottomrule}{\hline}
%\end{htmlonly}

% net wirklich notwendig -- h�ngt von code generierung ab
%\begin{htmlonly}
%\renewcommand{\javatt}{\texttt}
%\renewcommand{\javattb}{\texttt\bfseries}
%\end{htmlonly}

%\code{\hyphenchar\font=-1}

\newcommand{\mycomment}[1]{}

\newcommand{\instr}[6]{
    \begin{table}
    \index{microcode!{\textsf{#1}}}
        \begin{tabular}{ll}
            \emph{\large\textbf{#1}} & \\
            \\ \\
            \textbf{Operation} & #2 \\ \\
            \textbf{Opcode} & \texttt{#3} \\ \\
            \textbf{Dataflow} & \parbox[t]{10.5cm}{\(#4\)}\\ \\
            \textbf{JVM equivalent} & \parbox[t]{10.5cm}{\code{#5}} \\ \\
            \textbf{Description} & \parbox[t]{10.5cm}{#6}\\
        \end{tabular}
    \end{table}
}


The instruction set of JOP, the so-called microcode, is described in
this appendix. Each instruction consists of a single instruction
word (8 bits) without extra operands and executes in a single
cycle\footnote{The only multicycle instruction is \codefoot{wait}
and depends on the access time of the external memory}.
\tablename~\ref{tab:appendix:hwreg} lists the registers and internal
memory areas that are used in the dataflow description.

\begin{table}[h]
  \centering
  \begin{tabular}{ll}
    \toprule
    Name & Description \\
    \midrule
    A & Top of the stack\\
    B & The element one below the top of stack\\
    stack[] & The stack buffer for the rest of the stack\\
    sp & The stack pointer for the stack buffer\\
    vp & The variable pointer. Points to the first local in
    the stack buffer\\
    ar & Address register for indirect stack access\\
    pc & Microcode program counter\\
    jpc & Program counter for the Java bytecode\\
    opd & 8 bit operand from the bytecode fetch unit\\
    opd$_{16}$ & 16 bit operand from the bytecode fetch unit\\
    memrda & Read address register of the memory subsystem\\
    memwra & Write address register of the memory subsystem\\
    memrdd & Read data register of the memory subsystem\\
    memwrd & Write data register of the memory subsystem\\
    mula, mulb & Operands of the hardware multiplier\\
    mulr & Result register of the hardware multiplier\\
    membcr & Bytecode address and length register of the memory
    subsystem\\
    bcstart & Method start address register in the method cache\\
	memidx & Index register for native field access \\
    \bottomrule
  \end{tabular}
  \caption{JOP hardware registers and memory areas}\label{tab:appendix:hwreg}
\end{table}

\clearpage
\input{appendix/microcode}


%\end{document}


\chapter{Bytecode Execution Time} \label{appx:bytecode}
%\documentclass[%draft,
    11pt, % use explicit paper size
    headinclude, footexclude,
    twoside, % this produces strange margins!
    openright, % for new chapters
    notitlepage,
    cleardoubleempty,
    headsepline,
    pointlessnumbers,
    bibtotoc, idxtotoc,
    ]{scrbook}


% that's for virtualbookworm trim size
%\setlength{\paperwidth}{6in} \setlength{\paperheight}{9in}
% that's for CreateSpace trim size
\setlength{\paperwidth}{7.5in} \setlength{\paperheight}{9.25in}
\usepackage{pslatex} % -- times instead of computer modern
% pslatex should be replaced by this:
%\usepackage{mathptmx}
%\usepackage[scaled=.70]{helvet}
%\usepackage{courier}
% pslatex does not work with T1 encoding. <> Problem?

% typeare{calc} without BCOR results to a DIV of 8 for 11pt
%\typearea[0.465in]{15}
% the decision is 14
\typearea[0.5in]{14}
%\typearea[0.535in]{13}
%\typearea[0.57in]{12}

% headings
\usepackage{scrpage2} % for headers
 \setkomafont{pagehead}{\scshape\small}
 \setkomafont{pagenumber}{\scshape\small}
 \automark[section]{chapter}
 \ohead[]{\pagemark}
 \chead[]{}
 \ihead[]{\headmark}
 \ofoot[]{} \cfoot[]{} \ifoot[]{}

%\tolerance=500 % to avoid lines sticking out into the margin
               % needed for 'high-performance' in Intro - contributions
\emergencystretch=2em
% or tol. to 500 and emerg. to 1em?
% pagebreak was ok with 500 and 1em
\interfootnotelinepenalty=10000


% use BCOR = (paperwidth-textwidth)/4
% A4: 210mm x 297mm
% B5: 176mm x 250mm
% Java book: 185mm x 232mm
% Engblom: 120x188 (without head)
% Java: 127x187 (without head)
% 1pt = 1/72.27 in = 0.351 mm
% Thesis is 232mm x 185mm (9,1in x 7,28in) with 11pt
% Springer is 9,2in x 6in with 10pt
% Altera Quaruts handbook is 9in x 7in with 9pt


% for book
% 'Java-format' 526pt x 660pt (Ghostscript)
%\setlength{\paperwidth}{185mm} \setlength{\paperheight}{232mm}
% use that BCOR setting with twoside to compensate the margin
%\areaset[13.75mm]{130mm}{200mm} % Java book format



% use 10pt for code instead of 11pt - but I still would prefer Lucida Typewriter
%\newfont{\myttfont}{cmss10 scaled 1000}
%\newfont{\myttbfont}{cmssdc10 scaled 1000}
%
% This IS Lucida Typewriter
%\newfont{\myttfont}{plsr8r scaled 950}
%\newfont{\myttbfont}{plsb8r scaled 950}
%\newfont{\myttifont}{plsro8r scaled 950}
%%\newfont{\mytttextfont}{plsr8r}

% Lucida is perhaps available in the new Tex installation!!!!
% does not really work!!!
%\newfont{\myttfont}{hlsrt8r scaled 950}
%\newfont{\myttbfont}{hlsbt8r scaled 950}
%\newfont{\myttifont}{hlsrot8r scaled 950}

% I used these .ttf for the official Thesis
%..\ttf2pt1 -e -b LucidaTypewriterRegular.ttf plsr8a
%..\ttf2pt1 -e -b LucidaTypewriterBold.ttf plsb8a
%..\ttf2pt1 -e -b LucidaTypewriterOblique.ttf plsro8a
%..\ttf2pt1 -e -b LucidaTypewriterBoldOblique.ttf plsbo8a

%\newcommand{\javatt}{\myttfont}
%\newcommand{\javattb}{\myttbfont}
%\newcommand{\javatti}{\myttifont}
%\newcommand{\javatext}{\myttfont}
%
%\newcommand{\picscale}{0.909}
%\newcommand{\excelwidth}{11cm}

% end book

% for B5
%\newfont{\javatt}{cmss10}
%\newfont{\javattb}{cmssdc10}
%\newcommand{\picscale}{0.833}
%\newcommand{\excelwidth}{10cm}


%% was set to 9 for the green book
%% should be ok with 11 for the orange book
%\newfont{\javatt}{cmss11}
%\newfont{\javattb}{cmssdc11}
%% TODO find an italic
%\newfont{\javatti}{cmss11}
%\newcommand{\javatext}{\javatt}

\newcommand{\picscale}{1}
\newcommand{\excelwidth}{12cm}

% for SimpCon description
\newcommand{\sign}[1]{{\code{#1}}}

% for chapter head without a number
% \renewcommand{\chaptermark}[1]{\def\myleftmark{#1}}
% \ihead{\myleftmark} \chead{} \ohead{{\rightmark}}

\setkomafont{captionlabel}{\sffamily\bfseries}

\usepackage{latexsym}
\usepackage{graphicx}
\usepackage{amsmath}
\usepackage{amsthm}
\usepackage{longtable}
\usepackage{booktabs}

% I would need Lucida Console!!!
%
%\newfont{\javatt}{pltt12} % lucida teletype, better than normal but with serifs
%\newfont{\javatt}{plss12} % lucida no serifes, but variable spacing
%\newfont{\javatt}{plss10 scaled 1200}
%\newfont{\javattb}{plssdc10 scaled 1200}
% cmss is NOT a tt font....

\usepackage{listings}
%\lstset{language=Java,keywordstyle=,
%basicstyle=\small\javatt,emphstyle=\small\javattb,commentstyle=\small\javatti,
%%basicstyle=\small,
%%
%showstringspaces=false,captionpos=b,columns=flexible}
%%\lstset{basicstyle=\small,language=Java,columns=flexible}

\lstset{basicstyle=\sffamily\small,keywordstyle=\sffamily\small,language=Java,captionpos=b,columns=flexible,showstringspaces=false}

\usepackage{array}
\usepackage{dcolumn}
\newcommand{\cc}[1]{\multicolumn{1}{c}{#1}}
\newcolumntype{d}[1]{D{.}{.}{#1}}

\usepackage{capt-of}
\usepackage[colorlinks=true,urlcolor=black,linkcolor=black,citecolor=black]{hyperref}

% ----------------------

\usepackage{makeidx}
\makeindex


\usepackage{import} % for subimport text and graphics from subdirectory
% does not work with latex2html!


%\newcommand{\codefoot}{\textsf}
%\newcommand{\code}[1]{{\javatext#1}}
%\newcommand{\codeb}[1]{{\javattb#1}}

\newcommand{\codefoot}{\sffamily}
\newcommand{\code}[1]{{\small\sffamily{#1}}}


%\newcommand{\cmd}[1]{{\texttt{#1}}}
%\newcommand{\dirent}[1]{{\texttt{#1}}}
%\newcommand{\menuitem}[1]{\textsf{\textbf{#1}}}
\newcommand{\cmd}[1]{{\code{#1}}\index{#1}}
\newcommand{\cmdb}[1]{{\textsf{\textbf{#1}}}\index{#1}}
\newcommand{\dirent}[1]{{\code{#1}}}
\newcommand{\menuitem}[1]{\textsf{\textsl{#1}}}

\newcommand{\idx}[1]{#1\index{#1}}

% for flow.tex - part of index helper
\newcommand{\eei}[1]{%
\index{extension!\texttt{#1}}\texttt{#1}}

% JVs et al
%\newcommand{\ea}{et al.\xspace}
\newcommand{\ea}{et al.\ }

% used in GC period calculation
\newtheorem{lemma}{Lemma}
\newtheorem{theorem}{Theorem}

%\begin{htmlonly}
%\renewcommand{\code}[1]{{\texttt{#1}}} % for html2LaTeX
%\newcommand{\toprule}{\hline}
%\newcommand{\midrule}{\hline}
%\newcommand{\bottomrule}{\hline}
%\end{htmlonly}

% net wirklich notwendig -- h�ngt von code generierung ab
%\begin{htmlonly}
%\renewcommand{\javatt}{\texttt}
%\renewcommand{\javattb}{\texttt\bfseries}
%\end{htmlonly}

%\code{\hyphenchar\font=-1}

\newcommand{\mycomment}[1]{}

\newcommand{\instr}[6]{
    \begin{table}
    \index{microcode!{\textsf{#1}}}
        \begin{tabular}{ll}
            \emph{\large\textbf{#1}} & \\
            \\ \\
            \textbf{Operation} & #2 \\ \\
            \textbf{Opcode} & \texttt{#3} \\ \\
            \textbf{Dataflow} & \parbox[t]{10.5cm}{\(#4\)}\\ \\
            \textbf{JVM equivalent} & \parbox[t]{10.5cm}{\code{#5}} \\ \\
            \textbf{Description} & \parbox[t]{10.5cm}{#6}\\
        \end{tabular}
    \end{table}
}


\tablename~\ref{tab:appendix:bytecode} lists the bytecodes of the
JVM with their opcode, mnemonics, the implementation type and the
execution time on JOP. In the implementation column \emph{hw} means
that this bytecode has a microcode equivalent, \emph{mc} means that
a microcode sequence implements the bytecode, \emph{Java} means the
bytecode is implemented in Java, and a `-' indicates that this
bytecode is not yet implemented. For bytecodes with a variable
execution time the minimum and maximum values are given.

\begin{longtable}{rllr}
    \toprule
    Opcode & Instruction & Implementation & Cycles \\
    \midrule
    \endhead
    \bottomrule
    \caption{Implemented bytecodes and execution time in cycles}
    \label{tab:appendix:bytecode}
    \endfoot
%   18 & ldc & mc & 3+m \\
    0 & nop & hw & 1 \\
1 & aconst\_null & hw & 1 \\
2 & iconst\_m1 & hw & 1 \\
3 & iconst\_0 & hw & 1 \\
4 & iconst\_1 & hw & 1 \\
5 & iconst\_2 & hw & 1 \\
6 & iconst\_3 & hw & 1 \\
7 & iconst\_4 & hw & 1 \\
8 & iconst\_5 & hw & 1 \\
9 & lconst\_0 & mc & 2 \\
10 & lconst\_1 & mc & 2 \\
11 & fconst\_0 & mc & 1 \\
12 & fconst\_1 & Java &  \\
13 & fconst\_2 & Java &  \\
14 & dconst\_0 & mc & 2 \\
15 & dconst\_1 & Java &  \\
16 & bipush & mc & 2 \\
17 & sipush & mc & 3 \\
18 & ldc & mc & 7+r \\
19 & ldc\_w & mc & 8+r \\
20 & ldc2\_w\footnotemark[20] & mc & 17+2*r \\
21 & iload & mc & 2 \\
22 & lload & mc & 11 \\
23 & fload & mc & 2 \\
24 & dload & mc & 11 \\
25 & aload & mc & 2 \\
26 & iload\_0 & hw & 1 \\
27 & iload\_1 & hw & 1 \\
28 & iload\_2 & hw & 1 \\
29 & iload\_3 & hw & 1 \\
30 & lload\_0 & mc & 2 \\
31 & lload\_1 & mc & 2 \\
32 & lload\_2 & mc & 2 \\
33 & lload\_3 & mc & 11 \\
34 & fload\_0 & hw & 1 \\
35 & fload\_1 & hw & 1 \\
36 & fload\_2 & hw & 1 \\
37 & fload\_3 & hw & 1 \\
38 & dload\_0 & mc & 2 \\
39 & dload\_1 & mc & 2 \\
40 & dload\_2 & mc & 2 \\
41 & dload\_3 & mc & 11 \\
42 & aload\_0 & hw & 1 \\
43 & aload\_1 & hw & 1 \\
44 & aload\_2 & hw & 1 \\
45 & aload\_3 & hw & 1 \\
%46 & iaload\footnotemark[46] & mc & 32+3*r \\
46 & iaload\footnotemark[46] & hw & 6+3*r \\
47 & laload & mc & 43+4*r \\
48 & faload\footnotemark[46] & hw & 6+3*r \\
49 & daload & hw & 43+4*r \\
50 & aaload\footnotemark[46] & hw & 6+3*r \\
51 & baload\footnotemark[46] & hw & 6+3*r \\
52 & caload\footnotemark[46] & hw & 6+3*r \\
53 & saload\footnotemark[46] & hw & 6+3*r \\
54 & istore & mc & 2 \\
55 & lstore & mc & 11 \\
56 & fstore & mc & 2 \\
57 & dstore & mc & 11 \\
58 & astore & mc & 2 \\
59 & istore\_0 & hw & 1 \\
60 & istore\_1 & hw & 1 \\
61 & istore\_2 & hw & 1 \\
62 & istore\_3 & hw & 1 \\
63 & lstore\_0 & mc & 2 \\
64 & lstore\_1 & mc & 2 \\
65 & lstore\_2 & mc & 2 \\
66 & lstore\_3 & mc & 11 \\
67 & fstore\_0 & hw & 1 \\
68 & fstore\_1 & hw & 1 \\
69 & fstore\_2 & hw & 1 \\
70 & fstore\_3 & hw & 1 \\
71 & dstore\_0 & mc & 2 \\
72 & dstore\_1 & mc & 2 \\
73 & dstore\_2 & mc & 2 \\
74 & dstore\_3 & mc & 11 \\
75 & astore\_0 & hw & 1 \\
76 & astore\_1 & hw & 1 \\
77 & astore\_2 & hw & 1 \\
78 & astore\_3 & hw & 1 \\
%79 & iastore\footnotemark[79] & mc & 35+2*r+w \\
79 & iastore\footnotemark[79] & hw & 10+2*r+w \\
80 & lastore\footnotemark[1] & mc & 48+2*r+2*w \\
81 & fastore\footnotemark[79] & hw & 10+2*r+w \\
82 & dastore\footnotemark[82] & mc & 48+2*r+w \\
83 & aastore & Java & \\
84 & bastore\footnotemark[79] & hw & 10+2*r+w \\
85 & castore\footnotemark[79] & hw & 10+2*r+w \\
86 & sastore\footnotemark[79] & hw & 10+2*r+w \\
87 & pop & hw & 1 \\
88 & pop2 & mc & 2 \\
89 & dup & hw & 1 \\
90 & dup\_x1 & mc & 5 \\
91 & dup\_x2 & mc & 7 \\
92 & dup2 & mc & 6 \\
93 & dup2\_x1 & mc & 8 \\
94 & dup2\_x2 & mc & 10 \\
95 & swap\footnotemark[2] & mc & 4 \\
96 & iadd & hw & 1 \\
97 & ladd & mc & 26 \\
98 & fadd & Java &  \\
99 & dadd & - &  \\
100 & isub & hw & 1 \\
101 & lsub & mc & 38 \\
102 & fsub & Java &  \\
103 & dsub & - &  \\
104 & imul & mc & 35 \\
105 & lmul & Java &  \\
106 & fmul & Java &  \\
107 & dmul & - &  \\
108 & idiv & Java &  \\
109 & ldiv & Java &  \\
110 & fdiv & Java &  \\
111 & ddiv & - &  \\
112 & irem & Java &  \\
113 & lrem & Java &  \\
114 & frem & Java &  \\
115 & drem & - &  \\
116 & ineg & mc & 4 \\
117 & lneg & mc & 34 \\
118 & fneg & Java &  \\
119 & dneg & - &  \\
120 & ishl & hw & 1 \\
121 & lshl & mc & 28 \\
122 & ishr & hw & 1 \\
123 & lshr & mc & 28 \\
124 & iushr & hw & 1 \\
125 & lushr & mc & 28 \\
126 & iand & hw & 1 \\
127 & land & mc & 8 \\
128 & ior & hw & 1 \\
129 & lor & mc & 8 \\
130 & ixor & hw & 1 \\
131 & lxor & mc & 8 \\
132 & iinc & mc & 8 \\
133 & i2l & mc & 5 \\
134 & i2f & Java &  \\
135 & i2d & - &  \\
136 & l2i & mc & 3 \\
137 & l2f & - &  \\
138 & l2d & - &  \\
139 & f2i & Java &  \\
140 & f2l & - &  \\
141 & f2d & - &  \\
142 & d2i & - &  \\
143 & d2l & - &  \\
144 & d2f & - &  \\
145 & i2b & Java &  \\
146 & i2c & mc & 2 \\
147 & i2s & Java &  \\
148 & lcmp & mc &  85 \\
149 & fcmpl & Java &  \\
150 & fcmpg & Java &  \\
151 & dcmpl & - &  \\
152 & dcmpg & - &  \\
153 & ifeq & mc & 4 \\
154 & ifne & mc & 4 \\
155 & iflt & mc & 4 \\
156 & ifge & mc & 4 \\
157 & ifgt & mc & 4 \\
158 & ifle & mc & 4 \\
159 & if\_icmpeq & mc & 4 \\
160 & if\_icmpne & mc & 4 \\
161 & if\_icmplt & mc & 4 \\
162 & if\_icmpge & mc & 4 \\
163 & if\_icmpgt & mc & 4 \\
164 & if\_icmple & mc & 4 \\
165 & if\_acmpeq & mc & 4 \\
166 & if\_acmpne & mc & 4 \\
167 & goto & mc & 4 \\
168 & jsr & \emph{not used} &  \\
169 & ret & \emph{not used} &  \\
170 & tableswitch\footnotemark[170] & Java & \\
171 & lookupswitch\footnotemark[171] & Java &  \\
172 & ireturn\footnotemark[172] & mc &  23+r+l \\
173 & lreturn\footnotemark[173] & mc &  25+r+l \\
174 & freturn\footnotemark[172] & mc &  23+r+l \\
175 & dreturn\footnotemark[173] & mc &  25+r+l \\
176 & areturn\footnotemark[172] & mc &  23+r+l \\
177 & return\footnotemark[177] & mc &  21+r+l \\
178 & getstatic & mc & 5+r \\
179 & putstatic & mc & 5+w \\
180 & getfield & hw & 8+2*r \\
181 & putfield & hw & 9+r+w \\
182 & invokevirtual\footnotemark[182] & mc & 98+4r+l \\
183 & invokespecial\footnotemark[183] & mc &  73+3*r+l \\
184 & invokestatic\footnotemark[183] & mc &  72+3*r+l \\
185 & invokeinterface\footnotemark[185] & mc &  111+6r+l \\
186 & unused\_ba & - &  \\
187 & new\footnotemark[187] & Java &   \\
188 & newarray\footnotemark[188] & Java &  \\
189 & anewarray & Java &  \\
190 & arraylength & mc & 6+r \\
191 & athrow\footnotemark[3] & Java &  \\
192 & checkcast & Java &  \\
193 & instanceof & Java &  \\
194 & monitorenter & mc & 20 \\
195 & monitorexit & mc & 22 \\
196 & wide & \emph{not used} &  \\
197 & multianewarray\footnotemark[4] & Java &  \\
198 & ifnull & mc & 4 \\
199 & ifnonnull & mc & 4 \\
200 & goto\_w & \emph{not used} &  \\
201 & jsr\_w & \emph{not used} &  \\
202 & breakpoint & - &  \\
203 & reserved & - &  \\
204 & reserved & - &  \\
205 & reserved & - &  \\
206 & reserved & - &  \\
207 & reserved & - &  \\
208 & reserved & - &  \\
209 & jopsys\_rd\footnotemark[209] & mc & 4+r \\
210 & jopsys\_wr & mc & 5+w \\
211 & jopsys\_rdmem & mc & 4+r \\
212 & jopsys\_wrmem & mc & 5+w \\
213 & jopsys\_rdint & mc & 3 \\
214 & jopsys\_wrint & mc & 3 \\
215 & jopsys\_getsp & mc & 3 \\
216 & jopsys\_setsp & mc & 4 \\
217 & jopsys\_getvp & hw & 1 \\
218 & jopsys\_setvp & mc & 2 \\
219 & jopsys\_int2ext\footnotemark[219] & mc & 14+r+n*(23+w) \\
220 & jopsys\_ext2int\footnotemark[220] & mc & 14+r+n*(23+r) \\
221 & jopsys\_nop & mc & 1 \\
222 & jopsys\_invoke & mc &  \\
223 & jopsys\_cond\_move & mc & 5 \\
224 & getstatic\_ref & mc & 5+r \\
225 & putstatic\_ref & Java & \\
226 & getfield\_ref & mc & 8+2*r \\
227 & putfield\_ref & Java & \\
228 & getstatic\_long & mc & \\
229 & putstatic\_long & mc & \\
230 & getfield\_long & mc & \\
231 & putfield\_long & mc & \\
232 & jopsys\_memcpy & mc &  \\
233 & reserved & - \\
234 & reserved & - \\
235 & reserved & - \\
236 & invokesuper & mc & - \\
237 & reserved & - \\
238 & reserved & - \\
239 & reserved & - \\
240 & sys\_int\footnotemark[240] & Java \\
241 & sys\_exc\footnotemark[240] & Java \\
242 & reserved & - \\
243 & reserved & - \\
244 & reserved & - \\
245 & reserved & - \\
246 & reserved & - \\
247 & reserved & - \\
248 & reserved & - \\
249 & reserved & - \\
250 & reserved & - \\
251 & reserved & - \\
252 & reserved & - \\
253 & reserved & - \\
254 & sys\_noimp & Java \\
255 & sys\_init & \emph{not used} \\

\end{longtable}

\footnotetext[1]{The exact value is given below.}

\footnotetext[2]{Not tested as javac does not emit the \code{swap}
bytecode.}

\footnotetext[3]{A simple version that stops the JVM. No catch
support.}

\footnotetext[4]{Only dimension 2 supported.}

\footnotetext[20]{The exact value is
    $17+\left\{\begin{array}{r@{\quad:\quad}l}
    r-2 & r>2 \\
    0   & r\le2
    \end{array} \right.
    +
    \left\{\begin{array}{r@{\quad:\quad}l}
    r-1 & r>1 \\
    0   & r\le1
    \end{array} \right.
    $
}

\footnotetext[46]{The exact value is
%    $19+r+\left\{\begin{array}{r@{\quad:\quad}l}
%    r-2 & r\ge6 \\
%    4   & r<6
%    \end{array} \right. $
 \emph{no hidden wait states at the moment.}
}

\footnotetext[79]{The exact value is
%    $22+\left\{\begin{array}{r@{\quad:\quad}l}
%    r-2 & r\ge6 \\
%    4   & r<6
%    \end{array} \right.
%    +w
%    $
 \emph{no hidden wait states at the moment.}
}

\footnotetext[82] {TODO: exact value}

\footnotetext[170]{\codefoot{tableswitch} execution time depends to
a great extent on the caching of the corresponding Java method or
the memory transfer time for the method.}

\footnotetext[171]{\codefoot{lookupswitch} execution time depends to
a great extent on the caching of the corresponding Java method or
the memory transfer time for the method. \codefoot{lookupswitch}
also depends on the argument as it performs a linear search in the
jump table.}

%172 & ireturn & mc &  23+r+b\footnotemark[172] \\
\footnotetext[172]{The exact value is:
    $
    23+\left\{\begin{array}{r@{\quad:\quad}l}
    r-3 & r>3 \\
    0   & r\le3
    \end{array} \right.
    +
% the saved cycles are counted from the instruction after stbcrd
% up to and including the last wait
    \left\{\begin{array}{r@{\quad:\quad}l}
    l-10 & l>10 \\
    0   & l\le10
    \end{array} \right.
    $
}

%173 & lreturn & mc &  25+r+b\footnotemark[173] \\
\footnotetext[173]{The exact value is:
    $
    25+\left\{\begin{array}{r@{\quad:\quad}l}
    r-3 & r>3 \\
    0   & r\le3
    \end{array} \right.
    +
    \left\{\begin{array}{r@{\quad:\quad}l}
    l-11 & l>11 \\
    0   & l\le11
    \end{array} \right.
    $
}



%177 & return & mc &  21+r+b\footnotemark[177] \\
\footnotetext[177]{ The exact value is:
    $
    21+\left\{\begin{array}{r@{\quad:\quad}l}
    r-3 & r>3 \\
    0   & r\le3
    \end{array} \right.
    +
    \left\{\begin{array}{r@{\quad:\quad}l}
    l-9 & l>9 \\
    0   & l\le9
    \end{array} \right.
    $
}

%182 & invokevirtual & mc & 82+4r+b\footnotemark[182] \\
\footnotetext[182]{The exact value is:
    $
    100+2r+
    \left\{\begin{array}{r@{\quad:\quad}l}
    r-3 & r>3 \\
    0   & r\le3
    \end{array} \right.
    +
    \left\{\begin{array}{r@{\quad:\quad}l}
    r-2 & r>2 \\
    0   & r\le2
    \end{array} \right.
    +
    \left\{\begin{array}{r@{\quad:\quad}l}
    l-37 & l>37 \\
    0   & l\le37
    \end{array} \right.
    $
}

%183 & invokespecial & mc &  72+3r+b\footnotemark[182] \\
%184 & invokestatic & mc &  72+3r+b\footnotemark[182] \\
\footnotetext[183]{The exact value is:
    $
    73+r+
    \left\{\begin{array}{r@{\quad:\quad}l}
    r-3 & r>3 \\
    0   & r\le3
    \end{array} \right.
    +
    \left\{\begin{array}{r@{\quad:\quad}l}
    r-2 & r>2 \\
    0   & r\le2
    \end{array} \right.
    +
    \left\{\begin{array}{r@{\quad:\quad}l}
    l-37 & l>37 \\
    0   & l\le37
    \end{array} \right.
    $
}

%185 & invokeinterface & mc &  112+6r+b\footnotemark[182] \\
\footnotetext[185]{The exact value is:
    $
    111+4r+
    \left\{\begin{array}{r@{\quad:\quad}l}
    r-3 & r>3 \\
    0   & r\le3
    \end{array} \right.
    +
    \left\{\begin{array}{r@{\quad:\quad}l}
    r-2 & r>2 \\
    0   & r\le2
    \end{array} \right.
    +
    \left\{\begin{array}{r@{\quad:\quad}l}
    l-37 & l>37 \\
    0   & l\le37
    \end{array} \right.
    $
}



\footnotetext[187]{\codefoot{new} execution time depends to a great
extent on the caching of the corresponding Java method or the memory
transfer time for the method. \codefoot{new} also depends on the
size of the created object as the memory for the object is filled
with zeros -- This will change with the GC}

%188 & newarray & mc & 12+w-7\footnotemark[188] \\
\footnotetext[188]{\codefoot{newarray} execution time depends to a
great extent on the caching of the corresponding Java method or the
memory transfer time for the method. \codefoot{newarray} also
depends on the size of the array as the memory for the object is
filled with zeros -- This will change with the GC}

\footnotetext[209]{The native instructions \codefoot{jopsys\_rd} and
\codefoot{jopsys\_wr} are alias to the \codefoot{jopsys\_rdmem} and
\codefoot{jopsys\_wrmem} instructions just for compatibility to
existing Java code. I/O devices are now memory mapped. In the case
for simple I/O devices there are no wait states and the exact values
are 4 and 5 cycles respective.}

%14+r+n*(23+w)
\footnotetext[219]{The exact value is
    $14+r+n(23+\left\{\begin{array}{r@{\quad:\quad}l}
    w-8 & w>8 \\
    0   & w\le8
    \end{array} \right. )$.
$n$ is the number of words transferred.}

%14+r+n*(23+w)
\footnotetext[220]{The exact value is
    $14+r+n(23+\left\{\begin{array}{r@{\quad:\quad}l}
    r-10 & r>10 \\
    0   & r\le10
    \end{array} \right. )$.
$n$ is the number of words transferred.}

%\footnotetext[240]{\emph{Is the interrupt and the exception still a
%bytecode or is it now inserted just as microcode address?}}


\subsection*{Memory Timing}

The external memory timing is defined in the top level VHDL file
(e.g.\ \code{jopcyc.vhd}) with \code{ram\_cnt} for the number of
cycles for a read and write access. At the moment there is no
difference for a read and write access. For the 100~MHz JOP with
15~ns SRAMs this access time is two cycles (\code{ram\_cnt}=2,
20~ns). Therefore the wait state $n_{ws}$ is 1 (\code{ram\_cnt-1}).
%
A basic memory read in microcode is as follows:
\begin{lstlisting}
    stmra    // start read with address store
    wait     // fill the pipeline with two
    wait     // wait instructions
    ldmrd    // push read result on TOS
\end{lstlisting}
%
In this sequence the \emph{last} \code{wait} executes for $1+n_{ws}$
cycles. Therefore the whole read sequence takes $4+n_{ws}$ cycles.
For the example with \code{ram\_cnt}=2 this basic memory read takes
5 cycles.

A memory write in microcode is as follows:
\begin{lstlisting}
    stmwa    // store address
    stmwd    // store data and start the write
    wait     // fill the pipeline with wait
    wait     // wait for the memory ready
\end{lstlisting}
The last wait again executes for $1+n_{ws}$ cycles and the basic
write takes $4+n_{ws}$ cycles. For the native bytecode \code
{jopsys\_wrmem} an additional \code{nop} instruction for the
\code{nxt} flag is necessary.

The read and write wait states $r_{ws}$ and $w_{ws}$ are:
\begin{equation*}
    r_{ws} = w_{ws} =
    \left\{\begin{array}{r@{\quad:\quad}l}
    ram\_cnt-1 & ram\_cnt>1 \\
    0   & ram\_cnt\le1
    \end{array} \right.
\end{equation*}

In the instruction timing we use $r$ and $w$ instead of $r_{ws}$ and
$w_{ws}$. The wait states can be hidden by other microcode
instructions between \code{stmra/wait} and \code{stmwd/wait}. The
exact value is given in the footnote.

\subsection*{Instruction Timing}

The bytecodes that access memory are indicated by an $r$ for a memory
read and an $w$ for a memory write at the cycles column ($r$ and $w$
are the additional wait states). The wait cycles for the memory
access have to be added to the execution time. These two values are
implementation dependent (clock frequency versus RAM access time,
data bus width); for the Cyclone EP1C6 board with 15~ns SRAMs and
100~MHz clock frequency these values are both 1 cycle
(\code{ram\_cnt}-1).

For some bytecodes, part of the memory latency can be hidden by
executing microcode during the memory access. However, these cycles
can only be subtracted when the wait states (\emph{r} or \emph{w})
are larger then 0 cycles. The exact execution time with the
subtraction of the saved cycles is given in the footnote.

\subsubsection*{Cache Load}


% We count the hidden cycles in the same way as for a read or write:
%   the instructions between stbcr and the first wait
%

Memory access time also determines the cache load time on a miss. For
the current implementation the cache load time is calculated as
follows: the wait state $c_{ws}$ for a single word cache load is:
\begin{equation*}
    c_{ws} =
    \left\{\begin{array}{r@{\quad:\quad}l}
    r_{ws} & r_{ws}>1 \\
    1   & r_{ws}\le1
    \end{array} \right.
\end{equation*}
%
On a method invoke or return, the respective method has to be loaded
into the cache on a cache miss. The load time $l$ is:
\[
    l =
    \left\{\begin{array}{r@{\quad:\quad}l}
    6+(n+1)(1+c_{ws}) & \mbox{cache miss} \\
    4   & \mbox{cache hit}
    \end{array} \right.
\]
where $n$ is the size of the method in number of 32-bit words. For
short methods, the load time of the method on a cache miss, or part
of it, is hidden by microcode execution. The exact value is given in
the footnote.

\subsubsection*{lastore}

% 48+2*r+2*w
\begin{equation*}
    t_{lastore} = 48+2r_{ws}+w_{ws} + \left\{\begin{array}{r@{\quad:\quad}l}
    w_{ws}-3 & w_{ws}>3 \\
    0   & w_{ws}\le3
    \end{array} \right.
\end{equation*}

%\subsubsection*{get/putfield/ref/long}
%
%TODO: add different values for 32-bit, 64-bit and reference type.
%
%TODO: add invokesuper - a special version of invokespecial

%\end{document}


\chapter{Benchmark Results} \label{appx:bench}
%\documentclass[%draft,
    11pt, % use explicit paper size
    headinclude, footexclude,
    twoside, % this produces strange margins!
    openright, % for new chapters
    notitlepage,
    cleardoubleempty,
    headsepline,
    pointlessnumbers,
    bibtotoc, idxtotoc,
    ]{scrbook}


% that's for virtualbookworm trim size
%\setlength{\paperwidth}{6in} \setlength{\paperheight}{9in}
% that's for CreateSpace trim size
\setlength{\paperwidth}{7.5in} \setlength{\paperheight}{9.25in}
\usepackage{pslatex} % -- times instead of computer modern
% pslatex should be replaced by this:
%\usepackage{mathptmx}
%\usepackage[scaled=.70]{helvet}
%\usepackage{courier}
% pslatex does not work with T1 encoding. <> Problem?

% typeare{calc} without BCOR results to a DIV of 8 for 11pt
%\typearea[0.465in]{15}
% the decision is 14
\typearea[0.5in]{14}
%\typearea[0.535in]{13}
%\typearea[0.57in]{12}

% headings
\usepackage{scrpage2} % for headers
 \setkomafont{pagehead}{\scshape\small}
 \setkomafont{pagenumber}{\scshape\small}
 \automark[section]{chapter}
 \ohead[]{\pagemark}
 \chead[]{}
 \ihead[]{\headmark}
 \ofoot[]{} \cfoot[]{} \ifoot[]{}

%\tolerance=500 % to avoid lines sticking out into the margin
               % needed for 'high-performance' in Intro - contributions
\emergencystretch=2em
% or tol. to 500 and emerg. to 1em?
% pagebreak was ok with 500 and 1em
\interfootnotelinepenalty=10000


% use BCOR = (paperwidth-textwidth)/4
% A4: 210mm x 297mm
% B5: 176mm x 250mm
% Java book: 185mm x 232mm
% Engblom: 120x188 (without head)
% Java: 127x187 (without head)
% 1pt = 1/72.27 in = 0.351 mm
% Thesis is 232mm x 185mm (9,1in x 7,28in) with 11pt
% Springer is 9,2in x 6in with 10pt
% Altera Quaruts handbook is 9in x 7in with 9pt


% for book
% 'Java-format' 526pt x 660pt (Ghostscript)
%\setlength{\paperwidth}{185mm} \setlength{\paperheight}{232mm}
% use that BCOR setting with twoside to compensate the margin
%\areaset[13.75mm]{130mm}{200mm} % Java book format



% use 10pt for code instead of 11pt - but I still would prefer Lucida Typewriter
%\newfont{\myttfont}{cmss10 scaled 1000}
%\newfont{\myttbfont}{cmssdc10 scaled 1000}
%
% This IS Lucida Typewriter
%\newfont{\myttfont}{plsr8r scaled 950}
%\newfont{\myttbfont}{plsb8r scaled 950}
%\newfont{\myttifont}{plsro8r scaled 950}
%%\newfont{\mytttextfont}{plsr8r}

% Lucida is perhaps available in the new Tex installation!!!!
% does not really work!!!
%\newfont{\myttfont}{hlsrt8r scaled 950}
%\newfont{\myttbfont}{hlsbt8r scaled 950}
%\newfont{\myttifont}{hlsrot8r scaled 950}

% I used these .ttf for the official Thesis
%..\ttf2pt1 -e -b LucidaTypewriterRegular.ttf plsr8a
%..\ttf2pt1 -e -b LucidaTypewriterBold.ttf plsb8a
%..\ttf2pt1 -e -b LucidaTypewriterOblique.ttf plsro8a
%..\ttf2pt1 -e -b LucidaTypewriterBoldOblique.ttf plsbo8a

%\newcommand{\javatt}{\myttfont}
%\newcommand{\javattb}{\myttbfont}
%\newcommand{\javatti}{\myttifont}
%\newcommand{\javatext}{\myttfont}
%
%\newcommand{\picscale}{0.909}
%\newcommand{\excelwidth}{11cm}

% end book

% for B5
%\newfont{\javatt}{cmss10}
%\newfont{\javattb}{cmssdc10}
%\newcommand{\picscale}{0.833}
%\newcommand{\excelwidth}{10cm}


%% was set to 9 for the green book
%% should be ok with 11 for the orange book
%\newfont{\javatt}{cmss11}
%\newfont{\javattb}{cmssdc11}
%% TODO find an italic
%\newfont{\javatti}{cmss11}
%\newcommand{\javatext}{\javatt}

\newcommand{\picscale}{1}
\newcommand{\excelwidth}{12cm}

% for SimpCon description
\newcommand{\sign}[1]{{\code{#1}}}

% for chapter head without a number
% \renewcommand{\chaptermark}[1]{\def\myleftmark{#1}}
% \ihead{\myleftmark} \chead{} \ohead{{\rightmark}}

\setkomafont{captionlabel}{\sffamily\bfseries}

\usepackage{latexsym}
\usepackage{graphicx}
\usepackage{amsmath}
\usepackage{amsthm}
\usepackage{longtable}
\usepackage{booktabs}

% I would need Lucida Console!!!
%
%\newfont{\javatt}{pltt12} % lucida teletype, better than normal but with serifs
%\newfont{\javatt}{plss12} % lucida no serifes, but variable spacing
%\newfont{\javatt}{plss10 scaled 1200}
%\newfont{\javattb}{plssdc10 scaled 1200}
% cmss is NOT a tt font....

\usepackage{listings}
%\lstset{language=Java,keywordstyle=,
%basicstyle=\small\javatt,emphstyle=\small\javattb,commentstyle=\small\javatti,
%%basicstyle=\small,
%%
%showstringspaces=false,captionpos=b,columns=flexible}
%%\lstset{basicstyle=\small,language=Java,columns=flexible}

\lstset{basicstyle=\sffamily\small,keywordstyle=\sffamily\small,language=Java,captionpos=b,columns=flexible,showstringspaces=false}

\usepackage{array}
\usepackage{dcolumn}
\newcommand{\cc}[1]{\multicolumn{1}{c}{#1}}
\newcolumntype{d}[1]{D{.}{.}{#1}}

\usepackage{capt-of}
\usepackage[colorlinks=true,urlcolor=black,linkcolor=black,citecolor=black]{hyperref}

% ----------------------

\usepackage{makeidx}
\makeindex


\usepackage{import} % for subimport text and graphics from subdirectory
% does not work with latex2html!


%\newcommand{\codefoot}{\textsf}
%\newcommand{\code}[1]{{\javatext#1}}
%\newcommand{\codeb}[1]{{\javattb#1}}

\newcommand{\codefoot}{\sffamily}
\newcommand{\code}[1]{{\small\sffamily{#1}}}


%\newcommand{\cmd}[1]{{\texttt{#1}}}
%\newcommand{\dirent}[1]{{\texttt{#1}}}
%\newcommand{\menuitem}[1]{\textsf{\textbf{#1}}}
\newcommand{\cmd}[1]{{\code{#1}}\index{#1}}
\newcommand{\cmdb}[1]{{\textsf{\textbf{#1}}}\index{#1}}
\newcommand{\dirent}[1]{{\code{#1}}}
\newcommand{\menuitem}[1]{\textsf{\textsl{#1}}}

\newcommand{\idx}[1]{#1\index{#1}}

% for flow.tex - part of index helper
\newcommand{\eei}[1]{%
\index{extension!\texttt{#1}}\texttt{#1}}

% JVs et al
%\newcommand{\ea}{et al.\xspace}
\newcommand{\ea}{et al.\ }

% used in GC period calculation
\newtheorem{lemma}{Lemma}
\newtheorem{theorem}{Theorem}

%\begin{htmlonly}
%\renewcommand{\code}[1]{{\texttt{#1}}} % for html2LaTeX
%\newcommand{\toprule}{\hline}
%\newcommand{\midrule}{\hline}
%\newcommand{\bottomrule}{\hline}
%\end{htmlonly}

% net wirklich notwendig -- h�ngt von code generierung ab
%\begin{htmlonly}
%\renewcommand{\javatt}{\texttt}
%\renewcommand{\javattb}{\texttt\bfseries}
%\end{htmlonly}

%\code{\hyphenchar\font=-1}

\newcommand{\mycomment}[1]{}

\newcommand{\instr}[6]{
    \begin{table}
    \index{microcode!{\textsf{#1}}}
        \begin{tabular}{ll}
            \emph{\large\textbf{#1}} & \\
            \\ \\
            \textbf{Operation} & #2 \\ \\
            \textbf{Opcode} & \texttt{#3} \\ \\
            \textbf{Dataflow} & \parbox[t]{10.5cm}{\(#4\)}\\ \\
            \textbf{JVM equivalent} & \parbox[t]{10.5cm}{\code{#5}} \\ \\
            \textbf{Description} & \parbox[t]{10.5cm}{#6}\\
        \end{tabular}
    \end{table}
}




\begin{table}[htb]
    \centering
{\small
    \begin{tabular}{lrrrrr}
        \toprule

& JOP & leJOS & TINI & Komodo & JStamp \\
        \midrule
Frequency [MHz] & 100 & 16 & 40 & 33 & 73.728 \\
        \midrule
%geom. Mean Micro & 5026748 & 9347 & 19820 & 554313 & 739697 \\
iload iadd  & 49,344,000 & 19,140 & 50,724 & 4,111,569 & 1,934,642 \\
iinc          & 9,078,000 & 37,925 & 103,044 & 8,318,030 & 1,789,378 \\
ldc           & 10,010,000 & 11,941 & 35,463 & 825,446 & 1,101,445 \\
if\_icmplt taken & 16,644,000 & 9,941 & 31,629 & 1,372,264 & 1,747,626 \\
if\_icmplt not taken & 16,710,000 & 10,529 & 33,032 & 1,375,754 & 1,833,174 \\
getfield & 4,002,000 & 8,515 & 16,684 & 687,877 & 518,071 \\
getstatic & 5,874,000 & 9,547 & 8,962 & 412,723 & 723,155 \\
iaload & 3,328,000 & 14,787 & 25,924 & 1,180,501 & 992,969 \\
invoke        & 781,935 & 3,362 & 6,159 & 85,874 & 211,406 \\
invoke static & 989,222 & 4,129 & 6,815 & 48,510 & 271,933 \\
invoke interface & 684,896 & 3,141 & 5,885 & 20402 & 138,847 \\
Sieve & 4,286 & 7 & 15 & 627 & 564 \\
Kfl & 14,222 & 25 & 64 & 924 & 2,221 \\
UDP/IP & 6,050 & 13 & 29 & 520 & 1,004 \\
        \midrule
geom. Mean App & 9,276 & 18 & 43 & 693 & 1,493 \\
        \midrule
geom. Mean App/MHz & 79 & 1 & 1 & 21 & 20 \\
        \bottomrule
    \end{tabular}
}
    \caption{Raw data of all benchmarks in [iterations/s] I.}
    \label{tab:appendix:bench:all1}

\end{table}

\begin{table}[htb]
    \centering
{\footnotesize
    \begin{tabular}{lrrrrr}
        \toprule

     & SaJe & EJC & Sun jvm & gcj & Xint \\
        \midrule
Frequency [MHz] & 103 & 74 & 266 & 266 & 266 \\
        \midrule

% geom. Mean Micro & 4265320 & 10040052 & 84020831 & 50399644 & 6679510 \\
iload iadd  & 12,710,000 & 72,315,000 & 84,307,000 & 248,551,000 & 15,363,000 \\
iinc          & 9,320,000 & 36,002,000 & 296,941,000 & 88,069,000 & 122,228,000 \\
ldc           & 11,275,000 & 23,967,000 & 132,626,000 &  & 8,719,000 \\
if\_icmplt taken & 5,652,000 & 35,925,000 & 128,561,000 & 86,480,000 & 7,449,000 \\
if\_icmplt not taken & 7,281,000 & 71,697,000 & 246,723,000 & 89,240,000 & 7,206,000 \\
getfield & 4,433,000 & 7,212,000 & 90,687,000 & 122,016,000 & 6,853,000 \\
getstatic & 6,786,000 & 17,962,000 & 86,703,000 & 241,398,000 & 6,700,000 \\
iaload & 7,854,000 & 5,966,000 & 65,536,000 & 23,967,000 & 8,962,000 \\
invoke       & 894,689 & 1,703,000 & 10,022,000 & 20,092,000 & 1,458,381 \\
invoke static & 1,084,359 & 309,132 & 270,600,000 & 7,898,000 & 1,620,673 \\
invoke interface & 674,759 & 1,598,000 & 10,010,000 & 5,588,000 & 1,381,523 \\

Sieve & 3,972 & 9,475 & 52,681 & 39,432 & 6,601 \\
Kfl & 14,148 & 9,893 & 212,952 & 139,884 & 17,310 \\
UDP/IP & 6,415 & 2,822 & 91,851 & 38,460 & 8,747 \\
        \midrule
geom. Mean App & 9,527 & 5,284 & 139,857 & 73,348 & 12,305 \\
        \midrule
App/MHz & 92 & 71 & 526 & 276 & 46 \\
        \bottomrule
    \end{tabular}
}
    \caption{Raw data of all benchmarks in [iterations/s] II.}
    \label{tab:appendix:bench:all2}

\end{table}

\begin{table}[htb]
    \centering
{\small
    \begin{tabular}{lcccccc}
    \toprule
    & & & & \multicolumn{3}{c}{Memory access time} \\
    \cmidrule{5-7}
    Type & Size & MBIB & MTIB & SRAM & SDRAM & DDR \\
    \midrule

    Prefetch buffer & 8 B & 1.37 & 0.342                 & 1.02 & 2.05 & 1.71 \\
    Single method cache & 1 KB & 2.32 & 0.021            & 1.18 & 0.69 & 0.39 \\
    Two block cache & 2 KB & 1.21 & 0.013                & 0.62 & 0.37 & 0.21 \\
    Four block cache & 4 KB & 0.90 & 0.010               & 0.46 & 0.27 & 0.16 \\
    Direct-mapped 8 bytes & 1 KB & 0.28 & 0.035          & 0.18 & 0.25 & 0.19 \\
    Direct-mapped 16 bytes & 1 KB & 0.38 & 0.024         & 0.22 & 0.22 & 0.16 \\
    Direct-mapped 32 bytes & 1 KB & 0.58 & 0.018         & 0.31 & 0.24 & 0.15 \\
    Direct-mapped 8 bytes & 2 KB & 0.17 & 0.022          & 0.11 & 0.15 & 0.12 \\
    Direct-mapped 16 bytes & 2 KB & 0.25 & 0.015         & 0.14 & 0.14 & 0.10 \\
    Direct-mapped 32 bytes & 2 KB & 0.41 & 0.013         & 0.22 & 0.17 & 0.11 \\
    Direct-mapped 8 bytes & 4 KB & 0.00 & 0.001          & 0.00 & 0.00 & 0.00 \\
    Direct-mapped 16 bytes & 4 KB & 0.01 & 0.000         & 0.00 & 0.00 & 0.00 \\
    Direct-mapped 32 bytes & 4 KB & 0.01 & 0.000         & 0.00 & 0.00 & 0.00 \\
    Variable block cache 8 blocks & 1 KB & 0.80 & 0.009  & 0.41 & 0.24 & 0.14 \\
    Variable block cache 16 blocks & 1 KB & 0.71 & 0.008 & 0.36 & 0.22 & 0.12 \\
    Variable block cache 32 blocks & 1 KB & 0.70 & 0.008 & 0.36 & 0.21 & 0.12 \\
    Variable block cache 64 blocks & 1 KB & 0.70 & 0.008 & 0.36 & 0.21 & 0.12 \\
    Variable block cache 8 blocks & 2 KB & 0.73 & 0.008  & 0.37 & 0.22 & 0.13 \\
    Variable block cache 16 blocks & 2 KB & 0.37 & 0.004 & 0.19 & 0.11 & 0.06 \\
    Variable block cache 32 blocks & 2 KB & 0.24 & 0.003 & 0.12 & 0.08 & 0.04 \\
    Variable block cache 64 blocks & 2 KB & 0.12 & 0.001 & 0.06 & 0.04 & 0.02 \\
    Variable block cache 8 blocks & 4 KB & 0.73 & 0.008  & 0.37 & 0.22 & 0.13 \\
    Variable block cache 16 blocks & 4 KB & 0.25 & 0.003 & 0.13 & 0.08 & 0.05 \\
    Variable block cache 32 blocks & 4 KB & 0.01 & 0.000 & 0.00 & 0.00 & 0.00 \\
    Variable block cache 64 blocks & 4 KB & 0.00 & 0.000 & 0.00 & 0.00 & 0.00 \\

    \bottomrule

    \end{tabular}
}
    \caption[Cache performance compared]{Cache performance in MBIB and MTIB of all variations of
    the method cache and a conventional direct-mapped cache. Average
    memory access time per instruction byte for three different main
    memory technologies. Memory access times are in clock cycles.
    }
    \label{tab_cache_all}
\end{table}


%\end{document}


\chapter{Cyclone FPGA Board} \label{appx:cycore}
\begin{figure}[htb]
    \centering
    \includegraphics[height=64mm]{appendix/cycore_top}
    \includegraphics[height=67mm]{appendix/cycore_bottom}
    \caption{Top and bottom side of the Cyclone FPGA board}
\end{figure}
\begin{figure}
    \centering
    \includegraphics[scale=0.87]{appendix/cycore_p1}
    \caption{Schematic of the Cyclone FPGA board, page 1}
\end{figure}
\begin{figure}
    \centering
    \includegraphics[scale=0.87]{appendix/cycore_p2}
    \caption{Schematic of the Cyclone FPGA board, page 2}
\end{figure}
\begin{figure}
    \centering
    \includegraphics[scale=0.87]{appendix/cycore_p3}
    \caption{Schematic of the Cyclone FPGA board, page 3}
\end{figure}


\input{classes/classes}

% uncomment for IKT book ;-)
%\chapter{TODO Lists}
%\section{TODO Handbook}

\begin{itemize}
    \item OOHW copy
    \item HWO copy
    \item Update Results section (and decide on a different name)
    \item rewrite user scheduler to a scheduler description
    \item restructure with background info and related work in each
    chapter
    \item Document design alternatives you didn't take and why you
    didn't take them
    \item jHISC to related work
    \item derived work
\end{itemize}

\subsection{Add Section}

\begin{itemize}
    \item VHDL Hello world with cycore
    \item RTS Introduction
    \item WCET analysis description + usage (remove from Results
    section)
    \item SCJ definition (+ implementation)
    \item Mission modes + implementation
    \item Examples section
    \begin{itemize}
        \item RtThread examples
        \item ejip examples
        \item low-level access
    \end{itemize}
    \item About boards (Baseio, dspio,...)
\end{itemize}

\subsection{Cleanup Following Sections}

\begin{itemize}
    \item Introduction
    \item Conclusions
    \item SC Java
\end{itemize}

\subsection{Addition or Cleanup Done}

\begin{itemize}
    \item Real-Time GC
    \item SimpCon description
    \item Javadoc of some classes
\end{itemize}

\subsection{IKT}

\begin{itemize}
    \item other papers?
\end{itemize}
\subsection{Notes}

Based in thesis:
\begin{description}
    \item[-] too much scientific stuff
    \item[+-] Change related work to comparison
    \item[+] Getting started
    \item[+] HOTOs
    \item[+] Internal docu
\end{description}

Copy stuff from JSA paper. However, only paragraphs to avoid
\emph{copyright issues}.

\subsection{Publishing the Book}

Target is real book with ISBN (= publication). The copyright should
still be open-source.

\begin{itemize}
    \item \url{http://www.virtualbookworm.com/} \$ 360 -- \$ 440, incl. Amazon only, 4 computer
    books, form mail sent, e-mail sent
    \item \url{http://www.authorhouse.com/} \$ 700.- incl. Amazon,
    no control on layout, form mail sent
    \item \url{http://www.wingspanpress.com/} \$ 500 incl. Amazon,
    no computer books, e-mail possible
    \item \url{http://www.pagefreepublishing.com/} form mail sent,
    e-mail sent
    \item \url{http://www.aventinepress.com/} e-mail sent
    \item \url{http://www.iuniverse.com/} \$ 399,- (\$ 599,- incl.
    Amazon), Author discount, not many computer related books (old
    topics) -- e-mail only possible with a Friends name and address!
    \item \url{http://trafford.com/}, EUR 750,-(\$ 1600,- incl. Amazon), EUR 11,- per
    book, some computer related books
    \item \url{http://www.booksurgepages.com/amzn/ondemand/}
    \item \url{http://www.booksurge.com/} e-mail sent
\end{itemize}

Wishlist:
\begin{itemize}
    \item ISBN
    \item A publisher who has serious technical books - no pornography!
    \item Cheap upfront
    \item Order possible from Amazon
    \item Cheap order for myself
    \item Right to sell it myself
\end{itemize}

Send an e-mail and ask following questions:
\begin{itemize}
    \item Can I provide my own layout (submitting a PDF)?
    \item Can I still provide a free PDF version of my book on my
    web site?
    \item Can I update the content of the book (new edition)?
    \item Can I sell the book myself?
    \item Add a link to the manuscript.
\end{itemize}

No go's:
\begin{itemize}
    \item \url{http://www.lulu.com/} cheap \$ 99, but not clearely
    listed, freedom (e.g.\ GNU licence), no Amazon
    \item \url{http://www2.xlibris.com/} fixed layout
    \item http://superiorbooks.com/
    \item http://www.ubooks.de/
\end{itemize}
\subsection{A Possible Structure of the Book}

\begin{itemize}
    \item Introduction
    \begin{itemize}
        \item Intro what JOP is
        \item A Quick Start (Hello World)
        \item A Short History
        \item About this handbook (based on, organization)
    \end{itemize}
    \item Architecture (HW)
    \item Architecture (SW)
    \item Build.pdf
    \item Source organization
    \item Background information
    \begin{itemize}
        \item JVM
        \item GC
        \item Related Work
        \item Real-time systems
    \end{itemize}
    \item Real-time threads
    \item JOPizer
    \item Library: util, ejip,...
    \item JDK + support
    \item Board descriptions
    \item Appendix
    \begin{itemize}
        \item Microcode
        \item Instruction timing
        \item Javadoc for classes
    \end{itemize}
    \item Further reading (related work) at the end of each chapter
    \item Browser other books for structural ideas
\end{itemize}

\section{TODO JOP}

\begin{itemize}
    \item Interrupt module for HW Objects paper
    \item Javadoc for the handbook
    \item move IO devices after memory module to enable array access
    for HW objects
\end{itemize}



\end{document}
