% just for now for the printout with old title page
\setcounter{page}{1}

\begin{flushleft}
\pagestyle{empty}
\ \\
\vspace{1cm}
{\usekomafont{title}\mdseries\huge JOP Reference Handbook\\
%\bigskip \Large Building Embedded Systems with a Java Processor
}
\cleardoublepage
\end{flushleft}


\begin{flushleft}
\pagestyle{empty}
\ \\
\vspace{1cm}
{\usekomafont{title}\Huge JOP Reference Handbook\\
\mdseries
{\Large Building Embedded Systems with a Java Processor}\\
\bigskip
\bigskip
%{\large\itshape Beta Edition}\\
\bigskip
{\usekomafont{title}\huge Martin Schoeberl}
\medskip\\
%{\large\itshape martin@jopdesign.com}

}


\vspace{10cm} \emph{Version: \today}
\newpage
\end{flushleft}




\thispagestyle{empty}
\begin{flushleft}
{\small

Copyright \copyright \ 2008 Martin Schoeberl
\medskip

Martin Schoeberl\\
Strausseng. 2-10/2/55\\
A-1050 Vienna, Austria\\
\medskip

Email: \url{martin@jopdesign.com}\\
Visit the accompanying web site on \url{http://www.jopdesign.com/}
and\\
the JOP Wiki at \url{http://www.jopwiki.com/}
\medskip

%Published 2007 by Virtualbookworm.com Publishing Inc.,\\
%P.O. Box 9949, College Station, TX 77842, US.
Published 2008 by CreateSpace,\\
\url{http://www.createspace.com/}



\medskip

%Published 2007, First edition 2008
%\medskip

All rights reserved. No part of this publication may be reproduced,
stored in a retrieval system, or transmitted in any form or by any
means, electronic, mechanical, recording or otherwise, without the
prior written permission of Martin Schoeberl.
\medskip

%``JOP Reference Handbook" by Martin Schoeberl. .

\textbf{Library of Congress Cataloging-in-Publication Data}
\medskip

Schoeberl, Martin
\begin{quote}
    JOP Reference Handbook: Building Embedded Systems\\
    with a Java Processor / Martin Schoeberl\\
    Includes bibliographical references and index.\\
    ISBN 978-1438239699
\end{quote}

\bigskip


Manufactured in the United States of America.

Typeset in 11pt Times by Martin Schoeberl}
\end{flushleft}


\addchap{Foreword}

This book is about JOP, the Java Optimized Processor. JOP began as a
research project for a PhD thesis. JOP has been used in several
industrial applications and, due to the fact that it is an
open-source project, has a growing user base. This book is written
for all of you who build this lively community.

For a long time the thesis, some research papers, and the web site
have been the only available documentation for JOP. A thesis is quite
different from a reference manual. Its focus is on research results
and implementation details are usually omitted. This book complements
the thesis and provides insight into the implementation of JOP and
the accompanying Java virtual machine (JVM). It also gives you an
idea how to build an embedded real-time system based on JOP.

\addchap{Acknowledgements}

Many users of JOP contributed to the design of JOP and to the tool
chain. I also want to thank for the discussions with the students at
the Vienna University of Technology during three years of the course
``The JVM in Hardware" and one semester in Copenhagen at an embedded
systems course in Java. Furthermore, the questions and discussions in
the Java processor mailing list provided valuable input for the
documentation now available in form of this book.

Ed Anuff wrote \code{testmon.asm} to perform a memory interface test
and \code{BlockGen.java} to convert Altera \code{.mif} files to
Xilinx memory blocks. \code{BlockGen.java} was the key tool to port
JOP to Xilinx FPGAs in general and the Spartan-3 specifically.
Flavius Gruian wrote the initial version of \cmd{JOPizer} to generate
the \code{.jop} file from the application classes. \cmd{JOPizer} is
based on the open source BCEL and is a substitute to the formerly
used \code{JavaCodeCompact} from Sun.
