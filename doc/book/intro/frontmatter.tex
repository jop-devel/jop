% just for now for the printout with old title page
\setcounter{page}{1}

\begin{flushleft}
\pagestyle{empty}
\ \\
\vspace{1cm}
{\usekomafont{title}\mdseries\huge JOP Reference Handbook\\
%\bigskip \Large Building Embedded Systems with a Java Processor
}
\cleardoublepage
\end{flushleft}


\begin{flushleft}
\pagestyle{empty}
\ \\
\vspace{1cm}
{\usekomafont{title}\Huge JOP Reference Handbook\\
\mdseries
{\Large Building Embedded Systems with a Java Processor}\\
\bigskip
\bigskip
%{\large\itshape Beta Edition}\\
\bigskip
{\usekomafont{title}\huge Martin Schoeberl}
\medskip\\
%{\large\itshape martin@jopdesign.com}

}


%\vspace{10cm} \emph{Version: \today}
\newpage
\end{flushleft}




\thispagestyle{empty}
\begin{flushleft}
{\small

Copyright \copyright \ 2009 Martin Schoeberl
\medskip

Martin Schoeberl\\
Strausseng. 2-10/2/55\\
A-1050 Vienna, Austria\\
\medskip

Email: \url{martin@jopdesign.com}\\
Visit the accompanying web site on \url{http://www.jopdesign.com/}
and\\
the JOP Wiki at \url{http://www.jopwiki.com/}
\medskip

%Published 2007 by Virtualbookworm.com Publishing Inc.,\\
%P.O. Box 9949, College Station, TX 77842, US.
Published 2009 by CreateSpace,\\
\url{http://www.createspace.com/}



\medskip

%Published 2007, First edition 2008
%\medskip

All rights reserved. No part of this publication may be reproduced,
stored in a retrieval system, or transmitted in any form or by any
means, electronic, mechanical, recording or otherwise, without the
prior written permission of Martin Schoeberl.
\medskip

%``JOP Reference Handbook" by Martin Schoeberl. .

\textbf{Library of Congress Cataloging-in-Publication Data}
\medskip

Schoeberl, Martin
\begin{quote}
    JOP Reference Handbook: Building Embedded Systems\\
    with a Java Processor / Martin Schoeberl\\
    Includes bibliographical references and index.\\
    ISBN 978-1438239699
\end{quote}

\bigskip


Manufactured in the United States of America.

Typeset in 11pt Times by Martin Schoeberl}
\end{flushleft}


\addchap{Foreword}

%\code{

This book is about JOP, the Java Optimized Processor. JOP is an
implementation of the Java virtual machine (JVM) in hardware. The
main implementation platform is a field-programmable gate array
(FPGA). JOP began as a research project for a PhD thesis. In the mean
time, JOP has been used in several industrial applications and as a
research platform. JOP is a time-predictable processor for hard
real-time systems implemented in Java.

JOP is open-source under the GNU GPL and has a growing user base.
This book is written for all of you who build this lively community.
For a long time the PhD thesis, some research papers, and the web
site have been the main documentation for JOP. A PhD thesis focus is
on research results and implementation details are usually omitted.
This book complements the thesis and provides insight into the
implementation of JOP and the accompanying JVM. Furthermore, it gives
you an idea how to build an embedded real-time system based on JOP.

%\textbf{Martin Schoeberl} is the main developer of the Java processor
%JOP. He has published more than 50 scientific papers on
%time-predictable computer architectures and real-time Java. Martin
%Schoeberl is member of the Expert Group for the Safety-Critical Java
%Specification. Before joining the Institute of Computer Engineering
%at the Vienna University of Technology as Assistant
%Professor in 2005, he has been self employed with projects in the automation
%industry. Martin Schoeberl will start as Associate Professor in
%System-on-Chip in the Department of Informatics and Mathematical
%Modelling at the Technical University of Denmark in early 2010.
%
%}

\addchap{Acknowledgements}

Many users of JOP contributed to the design of JOP and to the tool
chain. I also want to thank the students at the Vienna University of
Technology during the four years of the course ``The JVM in Hardware"
and the students from CBS, Copenhagen at an embedded systems course
in Java for the interesting questions and discussions. Furthermore,
the questions and discussions in the Java processor mailing list
provided valuable input for the documentation now available in form
of this book. The following list of contributors to JOP is roughly in
chronological order.

Ed Anuff wrote \code{testmon.asm} to perform a memory interface test
and \code{BlockGen.java} to convert Altera \code{.mif} files to
Xilinx memory blocks.
% \code{BlockGen.java} was the key tool to port JOP to Xilinx FPGAs in
% general and the Spartan-3 specifically.
Flavius Gruian wrote the initial version of \cmd{JOPizer} to generate
the \code{.jop} file from the application classes. \cmd{JOPizer} is
based on the open source library BCEL and is a substitute to the
formerly used \code{JavaCodeCompact} from Sun. Peter Schrammel and
Christof Pitter have implemented the first version of long bytecodes.
Rasmus Pedersen based a class on very small information systems on
JOP and invited my to co-teach this class in Copenhagen. During this
time the first version of the WCET analysis tool was developed by
Rasmus. Rasmus has also implemented an Eclipse plugin for the JOP
design flow. Alexander Dejaco and Peter Hilber have developed the I/O
interface board for the LEGO Mindstorms. Christof Pitter designed and
implemented the chip-multiprocessor (CMP) version of JOP during his
PhD thesis. Wolfgang Puffitsch first contribution to JOP was the
finalization of the floating point library SoftFloat. Wolfgang, now
an active developer of JOP, contributed several enhancements (e.g.,
exceptions, HW field access, data cache,...) and works towards
real-time garbage collection for the CMP version of JOP. Alberto
Andriotti contributed several JVM test cases. Stefan Hepp has
implemented an optimizer at bytecode level during his Bachelor thesis
work. Benedikt Huber has redesigned the WCET analysis tool for JOP
during his Master's thesis. Trevor Harmon, who implemented the WCET
tool Volta for JOP during his PhD thesis, helped me with proofreading
of the handbook.

Furthermore, I would like to thank Walter Wilhelm from EEG for taking
the risk to accept a JOP based hardware for the
\emph{Kippfahrleitung} project at a very early development stage of
JOP. The development of JOP has received funding from the Wiener
Innovationsf\"oderprogram (Call IKT 2004) and from the EU project
JEOPARD.
