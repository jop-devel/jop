\index{SimpCon}

\section{Introduction}

The intention of the following SoC interconnect standard is to be
simple and efficient with respect to implementation resources and
transaction latency.

SimpCon is a fully synchronous standard for on-chip
interconnections. It is a point-to-point connection between a master
and a slave. The master starts either a read or write transaction.
Master commands are single cycle to free the master to continue on
internal operations during an outstanding transaction. The slave has
to register the address when needed for more than one cycle. The
slave also registers the data on a read and provides it to the
master for more than a single cycle. This property allows the master
to delay the actual read if it is busy with internal operations.

The slave signals the end of the transaction through a novel
\emph{ready counter} to provide an early notification. This early
notification simplifies the integration of peripherals into pipelined
masters. Slaves can also provide several levels of pipelining. This
feature is announced by two static output ports (one for read and one
write pipeline levels).

Off-chip connections (e.g.\ main memory) are device specific and
need a slave to perform the translation. Peripheral interrupts are
not covered by this specification.

\subsection{Features}

SimpCon provides following main features:

\begin{itemize}
    \item Master/slave point-to-point connection
    \item Synchronous operation
    \item Read and write transactions
    \item Early pipeline release for the master
    \item Pipelined transactions
    \item Open-source specification
    \item Low implementation overheads
\end{itemize}

\subsection{Basic Read Transaction}

\index{SimpCon!read transaction}

Figure~\ref{fig:sc:basic:rd} shows a basic read transaction for a
slave with one cycle latency. In the first cycle, the address phase,
the \sign{rd} signals the slave to start the read transaction. The
address is registered by the slave. During the following cycle, the
read phase,\footnote{It has to be noted that the read phase can be
longer for devices with a high latency. For simple on-chip I/O
devices the read phase can be omitted completely (0 cycles). In that
case \sign{rdy\_cnt} will be zero in the cycle following the address
phase.} the slave performs the read and registers the data. Due to
the register in the slave, the data is available in the third cycle,
the result phase. To simplify the master, \sign{rd\_data} stays valid
until the next read request response. It is therefore possible for a
master to issue a pre-fetch command early. When the pre-fetched data
arrives too early it is still valid when the master actually wants to
read it. Keeping the read data stable in the slave is
\emph{mandatory}.

\begin{figure}
    \centering
    \includegraphics[scale=\scgrsc]{\scgrp/sc_basic_rd}
    \caption{Basic read transaction}
    \label{fig:sc:basic:rd}
\end{figure}

\subsection{Basic Write Transaction}

\index{SimpCon!write transaction}

A write transaction consists of a single cycle address/command phase
started by assertion of \sign{wr} where the address and the write
data are valid. \sign{address} and \sign{wr\_data} are usually
registered by the slave. The end of the write cycle is signalled to
the master by the slave with \sign{rdy\_cnt}. See
Section~\ref{sec:ack} and an example in Figure~\ref{fig:sc:wr:ws}.

\section{SimpCon Signals}
\index{SimpCon!signals}

This sections defines the signals used by the SimpCon connection.
Some of the signals are optional and may not be present on a
peripheral device.

All signals are a single direction point-to-point connection between
a master and a slave. The signal details are described by the device
that drives the signal. Table~\ref{tab:sc:signals} lists the signals
that define the SimpCon interface. The column Direction indicates
whether the signal is driven by the master or the slave.

\begin{table}
    \centering

    \begin{tabular}{lrlll}
        \toprule
        Signal & Width & Direction & Required & Description \\
        \midrule
        \sign{address} & 1--32 & Master & No & Address lines from the
        master\\
        & & & & to the slave port\\
        \sign{wr\_data} & 32 & Master & No & Data lines from the
        master\\
        & & & & to the slave port\\
        \sign{rd} & 1 & Master & No & Start of a read transaction \\
        \sign{wr} & 1 & Master & No & Start of a write transaction \\
        \sign{rd\_data} & 32 & Slave & No & Data lines from the
        slave\\
        & & & & to the master port\\
        \sign{rdy\_cnt} & 2 & Slave & Yes & Transaction end signalling \\
        \sign{rd\_pipeline\_level} & 2 & Slave & No & Maximum pipeline
        level\\
        & & & & for read transactions \\
        \sign{wr\_pipeline\_level} & 2 & Slave & No & Maximum pipeline
        level\\
        & & & & for write transactions \\
        \sign{sel\_byte} & 2 & Master & No & Reserved for future use\\
        \sign{uncached}  & 1 & Master & No & Non cached access\\
        \sign{cache\_flash} & 1 & Master & No & Flush/invalidate a cache\\

        \bottomrule

    \end{tabular}
    \caption{SimpCon port signals}
    \label{tab:sc:signals}

\end{table}



\subsection{Master Signal Details}

This section describes the signals that are driven by the master to
initiate a transaction.

\subsubsection{address}

Master addresses represent word addresses as offsets in the slave's
address range. \sign{address} is valid a single cycle either with
\sign{rd} for a read transaction or with \sign{wr} and
\sign{wr\_data} for a write transaction. The number of bits for
\sign{address} depends on the slave's address range. For a single
port slave, \sign{address} can be omitted.

\subsubsection{wr\_data}

The \sign{wr\_data} signals carry the data for a write transaction.
It is valid for a single cycle together with \sign{address} and
\sign{wr}. The signal is typically 32 bits wide. Slaves can ignore
upper bits when the slave port is less than 32 bits.

\subsubsection{rd}

The \sign{rd} signal is asserted for a single clock cycle to start a
read transaction. \sign{address} has to be valid in the same cycle.

\subsubsection{wr}

The \sign{wr} signal is asserted for a single clock cycle to start a
write transaction. \sign{address} and \sign{wr\_data} have to be
valid in the same cycle.

\subsubsection{sel\_byte}

The \sign{sel\_byte} signal is reserved for future versions of the
SimpCon specification to add individual byte enables.

\subsubsection{uncached}

The \sign{uncached} signal is asserted for a single clock cycle
during a read or write transaction to signal that a cache, connected
in the SimpCon pipeline, shall not cache the read or write access.

\subsubsection{cache\_flash}

The \sign{cache\_flash} signal is asserted for a single clock cycle
invalidates a cache connected in the SimpCon pipeline.

\subsection{Slave Signal Details}

This section describes the signals that are driven by the slave as a
response to transactions initiated by the master.

\subsubsection{rd\_data}

The \sign{rd\_data} signals carry the result of a read transaction.
The data is valid when \sign{rdy\_cnt} reaches 0 and \emph{stays
valid} until a new read result is available. The signal is typically
32 bits wide. Slaves that provide less than 32 bits should pad the
upper bits with 0.

\subsubsection{rdy\_cnt}

The \sign{rdy\_cnt} signal provides the number of cycles until the
pending transaction will finish. A 0 means that either read data is
available or a write transaction has been finished. Values of 1 and 2
mean the transaction will finish in at least 1 or 2 cycles. The
maximum value is 3 and means the transaction will finish in 3 or
\emph{more} cycles. Note that not all values have to be used in a
transaction. Each monotonic sequence of \sign{rdy\_cnt} values is
legal.

\subsubsection{rd\_pipeline\_level}

The static \sign{rd\_pipeline\_level} provides the master with the
read pipeline level of the slave. The signal has to be constant to
enable the synthesizer to optimize the pipeline level dependent
state machine in the master.


\subsubsection{wr\_pipeline\_level}

The static \sign{wr\_pipeline\_level} provides the master with the
write pipeline level of the slave. The signal has to be constant to
enable the synthesizer to optimize the pipeline level dependent
state machine in the master.

\section{Slave Acknowledge}
\label{sec:ack} \index{SimpCon!ready count}

Flow control between the slave and the master is usually done by a
single signal in the form of \emph{wait} or \emph{acknowledge}. The
\sign{ack} signal, e.g.\ in the Wishbone specification, is set when
the data is available or the write operation has finished. However,
for a pipelined master it can be of interest to know it
\emph{earlier} when a transaction will finish.

For many slaves, e.g.\ an SRAM interface with fixed wait states, this
information is available inside the slave. In the SimpCon interface,
this information is communicated to the master through the two bit
ready counter (\sign{rdy\_cnt}). \sign{rdy\_cnt} signals the number
of cycles until the read data will be available or the write
transaction will be finished. A value of 0 is equivalent to an
\emph{ack} signal and 1, 2, and 3 are equivalent to a wait request
with the distinction that the master knows how long the wait request
will last.

To avoid too many signals at the interconnect, \sign{rdy\_cnt} has a
width of two bits. Therefore, the maximum value of 3 has the special
meaning that the transaction will finish in 3 or \emph{more} cycles.
As a result the master can only use the values 0, 1, and 2 to release
actions in its pipeline. If necessary, an extension for a longer
pipeline is straightforward with a larger
\sign{rdy\_cnt}.\footnote{The maximum value of the ready counter is
relevant for the early restart of a waiting master. A longer latency
from the slave e.g., for DDR SDRAM, will map to the maximum value of
the counter for the first cycles.}

Idle slaves will keep the former value of 0 for \sign{rdy\_cnt}.
Slaves that do not know in advance how many wait states are needed
for the transaction can produce sequences that omit any of the
numbers 3, 2, and 1. A simple slave can hold \sign{rdy\_cnt} on 3
until the data is available and set it then directly to 0. The master
has to handle those situations. Practically, this reduces the
possibilities of pipelining and therefore the performance of the
interconnect. The master will read the data later, which is not an
issue as the data stays valid.

Figure~\ref{fig:sc:rd:ws} shows an example of a slave that needs
three cycles for the read to be processed. In cycle 1, the read
command and the address are set by the master. The slave registers
the address and sets \sign{rdy\_cnt} to 3 in cycle 2. The read takes
three cycles (2--4) during which \sign{rdy\_cnt} gets decremented. In
cycle 4 the data is available inside the slave and gets registered.
It is available in cycle 5 for the master and \sign{rdy\_cnt} is
finally 0. Both, the \sign{rd\_data} and \sign{rdy\_cnt} will keep
their value until a new transaction is requested.

\begin{figure}
    \centering
    \includegraphics[scale=\scgrsc]{\scgrp/sc_rd_ws}
    \caption{Read transaction with wait states}
    \label{fig:sc:rd:ws}
\end{figure}


Figure~\ref{fig:sc:wr:ws} shows an example of a slave that needs
three cycles for the write to be processed. The address, the data to
be written, and the write command are valid during cycle 1. The slave
registers the address and the write data during cycle 1 and performs
the write operation during cycles 2--4. The \sign{rdy\_cnt} is
decremented and a non-pipelined slave can accept a new command after
cycle 4.

\begin{figure}
    \centering
    \includegraphics[scale=\scgrsc]{\scgrp/sc_wr_ws}
    \caption{Write transaction with wait states}
    \label{fig:sc:wr:ws}
\end{figure}



\section{Pipelining}
\index{SimpCon!pipelining}

Figure~\ref{fig:sc:pipe:level} shows a read transaction for a slave
with four clock cycles latency. Without any pipelining, the next read
transaction will start in cycle 7 after the data from the former read
transaction is read by the master. The three bottom lines show when
new read transactions (only the \sign{rd} signal is shown, address
lines are omitted from the figure) can be started for different
pipeline levels. With pipeline level 1, a new transaction can start
in the same cycle when the former read data is available (in this
example in cycle 6). At pipeline level 2, a new transaction (either
read or write) can start when \sign{rdy\_cnt} is 1, for pipeline
level 3 the next transaction can start at a \sign{rdy\_cnt} of 2.

\begin{figure}
    \centering
    \includegraphics[scale=\scgrsc]{\scgrp/sc_pipe_level}
    \caption{Different pipeline levels for a read transaction}
    \label{fig:sc:pipe:level}
\end{figure}

The implementation of level 1 in the slave is trivial (just two more
transitions in the state machine). It is recommended to provide  at
least level 1 for read transactions. Level 2 is a little bit more
complex but usually no additional address or data registers are
necessary.

To implement level 3 pipelining in the slave, at least one additional
address register is needed. However, to use level 3 the master has to
issue the request in the same cycle as \sign{rdy\_cnt} goes to 2.
That means that this transition is combinatorial. We see in
Figure~\ref{fig:sc:pipe:level} that \sign{rdy\_cnt} value of 3 means
three or more cycles until the data is available and can therefore
not be used to trigger a new transaction. Extension to an even deeper
pipeline needs a wider \sign{rdy\_cnt}.


\section{Interconnect}
\index{SimpCon!interconnection}

Although the definition of SimpCon is from a single master/slave
point-to-point viewpoint, all variations of multiple slave and
multiple master devices are possible.

\subsubsection{Slave Multiplexing}

To add several slaves to a single master, \sign{rd\_data} and
\sign{rdy\_cnt} have to be multiplexed. Due to the fact that all
\sign{rd\_data} signals are already registered by the slaves, a
single pipeline stage will be enough for a large multiplexer. The
selection of the multiplexer is also known at the transaction start
but at least needed one cycle later. Therefore it can be registered
to further speed up the multiplexer.

%TODO: add a schematic for the master \sign{rd\_data} multiplexer.

\subsubsection{Master Multiplexing}
\index{SimpCon!CMP}

SimpCon defines no signals for the communication between a master and
an arbiter. However, it is possible to build a multi-master system
with SimpCon. The SimpCon interface can be used as an interconnect
between the masters and the arbiter and the arbiter and the slaves.
In this case the arbiter acts as a slave for the master and as a
master for the peripheral devices. An example of an arbiter for
SimpCon, where JOP and a VGA controller are two masters for a shared
main memory, can be found in \cite{jop:dma}. The same arbiter is also
used to build a chip-multiprocessor version of JOP
\cite{jop:cmp:eval}.

The missing arbitration protocol in SimpCon results in the need to
queue $n-1$ requests in an arbiter for $n$ masters. However, this
additional hardware results in a zero cycle bus grant. The master,
which gets the bus granted, starts the slave transaction in the same
cycle as the original read/write request.

%TODO: add a timing diagram to explain this concept.

To add several slaves to a single master the \sign{rd\_data} and
\sign{rdy\_cnt} have to be multiplexed. Due to the fact that all
\sign{rd\_data} signals are registered by the slaves a single
pipeline stage will be enough for a large multiplexer. The selection
of the multiplexer is also known at the transaction start but needed
at most in the next cycle. Therefore it can be registered to further
speed up the multiplexer.
%\ \\
%\ \\
%TODO: add a schematic for the master \sign{rd\_data} multiplexer.


\section{Examples}

This section provides some examples for the application of the
SimpCon definition.

\begin{figure}
    \centering
    \includegraphics{\scgrp/ioport}
    \caption{A simple input/output port with a SimpCon interface}
    \label{fig:sc:ioport}
\end{figure}

\subsection{I/O Port}
\index{SimpCon!I/O port example}

Figure~\ref{fig:sc:ioport} shows a simple I/O port with a minimal
SimpCon interface. As address decoding is omitted for this simple
device signal \sign{address} is not needed. Furthermore we can tie
\sign{rdy\_cnt} to 0. We only need the \sign{rd} or \sign{wr} signal
to enable the port. Listing~\ref{lst:sc:ioport} shows the VHDL code
for this I/O port.

\begin{lstlisting}[float,caption={VHDL source for the simple input/output port},
label=lst:sc:ioport]

entity sc_test_slave is generic (addr_bits : integer);

port (
    clk      : in std_logic;
    reset    : in std_logic;
-- SimpCon interface
    wr_data        : in std_logic_vector(31 downto 0);
    rd, wr         : in std_logic;
    rd_data        : out std_logic_vector(31 downto 0);
    rdy_cnt        : out unsigned(1 downto 0);
-- input/output ports
    in_data        : in std_logic_vector(31 downto 0)
    out_data       : out std_logic_vector(31 downto 0)

); end sc_test_slave;

architecture rtl of sc_test_slave is

begin

    rdy_cnt <= "00";    -- no wait states

process(clk, reset) begin

    if (reset='1') then
        rd_data <= (others => '0');
        out_data <= (others => '0');
    elsif rising_edge(clk) then
        if rd='1' then
            rd_data <= in_data;
        end if;
        if wr='1' then
            out_data <= wr_data;
        end if;
    end if;

end process;

end rtl;
\end{lstlisting}

\subsection{SRAM interface}
\index{SimpCon!SRAM interface}

The following example assumes a master (processor) clocked at 100~MHz
and an static RAM (SRAM) with 15~ns access time. Therefore the
minimum access time for the SRAM is two cycles. The slack time of
5~ns forces us to use output registers for the SRAM address and write
data and input registers for the read data in the I/O cells of the
FPGA. These registers fit nicely with the intention of SimpCon to use
registers inside the slave.

Figure~\ref{fig:sc:sram} shows the memory interface for a
non-pipelined read access followed by a write access. Four signals
are driven by the master and two signals by the slave. The lower half
of the figure shows the signals at the FPGA pins where the SRAM is
connected.


\begin{figure}
    \centering
    \includegraphics[scale=\scgrsc]{\scgrp/sc_sram}
    \caption{Static RAM interface without pipelining}
    \label{fig:sc:sram}
\end{figure}

In cycle~1 the read transaction is started by the master and the
slave registers the address. The slave also sets the registered
control signals \sign{ncs} and \sign{noe} during cycle~1. Due to the
placement of the registers in the I/O cells, the address and control
signals are valid at the FPGA pins very early in cycle~2. At the end
of cycle~3 (15~ns after \sign{address}, \sign{ncs} and \sign{noe} are
stable) the data from the SRAM is available and can be sampled with
the rising edge for cycle~4. The setup time for the read register is
short, as the register can be placed in the I/O cell. The master
reads the data in cycle~4 and starts a write transaction in cycle~5.
Address and data are again registered by the slave and are available
for the SRAM at the beginning of cycle~6. To perform a write in two
cycles the \sign{nwr} signal is registered by a negative triggered
flip-flop.

In Figure~\ref{fig:sc:sram:prd} we see a pipelined read from the SRAM
with pipeline level 2. With this pipeline level and the two cycles
read access time of the SRAM we achieve the maximum possible
bandwidth.

\begin{figure}
    \centering
    \includegraphics[scale=\scgrsc]{\scgrp/sc_sram_prd}
    \caption{Pipelined read from a static RAM}
    \label{fig:sc:sram:prd}
\end{figure}

We can see the start of the second read transaction in cycle 3 during
the read of the first data from the SRAM. The new address is
registered in the same cycle and available for the SRAM in the
following cycle 4. Although we have a pipeline level of 2 we need no
additional address or data register. The read data is available for
two cycles (\sign{rdy\_cnt} 2 or 1 for the next read) and the master
has the freedom to select one of the two cycles to read the data.

It has to be noted that pipelining with one read per cycle is
possible with SimpCon. We just showed a 2 cycle slave in this
example. For a SDRAM memory interface the ready counter will stay
either at 2 or 1 during the single cycle reads (depending on the
slave pipeline level). It will go down to 0 only for the last data
word to read.


\section{Available VHDL Files}

Besides the SimpCon documentation, some example VHDL files for slave
devices and bridges are available from
\url{http://opencores.org/?do=project&who=simpcon}. All components
are also part of the standard JOP distribution.


\subsection{Components}

\begin{itemize}
    \item \code{sc\_pack.vhd} defines VHDL records and some
    constants.
    \item \code{sc\_test\_slave.vhd} is a very simple SimpCon
        device. A counter to be read out and a register that can
        be written and read. There is no connection to the outer
        world. This example can be used as basis for a new
        SimpCon device.
    \item \code{sc\_sram16.vhd} is a memory controller for 16-bit
    SRAM.
    \item \code{sc\_sram32.vhd} is a memory controller for 32-bit
    SRAM.
    \item \code{sc\_sram32\_flash.vhd} is a memory controller for 32-bit
    SRAM, a NOR Flash, and a NAND Flash as used in the Cycore FPGA board for JOP.
    \item \code{sc\_uart.vhd} is a simple UART with configurable
    baud rate and FIFO width.
    \item \code{sc\_usb.vhd} is an interface to the parallel port of
    the FTDI 2232 USB chip. The register definition is identical to
    the UART and the USB connection can be used as a drop in
    replacement for a UART.
    \item \code{sc\_isa.vhd} interfaces the old ISA bus. It can be used
    for the popular CS8900 Ethernet chip.
    \item \code{sc\_sigdel.vhd} is a configurable sigma-delta ADC
        for an FPGA that needs just two external components: a
        capacitor and a resistor.
    \item \code{sc\_fpu.vhd} provides an interface to the 32-bit FPU available
    at \url{www.opencores.org}.
    \item \code{sc\_arbiter\_*.vhd} different zero cycle SimpCon
        arbiters for CMP systems written by Christof Pitter
        \cite{jop:cmp}.
\end{itemize}

\subsection{Bridges}

\begin{itemize}
    \item \code{sc2wb.vhd} is a SimpCon/Wishbone \cite{soc:wishbone}
    bridge.
    \item \code{sc2avalon.vhd} is a SimpCon/Avalon \cite{soc:avalon}
    bridge to integrate a SimpCon based design with Altera's SOPC
    Builder \cite{quartus}.
    \item \code{sc2ahbsl.vhd} provides an interface to AHB slaves as
    defined in Gaisler's GRLIB \cite{grlib}. Many of the available
    GPL AHB modules from the GRLIB can be used in a SimpCon based
    design.
\end{itemize}


\section{Why a New Interconnection Standard?}

There are many interconnection standards available for SoC designs.
The natural question is: Why propose a new one? The answer is given
in the following section. In summary, the available standards are
still in the tradition of backplane busses and do not fit very well
for pipelined on-chip interconnections.

\subsection{Common SoC Interconnections}


Several point-to-point and bus standards have been proposed. The
following section gives a brief overview of common SoC
interconnection standards.

The Advanced Microcontroller Bus Architecture (AMBA) \cite{soc:amba}
is the interconnection definition from ARM. The specification defines
three different busses: Advanced High-performance Bus (AHB), Advanced
System Bus (ASB), and Advanced Peripheral Bus (APB). The AHB is used
to connect on-chip memory, cache, and external memory to the
processor. Peripheral devices are connected to the APB. A bridge
connects the AHB to the lower bandwidth APB. An AHB bus transfer can
be one cycle at burst operation. With the APB a bus transfer requires
two cycles and no burst mode is available. Peripheral bus cycles with
wait states are added in the version 3 of the APB specification. ASB
is the predecessor of AHB and is not recommended for new designs (ASB
uses both clock phases for the bus signals -- very uncommon for
today's synchronous designs). The AMBA 3 AXI (Advanced eXtensible
Interface) \cite{soc:amba3} is the latest extension to AMBA. AXI
introduces out-of-order transaction completion with the help of a 4
bit transaction ID tag. A ready signal acknowledges the transaction
start. The master has to hold the transaction information (e.g.\
address) until the interconnect signals ready. This enhancement ruins
the elegant single cycle address phase from the original AHB
specification.

Wishbone \cite{soc:wishbone} is a public domain standard used by
several open-source IP cores. The Wishbone interface specification is
still in the tradition of microcomputer or backplane busses. However,
for a SoC interconnect, which is usually
point-to-point,\footnote{Multiplexers are used instead of busses to
connect several slaves and masters.} this is not the best approach.
The master is requested to hold the address and data valid through
the whole read or write cycle. This complicates the connection to a
master that has the data valid only for one cycle. In this case the
address and data have to be registered \emph{before} the Wishbone
connect or an expensive (time and resources) multiplexer has to be
used. A register results in one additional cycle latency. A better
approach would be to register the address and data in the slave. In
that case the address decoding in the slave can be performed in the
same cycle as the address is registered. A similar issue, with
respect to the master, exists for the output data from the slave: As
it is only valid for a single cycle, the data has to be registered by
the master when the master is not reading it immediately. Therefore,
the slave should keep the last valid data at its output even when the
Wishbone strobe signal (\emph{wb.stb}) is not assigned anymore.
Holding the data in the slave is usually \emph{for free} from the
hardware complexity -- it is \emph{just} a specification issue. In
the Wishbone specification there is no way to perform pipelined read
or write. However, for blocked memory transfers (e.g. cache load)
this is the usual way to achieve good performance.


The Avalon \cite{soc:avalon} interface specification is provided by
Altera for a system-on-a-programmable-chip (SOPC) interconnection.
Avalon defines a great range of interconnection devices ranging from
a simple asynchronous interface intended for direct static RAM
connection up to sophisticated pipeline transfers with variable
latencies. This great flexibility provides an easy path to connect a
peripheral device to Avalon. How is this flexibility possible? The
\emph{Avalon Switch Fabric} translates between all those different
interconnection types. The switch fabric is generated by Altera's
SOPC Builder tool. However, it seems that this switch fabric is
Altera proprietary, thus tying this specification to Altera FPGAs.

The On-Chip Peripheral Bus (OPB) \cite{soc:opb} is an open standard
provided by IBM and used by Xilinx. The OPB specifies a bus for
multiple masters and slaves. The implementation of the bus is not
directly defined in the specification. A distributed ring, a
centralized multiplexer, or a centralized AND/OR network are
suggested. Xilinx uses the AND/OR approach and all masters and
slaves must drive the data busses to zero when inactive.


Sonics Inc. defined the Open Core Protocol (OCP) \cite{soc:ocp} as
an open, freely available standard. The standard is now handled by
the OCP International Partnership\footnote{\url{www.ocpip.org}}.




\subsection{What's Wrong with the Classic Standards?}

All SoC interconnection standards, which are widely in use, are still
in the tradition of a backplane bus. They force the master to hold
the address and control signals until the slave provides the data or
acknowledges the write request. However, this is not necessary in a
clocked, synchronous system. Why should we force the master to hold
the signals? Let the master move on after submitting the request in a
single cycle. Forcing the address and control valid for the complete
request disables any form of pipelined requests.


\begin{figure}
    \centering
    \includegraphics[scale=\scgrsc]{\scgrp/wb_basic_rd}
    \caption{Classic basic read transaction}
    \label{fig:wb:basic:rd}
\end{figure}

Figure~\ref{fig:wb:basic:rd} shows a read transaction with wait
states as defined in Wishbone \cite{soc:wishbone}, Avalon
\cite{soc:avalon}, OPB \cite{soc:opb}, and OCP
\cite{soc:ocp}.\footnote{The signal names are different, but the
principle is the same for all mentioned busses.} The master issues
the read request and the address in cycle 1. The slave has to reset
the \sign{ack} in the same cycle. When the slave data is available,
the acknowledge signal is set (\sign{ack} in cycle 3). The master has
to read the data and register them within the same clock cycle. The
master has to hold the address, write data, and control signal active
until the acknowledgement from the slave arrives. For pipelined
reads, the \sign{ack} signal can be split into two signals (available
in Avalon and OCP): one to accept the request and a second one to
signal the available data.

The master is blind about the status of the outstanding transaction
until it is finished. It could be possible that the slave informs the
master in how many cycles the result will be available. This
information can help in building deeply pipelined masters.

Only the AMBA AHB \cite{soc:amba} defines a different protocol. A
single cycle address phase followed by a variable length data phase.
The slave acknowledgement (HREADY) is only necessary in the data
phase avoiding the combinatorial path from address/command to the
acknowledgement. Overlapping address and data phase is allowed and
recommended for high performance. Compared to SimpCon, AMBA AHB
allows for single stage pipelining, whereas SimpCon makes multi-stage
pipelining possible using the ready counter (\sign{rdy\_cnt}). The
\sign{rdy\_cnt} signal defines the delay between the address and the
data on a read, signalled by the slave. Therefore, the pipeline depth
of the bus and the slaves is only limited by the bit width of
\sign{rdy\_cnt}.


Another issue with all interconnection standards is the single cycle
availability of the read data at the slaves. Why not keep the read
data valid as long as there is no new read data available? This
feature would allow the master to be more flexible when to read the
data. It would allow issuing a read command and then continuing with
other instructions -- a feature known as data pre-fetching to hide
long latencies.

The last argument sounds contradictory to the first argument: provide
the transaction data at the master just for a single cycle, but
request the slave to hold the data for several cycles. However, it is
motivated by the need to free up the master, keep it \emph{moving},
and move the data hold (register) burden into the slave. As data
processing bottlenecks are usually found in the master devices, it
sounds natural to move as much work as possible to the slave devices
to free up the master.

Avalon, Wishbone, and OPB provide a single cycle latency access to
slaves due to the possibility of acknowledging a request in the same
cycle. However, this feature is a scaling issue for larger systems.
There is a combinatorial path from master address/command to address
decoding, slave decision on \sign{ack}, slave \sign{ack} multiplexing
back to the master and the master decision to hold address/command or
read the data and continue. Also, the slave output data multiplexer
is on a combinatorial path from the master address.

AMBA, AHB, and SimpCon avoid this scaling issue by requesting the
acknowledge in the cycle following the command. In SimpCon and AMBA,
the select for the read data multiplexer can be registered as the
read address is known at least one cycle before the data is
available. The later acknowledgement results in a minor drawback on
SimpCon and AMBA (nothing is for free): It is not possible to perform
a single cycle read or write without pipelining. A single, non
pipelined transaction takes two cycles without a wait state. However,
a single cycle read transaction is only possible for very simple
slaves. Most non-trivial slaves (e.g.\ memory interfaces) will not
allow a single cycle access anyway.

\subsection{Evaluation}

We compare the SimpCon interface with the AMBA and the Avalon
interface as two examples of common interconnection standards. As an
evaluation example, we interface an external asynchronous SRAM with a
tight timing. The system is clocked at 100~MHz and the access time
for the SRAM is 15~ns. Therefore, there are 5~ns available for
on-chip register to SRAM input and SRAM output to on-chip register
delays. As an SoC, we use an actual low-cost FPGA (Cyclone EP1C6
\cite{AltCyc} and a Cyclone II).

The master is a Java processor (JOP \cite{jop:thesis,
jop:jnl:jsa2007}). The processor is configured with a 4~KB
instruction cache and a 512 byte on-chip stack cache. We run a
complete application benchmark on the different systems. The embedded
benchmark (\emph{Kfl} as described in \cite{jop:austrochip05}) is an
industrial control application already in production.


%For a different benchmark (a UDP/IP application with IP processing -
%lot of buffer access) the difference is larger.

\begin{table}
    \centering

    \begin{tabular}{rll}
        \toprule
        Performance &      Memory     & Interconnect \\
        \midrule
        16,633 &  32 bit SRAM & SimpCon \\

        14,259 &  32 bit SRAM & AMBA \\

        14,015 &  32 bit SRAM & Avalon/PTF \\
        13,920 &  32 bit SRAM & Avalon/VHDL \\
        15,762 & 32 bit on-chip & Avalon\\

        14,760 &  16 bit SRAM & SimpCon \\

        11,322 &  16 bit SRAM & Avalon \\


         7,288  & 16 bit SDRAM & Avalon \\


%        Kfl & UDP/IP   &  Memory     & Interconnect \\
%        \midrule
%        16,633 & 6,537 & 32 bit SRAM & SimpCon \\
%
%        14,015 &       & 32 bit SRAM & Avalon/PTF \\
%        13,920 &       & 32 bit SRAM & Avalon/VHDL \\
%
%        14,760 & 5,716 & 16 bit SRAM & SimpCon \\
%
%        11,322 & 4,302 & 16 bit SRAM & Avalon \\
%        15,762 & --    & 32 bit on-chip & Avalon\\
%
%
%         7,288 & 2,677 & 16 bit SDRAM & Avalon \\


        \bottomrule

    \end{tabular}
    \caption{JOP performance with different interconnection types}
    \label{tab:perf:diff}

%         5,630 & 2,140 &  50 MHz & 16 bit SRAM (relaxed timing) & Avalon \\
%         6,243 & 2,389 &  50 MHz & 16 bit SRAM & Avalon \\

\end{table}

Table~\ref{tab:perf:diff} shows the performance numbers of this
JOP/SRAM interface on the embedded benchmark. It measures iterations
per second and therefore higher numbers are better. One iteration is
the execution of the main control loop of the \emph{Kfl} application.
For a 32 bit SRAM interface, we compare SimpCon against AMBA and
Avalon. SimpCon outperforms AMBA by 17\% and Avalon by
19\%\footnote{The performance is the measurement of the execution
time of the whole application, not only the difference between the
bus transactions.} on a 32 bit SRAM.

The AMBA experiment uses the SRAM controller provided as part of
GRLIB \cite{grlib} by Gaisler Research. We avoided writing our own
AMBA slave to verify that the AMBA implementation on JOP is correct.
To provide a fair comparison between the single master solutions with
SimpCon and Avalon, the AMBA bus was configured without an arbiter.
JOP is connected directly to the AMBA memory slave. The difference
between the SimpCon and the AMBA performance can be explained by two
facts: (1) as with the Avalon interconnect, the master has the
information when the slave request is ready at the same cycle when
the data is available (compared to the \sign{rdy\_cnt} feature); (2)
the SRAM controller is conservative as it asserts \sign{HREADY} one
cycle later than the data is available in the read register
(\sign{HRDATA}). The second issue can be overcome by a better
implementation of the SRAM AMBA slave.


The Avalon experiment considers two versions: an SOPC Builder
generated interface (PTF) to the memory and a memory interface
written in VHDL. The SOPC Builder interface performs slightly better
than the VHDL version that generates the Avalon \sign{waitrequest}
signal. It is assumed that the SOPC Builder version uses fixed wait
states within the switch fabric.

We also implemented an Avalon interface to a single-cycle on-chip
memory. SimpCon is even faster with the 32 bit off-chip SRAM than
with the on-chip memory connected via Avalon. Furthermore, we also
performed experiments with a 16 bit memory interface to the same
SRAM. With this smaller data width the pressure on the
interconnection and memory interface is higher. As a result the
difference between SimpCon and Avalon gets larger (30\%) on the 16
bit SRAM interface. To complete the picture we also measured the
performance with an SDRAM memory connected to the Avalon bus. We see
that the large latency of an SDRAM is a big performance issue for the
Java processor.
