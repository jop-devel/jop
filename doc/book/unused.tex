
\documentclass[%draft,
    11pt, % use explicit paper size
    headinclude, footexclude,
    twoside, % this produces strange margins!
    openright, % for new chapters
    notitlepage,
    cleardoubleempty,
    headsepline,
    pointlessnumbers,
    bibtotoc, idxtotoc,
    ]{scrbook}


% that's for virtualbookworm trim size
%\setlength{\paperwidth}{6in} \setlength{\paperheight}{9in}
% that's for CreateSpace trim size
\setlength{\paperwidth}{7.5in} \setlength{\paperheight}{9.25in}
\usepackage{pslatex} % -- times instead of computer modern
% pslatex should be replaced by this:
%\usepackage{mathptmx}
%\usepackage[scaled=.70]{helvet}
%\usepackage{courier}
% pslatex does not work with T1 encoding. <> Problem?

% typeare{calc} without BCOR results to a DIV of 8 for 11pt
%\typearea[0.465in]{15}
% the decision is 14
\typearea[0.5in]{14}
%\typearea[0.535in]{13}
%\typearea[0.57in]{12}

% headings
\usepackage{scrpage2} % for headers
 \setkomafont{pagehead}{\scshape\small}
 \setkomafont{pagenumber}{\scshape\small}
 \automark[section]{chapter}
 \ohead[]{\pagemark}
 \chead[]{}
 \ihead[]{\headmark}
 \ofoot[]{} \cfoot[]{} \ifoot[]{}

%\tolerance=500 % to avoid lines sticking out into the margin
               % needed for 'high-performance' in Intro - contributions
\emergencystretch=2em
% or tol. to 500 and emerg. to 1em?
% pagebreak was ok with 500 and 1em
\interfootnotelinepenalty=10000


% use BCOR = (paperwidth-textwidth)/4
% A4: 210mm x 297mm
% B5: 176mm x 250mm
% Java book: 185mm x 232mm
% Engblom: 120x188 (without head)
% Java: 127x187 (without head)
% 1pt = 1/72.27 in = 0.351 mm
% Thesis is 232mm x 185mm (9,1in x 7,28in) with 11pt
% Springer is 9,2in x 6in with 10pt
% Altera Quaruts handbook is 9in x 7in with 9pt


% for book
% 'Java-format' 526pt x 660pt (Ghostscript)
%\setlength{\paperwidth}{185mm} \setlength{\paperheight}{232mm}
% use that BCOR setting with twoside to compensate the margin
%\areaset[13.75mm]{130mm}{200mm} % Java book format



% use 10pt for code instead of 11pt - but I still would prefer Lucida Typewriter
%\newfont{\myttfont}{cmss10 scaled 1000}
%\newfont{\myttbfont}{cmssdc10 scaled 1000}
%
% This IS Lucida Typewriter
%\newfont{\myttfont}{plsr8r scaled 950}
%\newfont{\myttbfont}{plsb8r scaled 950}
%\newfont{\myttifont}{plsro8r scaled 950}
%%\newfont{\mytttextfont}{plsr8r}

% Lucida is perhaps available in the new Tex installation!!!!
% does not really work!!!
%\newfont{\myttfont}{hlsrt8r scaled 950}
%\newfont{\myttbfont}{hlsbt8r scaled 950}
%\newfont{\myttifont}{hlsrot8r scaled 950}

% I used these .ttf for the official Thesis
%..\ttf2pt1 -e -b LucidaTypewriterRegular.ttf plsr8a
%..\ttf2pt1 -e -b LucidaTypewriterBold.ttf plsb8a
%..\ttf2pt1 -e -b LucidaTypewriterOblique.ttf plsro8a
%..\ttf2pt1 -e -b LucidaTypewriterBoldOblique.ttf plsbo8a

%\newcommand{\javatt}{\myttfont}
%\newcommand{\javattb}{\myttbfont}
%\newcommand{\javatti}{\myttifont}
%\newcommand{\javatext}{\myttfont}
%
%\newcommand{\picscale}{0.909}
%\newcommand{\excelwidth}{11cm}

% end book

% for B5
%\newfont{\javatt}{cmss10}
%\newfont{\javattb}{cmssdc10}
%\newcommand{\picscale}{0.833}
%\newcommand{\excelwidth}{10cm}


%% was set to 9 for the green book
%% should be ok with 11 for the orange book
%\newfont{\javatt}{cmss11}
%\newfont{\javattb}{cmssdc11}
%% TODO find an italic
%\newfont{\javatti}{cmss11}
%\newcommand{\javatext}{\javatt}

\newcommand{\picscale}{1}
\newcommand{\excelwidth}{12cm}

% for SimpCon description
\newcommand{\sign}[1]{{\code{#1}}}

% for chapter head without a number
% \renewcommand{\chaptermark}[1]{\def\myleftmark{#1}}
% \ihead{\myleftmark} \chead{} \ohead{{\rightmark}}

\setkomafont{captionlabel}{\sffamily\bfseries}

\usepackage{latexsym}
\usepackage{graphicx}
\usepackage{amsmath}
\usepackage{amsthm}
\usepackage{longtable}
\usepackage{booktabs}

% I would need Lucida Console!!!
%
%\newfont{\javatt}{pltt12} % lucida teletype, better than normal but with serifs
%\newfont{\javatt}{plss12} % lucida no serifes, but variable spacing
%\newfont{\javatt}{plss10 scaled 1200}
%\newfont{\javattb}{plssdc10 scaled 1200}
% cmss is NOT a tt font....

\usepackage{listings}
%\lstset{language=Java,keywordstyle=,
%basicstyle=\small\javatt,emphstyle=\small\javattb,commentstyle=\small\javatti,
%%basicstyle=\small,
%%
%showstringspaces=false,captionpos=b,columns=flexible}
%%\lstset{basicstyle=\small,language=Java,columns=flexible}

\lstset{basicstyle=\sffamily\small,keywordstyle=\sffamily\small,language=Java,captionpos=b,columns=flexible,showstringspaces=false}

\usepackage{array}
\usepackage{dcolumn}
\newcommand{\cc}[1]{\multicolumn{1}{c}{#1}}
\newcolumntype{d}[1]{D{.}{.}{#1}}

\usepackage{capt-of}
\usepackage[colorlinks=true,urlcolor=black,linkcolor=black,citecolor=black]{hyperref}

% ----------------------

\usepackage{makeidx}
\makeindex


\usepackage{import} % for subimport text and graphics from subdirectory
% does not work with latex2html!


%\newcommand{\codefoot}{\textsf}
%\newcommand{\code}[1]{{\javatext#1}}
%\newcommand{\codeb}[1]{{\javattb#1}}

\newcommand{\codefoot}{\sffamily}
\newcommand{\code}[1]{{\small\sffamily{#1}}}


%\newcommand{\cmd}[1]{{\texttt{#1}}}
%\newcommand{\dirent}[1]{{\texttt{#1}}}
%\newcommand{\menuitem}[1]{\textsf{\textbf{#1}}}
\newcommand{\cmd}[1]{{\code{#1}}\index{#1}}
\newcommand{\cmdb}[1]{{\textsf{\textbf{#1}}}\index{#1}}
\newcommand{\dirent}[1]{{\code{#1}}}
\newcommand{\menuitem}[1]{\textsf{\textsl{#1}}}

\newcommand{\idx}[1]{#1\index{#1}}

% for flow.tex - part of index helper
\newcommand{\eei}[1]{%
\index{extension!\texttt{#1}}\texttt{#1}}

% JVs et al
%\newcommand{\ea}{et al.\xspace}
\newcommand{\ea}{et al.\ }

% used in GC period calculation
\newtheorem{lemma}{Lemma}
\newtheorem{theorem}{Theorem}

%\begin{htmlonly}
%\renewcommand{\code}[1]{{\texttt{#1}}} % for html2LaTeX
%\newcommand{\toprule}{\hline}
%\newcommand{\midrule}{\hline}
%\newcommand{\bottomrule}{\hline}
%\end{htmlonly}

% net wirklich notwendig -- h�ngt von code generierung ab
%\begin{htmlonly}
%\renewcommand{\javatt}{\texttt}
%\renewcommand{\javattb}{\texttt\bfseries}
%\end{htmlonly}

%\code{\hyphenchar\font=-1}

\newcommand{\mycomment}[1]{}

\newcommand{\instr}[6]{
    \begin{table}
    \index{microcode!{\textsf{#1}}}
        \begin{tabular}{ll}
            \emph{\large\textbf{#1}} & \\
            \\ \\
            \textbf{Operation} & #2 \\ \\
            \textbf{Opcode} & \texttt{#3} \\ \\
            \textbf{Dataflow} & \parbox[t]{10.5cm}{\(#4\)}\\ \\
            \textbf{JVM equivalent} & \parbox[t]{10.5cm}{\code{#5}} \\ \\
            \textbf{Description} & \parbox[t]{10.5cm}{#6}\\
        \end{tabular}
    \end{table}
}


\begin{document}




\chapter{Introduction}
\label{chap:intro}


\emph{Copy some stuff from the thesis and rewrite it.}
    \input{intro/intro_diss}

\subsection{Real-Time Performance}
\label{subsec:rt:perf}

In this section, the implementation of the simple real-time profile
(from Section~\ref{sec:rtprof}) with JOP is compared with the
Reference Implementation (RI) of the RTSJ (see
Section~~\ref{sec:rtsj}) on top of Linux. We use the Linux platform
for the comparison, as it is the only platform for which the RTSJ is
available. The RI is an interpreting implementation of the JVM that
is, however, not optimized for performance. A commercial version of
the RTSJ, JTime by TimeSys, should perform better. However, it was
not possible to get a license of JTime for research purposes. JOP is
implemented in Altera's low-cost Cyclone EP1C6 FPGA, and clocked
with 100MHz. The test results for the RI were obtained on an Intel
Pentium MMX 266MHz, running Linux with two different kernels: a
generic kernel version 2.4.20 and the real-time kernel from TimeSys
\cite{TimeSysLinux}, as recommended for the RI. For each test, 500
measurements were taken. Time was measured using a hardware counter
in JOP and the time stamp counter of the Pentium processor under
Linux.

\subsubsection{Periodic Threads}


Many activities in real-time systems must be performed periodically.
Low release jitter is of major importance for tasks such as control
loops. The test setting is similar to the periodic thread test in
\cite{828497}. A single real-time thread only calls
\code{waitForNextPeriod()} in a loop and records the time between
subsequent calls. A second idle thread, with a lower priority,
merely consumes processing time. This test setting results in two
context switches per period.
\tablename~\ref{tab_results_periodic_jop} shows the average,
standard deviation and extreme values for different period times on
JOP. The same values are shown in
\tablename~\ref{tab_results_periodic_ri} for the RI. Please note
that the values are in $\mu$s for JOP and in ms for the RI.

\begin{table}
    \centering
    \begin{tabular}{rrd{2.0}rr}
        \toprule
        \cc{Period} & \cc{Avg.} & \cc{Std. Dev.} &  \cc{Min.} &   \cc{Max.} \\
        \cc{[$\mu$s]} & \cc{[$\mu$s]} & \cc{[$\mu$s]} & \cc{[$\mu$s]} & \cc{[$\mu$s]} \\
        \midrule
%        50   & 50  & 14 &  36  &  66\\ & pre cache version
        50   & 50  & 13 &  35  &  63\\
        70   & 70  &  0   & 70 & 70\\
        100  & 100  & 0   & 100&  100\\
        500  & 500  & 0   & 500 & 500\\
        1,000 & 1,000&  0   &  1,000 &  1,000\\
        \bottomrule
    \end{tabular}
    \caption{Jitter of periodic threads with JOP}
    \label{tab_results_periodic_jop}
\end{table}



\begin{table}
    \centering
    \begin{tabular}{d{1}d{2}d{2.3}d{3}d{2}}
        \toprule
        \cc{Period} & \cc{Avg.} & \cc{Std. Dev.} &  \cc{Min.} &   \cc{Max.} \\
        \cc{[ms]} & \cc{[ms]} & \cc{[ms]} & \cc{[ms]} & \cc{[ms]} \\
        \midrule
        5  & 4.0    & 7.92  &0.017  &   19.90 \\
        10  & 6.6   & 9.34  &0.019  &   19.94 \\
        20  & 20.0   & 0.015  &   19.87  &   20.14 \\
        35  & 35.0   & 5.001  &   29.75  &   40.25 \\
        50  & 50.0   & 0.018  &   49.95  &   50.06 \\
        100  & 100.0   & 0.002  &   99.94  &   100.1 \\
        \bottomrule
    \end{tabular}
    \caption{Jitter of periodic threads with RI/RTSJ}
    \label{tab_results_periodic_ri}
\end{table}


Using microsecond accurate timer interrupts, programmed by the
scheduler, results in excellent performance of periodic threads in
JOP. No jitter from the scheduler can be seen with a single thread
at periods longer than 70$\mu$s.

The measurement for the RI excludes the first values measured. The
first values are misleading as the RI behaves unpredictably at
\emph{startup}. The RI performs inaccurately at periods below 20ms.
This effect has also been observed in \cite{701668}. Larger periods
that are multiples of 10ms have very low jitter. However, using a
period such as 35ms shows a standard deviation of five ms. A
detailed look into the collected samples only shows values of 30 and
40ms. This implies a timer tick of 10ms in the underlying operating
system. No significant difference is observed when running this test
on the generic Linux kernel and the TimeSys kernel. The commercial
version of the TimeSys Linux kernel should perform better as the
resolution of the timer tick is 1ms and a programmable time can be
used for periodic threads. However, it was not possible to obtain a
license to evaluate the combination of JTime on the commercial Linux
kernel. \tablename~\ref{tab_results_periodic_ri} represents the
measurements on the generic kernel. This comparison shows the
advantage of an adjustable timer interrupt over a fixed timer tick.

\subsubsection{Context Switch}

This test setting consists of two threads. A low priority thread
continuously stores the current time in a shared variable. A high
priority periodic thread measures the time difference between this
value and the time immediately after \code{waitForNextPeriod()}.
\tablename~\ref{tab_results_context} gives the times for the context
switch in processor clock cycles.


\begin{table}
    \centering
    \begin{tabular}{lrd{1}rr}
        \toprule
         & \cc{Avg.} & \cc{Std. Dev.} & \cc{Min.} & \cc{Max.} \\
        \midrule
%        JOP & 2,878 & 7.97 & 2,876 & 2,909 \\
%        JOP & 2,878 & 8 & 2,876 & 2,909 \\ % pre cache version
        JOP & 2,686 & 14 & 2,676 & 2,709 \\
        RI Linux & 4,253 & 1,239 & 3,232 & 19,628 \\
        RI TS Linux & 12,923 & 1,145 & 11,529 & 21,090 \\
        \bottomrule
    \end{tabular}
    \caption{Time for a thread switch in clock cycles}
    \label{tab_results_context}
\end{table}



This test did not produce the expected behavior from the RI on the
generic Linux kernel. When the low priority thread ran in this tight
loop, the high priority thread was not scheduled. After inserting a
\code{Thread.yield()} and an operating system call, such as
\code{System.out.print()}, in this loop, the test performed as
expected. This indicates a major problem in either the RI or the
operating system scheduler. This problem did not occur when the RI
was run on the TimeSys Linux kernel. However, the context switch
time on the TimeSys kernel is three times longer than on the
standard kernel.

\subsubsection{Asynchronous Event Handler}

In this test setting, a high priority event handler is triggered by
a low priority periodic thread. As \code{AsynchEventHandler}
performs poorly in the RI (see \cite{701668}), a
\code{BoundAsynchEventHandler} is used for the RI test program. The
time elapsed between the invocation of \code{fire()} and the first
statement of the event handler was measured.
\tablename~\ref{tab_results_event} shows the elapsed times in clock
cycles for JOP and the RTSJ RI.


\begin{table}
    \centering
    \begin{tabular}{lrd{1}rr}
        \toprule
         & \cc{Avg.} & \cc{Std. Dev.} & \cc{Min.} & \cc{Max.} \\
        \midrule
%        JOP & 2,986 & 7.3 & 2,822 & 2,986 \\
%        JOP & 2,986 & 7 & 2,822 & 2,986 \\ % pre cache version
        JOP & 2,935 & 7 & 2,773 & 2,935 \\
        RI Linux & 53,685 & 7,014 & 47,400 & 87,196 \\
        RI TS Linux & 69,273 & 7,832 & 63,060 & 101,292 \\
        \bottomrule
    \end{tabular}
    \caption{Dispatch latency of event handlers in clock cycles}
    \label{tab_results_event}
\end{table}


The time taken to dispatch an asynchronous event is similar to the
context switch time in JOP. This is to be expected as events are
scheduled and dispatched as threads. The minimum value only occurred
in the first event, all following events having been dispatched in
the maximum time.

In the RI, the dispatch time is about 12 times larger than a context
switch with a significant variation in time. This indicates that the
implementation of \code{fire()} and the communication of the event
to the underlying operating system are not optimal. The time factor
between context switch and event handling on the TimeSys kernel is
lower than on the standard kernel, but is nevertheless still
significant.

\subsubsection{Summary}

In this section, we have compared the RTSJ on top of Linux with the
implementation of a simple real-time profile on top of JOP. The RTSJ
addresses several issues relating to the use of Java for real-time
systems. However, the RTSJ is a specification too large and complex
to be implemented in small embedded systems. We therefore use the
simpler real-time profile for JOP. Tight integration of the
real-time scheduler with the supporting processor results in an
efficient platform for Java in embedded real-time systems. A
performance comparison between this implementation and the RTSJ
showed that a dedicated Java processor without an underlying
operating system is more predictable than trying to adopt a general
purpose OS for real-time systems. Time will show if an
implementation of the RTSJ on a \emph{real} RTOS will outperform the
presented solution.



\end{document}
